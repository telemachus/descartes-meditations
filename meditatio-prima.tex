% [[- Chapter title
\chapter{Meditatio Prima}
% -]] Chapter title

% [[- Meditatio prima

% [[- Introduction
Although his goal is knowledge, Descartes dedicates the first meditation to doubt. Starting from everyday mistakes, such as misjudging the size of a distant object, he rapidly builds towards radical, hyperbolic doubt as he imagines an all-powerful demon whose only goal is to confuse and mislead him. At the end of the first meditation, Descartes is so shaken that he fears he may be trapped forever in the unescapable darkness of error.

However, as Descartes himself says elsewhere, he is not like the sceptics who ``doubt only to doubt and pretend always to be undecided'' (\textbf{AT} VI 29). Descartes believes that he can use doubt and error strategically in order to achieve certainty and truth. Although the first meditation contains some of the most vivid writing and imagery in the work, Descartes would have been very sorry for readers to remember it best. His goal is to overcome doubt rather than to dwell on it.

One tip about what follows. Before each new section of Descartes argument, you will find a paragraph---sometimes more than one---meant to get you thinking. These paragraphs aren't exactly paraphrases or summaries. Sometimes I leave out some parts of what Descartes says, and sometimes I include material from other places. Also the paragraphs act as though \textit{we} were thinking or talking about the issues rather than Descartes. But they should help guide you along the main path of the arguments that Descartes makes, as least as I see it. Good luck and enjoy!

\clearpage
% -]] Introduction

% [[- Background
\begin{center}
    \beginnumbering
    \numberlinefalse
    \pstart
    \textit{De iis quae in dubium revocari possunt}\ledsidenote{17}
    \pend
    \endnumbering
\end{center}

\markright{§1}
\beginnumbering
\pstart
\begin{latin}
    \textenglish{\textbf{1.}} Animadverti iam ante aliquot annos quam multa, ineunte aetate, falsa pro veris admiserim, et quam dubia sint quaecunque istis postea superextruxi, ac proinde funditus omnia semel in vita esse evertenda, atque a primis fundamentis denuo inchoandum, si quid aliquando firmum et mansurum cupiam in scientiis stabilire; sed ingens opus esse videbatur, eamque aetatem expectabam, quae foret tam matura, ut capessendis disciplinis aptior nulla sequeretur. Quare tamdiu cunctatus sum ut deinceps essem in culpa, si quod temporis superest ad agendum, deliberando consumerem. Opportune igitur hodie mentem curis\at{18} omnibus exsolvi, securum mihi otium procuravi, solus secedo, serio tandem et libere generali huic mearum opinionum eversioni vacabo.
\end{latin}
\pend
\endnumbering

\prenotes

\textbf{§1.} Although humans may be ``rational animals'', our childhoods are not famous for their use of reason. In fact, by the time we begin to consciously evaluate our own beliefs---if we ever do---we inevitably have many false ideas that we picked up earlier in life, often unintentionally and unconsciously. If, however, we want to achieve certain knowledge, how can we allow ourselves to rely on such shaky foundations? Perhaps it's necessary to try something extreme: reject all of our beliefs and accept back only the ones that are certain.

\lem{1 Animadverti}, the main verb of the sentence, takes four objects: two indirect questions and two indirect statements. The two indirect questions are (i) \textit{quam multa\dots admiserim} and (ii) \textit{quam dubia sint}. The two indirect statements are (i) \textit{omnia\dots esse evertenda} and (ii) \textit{inchoandum <esse>}.

\lem{1 ante} usually means ``before'', but it can also mean ``ago''. In this use, it is followed by either an ablative or, as here, an accusative indicating how long ago.

\lem{1 ineunte aetate} is an ablative absolute that modifies \textit{admiserim}. Remember that an ablative absolute often stands in place of a clause with temporal, causal, concessive, conditional, or coordinate force (\textbf{AG} §420). In this case, a temporal clause makes the best sense.

\lemc{2 admiserim\dots sint} although the main verb \textit{animadverti} is secondary sequence (\textit{ante aliquot annos} guarantees this), these two verbs are primary sequence. This may be for vividness, but throughout this sentence Descartes shifts his perspective from his past realization and resolution to his present plans. It reads very naturally, but the grammar is a bit hard to pin down.

\lem{2 admiserim} assumes a particular view of belief. A person is thought of as reviewing impressions and then either granting them entrance (i.e., believing them) or turning them away (i.e., not believing them). This suggests that belief is voluntary, or at least subject to the will in some way. All of this is controversial, and Descartes will discuss it later.

\lemc{2 quam dubia sint quaecunque\dots } the relative clause serves as the subject of \textit{sint}. I.e., Descartes realized how doubtful \textit{whatever he built on top of those foundations} is.

\lemc{2 quaecunque istis postea superextruxi} throughout the \textit{Meditations} Descartes uses metaphors drawn from construction and demolition. He represents knowledge and science as buildings that can be firm and lasting or have weak foundations; they can be torn down and built back up again. (Descartes makes heavy use of this metaphor, and the related metaphor of the house, in his \textit{Discours de la méthode} as well. See \cite[22]{curtis1984}.)

\lem{2 quaecunque} = \textit{quaecumque}.

\lem{2 istis} is probably dative with the compound \textit{superextruxi}, and the pronoun refers back to the \textit{falsa} that Descartes believed to be true when he was younger. Descartes continues to use the implicit metaphor of construction.

\lemc{2 superextruxi} this verb does not appear in classical Latin, but \textit{ex(s)truo} does, and compound verbs with \textit{super}- are common in Latin.

\lemc{2 proinde} Descartes doesn't attempt to justify the inference from ``I'm aware I've made mistakes'' to ``I should overturn all my beliefs''. He moves quickly at this point, assuming that the reader is willing to go along to see where the argument leads. 

\lemc{3 esse evertenda\dots inchoandum} infinitives of the passive periphrastic. This periphrastic has the force of obligation, necessity, or propriety (\textbf{AG} §194b).

\lem{5 foret} = \textit{esset}, imperfect subjunctive in a relative clause of characteristic. Note \textit{\textbf{eam} aetatem}: ``\textit{that (kind of)} age that would\dots''.

\lemc{6 cappesendis disciplinis} the dative depends on \textit{aptior}, and the gerundive phrase expresses purpose.

\lemc{7 quod temporis} the genitive is partitive, and the phrase serves as the object of \textit{consumerem}: ``I would be at fault if I wasted what time remains\dots''. There's no antecedent for \textit{quod}. This is common for indefinite antecedents (\textbf{AG} §307c). Fully spelled out the thought is \textit{si illud temporis quod ad agendum superest deliberando consumerem}.

\lem{8 curis omnibus} ablative of separation with \textit{exsolvo}.

\lem{9--10 generali huic\dots eversioni} dative with \textit{vacabo}.

\lem{10 mearum opionum} objective genitive with \textit{eversioni}.

% -]] Background

% [[- A shortcut
\clearpage

\beginnumbering
\pstart
\begin{latin}
    \textenglish{\textbf{2.}} Ad hoc autem non erit necesse ut omnes esse falsas ostendam, quod nunquam fortassis assequi possem; sed quia iam ratio persuadet, non minus accurate ab iis quae non plane certa sunt atque indubitata, quam ab aperte falsis assensionem esse cohibendam, satis erit ad omnes reiiciendas, si aliquam rationem dubitandi in unaquaque reperero. Nec ideo etiam singulae erunt percurrendae, quod operis esset infiniti; sed quia, suffossis fundamentis, quidquid iis superaedificatum est sponte collabitur, aggrediar statim ipsa principia, quibus illud omne quod olim credidi nitebatur.
\end{latin}
\pend
\endnumbering

\prenotes

\textbf{§2.} It would be an impossibly long task to show that each and every one of our beliefs is uncertain, but perhaps this isn't necessary. Since we want absolutely secure knowledge, we should withold assent from anything uncertain just as much as from clear falsehoods. And if we undercut the beliefs at the foundation of our thinking, then any beliefs based on those foundations will collapse as well. Thus, we are quickly on our way to removing all of our beliefs.

\lem{1 hoc} refers to the overturning (\textit{eversio} §1.10) of Descartes' beliefs from the previous paragraph. Descartes uses the neuter rather than the feminine perhaps because he has in mind the general idea rather than the specific word.

\lemc{1 ut\dots ostendam} a substantive result clause, the subject of \textit{erit necesse}.

\lemc{1 quod} neuter like \textit{hoc} because it refers back to the entire clause \textit{ut omnes esse falsas ostendam}.

\lemc{2 possem} effectively the apodosis of a present contrary-to-fact conditional sentence, though the protasis is only implied. ``<Even if I were to try,> \textit{I would never be able\dots}''.

\lemc{3 iis} this form is a common alternative for \textit{eis}, the dative and ablative plural of \textit{is, ea, id} for all genders. Similarly \textit{ii} is an alternative for \textit{ei}, the masculine nominative plural. The dative singular \textit{ei} (all genders) has no such alternative. (These alternatives also appear often in compounds of \textit{is, ea, id}. E.g., \textit{iisdem}.

\lemc{3--4 assensionem esse cohibendam} The technical vocabulary of ``withholding assent'' derives from ancient debates between Sceptics and Stoics. Under normal circumstances, people don't necessarily believe that they should treat anything not overwhelmingly certain and indubitable as if it were clearly false. However, in Descartes's special situation---a single-minded search for truth and absolute certainty---this rule may be justified.

\lemc{5 quod} like the \textit{quod} in §2.11, the neuter refers back to the action of the previous clause \textit{singulae erunt percurrendae}.

\lem{5--6 operis\dots infiniti} is a predicative genitive (\textbf{AG} §343b): ``which thing (i.e., to run through each opinion one by one) would be an infinite task''.

\lemc{6 suffossis fundamentis} given the context, this ablative absolute is best taken as equivalent to a conditional clause.

\lem{6 iis} is dative with \textit{superaedificatum est}.

\lemc{7 quibus} ablative with \textit{nitebatur} (\textbf{AG} §431).

% -]] A shortcut

% [[- Attack the senses first
\clearpage

\beginnumbering
\pstart
\begin{latin}
    \textenglish{\textbf{3.}} Nempe quidquid hactenus ut maxime verum admisi, vel a sensibus, vel per sensus accepi; hos autem interdum fallere deprehendi, ac prudentiae est nunquam illis plane confidere qui nos vel semel deceperunt.
\end{latin}
\pend
\endnumbering

\beginnumbering
\pstart
\begin{latin}
    \textenglish{\textbf{4.}} Sed forte, quamvis interdum sensus circa minuta quaedam et remotiora nos fallant, pleraque tamen alia sunt de quibus dubitari plane non potest, quamvis ab iisdem hauriantur: ut iam me hic esse, foco assidere, hyemali toga esse indutum, chartam istam manibus contrectare, et similia. Manus vero has ipsas, totumque hoc corpus meum esse, qua ratione posset negari? nisi me forte comparem nescio quibus insanis,\at{19} quorum cerebella tam contumax vapor ex atra bile labefactat, ut constanter asseverent vel se esse reges, cum sunt pauperrimi, vel purpura indutos, cum sunt nudi, \edtext{vel caput habere fictile}{\Afootnote{The French translation omits this phrase.}}, vel se totos esse cucurbitas, vel ex vitro conflatos; sed amentes sunt isti, nec minus ipse demens viderer, si quod ab iis exemplum ad me transferrem.
\end{latin}
\pend
\endnumbering

\prenotes

\textbf{§3.} Our most basic beliefs rely on the senses, and so we should attack the senses first. A brisk argument will do: (i) the senses are sometimes wrong and (ii) we want absolute certainty; thus, (iii) we should never trust the senses.

\lem{1 quidquid} functions as the direct object of both \textit{admisi} and \textit{accepi}: ``whatever I admitted\dots I received''.

\lemc{1--2 vel a sensibus vel per sensus} In another work (\textbf{EB} 2), Descartes explains this as a distinction between information received directly from a sense (e.g., seeing a color or shape) and information received indirectly (e.g., learning about things by hearing other people speak).

\lemc{2 hos\dots fallere deprehendi} we can supply \textit{me} or \textit{nos} as the direct object of \textit{fallere}, or we can take it absolutely to mean simply ``deceive, be deceptive''.

\lemc{2 prudentiae} this genitive, which is sometimes called the genitive of characteristic or the genitive of the mark, is a type of possessive genitive (\textbf{AG} §343c). It belongs to wisdom to distrust anything that has been deceptive even once. I.e., it's wise to be suspicious in such a case.

\lem{3 vel} = ``even'', modifying \textit{semel}.

\textbf{§4.} But perhaps this is too fast. Maybe the senses only deceive us in certain kinds of cases. Moreover, anyone who doubts \textit{all} of their sensory beliefs appears to have lost their grip on sanity altogether. Perhaps we are moving away from secure truths, rather than towards them. 

\lem{3 ut} = ``for example, e.g.''. This use of \textit{ut} is a development of its adverbial meaning ``as, in such a manner''. In a sentence like this, the adverb introduces specific cases which exemplify some general point. The indirect statements following \textit{ut} are all examples of the kinds of beliefs that are not easily doubted, although they come from the senses.

\lemc{4--5 Manus\dots posset negari} the two infinitive plus accusative phrases (\textit{Manus has ipsas <meas esse>} and \textit{totum hoc corpus meum esse}) are both subjects of the main verb \textit{posset}. I.e., how could it be denied that these are my hands and that this is my body?

\lemc{4--5 Manus\dots has ipsas, totum\dots hoc corpus} it was already difficult to imagine being wrong about the previous examples, but these last two are meant to be show-stoppers. The repeated \textit{has\dots hoc} implies, as often in Latin, a gesture on the part of the speaker: \textit{these} hands\dots this body---the ones right in front of you. In a similar fashion, G.E. Moore \cite[165--166]{baldwin1993} famously argued as follows:
\begin{quote}
    I can prove now, for instance, that two human hands exist. How? By holding up my two hands, and saying, as I make a certain gesture with the right hand, ``Here is one hand'', and adding, as I make a certain gesture with the left, ``and here is another''.
\end{quote}
Do you think that an argument like this can settle the philosophical problem of scepticism?

\lem{5 qua ratione} = literally ``by what reasoning'', but idiomatically it means ``how''.

\lemc{5--6 nescio quibus} the verb \textit{nescio} can be joined with various words to create indefinite adjectives, pronouns, and adverbs. The verbal part is treated like an indeclinable prefix. Editors disagree over whether to write these forms as one word or two, but there's no difference in meaning either way. We can do a similar thing in English. Imagine someone telling a story and saying, ``Then I-don't-know-who comes along and\dots''. Here \textit{nescio quibus} is an indefinite adjective, modifying \textit{insanis}: ``some crazy people''.

\lemc{6 quorum cerebella\dots labefactat} Black bile is one of the four humors; the other three are blood, yellow bile, and phlegm. The balance or imbalance of these four was thought to control a person's physical and mental well-being.

\lemc{7 asseverent vel\dots vel\dots} these examples are extreme in two ways. First, the beliefs in question are not a little wrong, but wildly wrong. Second, they all involve mistakes about very basic facts, the kind of things that people simply don't make mistakes about, under ordinary circumstances. Hence, this objection poses a serious threat to Descartes' opening argument against the senses. He now needs to answer the challenge that his position is tantamount to insanity.
% -]] Attack the senses first

% [[- Dreams, part 1
\clearpage

\beginnumbering
\pstart
\begin{latin}
\textenglish{\textbf{5.}} Praeclare sane, tanquam non sim homo qui soleam noctu dormire, et eadem omnia in somnis pati, vel etiam interdum minus verisimilia, quam quae isti vigilantes. Quam frequenter vero usitata ista, me hic esse, toga vestiri, foco assidere, quies nocturna persuadet, cum tamen positis vestibus iaceo inter strata! Atqui nunc certe vigilantibus oculis intueor hanc chartam, non sopitum est hoc caput quod commoveo, manum istam prudens et sciens extendo et sentio; non tam distincta contingerent dormienti. Quasi scilicet non recorder a similibus etiam cogitationibus me alias in somnis fuisse delusum; quae dum cogito attentius, tam plane video nunquam certis indiciis vigiliam a somno posse distingui, ut obstupescam, et fere hic ipse stupor mihi opinionem somni confirmet.
\end{latin}
\pend
\endnumbering

\prenotes

\textbf{§5.} And yet what about our dreams? We hope that we are reasonable and not at all like people who imagine that they are have pumpkins for heads, but don't we regularly dream of things just as outrageous as those beliefs? Pick anything you like, anything you're certain of right now. Haven't you dreamed of exactly the same thing, or things just like it, only to realize the next morning that it was ``only a dream''? Pick any belief based on the senses that you like: how can you be sure that you're not merely dreaming it? For example, I see three books on a table in front of me now, but I could easily dream the exact same sight.

\lemc{1 Praeclare sane} these words respond with heavy sarcasm to the argument at the end of the previous section. There is an ellipsis where this phrase's verb would be. Something like ``you argued'' or ``that was argued'' would fill the ellipse, but simply saying ``Oh, brilliant!'' would be a more idiomatic English. An exaggerated eye-roll would help.

\lem{1 Tanquam} = \textit{tamquam}.

\lemc{1 Tanquam non sim} the subjunctive with \textit{tamquam} introduces a comparative clause. Usually the comparison is hypothetical or unreal, and so the subjunctive often appears in these clauses. E.g., you read Latin as if you were a native speaker. Note that the translation should sound counter-factual in English, even though the subjunctive in Latin is present tense rather than imperfect. (For a good discussion of the idiom and the tense, see \textbf{AG} §524, especially Note 2.)

\lem{2 quam} picks up both \textit{eadem} and \textit{minus verisimilia}. ``The same things \textit{as} those people\dots'', and ``things even less likely to be truth \textit{than} what those people\dots''.

\lemc{2--3 isti vigilantes} the pronoun refers back to the \textit{insani} of the previous paragraph; the participle should be taken circumstantially as a temporal clause. I.e., ``when they are awake''. This phrase lacks a verb. We can easily supply one (e.g., ``experience'') or leave it out in English as well.

\lemc{3--4 Quam\dots strata!} the \textit{quam} is exclamatory, modifying \textit{frequenter}: ``How often\dots !''; the subject of the sentence, \textit{quies nocturna} is a periphrasis for ``a night's sleep'', \textit{usitata ista} is the (internal) direct object of \textit{persuadet}, and the three accusative and infinitives are indrect statements in apposition to \textit{usitata ista}. I.e., sleep persuades me of these familiar things, namely that x, y, and z.

\lemc{5 vigilantibus oculis} this phrase can be taken as an ablative of means or an ablative absolute.

\lemc{6 prudens et sciens extendo et sentio} Latin idiom often uses an adjective in the nominative where English idiom would use an adverb or adverbial phrase. So here, ``I extend and feel this hand deliberately (\textit{prudens}) and with full knowledge (\textit{sciens})''.

\lemc{7 contingerent dormienti} the subjunctive here is conditional (TODO: tense?); the participle is dative with a compound verb.

\lemc{7 Quasi scilicet} heavily sarcastic, like ``Praeclare sane'' above.

\lemc{7--8 a similibus\dots cogitationibus} the ablative of means does not normally have a preposition. However, it's unnecessary to insist that the \textit{cogitationes} are personified (and thus ablative of personal agent) or that the phrase must be an ablative of cause. (The prepositions \textit{ab}, \textit{de}, or \textit{ex}, and \textit{in} appear with causal ablatives more often than prepositions appear with clear ablatives of means.)

\lem{8 quae dum cogito} = \textit{et dum haec cogito}. I.e., \textit{quae} is a connective relative (\textbf{AG} §308f).

\lem{9 nunquam} = \textit{numquam}.

\lemc{9--10 ut obstupescam\dots confirmet} result clauses. Note the \textit{tam} in the preceding clause.

\lemc{9 fere} be careful to distinguish this adverb (one ``r'', long final ``e'') from the infinitive \textit{ferre} (two ``r''s, short final ``e''). Although \textit{fere} is placed early in the clause, very likely for emphasis, it modifies the main verb at the end of the clause ``this amazement itself \textit{nearly confirms} in me the belief that I'm sleeping''.

\lemc{10 opinionem somni} the genitive is objective. The phrase ``thought of sleep'' refers to the hypothesis that the meditator may be sleeping. Thus the irony of the final sentence is that the intellectual vertigo (``stupor'') that this argument produces \textit{itself} (``ipse'') strengthens the worry that all our thoughts may be false dreams.
% -]] Dreams, part 1

% [[- Dreams, part 2
\clearpage

\beginnumbering
\pstart
\begin{latin}
\textenglish{\textbf{6.}} Age ergo somniemus, nec particularia ista vera sint, nos oculos aperire, caput movere, manus extendere, nec forte etiam nos habere tales manus, nec tale totum corpus; tamen profecto fatendum est visa per quietem esse veluti quasdam pictas imagines, quae non nisi ad similitudinem rerum verarum fingi potuerunt; ideoque saltem generalia haec, oculos, caput, manus, totumque corpus, res quasdam non imaginarias, sed veras existere. Nam sane pictores ipsi, ne tum qui\at{20}dem, cum Sirenas et Satyriscos maxime inusitatis formis fingere student, naturas omni ex parte novas iis possunt assignare, sed tantummodo diversorum animalium membra permiscent; vel si forte aliquid excogitent adeo novum, ut nihil omnino ei simile fuerit visum, atque ita plane fictitium sit et falsum, certe tamen ad minimum veri colores esse debent, ex quibus illud componant.
\end{latin}
\pend
\endnumbering

\prenotes

\textbf{§6.} In fact, it gets worse. We already agreed that I might be dreaming that I'm eating this meal when in fact I'm sleeping in bed. But what if there are no meals at all? Where can we stop this retreat? Well, at the very least, our dreams must be based on \textit{something}. Painters must use color, even if they avoid representing any specific thing from the real world. In the same way, even if there are no bodies as we imagine there are, there must exist more basic underlying realities that make up these imagined falsehoods.

\lemc{1 Age} the imperative of \textit{agere} is often used idiomatically to introduce a main verb. Very often the main verb is an imperative, and \textit{age}/\textit{agite} can be translated as idiomatically in English as ``C'mon'' or   ``Alright''. Here, the main verb is a concessive use of the jussive subjunctive, and \textit{age} might be translated idiomatically as ``Ok'' or ``Fine''. It introduces the concession that \textit{somniemus} makes.

\lemc{1 somniemus} this subjunctive introduces a command---subjunctives like this are usually called ``hortatory'' or ``jussive''. Idiomatically in Latin such subjunctives often introduce concessions made in the course of an argument (\textbf{AG} §440). In English, we often introduce such concessions with ``Grant that\dots''. Here, Descartes concedes that the previous objection may be true, ``Fine, therefore, grant that we're dreaming''.

\lemc{1 particularia ista} these words point forward to the indirect statements that follow. Those specific things are conceded to be false, namely that x, y, and z.

\lemc{3 tamen profecto} at this point, Descartes draws a line in the sand, so to speak. Even if we grant all the preceding claims, nevertheless the following must be the case.

\lem{3 fatendum est} a passive periphrastic expressing necessity. What must be admitted is the indirect statement that follows: \textit{visa\dots esse}

\lemc{3 visa per quietem} a very elaborate way of saying ``dreams'': the things which appear during sleep.

\lem{4 ad} means roughly ``with an eye towards''.

\lemc{4 rerum verarum} throughout the meditations, Descartes will use the words \textit{verus} and \textit{falsus} to mean both ``true'' and ``false'' but also ``real'' and ``unreal''. This is arguably a mistake, but it was a common thought from ancient Greek philosophy on that truth and reality were in some sense the same thing and thus that falsity and unreality were also the same.

\lemc{5--6 res\dots non imaginarias, sed veras} these phrases are predicate with \textit{existere}. I.e., these very general items exist not as imaginary but as true things.

\lemc{6 Nam sane pictores ipsi\dots} Descartes initially argues that even fantastical creatures of mythology are based in reality insofar as we can see features of everyday animals in them. But then he quickly concedes 

\lemc{6 ne tum quidem} the phrase \textit{ne X quidem} is Latin for ``not even X''. You must remember to adjust the word order for English idiom. Also, keep in mind that \textit{X} can be more than one word, which can make the Latin idiom harder to see.

\lemc{6--7 Sirenas et Satyriscos} two types of mythological creatures. Sirens were bird women whose singing bewitched sailors and led them to their deaths. Satyrs were part goat and part human. These examples support Descartes' claim that even imaginary creatures are just combinations of actual things.

\lemc{9--10 ut\dots fuerit visum\dots sit} subjunctives in a result clause. The tense of \textit{fuerit sit} is surprising. It probably should be merely \textit{sit visum}, a perfect subjunctive, rather than the pluperfect that it actually is.

\lemc{10 ad minimum} just like the English idiom ``at least''.

\lemc{10 veri colores} despite the word order, \textit{colores} is subject, and \textit{veri} is predicate. I.e., the colors should be real.

\lemc{11 ex quibus illud componant} the relative clause describes the \textit{colores}. I.e., ``the colors from which''. \textit{illud} refers back to the entirely new thing (\textit{aliquid\dots novum}) that Descartes concedes that someone might paint. Finally, \textit{componant} is subjunctive because this entire sentence refers to a hypothesis: even if they were to imagine such an entirely new thing, the colors from which \textit{they would compose} it would have to be real. (The subject of \textit{componant} is the hypothetical creative painters.)
% -]] Dreams, part 2

% [[- More basic truths
\clearpage

\beginnumbering
\pstart
\setline{11}
\begin{latin}
    \textenglish{\textbf{6. (cont.)}} Nec dispari ratione, quamvis etiam generalia haec, oculi, caput, manus, et similia, imaginaria esse possent, necessario tamen saltem alia quaedam adhuc magis simplicia et universalia vera esse fatendum est, ex quibus tanquam coloribus veris omnes istae, seu verae seu falsae, quae in cogitatione nostra sunt, rerum imagines effinguntur.
\end{latin}
\pend
\endnumbering

\beginnumbering
\pstart
\begin{latin}
   \textenglish{\textbf{7.}} Cuius generis esse videntur natura corporea in communi, eiusque extensio; item figura rerum extensarum; item quantitas, sive earumdem magnitudo et numerus; item locus in quo existant, tempusque per quod durent, et similia.
\end{latin}
\pend
\endnumbering

\prenotes

\lemc{11 Nec dispari ratione\dots effinguntur} this sentence is rather complex, so it will help to look at its overall structure before reading it more closely. \textit{quamvis\dots possent} is a concession: ``although these general things would be false (on the hypothesis under consideration)''. The main clause follows. \textit{necessario\dots vera esse fatendum est} = ``it must necessarily be admitted that other, more basic, things are real''. The rest of the sentence describes these \textit{alia} further: \textit{ex quibus\dots imagines effinguntur} = ``from which all those images that are in our thinking---whether true or false---are formed''.

\lemc{11 Nec dispari ratione} a litotes that modifies the main verb \textit{fatendum est}. I.e., ``and by exactly the same argument''.

\lemc{12--13 imaginaria esse\dots vera esse} in both these cases, the adjectives are predicative after \textit{esse}. These general things would be imaginary, but it must be admitted that even more basic things must be real.

\textbf{§7.} What might these more basic underlying items be? Well, as a start, the following: corporeality, i.e., the essential physicality of objects; shape; quantity and number; place; and time. Even if all the people---even if people in general---in our thoughts (our dreams?) don't exist, they all have physical natures. These physical natures have shapes, they have sizes, there are specific numbers of them, they exist in different locations, and they exist over time.

\lemc{§7.1 Cuius generis esse videntur natura corporea\dots} this sentence is hard to get into English exactly as it's written. Strictly speaking \textit{Cuius generis} is the predicate following \textit{esse videntur}, and the phrases that follow are all the subject of \textit{videntur}. To make matters even more complicated, \textit{Cuius} is a connective relative. Perhaps something like this, ``The following things seem to be of this type: bodily nature\dots''.

\lemc{1 natura corporea\dots eiusque extensio} the \textit{que} joins together very closely \textit{natura corporea} and \textit{eius extensio}. The nature of physical objects virtually is to be extended things in space, so the \textit{que} may even more more explanatory than connective.

\lemc{2--3 quantitas, sive\dots magnitudo et numerus} the \textit{sive} clause explains \textit{quantitas}. Quantity has two important aspects: size and number.

\lemc{3 et similia} it's unclear to me whether Descartes genuinely has other similar basic features of reality in mind or if this is simply an imprecise phrase.
% -]] More basic truths

% [[- Simple versus composite disciplines
\clearpage

\beginnumbering
\pstart
\begin{latin}
    \textenglish{\textbf{8.}} Quapropter ex his forsan non male concludemus Physicam, Astronomiam, Medicinam, disciplinasque alias omnes, quae a rerum compositarum consideratione dependent, dubias quidem esse; atqui Arithmeticam, Geometriam, aliasque eiusmodi, quae nonnisi de simplicissimis et maxime generalibus rebus tractant, atque utrum eae sint in rerum natura necne, parum curant, aliquid certi atque indubitati continere. Nam sive vigilem, sive dormiam, duo et tria simul iuncta sunt quinque, quadratumque non plura habet latera quam quatuor; nec fieri posse videtur ut tam perspicuae veritates in suspicionem falsitatis incurrant.
\end{latin}
\pend
\endnumbering

\prenotes

\textbf{§8.} So if we want to find absolutely certain truth, perhaps we should stick to these basic realities. Fields such as arithmetic and geometry deal in such facts. Perhaps they are safe. However, we will need to abandon many disciplines which seem fruitful, but rely on claims about higher-level objects---things we can no longer be certain about. The disciplines we must let go include astronomy and medicine.

\lem{1 ex his} refers back to the considerations in the previous paragraphs.

\lemc{1 non male} perhaps a litotes, suggesting that what follows is a very reasonable conclusion. But perhaps Descartes deliberately downplays the strength of this argument.

\lemc{1 concludemus} a potential subjunctive with \textit{forsan}. (The \textit{non} helps to see this since an independent subjuctive expressing a command or a wish would use \textit{ne} as its negative.)

\lemc{1--2 Physicam, Astronomiam, Medicinam} we get a better sense of why Descartes chooses these areas of study from the contrast with arithmetic and geometry in the next part of the sentence. Physics, astronomy, and medicine make claims about the world (\textit{utrum eae sint in rerum natura necne}) while arithmetic and geometry are, in a significant sense, purely theoretical. (Note that early modern physics was far more practical than contemporary physics.)

\lem{4 eiusmodi} = \textit{eius modi}, a genitive of description, ``of this sort''.

\lemc{4--5 quae\dots curant} the structure here is \textit{quae tractant atque parum curant utrum eae sint necne}. That is, \textit{tractant} and \textit{curant} are parallel verbs in the relative clause introduced by \textit{quae}, and \textit{utrum\dots} is an indirect question dependent on \textit{curant}.

\lemc{4 nonnisi} literally means ``not unless''. More idiomatically in English we can translate it here as ``only''. Latin is much more comfortable using such doubly negative phrases. For example, ``nonnulli'', which is literally ``not no people'', is often used to mean ``some people'', ``several people'', or ``a number of people''.

\lemc{5 utrum eae sint in rerum natura necne} first, \textit{utrum\dots necne} introduces a double indirect question: ``whether\dots or not''; second, \textit{sint} is existential here: the question that the composite disciplines care about but arithmetic and geometry do not care about is whether or not their subjects \textit{exist} in the world (\textit{rerum natura}). The idea is that we can consider the properties of addition and subtraction and the nature of triangles and squares without any concern for existing, real things to combine or separate or to measure and examine.

\lemc{5--6 aliquid certi atque indubitati} the genitives are partitive, but in English we would treat them simply as adjectives modifying \textit{aliquid}. That is, ``something certain and without doubt''.

\lemc{6--7 sive vigilem sive dormiam\dots sunt\dots habet} the mood of the verbs reinforces the argument. Descartes cannot say whether he is awake or sleeping, so he uses subjunctives (\textit{vigilem} and \textit{dormiam}) to present the two open possibilities. On the other hand, no matter whether he is asleep or awake, the facts about math and geometry are true. (At least, that's what he's arguing at this point. He will turn around and attack this position soon enough.) Descartes uses indicatives (\textit{sunt} and \textit{habet}) to emphasize the truth of those basic facts.

\lemc{7 quadratumque non plura habet latera quam quatuor} there is a harmless imprecision here. Descartes means to say that a square has precisely four sides not merely that it has no more than four---as if it might possibly have \textit{less} than four.

\lem{7 quatuor} = \textit{quattuor}

\lemc{8 ut tam\dots incurrant} a substantive clause of result and the subject of \textit{videtur}. In English we use ``it'' as a dummy subject, but the true subject is the noun clause introduced by ``that''. So in the sentence ``It seems impossible that such clear truths come into any suspicion of falsehood.'' the ``that'' clause is what seems impossible. (See \cite[218]{huddleston2005}.)
% -]] Simple versus composite disciplines

% [[- A deceptive god?
\clearpage

\beginnumbering
\pstart
\begin{latin}
    \textenglish{\textbf{9.}} \at{21}Verumtamen infixa quaedam est meae menti vetus opinio, Deum esse qui potest omnia, et a quo talis, qualis existo, sum creatus. Unde autem scio illum non fecisse ut nulla plane sit terra, nullum coelum, nulla res extensa, nulla figura, nulla magnitudo, nullus locus, et tamen haec omnia non aliter quam nunc mihi videantur existere? Imo etiam, quemadmodum iudico interdum alios errare circa ea quae se perfectissime scire arbitrantur, ita ego ut fallar quoties duo et tria simul addo, vel numero quadrati latera, vel si quid aliud facilius fingi potest? At forte noluit Deus ita me decipi, dicitur enim summe bonus; sed si hoc eius bonitati repugnaret, talem me creasse ut semper fallar, ab eadem etiam videretur esse alienum permittere ut interdum fallar; quod ultimum tamen non potest dici.
\end{latin}
\pend
\endnumbering

\prenotes

\textbf{§9.} It gets still worse. Up until now we've found three types of possibly secure truths: (i) categories of reality such as corporeality, quantity, space, and time; (ii) very basic disciplines such as mathematics which, at least in their pure forms, may address only items from (i); (iii) very basic facts from the diciplines in (ii) such as that two plus two equals four or that squares have four sides. And yet: an all-powerful but deceptive god might make us think that these three are certain when in fact they are false or don't exist.

It's important to be clear about what we're arguing here. We don't need to prove that there is such a deceptive god, nor that this is likely. All this argument requires is that the existence of such a god is logically possible and that it is consistent with the world seeming as it does to us even though all three types of certainty do not exist. If we meet these requirements, we're back to square one, left with no candidates for certainty at all.

\lemc{1 meae menti} dative with \textit{infixa}.

\lemc{1 Deum esse} an indirect statement dependent on \textit{vetus opinio}.

\lemc{2 potest omnia} supply \textit{facere} with \textit{omnia} as its direct object.

\lemc{2 talis qualis existo sum creatus} in English it's idiomatic to express only one half of the correlative pair \textit{talis qualis}, but in Latin it's idiomatic to express both halves. I.e., god made Descarts ``such as he is'' (in English), but ``of that sort which sort he is'' in Latin.

\lemc{3 fecisse ut} idiomatically means ``to have brought it about that, to have acted (such) that''. The \textit{ut} clause is a substantive clause of result and the direct object of \textit{fecisse}.

\lem{3 coelum} = \textit{caelum}.

\lemc{5 interdum} take this with \textit{errare} rather than \textit{iudico}. Descartes argues that people sometimes make mistakes; he's not concerned here with how often he thinks that they make mistakes.

\lemc{6 ita ego ut fallar} after \textit{ita}, which is correlative with \textit{quemadmodum} above, we must supply something like \textit{unde scio Deum non fecisse} from the previous sentence. Here's a paraphrase of the argument: even worse than the previous possibility (\textit{imo etiam}), how do I know that god doesn't deceive me every time (\textit{quoties}) I think about simple addition or geometry, just like cases where people make mistakes about things that they believe they understand perfectly.

\lemc{7 At forte\dots} Descartes imagines the objection that a supremely good god would not deceive him. In reply Descartes argues that (i) perhaps a supremely good god would not wish Descartes always to be mistaken, but (ii) there appears to be a similar conflict between god's goodness and Descartes sometimes being mistaken, and (iii) it cannot be denied that Descartes is sometimes mistaken. This is not a complete reply to the objection: how might you fill in the steps of the argument?

\lemc{8 dicitur\dots summe bonus} supply \textit{esse} and take \textit{summe bonus} as predicate. I.e., ``he is said (to be) supremely good''.

\lemc{8 hoc} anticipates the indirect statement that follows. Translate ``this, namely that\dots''.

\lemc{ab} Latin idiom says ``foreign from'' while English says ``foreign to''.

\lemc{9 eadem} refers back to \textit{bonitati} in line 8.

\lemc{10 quod ultimum} the wording here is somewhat imprecise. What is the ``final thing'' that cannot be denied? Probably thus: it cannot be said that god never permits Descartes to be mistaken. (This essentially follows the earliest French translation: ``I cannot doubt that he does allow this''.)
% -]] A deceptive god?

% [[- However I came to be, I make mistakes
\clearpage

\beginnumbering
\pstart
\begin{latin}
    \textenglish{\textbf{10.}} Essent vero fortasse nonnulli qui tam potentem aliquem Deum mallent negare, quam res alias omnes credere esse incertas. Sed iis non repugnemus, totumque hoc de Deo demus esse fictitium; at seu fato, seu casu, seu continuata rerum serie, seu quovis alio modo me ad id quod sum pervenisse supponant; quoniam falli et errare imperfectio quaedam esse videtur, quo minus potentem originis meae authorem assignabunt, eo probabilius erit me tam imperfectum esse ut semper fallar. Quibus sane argumentis non habeo quod respondeam, sed tandem cogor fateri nihil esse ex iis quae olim vera putabam, de quo non liceat dubitare, idque non per inconsiderantiam vel levitatem, sed propter validas et meditatas rationes; ideoque etiam ab iisdem, non minus quam ab aperte falsis,\at{22} accurate deinceps assensionem esse cohibendam, si quid certi velim invenire.
\end{latin}
\pend
\endnumbering

\prenotes

\textbf{§10.} Actually our argument doesn't even require a deceiving god. No matter how we were created, we makes mistakes. Therefore even our most basic beliefs can be doubted. The previous argument imagined an all-powerful god who created us and made us subject to error. But the argument won't collapse if someone rejects this premise. We can respond that if a person, force, or entity less powerful than god created us, it's even more likely that we're always mistaken. Hence, the conclusion of the previous paragraph holds even if we deny the existence of an all-powerful creator god.

\lemc{1 Essent} like the apodosis of a present contrary-to-fact conditional sentence, with an implied protasis. ``<If they heard the previous argument>, perhaps there would be some people who\dots''. See §2.2.

\lemc{1 mallent negare quam\dots credere} the verb \textit{malo, malle} often leads to an explicit comparison with \textit{quam}. You can translate as ``prefer to x <rather> than to y'' or ``rather x than y''.

\lemc{2 iis} dative with \textit{repugnemus}. Like the simple verb \textit{pugnare}, \textit{repugnare} tends to be intransitive. The person you resist or fight against goes into the dative or is placed into a prepositional phrase.

\lemc{repugnemus\dots demus} hortatory subjunctives.

\lemc{demus} acts like a verb of speech or thought here and takes an indirect statement since it means ``grant'' or ``concede''.

\lemc{3--6 at seu\dots fallar} this argument can be surprising at first. Descartes argues as follows: (i) making mistakes is a sign of imperfection, (ii) he makes mistakes, therefore (iii) the weaker an origin people attribute to him the more likely that he is always wrong. The argument relies on an unstated and unsupported assumption that a more ``powerful'' (in what sense the origin is powerful is also not explained) origin produces (or can produce) a more perfect creation.

\lemc{4 me ad id quod sum pervenisse} to have ``reached (to) that thing which I am'' is a way of expressing ``to have become what I am''.

\lemc{5--6 quo minus\dots eo probabilius} Latin expresses both halves of the correlation. ``by how much less powerful a creator\dots by that much more probable''. When you read Latin, you should note how both parts of the expression work together. But when you translate into English, you should simplify the phrasing.

\lemc{7 Quibus\dots argumentis} is dative indirect object with \textit{respondeam}. The relative ``quibus'' is a connective relative; translate as if the sentence began \textit{Et his} instead of \textit{Quod}.

\lemc{7 tandem} Descartes finally reaches the goal he set out to achieve in the first paragraph. He believes that he has shown that he has reason to doubt all of his former beliefs.

\lemc{nihil\dots ex iis} expresses a quasi-partitive idea. None of the things that Descartes previously believed to be true was beyond the reach of doubt.

\lemc{8 id} there is no precise syntax for this pronoun in the sentence, but it is perfectly clear what Descartes means. He has just said that that he must admit that there's nothing that he used to believe which is not open to doubt. He continues ``and this not because of lack of reflection or triviality but on account of strong and well thought-out reasons''. We could supply ``I have concluded'' before ``this'', but it's not absolutely necessary in order to grasp the meaning.

\lemc{8--9 non per inconsiderantiam vel levitatem sed propter validas et meditatas rationes} these words echo \textit{Discourse on the Method}, which Descartes published four years earlier. In a context similar to this paragraph, Descartes says that he strives to examine ideas ``not by weak conjectures but by clear and certain reasonings'' (\textbf{AT} VI 29). He stresses this in both works since his conclusions are liable to shock readers.

\lemc{10 iisdem} refers back to \textit{iis quae olim vera putabam}. I.e., everything that Descartes formerly thought was true.

\lemc{10 non minus quam ab aperte falsis} this part of Descartes' argument is rightfully controversial. However, it is essential to remember that he does not recommend this attitude for everyday life but only as part of the pursuit of truth. See the end of this sentence where he adds \textit{si quid certi velim invenire}.

\lemc{10--11 assensionem esse cohibendam} depends on \textit{cogor fateri} and is parallel to \textit{nihil esse\dots dubitare}. Descartes is forced to admit two things: (i) nothing that he formerly believed is beyond doubt and (ii) we should reject doubtful ideas no less than we reject clearly false ideas.

% -]] However I came to be, I make mistakes

% [[- I must counteract the inertia towards belief
\clearpage

\beginnumbering
\pstart
\begin{latin}
    \textenglish{\textbf{11.}} Sed nondum sufficit haec advertisse, curandum est ut recorder; assidue enim recurrunt consuetae opiniones, occupantque credulitatem meam tanquam longo usu et familiaritatis iure sibi devinctam, fere etiam me invito; nec unquam iis assentiri et confidere desuescam, quamdiu tales esse supponam quales sunt revera, nempe aliquo quidem modo dubias, ut iam iam ostensum est, sed nihilominus valde probabiles, et quas multo magis rationi consentaneum sit credere quam negare. Quapropter, ut opinor, non male agam, si, voluntate plane in contrarium versa, me ipsum fallam, illasque aliquandiu omnino falsas imaginariasque esse fingam, donec tandem, velut aequatis utrimque praeiudiciorum ponderibus, nulla amplius prava consuetudo iudicium meum a recta rerum perceptione detorqueat. Etenim scio nihil inde periculi vel erroris interim sequuturum, et me plus aequo diffidentiae indulgere non posse, quandoquidem nunc non rebus agendis, sed cognoscendis tantum incumbo.
\end{latin}
\pend
\endnumbering

\prenotes

\textbf{§11.} Once we have marked all of our beliefs as doubtful, we must strive not to forget this. Belief is an act of will; thus, we believe what we wish to believe. However, we must not forget that there are other forces can lead us astray. For example, we tend to keep believing whatever we already believe. Also, if a belief seems highly probably, we tend to treat it as certain. We need a counter-weight against these potentially deceptive psychological tendencies. Instead of thinking of the relevant beliefs as merely doubtful, we could lie to ourselves and pretend that they are wholly false. No harm will come of this since we are concerned only with knowledge and not with action. That is, we are in no immediate danger even if we deceive ourselves in this way.

\lemc{2 ut recorder} a substantive clause of result and the object of \textit{curandum est}.

\lemc{2 recurrunt\dots occupantque} a military metaphor. Habitual beliefs rush back and seize hold of belief, like soldiers who retreat and then attack again.

\lem{2 tanquam} = \textit{tamquam}. 

\lemc{3 longo usu et familiaritatis iure sibi devinctam} the analogy shifts from fighting soldiers to a social, quasi-legal description. Descartes' belief is ``obliged to'' these recurring beliefs because of ``long use and the law of familiarity''.

\lemc{3 sibi devinctam} the pronoun \textit{sibi} refers back to \textit{opiniones}, and \textit{devinctam} agrees with \textit{credulitatem}. Latin uses the reflextive pronoun since the reference is to the subject of the main verb. However, English would more naturally use ``to them'' not ``to themselves''. (There are a number of situations where Latin uses the third person reflexive pronoun, but English would use a non-reflexive one. The most common is indirect statement. \textit{Caesar se vicisse dixit} is simply ``Caesar said that he had won''. To say ``Caesar said that he himself had won'' in Latin, you would need to add a form of \textit{ipse, ipsa, ipsum}: ``Caesar se ipsum vicisse dixit''.)

\lemc{3 me invito} an ablative absolute, equivalent in meaning to an adverbial phrase like ``against my will'' or ``despite what I want''.

\lem{3 unquam} = \textit{umquam}.

\lemc{4--5 revera} a common phrase, sometimes written as two words \textit{re vera}. The phrase is an ablative of manner. Translate as ``truly'' or ``actually'', but compare the English ``in actual fact'' for the virtual redundancy.

\lemc{5 nempe\dots quidem} work together for emphasis. \textit{nempe} modifies the entire clause, while \textit{quidem} goes closely with \textit{aliquo modo}. Notice them both in the Latin, but if translating you only need one ``certainly'' or ``surely''.

\lemc{5 iam iam} emphatic ``just this very moment'', ``only a moment ago'', or the like.

\lemc{6 quas} functions as the object of both \textit{credere} and \textit{negare}. It goes into the accusative, the default object case, even though the closer verb \textit{credere} is intransitive and takes a dative as object. (In addition, a dative could have confused readers since \textit{consentaneum sit} leads to the dative \textit{rationi}.)

\lemc{7--8 voluntate\dots versa} ablative absolute.

\lemc{8 me ipsum} by itself \textit{me} is reflexive; hence \textit{ipsum} is reflexive. In this case Descartes will fool himself of his own accord rather than let habit lull him back into complancency.

\lem{8 aliquandiu} = \textit{aliquamdiu}. 

\lemc{9 velut\dots ponderibus} an analogy form balancing items in a scale. Descartes will lean too much in one direction intentionally in order to counter-balance a natural human flaw, the tendency to believe what we have always believed.

\lemc{9 praeiudiciorum} in his \textit{Discourse on the Method} (\textbf{AT} VI 18), Descartes warns against two key cognitive dangers: hasty judgment (la précipitation) and prejudice (la prévention).

\lemc{10 prava\dots recta} these two adjectives are natural opposites already in classical Latin: what is \textit{rectus} is straight, proper, good, correct while what is \textit{pravus} is crooked, bent, twisted, wrong. The contrast especially suits Descartes who habitually employs metaphors drawn from roads and travel where one's aim is to go the right way and not deviate from a proper path in order to reach one's goal.

\lem{11 sequuturum} = \textit{secuturum}. \textit{scio} introduces indirect statement, so supply \textit{esse} to form the future active infinitive. (The ommission of \textit{esse} in this infinitive is common in some classical authors as well.)

\lemc{11-13 me\dots incumbo} this is essential for Descartes. It is \textit{impossible} to be too cautious when pursuing truth and knowledge.
% -]] I must counteract the inertia towards belief

% [[- The evil demon
\clearpage

\beginnumbering
\pstart
\begin{latin}
    \textenglish{\textbf{12.}} Supponam igitur non optimum Deum, fontem veritatis, sed genium aliquem malignum, eundemque summe potentem et callidum, omnem suam industriam in eo posuisse, ut me falleret: putabo coelum, aërem, terram, colores, figuras, sonos, cunctaque externa nihil aliud esse quam ludificationes somniorum, quibus insidias credulitati meae tetendit: considerabo\at{23} meipsum tanquam manus non habentem, non oculos, non carnem, non sanguinem, non aliquem sensum, sed haec omnia me habere falso opinantem: manebo obstinate in hac meditatione defixus, atque ita, siquidem non in potestate mea sit aliquid veri cognoscere, at certe hoc quod in me est, ne falsis assentiar, nec mihi quidquam iste deceptor, quantumvis potens, quantumvis callidus, possit imponere, obfirmata mente cavebo. Sed laboriosum est hoc institutum, et desidia quaedam ad consuetudinem vitae me reducit. Nec aliter quam captivus, qui forte imaginaria libertate fruebatur in somnis, cum postea suspicari incipit se dormire, timet excitari, blandisque illusionibus lente connivet: \vvar{sic}{\nth{2}}{hic}{\nth{1}} sponte relabor in veteres opiniones, vereorque expergisci, ne placidae quieti laboriosa vigilia succedens, non in aliqua luce, sed inter inextricabiles iam motarum difficultatum tenebras, in posterum sit degenda.
\end{latin}
\pend
\endnumbering

\prenotes

\textbf{§12.} In order to counteract the force of habit, we can employ a thought experiment. We will imagine that an all-powerful evil demon exists and that the demon uses all of its efforts to deceive us. If this is so, we had better remain far more wary of ordinary belief than we would normally. This thought experiment solves an additional problem for us. We can return to the questions concerning an all-powerful god from §9 without risking censure or anger from religious readers. For a contemporary equivalent to the evil demon, compare the thought experiment that we may be merely ``brains in a vat'' \cite[5]{harman1973}.

\lemc{1 Supponam} a future indicative rather than a hortatory present subjunctive, as the subsequent verbs \textit{putabo}, \textit{considerabot}, \textit{manebo}, and \textit{cavebo} show.

\lemc{3 eo} anticipates the following substantive \textit{ut} clause. The evil demon has worked so hard ``on this: that he deceive me''.

\lemc{4 quibus} ablative of means. All external things are merely dream nonsense ``by means of which'' the demon has set a trap for Descartes.

\lemc{5 insidias\dots tetendit} a metaphor from hunting. The demon has laid out a snare to catch Descartes' gullibility.

\lem{5 meipsum} = \textit{me ipsum}.

\lemc{7 haec omnia me habere} an indirect statement dependent on \textit{opinantem}. The main structure is \textit{considerabo me ipsum tanquam falso opinantem}: ``I will consider myself like (someone) falsely believing''. Then \textit{me} is the subject of \textit{habere}, and \textit{haec omnia} is the direct object of the same verb.

\lemc{8 siquidem\dots cognoscere} at this point in the work, the meditator may or may not discover any certain truths.

\lemc{8--10 at\dots cavebo} the structure of this last sentence is complicated and difficult. The main verb is \textit{cavebo} and its initial object is \textit{hoc} which is continued in the relative clause \textit{quod in me est}. That gives us ``But at least I will guard against this which is in my power''. The two clauses \textit{ne\dots assentiar} and \textit{nec\dots possit imponere} are substantive clauses in apposition to \textit{hoc}, and in sense they are the true direct objects of \textit{cavebo}. ``I will certainly take guard that I not assent\dots and that\dots is not possible''. \textit{nec} before the second clause stands for \textit{et ne} or \textit{neve} rather than the more common \textit{et non}.

\lemc{9--10 nec\dots possit imponere} as we say in English, the demon will ``not be able to put anything over on'' Descartes. In Latin ``to impose something on someone'' (\textit{alicui aliquid imponere}) is an idiom meaning ``to deceive, to cheat, to trick''.

\lemc{9--10 quantumvis potens, quantumvis callidus} modifies the \textit{deceptor}. The deceiver will fail ``however powerful, however clever (he may be)''.

\lemc{10 obfirmata mente} for Descartes freedom from error is an ongoing struggle not simply a state one achieves and then remains in without effort. Hence, in part, the meditative nature of this work.

\lemc{12 Nec aliter quam} litotes for ``exactly like''.

\lemc{12--14 captivus\dots connivet} Descartes, who has worried that his whole life might be a dream, now compares himself to a slave who dreams of freedom and resists waking up. It's familiar that we can be simultaneously waking from a dream and trying to remain in it a while longer. But Descartes makes good use of this analogy here since there appears to be no good option. If Descartes remains asleep, he knowingly lets himself fall into error. But if wakes up, he runs the risk of never finding any certain truth and spending his whole life in the darkness of doubt.

\lemc{14 vereor\dots ne} remember that after a word of fearing, \textit{ne} introduces a clause that the writer or speaker fears will happen. All the negative force is lost. (Clauses expressing what the speaker or writer fears will not happen are introduced by \textit{ut} or sometimes \textit{ne\dots non}.

\lemc{15--16 luce\dots tenebras} the imagery of a future lived in light or shadow suits a work that deliberately invokes the style of religious meditations. Note also that this first meditation ends on a rather ominous cliffhanger.
% -]] The evil demon

% -]] Meditatio prima
