% [[- Chapter title
\chapter{Meditatio Prima}
% -]] Chapter title

% [[- Meditatio prima

% [[- Introduction
Although his goal is knowledge, Descartes dedicates the first meditation to doubt. Starting from everyday mistakes, such as misjudging the size of a distant object, he rapidly builds towards radical, hyperbolic doubt as he imagines an all-powerful demon whose only goal is to confuse and mislead him. At the end of the first meditation, Descartes is so shaken that he fears he may be trapped in the unescapable darkness of error.

However, as Descartes himself says elsewhere, he is not like the sceptics who `doubt only to doubt and pretend always to be undecided' (AT VI 29). Descartes believes that he can use doubt and error strategically in order to achieve certainty and truth. Although the first meditation contains some of the most vivid writing and imagery in the work, Descartes would have been very sorry for readers to remember it best. His goal is to overcome doubt rather than to dwell on it.
% -]] Introduction

% [[- Background
\clearpage
\begin{center}
    \beginnumbering
    \numberlinefalse
    \pstart
    \textit{De iis quae in dubium revocari possunt}\ledsidenote{17}
    \pend
    \endnumbering
\end{center}

\beginnumbering
\pstart
\begin{latin}
    \textenglish{\textbf{1.}} Animadverti iam ante aliquot annos quam multa, ineunte aetate, falsa pro veris admiserim et quam dubia sint quaecunque istis postea superextruxi, ac proinde funditus omnia semel in vita esse evertenda atque a primis fundamentis denuo inchoandum, si quid aliquando firmum et mansurum cupiam in scientiis stabilire; sed ingens opus esse videbatur, eamque aetatem expectabam quae foret tam matura ut capessendis disciplinis aptior nulla sequeretur. Quare tamdiu cunctatus sum ut deinceps essem in culpa, si quod temporis superest ad agendum, deliberando consumerem. Opportune igitur hodie mentem curis\ledsidenote{18} omnibus exsolvi, securum mihi otium procuravi, solus secedo, serio tandem et libere generali huic mearum opinionum eversioni vacabo.
\end{latin}
\pend
\endnumbering

\prenotes

Descartes starts with what can be doubted, but not because he wants to shake our beliefs. He hopes to strengthen our understanding by clearing away anything that is not solid and reliable. After we remove everything that \textit{can} be doubted, what remains \textit{cannot} be doubted. In this way, Descartes intends to use doubt in order to reach certainty.

\lem{§1.1 Animadverti}, the main verb of the sentence, takes four objects: two indirect questions and two indirect statements. The two indirect questions are (i) \textit{quam multa\dots admiserim} and (ii) \textit{quam dubia sint}. The two indirect statements are (i) \textit{omnia\dots esse evertenda} and (ii) \textit{inchoandum <esse>}.

\lem{1 ante} usually means `before', but it can also mean `ago'. In this use, it is followed by either an ablative or, as here, an accusative indicating how long ago.

\lemc{1--2 quam multa\dots quam dubia} as an interrogative adverb, \textit{quam} means `how'.

\lem{1 ineunte aetate} is an ablative absolute that modifies \textit{admiserim}. Remember that an ablative absolute often stands in place of a clause with temporal, causal, concessive, conditional, or coordinate force (\textbf{AG} §420). In this case, a temporal clause makes the best sense.

\lemc{2 admiserim\dots sint} although the main verb \textit{animadverti} is secondary sequence (\textit{ante aliquot annos} guarantees this), these two verbs are primary sequence. This may be for vividness, but throughout this sentence Descartes shifts his perspective from his past realization and resolution to his present plans. It reads very naturally, but the grammar is a bit hard to pin down.

\lem{2 admiserim} assumes a particular view of belief. A person is thought of as reviewing impressions and then either granting them entrance (i.e., believing them) or turning them away (i.e., not believing them). This suggests, though it does not necessarily require, that belief is voluntary, or at least subject to the will in some way. All of this is controversial, and Descartes will discuss it later in the work.

\lemc{2 quam dubia sint quaecunque\dots } the relative clause serves as the subject of \textit{sint}. I.e., Descartes realized how doubtful \textit{whatever he built on top of those foundations} is.

\lemc{2 quaecunque istis postea superextruxi} throughout the \textit{Meditations} Descartes uses metaphors drawn from construction and demolition. He represents knowledge and science as buildings that can be firm and lasting or have weak foundations; they can be torn down and built back up again. (Descartes makes heavy use of this metaphor, and the related metaphor of the house, in his \textit{Discours de la méthode} as well. See \citet[22]{curtis1984}.)

\lem{2 quaecunque} = \textit{quaecumque}.

\lem{2 istis} is probably dative with the compound \textit{superextruxi}, and the pronoun refers back to the \textit{falsa} that Descartes believed to be true when he was younger. Descartes continues to use the implicit metaphor of construction.

\lemc{2 superextruxi} this verb does not appear in classical Latin, but \textit{ex(s)truo} does, and compound verbs with \textit{super}- are common in Latin.

\lemc{2 proinde} Descartes doesn't attempt to justify the inference from `I'm aware I've made mistakes' to `I should overturn all my beliefs'. He moves quickly at this point, assuming that the reader is willing to go along to see where the argument leads. 

\lemc{3 esse evertenda\dots inchoandum} infinitives of the passive periphrastic. This periphrastic has the force of obligation, necessity, or propriety (\textbf{AG} §194b).

\lem{5 foret} = \textit{esset}, imperfect subjunctive in a relative clause of characteristic. Note \textit{\textbf{eam} aetatem}: `\textit{that (kind of)} age that would\dots'.

\lemc{6 cappesendis disciplinis} the dative depends on \textit{aptior}, and the gerundive phrase expresses purpose.

\lemc{7 quod temporis} the genitive is partitive, and the phrase serves as the object of \textit{consumerem}: `I would be at fault if I wasted what time remains\dots'. There's no antecedent for \textit{quod}. This is common for indefinite antecedents (\textbf{AG} §307c). Fully spelled out the thought is \textit{si illud temporis quod ad agendum superest deliberando consumerem}.

\lem{8 curis omnibus} ablative of separation with \textit{exsolvo}.

\lem{9--10 generali huic\dots eversioni} dative with \textit{vacabo}.

\lem{10 mearum opionum} objective genitive with \textit{eversioni}.

% -]] Background

% [[- A shortcut
\clearpage

\beginnumbering
\pstart
\begin{latin}
    \textenglish{\textbf{2.}} Ad hoc autem non erit necesse ut omnes esse falsas ostendam---quod nunquam fortassis assequi possem; sed quia iam ratio persuadet non minus accurate ab iis quae non plane certa sunt atque indubitata quam ab aperte falsis assensionem esse cohibendam, satis erit ad omnes reiiciendas si aliquam rationem dubitandi in unaquaque reperero. Nec ideo etiam singulae erunt percurrendae---quod operis esset infiniti; sed quia, suffossis fundamentis, quidquid iis superaedificatum est sponte collabitur, aggrediar statim ipsa principia quibus illud omne quod olim credidi nitebatur.
\end{latin}
\pend
\endnumbering

\prenotes

Descartes wishes to test all of his beliefs, but he would rather not have to examine each of them. In order to speed things us, he argues that (i) assent should be withheld from anything uncertain just as much as from clear falsehoods and (ii) if he can undercut the beliefs at the foundation of his thinking, then any beliefs based on those foundations will collapse as well. In these two ways Descartes hopes to make his task easier.

\lem{§2.1 hoc} refers to the overturning (\textit{eversio} §1.10) of Descartes' beliefs from the previous paragraph. Descartes uses the neuter rather than the feminine perhaps because he has in mind the general idea rather than the specific word.

\lemc{1 ut\dots ostendam} a substantive result clause, the subject of \textit{erit necesse}.

\lemc{1 quod} neuter like \textit{hoc} because it refers back to the entire clause \textit{ut omnes esse falsas ostendam}.

\lemc{2 possem} effectively the apodosis of a present contrary-to-fact conditional sentence, though the protasis is only implied. `<Even if I were to try,> \textit{I would never be able\dots}'.

\lemc{3 iis} this form is a common alternative for \textit{eis}, the dative and ablative plural of \textit{is, ea, id} for all genders. Similarly \textit{ii} is an alternative for \textit{ei}, the masculine nominative plural. The dative singular \textit{ei} (all genders) has no such alternative. (These alternatives also appear often in compounds of \textit{is, ea, id}. E.g., \textit{iisdem}.

\lemc{3--4 assensionem esse cohibendam} The technical vocabulary of `withholding assent' derives from ancient debates between Sceptics and Stoics. Under normal circumstances, people don't necessarily believe that they should treat anything not overwhelmingly certain and indubitable as if it were clearly false. However, in Descartes' special situation---a single-minded search for truth and absolute certainty---this rule may be justified.

\lemc{5 quod} like the \textit{quod} in §2.11, the neuter refers back to the action of the previous clause \textit{singulae erunt percurrendae}.

\lem{5--6 operis\dots infiniti} is a predicative genitive (\textbf{AG} §343b): `which thing (i.e., to run through each opinion one by one) would be an infinite task'.

\lemc{6 suffossis fundamentis} given the context, this ablative absolute is best taken as equivalent to a conditional clause.

\lem{6 iis} is dative with \textit{superaedificatum est}.

\lemc{7 quibus} ablative with \textit{nitebatur} (\textbf{AG} §431).

% -]] A shortcut

% [[- Attack the senses first
\clearpage

\beginnumbering
\pstart
\begin{latin}
    \textenglish{\textbf{3.}} Nempe quidquid hactenus ut maxime verum admisi, vel a sensibus vel per sensus accepi; hos autem interdum fallere deprehendi, ac prudentiae est nunquam illis plane confidere qui nos vel semel deceperunt.
\end{latin}
\pend
\endnumbering

\beginnumbering
\pstart
\begin{latin}
    \textenglish{\textbf{4.}} Sed forte, quamvis interdum sensus circa minuta quaedam et remotiora nos fallant, pleraque tamen alia sunt de quibus dubitari plane non potest, quamvis ab iisdem hauriantur: ut iam me hic esse, fovo assidere, hyemali toga esse indutum, chartam istam manibus contrectare, et similia. Manus vero has ipsas, totumque hoc corpus meum esse, qua ratione posset negari? nisi me forte comparem nescio quibus insanis,\ledsidenote{19} quorum cerebella tam contumax vapor ex atra bile labefactat ut constanter asseverent vel se esse reges, cum sunt pauperrimi, vel purpura indutos, cum sunt nudi, \edtext{vel caput habere fictile}{\Afootnote{The French translation omits this phrase.}}, vel se totos esse cucurbitas, vel ex vitro conflatos; sed amentes sunt isti, nec minus ipse demens viderer, si quod ab iis exemplum ad me transferrem.
\end{latin}
\pend
\endnumbering

\prenotes

Descartes proposes a shortcut to the kind of radical scepticism he wants to reach: undermine the senses, which are at the base of all our beliefs, and everything based on the senses will be undermined at the same time. His argument against the senses is brisk: they sometimes deceive, and therefore they can never be trusted. However, Descartes then raises an objection that even if \textit{some} sense-based beliefs can deceive us, others are so basic that they can never be doubted. Indeed to doubt certain very basic beliefs that derive from the senses would be tantamount to giving up reason altogether---or so claims the objection.

\lem{§3.1 quidquid} functions as the direct object of both \textit{admisi} and \textit{accepi}: `whatever I admitted\dots I received'.

\lemc{1--2 vel a sensibus vel per sensus} In another work (\textbf{EB} 2), Descartes explains this as a distinction between information received directly from a sense (e.g., seeing a color or shape) and information received indirectly (e.g., learning about things by hearing other people speak).

\lemc{2 hos\dots fallere deprehendi} we can supply \textit{me} or \textit{nos} as the direct object of \textit{fallere}, or we can take it absolutely to mean simply `deceive, be deceptive'.

\lemc{2 prudentiae} this genitive, which is sometimes called the genitive of characteristic or the genitive of the mark, is a type of possessive genitive (\textbf{AG} §343c). It belongs to wisdom to distrust anything that has been deceptive even once. I.e., it's wise to be suspicious in such a case.

\lem{3 vel} = `even', modifying \textit{semel}.

\lem{§4.3 ut} = `for example, e.g.'. This use of \textit{ut} is a development of its adverbial meaning `as, in such a manner'. In a sentence like this, the adverb introduces specific cases which exemplify some general point. The indirect statements following \textit{ut} are all examples of the kinds of beliefs that are not easily doubted, although they come from the senses.

\lemc{4--5 Manus\dots posset negari} the two infinitive plus accusative phrases (\textit{Manus has ipsas <meas esse>} and \textit{totum hoc corpus meum esse}) are both subjects of the main verb \textit{posset}. I.e., how could it be denied that these are my hands and that this is my body?

\lemc{4--5 Manus\dots has ipsas, totum\dots hoc corpus} it was already difficult to imagine being wrong about the previous examples, but these last two are meant to be show-stoppers. The repeated \textit{has\dots hoc} implies, as often in Latin, a gesture on the part of the speaker: \textit{these} hands\dots this body---the ones right in front of you. In a similar fashion, G.E. Moore \citep[165--166]{baldwin1993} famously argued as follows:
\begin{quote}
    I can prove now, for instance, that two human hands exist. How? By holding up my two hands, and saying, as I make a certain gesture with the right hand, `Here is one hand', and adding, as I make a certain gesture with the left, `and here is another'.
\end{quote}
Do you think that an argument like this can settle the philosophical problem of scepticism?

\lem{5 qua ratione} = literally `by what reasoning', but idiomatically it means `how'.

\lemc{5--6 nescio quibus} the verb \textit{nescio} can be joined with various words to create indefinite adjectives, pronouns, and adverbs. The verbal part is treated like an indeclinable prefix. Editors disagree over whether to write these forms as one word or two, but there's no difference in meaning either way. We can do a similar thing in English. Imagine someone telling a story and saying, `Then I-don't-know-who comes along and\dots'. Here \textit{nescio quibus} is an indefinite adjective, modifying \textit{insanis}: `some crazy people'.

\lemc{6 quorum cerebella\dots labefactat} Black bile is one of the four humors; the other three are blood, yellow bile, and phlegm. The balance or imbalance of these four was thought to control a person's physical and mental well-being.

\lemc{7 asseverent vel\dots vel\dots} these examples are extreme in two ways. First, the beliefs in question are not a little wrong, but wildly wrong. Second, they all involve mistakes about very basic facts, the kind of things that people simply don't make mistakes about, under ordinary circumstances. Hence, this objection poses a serious threat to Descartes' opening argument against the senses. He now needs to answer the challenge that his position is tantamount to insanity.
% -]] Attack the senses first

% [[- Dreams, part 1
\clearpage

\beginnumbering
\pstart
\begin{latin}
\textenglish{\textbf{5.}} Praeclare sane! Tanquam non sim homo qui soleam noctu dormire et eadem omnia in somnis pati, vel etiam interdum minus verisimilia, quam quae isti vigilantes. Quam frequenter vero usitata ista, me hic esse, toga vestiri, foco assidere, quies nocturna persuadet, cum tamen positis vestibus iaceo inter strata! Atqui nunc certe vigilantibus oculis intueor hanc chartam, non sopitum est hoc caput quod commoveo, manum istam prudens et sciens extendo et sentio; non tam distincta contingerent dormienti. Quasi scilicet non recorder a similibus etiam cogitationibus me alias in somnis fuisse delusum; quae dum cogito attentius, tam plane video nunquam certis indiciis vigiliam a somno posse distingui ut obstupescam, et fere hic ipse stupor mihi opinionem somni confirmet.
\end{latin}
\pend
\endnumbering

\prenotes

Descartes introduces dreaming in order to blunt the force of the previous objection. The argument has become complex, so let's restate where we are. Descartes hopes to undermine a vast number of our beliefs by attacking the senses as unreliable. However, he considers an objection that there is a class of sense-based beliefs (such as that that these are my hands and this is my body) that only insane people would doubt, even if some other sense-based beliefs are unreliable. In this paragraph, Descartes responds to that defense of the senses by arguing that, actually, dreams show that even such basic sense-based beliefs cannot be relied upon.

\lemc{§5.1 Praeclare sane} these words respond with heavy sarcasm to the argument at the end of the previous section. There is an ellipsis where this phrase's verb would be. Something like `you argued' or `that was argued' would fill the ellipse, but simply saying `Oh, brilliant!' would be a more idiomatic English. A blantant eye-roll would help.

\lem{1 Tanquam} = \textit{tamquam}.

\lemc{1 Tanquam non sim} the subjunctive with \textit{tamquam} introduces a comparative clause. Usually the comparison is hypothetical or unreal, and so the subjunctive often appears in these clauses. E.g., you read Latin as if you were a native speaker. Note that the translation should sound counter-factual in English, even though the subjunctive in Latin is present tense rather than imperfect. (For a good discussion of the idiom and the tense, see \textbf{AG} §524, especially Note 2.)

\lem{2 quam} picks up both \textit{eadem} and \textit{minus verisimilia}. `The same things \textit{as} those people\dots', and `things even less likely to be truth \textit{than} what those people\dots'.

\lemc{2--3 isti vigilantes} the pronoun refers back to the \textit{insani} of the previous paragraph; the participle should be taken circumstantially as a temporal clause. I.e., `when they are awake'. This phrase lacks a verb. We can easily supply one (e.g., `experience') or leave it out in English as well.

\lemc{3--4 Quam\dots strata!} the \textit{quam} is exclamatory, modifying \textit{frequenter}: `How often\dots !'; the subject of the sentence, \textit{quies nocturna} is a periphrasis for `a night's sleep', \textit{usitata ista} is the (internal) direct object of \textit{persuadet}, and the three accusative and infinitives are indrect statements in apposition to \textit{usitata ista}. I.e., sleep persuades me of these familiar things, namely that x, y, and z.

\lemc{5 vigilantibus oculis} this phrase can be taken as an ablative of means or an ablative absolute.

\lemc{6 prudens et sciens extendo et sentio} Latin idiom often uses an adjective in the nominative where English idiom would use an adverb or adverbial phrase. So here, `I extend and feel this hand deliberately (\textit{prudens}) and with full knowledge (\textit{sciens})'.

\lemc{7 contingerent dormienti} the subjunctive here is conditional (TODO: tense?); the participle is dative with a compound verb.

\lemc{7 Quasi scilicet} heavily sarcastic, like `Praeclare sane' above.

\lemc{7--8 a similibus\dots cogitationibus} the ablative of means does not normally have a preposition. However, it's unnecessary to insist that the \textit{cogitationes} are personified (and thus ablative of personal agent) or that the phrase must be an ablative of cause. (The prepositions \textit{ab}, \textit{de}, or \textit{ex}, and \textit{in} appear with causal ablatives more often than prepositions appear with clear ablatives of means.)

\lem{8 quae dum cogito} = \textit{et dum haec cogito}. I.e., \textit{quae} is a connective relative (\textbf{AG} §308f).

\lem{9 nunquam} = \textit{numquam}.

\lemc{9--10 ut obstupescam\dots confirmet} result clauses. Note the \textit{tam} in the preceding clause.

\lemc{9 fere} be careful to distinguish this adverb (one `r', long final `e') from the infinitive \textit{ferre} (two `r's, short final `e'). Although \textit{fere} is placed early in the clause, very likely for emphasis, it modifies the main verb at the end of the clause `this amazement itself \textit{nearly confirms} in me the belief that I'm sleeping'.

\lemc{10 opinionem somni} the genitive is objective. The phrase `thought of sleep' refers to the hypothesis that the meditator may be sleeping. Thus the irony of the final sentence is that the intellectual vertigo (`stupor') that this argument produces \textit{itself} (`ipse') strengthens the worry that all our thoughts may be false dreams.
% -]] Dreams, part 1

% [[- Dreams, part 2
\clearpage

\beginnumbering
\pstart
\begin{latin}
\textenglish{\textbf{6.}} Age ergo somniemus, nec particularia ista vera sint: nos oculos aperire, caput movere, manus extendere, nec forte etiam nos habere tales manus, nec tale totum corpus; tamen profecto fatendum est visa per quietem esse veluti quasdam pictas imagines, quae non nisi ad similitudinem rerum verarum fingi potuerunt; ideoque saltem generalia haec---oculos, caput, manus, totumque corpus---res quasdam non imaginarias, sed veras existere. Nam sane pictores ipsi, ne tum qui\at{20}dem, cum Sirenas et Satyriscos maxime inusitatis formis fingere student, naturas omni ex parte novas iis possunt assignare, sed tantummodo diversorum animalium membra permiscent; vel si forte aliquid excogitent adeo novum ut nihil omnino ei simile fuerit visum, atque ita plane fictitium sit et falsum, certe tamen ad minimum veri colores esse debent ex quibus illud componant.
\end{latin}
\pend
\endnumbering

\prenotes

This paragraph takes a surprising turn and concedes more to the sceptic than many readers will expect. At first Descartes seems prepared to argue that even if this or that belief about, say, our hands or head is false, nevertheless even our false beliefs start from the truth that we have hands and heads. However, Descartes does not dig in and insist on this after all. Instead, he grants that we might not even have a body as we normally think of it, but that our false beliefs must rely on some more basic reality. Descartes uses an analogy here: however abstract a painting may be, nevertheless it must rely on colors; in the same way, however false our beliefs may be, they must rely on some truths. This analogy may trouble readers rather than comfort them since it appears to grant far too much to the sceptic.

\lemc{§6.1 Age} the imperative of \textit{agere} is often used idiomatically to introduce a main verb. Very often the main verb is an imperative, and \textit{age}/\textit{agite} can be translated as idiomatically in English as `C'mon' or `Alright'. Here, the main verb is a concessive use of the jussive subjunctive, and \textit{age} might be translated idiomatically as `Ok' or `Fine'. It introduces the concession that \textit{somniemus} makes.

\lemc{1 somniemus} this subjunctive introduces a command---subjunctives like this are usually called `hortatory' or `jussive'. Idiomatically in Latin such subjunctives often introduce concessions made in the course of an argument (\textbf{AG} §440). In English, we often introduce such concessions with `Grant that\dots'. Here, Descartes concedes that the previous objection may be true, `Fine, therefore, grant that we're dreaming'.

\lemc{1 particularia ista} these words point forward to the indirect statements that follow. Those specific things are conceded to be false, namely that x, y, and z.

\lemc{3 tamen profecto} at this point, Descartes draws a line in the sand, so to speak. Even if we grant all the preceding claims, nevertheless the following must be the case.

\lem{3 fatendum est} a passive periphrastic expressing necessity. What must be admitted is the indirect statement that follows: \textit{visa\dots esse}

\lemc{3 visa per quietem} a very elaborate way of saying `dreams': the things which appear during sleep.

\lem{4 ad} means roughly `with an eye towards'.

\lemc{4 rerum verarum} throughout the meditations, Descartes will use the words \textit{verus} and \textit{falsus} to mean both `true' and `false' but also `real' and `unreal'. This is arguably a mistake, but it was a common thought from ancient Greek philosophy on that truth and reality were in some sense the same thing and thus that falsity and unreality were also the same.

\lemc{5--6 res\dots non imaginarias, sed veras} these phrases are predicate with \textit{existere}. I.e., these very general items exist not as imaginary but as true things.

\lemc{6 Nam sane pictores ipsi\dots} Descartes initially argues that even fantastical creatures of mythology are based in reality insofar as we can see features of everyday animals in them. But then he quickly concedes 

\lemc{6 ne tum quidem} the phrase \textit{ne X quidem} is Latin for `not even X'. You must remember to adjust the word order for English idiom. Also, keep in mind that \textit{X} can be more than one word, which can make the Latin idiom harder to see.

\lemc{6--7 Sirenas et Satyriscos} two types of mythological creatures. Sirens were bird women whose singing bewitched sailors and led them to their deaths. Satyrs were part goat and part human. These examples support Descartes' claim that even imaginary creatures are just combinations of actual things.

\lemc{9--10 ut\dots fuerit visum\dots sit} subjunctives in a result clause. The tense of \textit{fuerit sit} is surprising. It probably should be merely \textit{sit visum}, a perfect subjunctive, rather than the pluperfect that it actually is.

\lemc{10 ad minimum} just like the English idiom `at least'.

\lemc{10 veri colores} despite the word order, \textit{colores} is subject, and \textit{veri} is predicate. I.e., the colors should be real.

\lemc{11 ex quibus illud componant} the relative clause describes the \textit{colores}. I.e., `the colors from which'. \textit{illud} refers back to the entirely new thing (\textit{aliquid\dots novum}) that Descartes concedes that someone might paint. Finally, \textit{componant} is subjunctive because this entire sentence refers to a hypothesis: even if they were to imagine such an entirely new thing, the colors from which \textit{they would compose} it would have to be real. (The subject of \textit{componant} is the hypothetical creative painters.)
% -]] Dreams, part 2

% [[- More basic truths
\clearpage

\beginnumbering
\pstart
\setline{11}
\begin{latin}
    \textenglish{\textbf{6. (cont.)}} Nec dispari ratione, quamvis etiam generalia haec---oculi, caput, manus, et similia---imaginaria esse possent, necessario tamen saltem alia quaedam adhuc magis simplicia et universalia vera esse fatendum est ex quibus tanquam coloribus veris omnes istae---seu verae seu falsae---quae in cogitatione nostra sunt rerum imagines effinguntur.
\end{latin}
\pend
\endnumbering

\beginnumbering
\pstart
\begin{latin}
   \textenglish{\textbf{7.}} Cuius generis esse videntur natura corporea in communi eiusque extensio; item figura rerum extensarum; item quantitas, sive earumdem magnitudo et numerus; item locus in quo existant, tempusque per quod durent, et similia.
\end{latin}
\pend
\endnumbering

\prenotes

\lemc{§6.11 Nec dispari ratione\dots effinguntur} this sentence is rather complex, so it will help to look at its overall structure before reading it more closely. \textit{quamvis\dots possent} is a concession: `although these general things would be false (on the hypothesis under consideration)'. The main clause follows. \textit{necessario\dots vera esse fatendum est} = `it must necessarily be admitted that other, more basic, things are real'. The rest of the sentence describes these \textit{alia} further: \textit{ex quibus\dots imagines effinguntur} = `from which all those images that are in our thinking---whether true or false---are formed'.

\lemc{11 Nec dispari ratione} a litotes that modifies the main verb \textit{fatendum est}. I.e., `and by exactly the same argument'.

\lemc{12--13 imaginaria esse\dots vera esse} in both these cases, the adjectives are predicative after \textit{esse}. These general things would be imaginary, but it must be admitted that even more basic things must be real.

Descartes lists here the qualities that are as basic to mental images as color is to painting. The list includes corporeality, i.e., the essential physicality of objects, shape, quantity and number, place, and time. Descartes does not discuss or explain the precise nature of these features because they are not his primary concern at this point of the meditations. He is not, à la Kant, trying to trace out the limits and structure of our thinking. Descartes is searching for something certain, and it will turn out that these alleged prerequisites of thought cannot be what he is looking for. Or at least so Descartes will argue.

\lemc{§7.1 Cuius generis esse videntur natura corporea\dots} this sentence is hard to get into English exactly as it's written. Strictly speaking \textit{Cuius generis} is the predicate following \textit{esse videntur}, and the phrases that follow are all the subject of \textit{videntur}. To make matters even more complicated, \textit{Cuius} is a connective relative. Perhaps something like this, `The following things seem to be of this type: bodily nature\dots'.

\lemc{1 natura corporea\dots eiusque extensio} the \textit{que} joins together very closely \textit{natura corporea} and \textit{eius extensio}. The nature of physical objects virtually is to be extended things in space, so the \textit{que} may even more more explanatory than connective.

\lemc{2--3 quantitas, sive\dots magnitudo et numerus} the \textit{sive} clause explains \textit{quantitas}. Quantity has two important aspects: size and number.

\lemc{3 et similia} it's unclear to me whether Descartes genuinely has other similar basic features of reality in mind or if this is simply an imprecise phrase.
% -]] More basic truths

% -]] Meditatio prima
