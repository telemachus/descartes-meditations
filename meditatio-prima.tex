% [[- Chapter title
\chapter{Meditatio Prima}
% -]] Chapter title

% [[- Meditatio prima

% [[- Introduction
Although his goal is knowledge, Descartes dedicates the first meditation to doubt. Starting from everyday mistakes, such as misjudging the size of a distant object, he rapidly builds towards radical, hyperbolic doubt as he imagines an all-powerful demon whose only goal is to confuse and mislead him. At the end of the first meditation, Descartes is so shaken that he fears he may be trapped in the unescapable darkness of error.

However, as Descartes himself says elsewhere, he is not like the sceptics who ``doubt only to doubt and pretend always to be undecided'' (AT VI 29). Descartes believes that he can use doubt and error strategically in order to achieve certainty and truth. Although the first meditation contains some of the most vivid writing and imagery in the work, Descartes would have been very sorry for readers to remember it best. His goal is to overcome doubt rather than to dwell on it.
% -]] Introduction

% [[- Background
\clearpage
\begin{center}
    \beginnumbering
    \numberlinefalse
    \pstart
    \textit{De iis quae in dubium revocari possunt}\ledsidenote{17}
    \pend
    \endnumbering
\end{center}

\beginnumbering
\numberpstarttrue
\pstart
\begin{latin}
    Animadverti iam ante aliquot annos quam multa, ineunte aetate, falsa pro veris admiserim, et quam dubia sint quaecunque istis postea superextruxi, ac proinde funditus omnia semel in vita esse evertenda, atque a primis fundamentis denuo inchoandum, si quid aliquando firmum et mansurum cupiam in scientiis stabilire; sed ingens opus esse videbatur, eamque aetatem expectabam, quae foret tam matura, ut capessendis disciplinis aptior nulla sequeretur. Quare tamdiu cunctatus sum ut deinceps essem in culpa, si quod temporis superest ad agendum, deliberando consumerem. Opportune igitur hodie mentem curis\ledsidenote{18} omnibus exsolvi, securum mihi otium procuravi, solus secedo, serio tandem et libere generali huic mearum opinionum eversioni vacabo.
\end{latin}
\pend
\endnumbering

\prenotes

Descartes starts with what can be doubted, but not because he wants to shake our beliefs. He hopes to strengthen our understanding by clearing away anything that is not solid and reliable. After we remove everything that \textit{can} be doubted, what remains \textit{cannot} be doubted. In this way, Descartes intends to use doubt in order to reach certainty.

\lem{Animadverti}, the main verb of the sentence, takes four objects: two indirect questions and two indirect statements. The two indirect questions are (i) \textit{quam multa\dots admiserim} and (ii) \textit{quam dubia sint}. The two indirect statements are (i) \textit{omnia\dots esse evertenda} and (ii) \textit{inchoandum <esse>}.

\lem{ante} usually means `before', but it can also mean `ago'. In this use, it is followed by either an ablative or, as here, an accusative indicating how long ago.

\lemc{quam multa\dots quam dubia} as an interrogative adverb, \textit{quam} means `how'.

\lem{ineunte aetate} is an ablative absolute that modifies \textit{admiserim}. Remember that an ablative absolute often stands in place of a clause with temporal, causal, concessive, conditional, or coordinate force (\textbf{AG §420}). In this case, a temporal clause makes the best sense.

\lemc{admiserim\dots sint} although the main verb \textit{animadverti} is secondary sequence (\textit{ante aliquot annos} guarantees this), these two verbs are primary sequence. This may be for vividness, but throughout this sentence Descartes shifts his perspective from his past realization and resolution to his present plans. It reads very naturally, but the grammar is a bit hard to pin down.

\lem{admiserim} assumes a particular view of belief. A person is thought of as reviewing impressions and then either granting them entrance (i.e., believing them) or turning them away (i.e., not believing them). This suggests, though it does not necessarily require, that belief is voluntary, or at least subject to the will in some way. All of this is controversial, and Descartes will discuss it later in the work.

\lemc{quam dubia sint quaecunque\dots } the relative clause serves as the subject of \textit{sint}. That is, Descartes realized how doubtful \textit{whatever he built on top of those foundations} is.

\lemc{quaecunque istis postea superextruxi} throughout the \textit{Meditations} Descartes uses metaphors drawn from construction and demolition. He represents knowledge and science as buildings that can be firm and lasting or have weak foundations; they can be torn down and built back up again. (Descartes makes heavy use of this metaphor, and the related metaphor of the house, in his \textit{Discours de la méthode} as well. See \citet[22]{curtis1984}.)

\lem{quaecunque} = \textit{quaecumque}.

\lem{istis} is probably dative with the compound \textit{superextruxi}, and the pronoun refers back to the \textit{falsa} that Descartes believed to be true when he was younger. Descartes continues to use the implicit metaphor of construction.

\lemc{superextruxi} this verb does not appear in classical Latin, but \textit{ex(s)truo} does, and compound verbs with \textit{super}- are common in Latin.

\lemc{proinde} Descartes doesn't attempt to justify the inference from ``I'm aware I've made mistakes'' to ``I should overturn all my beliefs''. He moves quickly at this point, assuming that the reader is willing to go along to see where the argument leads. 

\lem{foret} = \textit{esset}, imperfect subjunctive in a relative clause of characteristic. Note \textit{\textbf{eam} aetatem}: `\textit{that (kind of)} age that would\dots'.

\lemc{cappesendis disciplinis} the dative depends on \textit{aptior}, and the gerundive phrase expresses purpose.

\lemc{quod temporis} the genitive is partitive, and the phrase serves as the object of \textit{consumerem}: ``I would be at fault if I wasted what time remains\dots''. There's no antecedent for \textit{quod}. This is common for indefinite antecedents (\textbf{AG §307c}). Fully spelled out the thought is \textit{si illud temporis quod ad agendum superest deliberando consumerem}.

\lem{curis omnibus} ablative of separation with \textit{exsolvo}.

\lem{generali huic\dots eversioni} dative with \textit{vacabo}.

\lem{mearum opionum} objective genitive with \textit{eversioni}.

% -]] Background

% -]] Meditatio prima
