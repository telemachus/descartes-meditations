% [[- Chapter title
\chapter{Meditatio Prima}
% -]] Chapter title

% [[- Meditatio prima

% [[- Introduction
Although his goal is knowledge, Descartes dedicates the first meditation to doubt. Starting from everyday mistakes, such as misjudging the size of a distant object, he rapidly builds towards radical, hyperbolic doubt as he imagines an all-powerful demon whose only goal is to confuse and mislead him. At the end of the first meditation, Descartes is so shaken that he fears he may be trapped in the unescapable darkness of error.

However, as Descartes himself says elsewhere, he is not like the sceptics who ``doubt only to doubt and pretend always to be undecided'' (AT VI 29). Descartes believes that he can use doubt and error strategically in order to achieve certainty and truth. Although the first meditation contains some of the most vivid writing and imagery in the work, Descartes would have been very sorry for readers to remember it best. His goal is to overcome doubt rather than to dwell on it.
% -]] Introduction

% [[- Background
\clearpage
\begin{center}
    \beginnumbering
    \numberlinefalse
    \pstart
    \textit{De iis quae in dubium revocari possunt}\ledsidenote{17}
    \pend
    \endnumbering
\end{center}

\beginnumbering
\pstart
\begin{latin}
    \textenglish{\textbf{1.}} Animadverti iam ante aliquot annos quam multa, ineunte aetate, falsa pro veris admiserim et quam dubia sint quaecunque istis postea superextruxi, ac proinde funditus omnia semel in vita esse evertenda atque a primis fundamentis denuo inchoandum, si quid aliquando firmum et mansurum cupiam in scientiis stabilire; sed ingens opus esse videbatur, eamque aetatem expectabam quae foret tam matura ut capessendis disciplinis aptior nulla sequeretur. Quare tamdiu cunctatus sum ut deinceps essem in culpa, si quod temporis superest ad agendum, deliberando consumerem. Opportune igitur hodie mentem curis\ledsidenote{18} omnibus exsolvi, securum mihi otium procuravi, solus secedo, serio tandem et libere generali huic mearum opinionum eversioni vacabo.
\end{latin}
\pend
\endnumbering

\prenotes

Descartes starts with what can be doubted, but not because he wants to shake our beliefs. He hopes to strengthen our understanding by clearing away anything that is not solid and reliable. After we remove everything that \textit{can} be doubted, what remains \textit{cannot} be doubted. In this way, Descartes intends to use doubt in order to reach certainty.

\lem{Animadverti}, the main verb of the sentence, takes four objects: two indirect questions and two indirect statements. The two indirect questions are (i) \textit{quam multa\dots admiserim} and (ii) \textit{quam dubia sint}. The two indirect statements are (i) \textit{omnia\dots esse evertenda} and (ii) \textit{inchoandum <esse>}.

\lem{ante} usually means `before', but it can also mean `ago'. In this use, it is followed by either an ablative or, as here, an accusative indicating how long ago.

\lemc{quam multa\dots quam dubia} as an interrogative adverb, \textit{quam} means `how'.

\lem{ineunte aetate} is an ablative absolute that modifies \textit{admiserim}. Remember that an ablative absolute often stands in place of a clause with temporal, causal, concessive, conditional, or coordinate force (\textbf{AG} §420). In this case, a temporal clause makes the best sense.

\lemc{admiserim\dots sint} although the main verb \textit{animadverti} is secondary sequence (\textit{ante aliquot annos} guarantees this), these two verbs are primary sequence. This may be for vividness, but throughout this sentence Descartes shifts his perspective from his past realization and resolution to his present plans. It reads very naturally, but the grammar is a bit hard to pin down.

\lem{admiserim} assumes a particular view of belief. A person is thought of as reviewing impressions and then either granting them entrance (i.e., believing them) or turning them away (i.e., not believing them). This suggests, though it does not necessarily require, that belief is voluntary, or at least subject to the will in some way. All of this is controversial, and Descartes will discuss it later in the work.

\lemc{quam dubia sint quaecunque\dots } the relative clause serves as the subject of \textit{sint}. That is, Descartes realized how doubtful \textit{whatever he built on top of those foundations} is.

\lemc{quaecunque istis postea superextruxi} throughout the \textit{Meditations} Descartes uses metaphors drawn from construction and demolition. He represents knowledge and science as buildings that can be firm and lasting or have weak foundations; they can be torn down and built back up again. (Descartes makes heavy use of this metaphor, and the related metaphor of the house, in his \textit{Discours de la méthode} as well. See \citet[22]{curtis1984}.)

\lem{quaecunque} = \textit{quaecumque}.

\lem{istis} is probably dative with the compound \textit{superextruxi}, and the pronoun refers back to the \textit{falsa} that Descartes believed to be true when he was younger. Descartes continues to use the implicit metaphor of construction.

\lemc{superextruxi} this verb does not appear in classical Latin, but \textit{ex(s)truo} does, and compound verbs with \textit{super}- are common in Latin.

\lemc{esse evertenda\dots inchoandum} infinitives of the passive periphrastic. This periphrastic has the force of obligation, necessity, or propriety (\textbf{AG} §194b).

\lemc{proinde} Descartes doesn't attempt to justify the inference from ``I'm aware I've made mistakes'' to ``I should overturn all my beliefs''. He moves quickly at this point, assuming that the reader is willing to go along to see where the argument leads. 

\lem{foret} = \textit{esset}, imperfect subjunctive in a relative clause of characteristic. Note \textit{\textbf{eam} aetatem}: `\textit{that (kind of)} age that would\dots'.

\lemc{cappesendis disciplinis} the dative depends on \textit{aptior}, and the gerundive phrase expresses purpose.

\lemc{quod temporis} the genitive is partitive, and the phrase serves as the object of \textit{consumerem}: ``I would be at fault if I wasted what time remains\dots''. There's no antecedent for \textit{quod}. This is common for indefinite antecedents (\textbf{AG} §307c). Fully spelled out the thought is \textit{si illud temporis quod ad agendum superest deliberando consumerem}.

\lem{curis omnibus} ablative of separation with \textit{exsolvo}.

\lem{generali huic\dots eversioni} dative with \textit{vacabo}.

\lem{mearum opionum} objective genitive with \textit{eversioni}.

% -]] Background

% [[- A shortcut
\clearpage

\beginnumbering
\pstart
\begin{latin}
    \textenglish{\textbf{2.}} Ad hoc autem non erit necesse ut omnes esse falsas ostendam---quod nunquam fortassis assequi possem; sed quia iam ratio persuadet non minus accurate ab iis quae non plane certa sunt atque indubitata quam ab aperte falsis assensionem esse cohibendam, satis erit ad omnes reiiciendas si aliquam rationem dubitandi in unaquaque reperero. Nec ideo etiam singulae erunt percurrendae---quod operis esset infiniti; sed quia, suffossis fundamentis, quidquid iis superaedificatum est sponte collabitur, aggrediar statim ipsa principia quibus illud omne quod olim credidi nitebatur.
\end{latin}
\pend
\endnumbering

\prenotes

Descartes wishes to test all of his beliefs, but he would rather not have to examine each of them. In order to speed things us, he argues that (i) assent should be withheld from anything uncertain just as much as from clear falsehoods and (ii) if he can undercut the beliefs at the foundation of his thinking, then any beliefs based on those foundations will collapse as well. In these two ways Descartes hopes to make his task easier.

\lem{hoc} refers to the overturning (\textit{eversio} §1.10) of Descartes' beliefs from the previous paragraph. Descartes uses the neuter rather than the feminine perhaps because he has in mind the general idea rather than the specific word.

\lemc{ut\dots ostendam} a substantive result clause, the subject of \textit{erit necesse}.

\lemc{quod} neuter like \textit{hoc} because it refers back to the entire clause \textit{ut omnes esse falsas ostendam}.

\lemc{possem} effectively the apodosis of a present contrary-to-fact conditional sentence, though the protasis is only implied. ``<Even if I were to try,> \textit{I would never be able\dots}''.

\lemc{assensionem esse cohibendam} The technical vocabulary of ``withholding assent'' derives from ancient debates between Sceptics and Stoics. Under normal circumstances, people don't necessarily believe that they should treat anything not overwhelmingly certain and indubitable as if it were clearly false. However, in Descartes' special situation---a single-minded search for truth and absolute certainty---this rule may be justified.

\lemc{iis} this form is a common alternative for \textit{eis}, the dative and ablative plural of \textit{is, ea, id} for all genders. Similarly \textit{ii} is an alternative for \textit{ei}, the masculine nominative plural. The dative singular \textit{ei} (all genders) has no such alternative. (These alternatives also appear often in compounds of \textit{is, ea, id}. E.g., \textit{iisdem}.

\lemc{quod} like the \textit{quod} in §2.11, the neuter refers back to the action of the previous clause \textit{singulae erunt percurrendae}.

\lem{operis\dots infiniti} is a predicative genitive (\textbf{AG} §343b): ``which thing (i.e., to run through each opinion one by one) would be an infinite task''.

\lemc{suffossis fundamentis} given the context, this ablative absolute is best taken as equivalent to a conditional clause.

\lem{iis} is dative with \textit{superaedificatum est}.

\lemc{quibus} ablative with \textit{nitebatur} (\textbf{AG} §431).

% -]] A shortcut

% [[- Attack the senses first
\clearpage

\beginnumbering
\pstart
\begin{latin}
    \textenglish{\textbf{3.}} Nempe quidquid hactenus ut maxime verum admisi, vel a sensibus vel per sensus accepi; hos autem interdum fallere deprehendi, ac prudentiae est nunquam illis plane confidere qui nos vel semel deceperunt.
\end{latin}
\pend
\endnumbering

\beginnumbering
\pstart
\begin{latin}
    \textenglish{\textbf{4.}} Sed forte, quamvis interdum sensus circa minuta quaedam et remotiora nos fallant, pleraque tamen alia sunt de quibus dubitari plane non potest, quamvis ab iisdem hauriantur: ut iam me hic esse, fovo assidere, hyemali toga esse indutum, chartam istam manibus contrectare, et similia. Manus vero has ipsas, totumque hoc corpus meum esse, qua ratione posset negari? nisi me forte comparem nescio quibus insanis,\ledsidenote{19} quorum cerebella tam contumax vapor ex atra bile labefactat ut constanter asseverent vel se esse reges, cum sunt pauperrimi, vel purpura indutos, cum sunt nudi, \edtext{vel caput habere fictile}{\Afootnote{The French translation omits this phrase.}}, vel se totos esse cucurbitas, vel ex vitro conflatos; sed amentes sunt isti, nec minus ipse demens viderer, si quod ab iis exemplum ad me transferrem.
\end{latin}
\pend
\endnumbering

\prenotes

Descartes proposes a shortcut to the kind of radical scepticism he wants to reach: undermine the senses, which are at the base of all our beliefs, and everything based on the senses will be undermined at the same time. His argument against the senses is brisk: they sometimes deceive, and therefore they can never be trusted. However, Descartes then raises an objection that even if \textit{some} sense-based beliefs can deceive us, others are so basic that they can never be doubted. Indeed to doubt certain very basic beliefs that derive from the senses would be tantamount to giving up reason altogether---or so claims the objection.

\lem{quidquid} functions as the direct object of both \textit{admisi} and \textit{accepi}: ``whatever I admitted\dots I received''.

\lemc{vel a sensibus vel per sensus} In another work (\textbf{CB} §1), Descartes explains this as a distinction between information received directly from a sense (e.g., seeing a color or shape) and information received indirectly (e.g., learning about things by hearing other people speak).

\lemc{hos\dots fallere deprehendi} we can supply \textit{me} or \textit{nos} as the direct object of \textit{fallere}, or we can take it absolutely to mean simply `deceive, be deceptive'.

\lemc{prudentiae} this genitive, which is sometimes called the genitive of characteristic or the genitive of the mark, is a type of possessive genitive (\textbf{AG} §343c). It belongs to wisdom to distrust anything that has been deceptive even once. That is, it's wise to be suspicious in such a case.

\lem{vel} = `even', modifying \textit{semel}.

\lem{ut} = `for example, e.g.'. This use of \textit{ut} is a development of its adverbial meaning `as, in such a manner'. In a sentence like this, the adverb introduces specific cases which exemplify some general point. The indirect statements following \textit{ut} are all examples of the kinds of beliefs that are not easily doubted, although they come from the senses.

\lemc{Manus\dots posset negari} the two infinitive plus accusative phrases (\textit{Manus has ipsas <meas esse>} and \textit{totum hoc corpus meum esse}) are both subjects of the main verb \textit{posset}. I.e., how could it be denied that these are my hands and that this is my body?

\lemc{Manus\dots has ipsas, totum\dots hoc corpus} it was already difficult to imagine being wrong about the previous examples, but these last two are meant to be show-stoppers. The repeated \textit{has\dots hoc} implies, as often in Latin, a gesture on the part of the speaker: \textit{these} hands\dots this body---the ones right in front of you. In a similar fashion, G.E. Moore \citep[165--166]{baldwin1993} famously argued as follows:
\begin{quote}
    I can prove now, for instance, that two human hands exist. How? By holding up my two hands, and saying, as I make a certain gesture with the right hand, `Here is one hand', and adding, as I make a certain gesture with the left, `and here is another'.
\end{quote}
Do you think an argument like this can settle the philosophical problem of scepticism?

\lem{qua ratione} = literally `by what reasoning', but idiomatically it means `how'.

\lemc{nescio quibus} the verb \textit{nescio} can be joined with various words to create indefinite adjectives, pronouns, and adverbs. The verbal part is treated like an indeclinable prefix. Editors disagree over whether to write these forms as one word or two, but there's no difference in meaning either way. We can do a similar thing in English. Imagine someone telling a story, saying ``And then I-don't-know-who comes along and\dots''. Here \textit{nescio quibus} is an indefinite adjective, modifying \textit{insanis}: `some crazy people'.

\lemc{quorum cerebella\dots labefactat} Black bile is one of the four humors; the other three are blood, yellow bile, and phlegm. The balance or imbalance of these four was thought to control a person's physical and mental well-being.

\lemc{asseverent vel\dots vel\dots} these examples are extreme in two ways. First, the beliefs in question are not a little wrong, but wildly wrong. Second, they all involve mistakes about very basic facts, the kind of things that people simply don't make mistakes about, under ordinary circumstances. Hence, this objection poses a serious threat to Descartes' opening argument against the senses. He now needs to answer the challenge that his position is tantamount to insanity.
% -]] Attack the senses first


% -]] Meditatio prima
