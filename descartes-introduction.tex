% [[- Chapter title
\chapter{Introduction}
% -]] Chapter title

% [[- Descartes's life
\section{Life}

René Descartes was born on March 31, 1596 to Jeanne and Joachim Descartes. His birthplace was La Haye, a small town near Tours in west-central France. (In honor of the philosopher, the town is now officially named Descartes.) Scholars disagree about the precise social status of Descartes's family, but they were certainly well educated and of significant means. There were doctors, lawyers, and civil servants on both sides of Descartes's family, and an inheritance allowed the frugal philosopher to live nearly his whole life without any need for outside income. Joachim Descartes was a magistrate and lawyer for the \textit{parlement} at Brittany, approximately 200 miles away from the family home, and so he was away for part of every year. Jeanne Descartes passed away during childbirth when Descartes was just over a year old. He and his siblings, an older brother Pierre and an older sister Jeanne, grew up with their maternal grandmother in La Haye.

Descartes received a demanding, elite education. He attended the Jesuit-run Collège Royal Henry-IV de La Flèche for eight years. The exact dates are uncertain, but the current consensus is that he started in 1607, when he was 10, and left in 1615. At La Flèche, Descartes received oustanding training in Latin, rhetoric, and classical literature. He also studied logic, mathematics, science, and philosophy in his last three years there. After finishing at La Flèche, Descartes went to the University of Poitiers, where he received a Baccalaureat and a degree in Law.

Unlike his father and older brother, Descartes did not choose the legal profession. Instead in 1618, Descartes left for Holland and he joined a volunteer army. It is not clear what his duties were, and the suggestions of biographers are very varied. Some believe that he was a soldier, others that he worked as a military engineer, and yet others that Descartes was a student at a military academy and saw no active service as a soldier or engineer. In later years, Descartes himself says only that he wanted to travel and see more of the world.

The years 1618 and 1619 were enormously important for Descartes. In 1618, Descartes met Isaac Beeckman (1588-1637), a Dutch scientist and philosopher. Beeckman helped to inspire Descartes towards what became a lifelong devotion to mathematics and science. In November of 1619, Descartes had an epiphany. While alone in winter quarters, he spent a day thinking intensely and that night he had a series of three, extraordinarily powerful dreams. From this time on, he strove to produce a universal science that would encompass the natural sciences, mathematics, and philosophy. He spent the rest of his life working on this plan in one form or another. Although his precise view of what he was doing changed over the next thirty years, the underlying goals and principles of the plan remained remarkably consistent.

Descartes lived most of his adult life outside of France. In 1628, Descartes moved to the Netherlands. He moved around a great deal within the country, but the Netherlands remained his home for most of the next twenty years. After a brief return to France in 1647, Descartes moved to Sweden in 1649 at the invitation of Queen Christina. Unfortunately, he caught pneumonia in February of 1650 and died on February 11 of that year.
% -]] Descartes's life

% [[- Descartes's two projects
\section{Science and Philosophy}

Descartes is best known now as a philosopher with a serious sideline in mathematics, but he also studied a wide variety of questions in the natural sciences.\footnote{I rely heavily on \textcite{gaukroger2011} in this section.} In the long run of history, his work in philosophy and math have been much more influential than his scientific work. But during his own life and in the hundred years after he died, this was not the case. Over time, his reputation in philosophy grew and his reputation as a scientist faded. Before discussing the \textit{Meditations} specifically, I want to give an overview of Descartes's career and describe how his different projects fit together.

After meeting Isaac Beeckman in 1618, Descartes initially devoted himself to mathematics. He worked intermittently from 1619 to 1628 on a book he eventually left unfinished: \textit{Regulae ad directionem ingenii} (\textit{Rules for the direction of the mind}). In this work, Descartes attempted to describe a universal method of inquiry based on mathematics. The work was inspired by his conviction that researchers could establish proofs in math that were entirely certain. Although he eventually realized that he could not use the method he described even to represent all mathematical knowledge, much less all knowledge as a whole, he retained a lifelong interest in methodology. He strove to discover the universal method that eluded him early on. The \textit{Regulae} also foreshadowed Descartes's ongoing fascination with certainty and his use of mathematics as a model for other disciplines.

Beginning in 1629, Descartes conceived of an even grander project, which again he never finished. He intended to write a universal science in three parts: the first part \textit{Le Monde} (\textit{The World}) on inanimate nature; a second \textit{L'Homme} (\textit{Man}) on the nature of living creatures, especially humans; and a third part on thought and the mind. Descartes wrote quite a lot of \textit{Le Monde} and \textit{L'Homme}, but he never began the third part. Nor did he publish the first two sections in their original forms for reasons explained below.

Without going into too much detail, the key fact about this project is how well it fits into the scientific revolution of the seventeenth century. Like Galileo Galilei, Descartes argued for a heliocentric view of the universe. Like Robert Boyle, Descartes explained the workings of inanimate nature by means of atomism, a complex theory of microscopic particles in motion. Like William Harvey, Descartes studied the circulatory and nervous systems of animals and humans; they treated the bodies of living creatures like machines, subject to the same laws of matter as the rest of the universe. Descartes clashed with other thinkers over countless details, but in a larger sense, he fits into his times perfectly.

In 1633 the Catholic Church condemned Galileo for supporting heliocentrism, and this led Descartes to supress his scientific work temporarily. Descartes knew that he was liable to censure or condemnation as well since he advanced essentially the same controversial views as Galileo. They both followed Copernicus and argued for heliocentrism; they both rejected the view of an unmoving Earth at the center of the universe. Throughout his life, Descartes avoided quarrels with the Catholic Church wherever he could. At the same time, he hoped that his new science might supplant the traditional Aristotelian curriculum at Catholic institutions like La Flèche where he studied as a young man. But Descartes preferred to work through indirect persuasion rather than challenge the Catholic Church openly.

At the urging of friends, Descartes published his first mature work in 1637, \textit{Discours de la méthode} (\textit{Discourse on the method}). In this work, Descartes returned to his interests in methodology and certainty, but the \textit{Discourse} was innovative in two important ways. First, although Descartes still aimed at certainty, he attempted to reach it in a new manner. Descartes used skepticism, paradoxically, to achieve certainty. In a nutshell, he applied skeptical doubt to \textit{everything} he believed. Whatever survives this doubt has been implicitly proven to be immune to doubt and thus certain. The second innovation was formal. Instead of a traditional essay or set of theses followed by supporting arguments, the \textit{Discourse} was essentially an autobiography. Descartes employed his own life story as a ``fable'' (\textbf{CSM} I 112; \textbf{AT} VI 4) in order to make his abstract arguments more concrete and engaging. We will see both of these features, the use of skepticism and formal innovation, again in the \textit{Meditations}.

Descartes also used this opportunity to repurpose some of his work from \textit{The World} and \textit{Man}. He reorganized three large chunks of material from those works, removed any connection to heliocentrism, and published them together with the \textit{Discourse} as ``essays that exemplify the method.'' These detailed scientific essays demonstrated what you could accomplish if you followed Descartes's method. At the same time, they allowed Descartes to publish some of his scientific work without running the risk of condemnation by the Catholic Church.

In 1641 Descartes published what would prove to be his best known work, \textit{Meditationes de prima philosophia} (\textit{Metaphysical Meditations}). The \textit{Meditations} covered the same topics as the fourth section of the \textit{Discourse}, but at greater length. Both works used skepticism in order to reach certainty, argued for existence of god, discussed the human soul, and argued that intellectual knowledge is logically prior to and more important than sensory knowledge. The \textit{Meditations} provided additional argument and explanation for each of these topics, and they also explained Descartes's theory of matter, the nature of the human mind and thought, and the relationship between a person's body and soul. (In this way, the \textit{Meditations} served as the previously unwritten third part of the grand project that Descartes began in 1619.)

Descartes spent the remaining nine years of his life largely consolidating his views, responding to critics, and offering clarifications where needed. In 1644 Descartes published \textit{Principia Philosophiae} (\textit{Principles of Philosophy}). Descartes hoped that this work could become a textbook, and in it he brought together the epistemology and metaphysics of his \textit{Discourse} and \textit{Meditations} with his earlier scientific work from \textit{The World} and \textit{Man}. In 1649, Descartes released the last work published in his lifetime, \textit{Les Passions de l'âme} (\textit{The Passions of the Soul}). This work was sparked by a long period of correspondence between Descartes and Princess Elizabeth of Bohemia. In response to her probing questions and criticisms, Descartes attempted to clarify the connection between mind and body in his system. He focused on human emotions since these are an especially obvious and important case of interaction between mind and body: what you think affects how you feel, and how you feel affects what you think.

After his death, many of Descartes's previously unpublished works and letters were published. In addition, his executors discovered an entirely unknown manuscript: \textit{La Recherche de la Verité par la lumière naturelle} (\textit{The Search for Truth by Means of Natural Light}\footnote{Descartes often uses the phrase \textit{natural light} to refer to higher intellectual capabilities that human beings share. These capabilities, which he also called \textit{reason} and \textit{good sense}, are what allows people to separate truth from falsehood and to reason. In his opinion, animals other than humans lack these capabilities. Although they enjoy sensory perception and have physical desires, animals are only complex machines without reason or free will.}). What remains of this work is unfortunately very brief---only around fifteen pages survive. In these pages, Descartes covered familiar questions about truth, certainty, and knowledge, but he employed a new form. In \textit{The Search}, Descartes imitated ancient philosophers such as Plato and Cicero: he wrote a dramatic dialogue in which three characters exchange views and argue with one another.

Although he died relatively young, just shy of his \nth{54} birthday, Descartes had an exceedingly wide-ranging career as a thinker and a writer. He studied atomic physics (as it was understood in the sixteenth century), the refraction of light, the mechanics of telescopic lenses, weather, animal and human physiology, the existence and nature of god, epistemology, morality, free will, the mind, and the emotions. He wrote elegantly in both French and Latin, using French when he wanted to reach a wider and more varied audience and Latin when he was concerned primarily with scholars and the Catholic Church. However, he also allowed translations of his French works into Latin and his Latin works into French, in order to reach the largest possible overall audience. Descartes also devoted significant effort to formal innovations as a writer, frequently varying the style and approach of his works and adapting different genres, again in an attempt to engage and expand his readership.
% -]] Descartes's two projects

% [[- Metaphysical Meditations
\section{\textit{Metaphysical Meditations}}

\subsection*{Form}

The form of Descartes's \textit{Meditations} is unusual and requires some explanation. Instead of writing a treatise, essays, or even a dialogue---all standard forms of philosophical writing in the 1600s---Descartes chose to present his ideas in what was traditionally a religious form: a book of meditations written in the first person. Contemporary readers would have associated the title and the first person point of view with books of spiritual guidance. So we need to consider why Descartes made this unusual choice and what it means for us.

Descartes's contemporary readers expected specific things from a book of meditations, and we should be on the lookout for these features.\footnote{I have drawn on \textcites{rorty1986a}{williams1986}{mercer2014} in order to understand how Descartes used and adapted meditative literature.} First, a book of meditations was practical as well as theoretical. Meditation books tried to change their readers behavior as well as their beliefs. Second, the \textit{I} in a book of meditations does not stand necessarily or exclusively for the author. Instead, each reader can become the anonymous \textit{I} who meditates.

We find both of these features in Descartes's \textit{Meditations}. In the preface, Descartes says that he wants only readers who are prepared ``to meditate seriously with me and to draw their mind away from the senses and at the same time away from all prejudices'' (\textbf{CSM} II 8; \textbf{AT} VII 9). Descartes demands readers who will be active, change their way of thinking, and give up previous false beliefs. In addition, although many modern readers associate the speaking \textit{I} with the author Descartes, we should not do so. Descartes appears to have modeled some of the speaker's experiences on parts of his own life, but he alters the details in many ways and he ascribes to the narrating \textit{I} several views that we know the actual Descartes did not hold in 1641 and very likely had not held in 1619 when he first conceived of his grand project.

Although Descartes uses the meditative form, he also adapts that form. Most importantly, the narrating \textit{I} is not actually meditating. Instead, she\footnote{The anonymous \textit{I} has no explicit gender, but we need to make some decision, so I use feminine pronouns.} is a philosopher. For this reason, I will not follow the common convention of calling the speaker \textit{the meditator}. There is no obvious or perfect way to refer to the speaker, but I will call her \textit{the inquirer}.\footnote{The inspiration for this choice is the subtitle of Bernard William's book \textit{Descartes: The Project of Pure Enquiry} \parencite{williams1978}.} The inquirer pursues certainty in an effort to discover knowledge as a preliminary to scientific progress.

By placing an anonymous inquirer at the center of the work, Descartes frees the arguments in the \textit{Meditations} from anyone's authority, even his own. It was extremely common in Descartes's time to base arguments on appeals to the authority of ancient philosophers, especially Aristotle, or the Catholic church.\footnote{\textcites[4--7]{cottingham1986}[4--5]{garber1998} discuss the importance of authority in the seventeenth century. They both also consider why and how Descartes rejected authority.} Descartes believed, however, that each person should work through his arguments for themselves. If we read the \textit{Meditations} in the right way, we will not simply accept (or reject) their claims because of someone's authority. Instead, we will make the arguments our own and decide for ourselves what to believe.

\subsection*{Goals}

Descartes set out two goals very clearly in the subtitle to the second edition of the \textit{Meditations}.\footnote{The first edition had no subtitle.}

\begin{quote}
    In quibus Dei existentia et animae a corpore distinctio demonstrantur

    In which the existence of god and the distinction of soul from body are proven
\end{quote}
We should expect the \textit{Meditations} to provide (i) proof of the existence of god and (ii) an account of soul and body that explains how they function and how they differ.

The \textit{Meditations} lives up to the promises of the subtitle, but first-time readers are likely to have questions about the book's organization and purpose. First, it is not immediately clear how the two goals of the subtitle are connected. What unifies a book with these two topics? Second, there appears to be far more to the \textit{Meditations} than just these two goals. We find proof of god's existence in the third meditation (as well as a secondary proof in the fifth meditation), and we learn about the distinction between soul and body in the sixth meditation. But what is the purpose of the other meditations? How are their arguments related to Descartes's primary goals?

Briefly, I would argue that in addition to his two stated goals, Descartes also had two unstated goals. First, Descartes wanted his proofs to be clear and certain. Therefore, Descartes began the \textit{Meditations} with a skeptical purification of his beliefs, as in his earlier \textit{Discourse}. During the first and second meditations, Descartes prepared the way for his proof of god's existence, and he also implicitly taught readers how certainty and clarity can be achieved in argument. Second, Descartes implicitly included the principles of his physics in the \textit{Meditations}. In a letter Descartes wrote to a friend in 1641, we find the following:

\begin{quote}
    I will tell you, just between the two of us, that these six \textit{Meditations} contain all the foundations of my physics. But, please, it is essential not to say anything about this; for the Aristotelians might have more trouble approving of them.\footnote{I.e., Aristotelian scholars would be less likely to approve of Descartes's \textit{Meditations}, if they knew how revolutionary they really were.} And I hope that those who read them will become accustomed to my principles without realizing it, and that they will acknowledge their truth before they realize that <my principles> destroy those of Aristotle.\footnote{Descartes to Father Mersenne, January 28, 1641 (\textbf{CSMK} 173; \textbf{AT} III 298).}
\end{quote}

Descartes had to be very careful how he presented his views. As mentioned earlier, he was very very concerned that the Catholic Church not sanction him for his heliocentric views, as they had sanctioned Galileo. In addition, Descartes hoped that his physics and philosophy would replace Aristotelianism in schools run by the Catholic Church. Thus, he took pains over presentation and thought a great deal about how best to convince his audience. He believed that the \textit{Meditations} had a better chance of persuading unprejudiced readers. As a result, the \textit{Meditations} handle some subjects only implicitly, so that Descartes could, as he wrote to Mersenne, accustom his contemporaries to his ``principles without [their] realizing it.''

To return to the explicit topics, Descartes began the \textit{Meditations} with a radical and justly famous skeptical critique of all beliefs. The inquirer rejects an increasingly wide collection of beliefs using an increasingly corrosive series of skeptical arguments. These arguments culminate in the ``evil demon'' thought experiment, in which the inquirer imagines that an all-powerful malevolent demon devotes all its energy to deceiving her. Faced with such a possibility, the inquirer doubts everything until reaching a single truth that is impossible to doubt: \textit{I think therefore I exist}.\footnote{The inquirer does not use these exact words in the \textit{Meditations}, but this is the traditional formula, and it will do for now.} Scholars call this argument the \textit{cogito} since \textit{cogito} is Latin for ``I think.''

The inquirer employs the \textit{cogito} in pursuit of Descartes's stated goals. A complex series of arguments carries the inquirer, in the second meditation, from the \textit{cogito} to the innate idea of god and from there to a proof that god exists in the third meditation. A second complex series of arguments leads to a second proof of god's existence in the fifth meditation and, finally, in the sixth meditation to a proof that the human soul is distinct from its body.

At the same time, Descartes scattered the \textit{Meditations} with key elements of his epistemology and his physics. In the second meditation, the inquirer realizes that humans understand their own minds better and more intimately than they understand physical nature. After the proof of god's existence in the third meditation, the inquirer spends the fourth meditation analyzing the nature of truth and falsity and explaining the role of free will in human knowledge and error. In the fifth and sixth meditations, the inquirer investigates the essential nature of physical things as opposed to minds, whose essense is thought.

If we keep in mind both his stated and his unstated goals, we will best understand the apparent detours and digressions in Descartes's \textit{Meditations}. The work is quite unified and well organized, but some of the principles of unity and organization are only implicit.
% -]] Metaphysical Meditations

% [[- Skepticism
\section{Skepticism}

Descartes is not a skeptic, but we need to think about skepticism in order to understand his \textit{Meditations}. Since Descartes inherited a great number of skeptical arguments and problems from the ancient world, this section looks first at Greco-Roman skepticism and then at the early modern rediscovery of skepticism.

\subsection*{Ancient Skepticism}

The ancient skeptics who had the most influence on early modern philosophy and Descartes belonged to the Pyrrhonian school. Our most important source for Pyrrhonian skepticism is Sextus Empiricus, a philosopher and doctor. Scholars do not know exactly when Sextus was born or died, or where he lived, but he was probably active in the \nth{2} century CE.

The word \textit{skeptic} comes from a Greek root (\textgreek{σκεπτ}- or in English transliteration \textit{skept}-) that means \textit{inquire} or \textit{investigate}. According to Sextus, the key difference between skeptics and all other philosophers is that the skeptics ``are still investigating'' \parencite[I 3]{annasbarnes1994}. Other philosophers believe that they have discovered the truth, but skeptics are not yet convinced.

To explain what it means for the skeptics to still be investigating, consider people's views about religion. Believers and atheists both think that they have answered the question ``Does god exist?'' The believer answers yes, and the atheist no. Most philosophers in antiquity were like believers: they thought that they had various true theories about the world. Some ancient philosophers were like atheists: they were prepared to declare that there was no knowledge or that knowledge was impossible. If we take Sextus seriously, however, Pyrrhonian skeptics are like agnostics. Pyrrhonians do not think that they know anything, not even \textit{that nothing can be known}. Instead, they make the more restricted assertion that thus far in their search they have not found any answers.\footnote{Thus, Pyrrhonians avoid the naive inconsistency of saying, ``We know that we know nothing.''} This also explains why Pyrrhonian skeptics are \textit{still} investigating: they have not yet found any conclusive answers.

Over time Pyrrhonian skeptics developed what they called \textit{modes}. A mode is a generic argument form that a skeptic can apply by plugging in specifics as needed.\footnote{For an excellent introduction to the modes and Pyrrhonian skepticism in general, see \textcite{annasbarnes1994}.} There are several different lists of modes, but the most useful for Descartes is the following list of five:

\begin{description}[leftmargin=*,widest="Infinite Regress"]
    \item[Disagreement] While trying to answer some question \textit{q}, we discover disagreement. Since we cannot resolve the disagreement, we suspend judgment.
    \item[Infinite regress] While trying to answer \textit{q}, someone offers a reason \textit{r}. But \textit{r} itself needs support from some other reason \textit{s}, and so on. This leads to an infinite regress. Since we cannot work through an infinite series of reasons, we suspend judgment.
    \item[Relativity] While trying to answer \textit{q}, we discover that the matter appears one way to one person (or in one situation), but another way to another person (or in another situation). Since we cannot decide between the various appearances, we suspend judgment.
    \item[Hypothesis] While trying to answer \textit{q}, a person will sometimes offer an assumption. The assumption is meant to provide a foundation for other arguments. Since the assumption itself is not supported, we do not know if it is true, and so we suspend judgment.
    \item[Circularity] While trying to answer \textit{q}, we offer \textit{r} as a supporting reason. However, \textit{r} in turn requires the support of \textit{q}. Since the argument is circular, we suspend judgment.
\end{description}

Notice that all of these modes end in the same way: suspension of judgment.\footnote{The Greek term for this is \textgreek{ἐποχή} or \textit{epochê} when transliterated into English.} As an example, imagine you are trying to decide whether a certain genetically modified food is healthy. You are no expert, so you read several academic studies. The studies disagree, and you have no way to choose between them: the authors of the studies are all equally qualified, none of them works for the corporation responsible for the modified food, and so on. In such a case, it seems that the only reasonable thing to do is stay neutral, not pick either side, and continue to investigate if possible. This is what skeptics mean by \textit{suspend judgment}. The skeptics are ``still investigating'' since they discover that the result of \textit{all} investigations is equally weighted arguments on either side of the question.\footnote{Pyrrhonians call this \textgreek{ἰσοθενία} or \textit{isothenia} when transliterated into English. Scholars and translators often use \textit{equipollence} to represent this idea in English.}

These skeptical arguments suggest a strikingly strong conclusion: in (nearly?) all cases, we have no reason to believe \textit{anything}. In such cases, if we suspend judgment as the skeptics recommend, we will end up with no opinion on the matter. In order to show how unsettling this result is, consider the following. If we asked ``Is the total number of stars in the sky odd or even?'', surely most people would answer that they had no idea. There is no practical way to count or to guess. Any answer would be no better than picking a very large number at random. In this case, the question has no bearing on our lives, so it is easy to respond ``I have no idea.'' The skeptics suggest, however, that we should be unsure about nearly \textit{all} questions in life. This should shock us.

In response to this radical attack, non-skeptical philosophers---whom the skeptics called \textit{dogmatists}---accused the Pyrrhonians of two kinds of inconsistency. First, the dogmatists argued that the Pyrrhonians were logically inconsistent. On the one hand, they claimed to have no beliefs, but on the other hand, they made numerous assertions. According to the dogmatists, this places skeptics in a bind: either the skeptics believe in their assertions or they do not. If they believe their own assertions, then they \textit{do} have some beliefs, and they are inconsistent. If they do not believe their own assertions, then we do not need to take their arguments seriously. After all, when we are having serious discussions, we normally dismiss things people say at random or for no particular reason. Second, the dogmatists accused Pyrrhonians of a practical inconsistency. The skeptics say that they have no beliefs, but they still make choices, they still act. What motivates their actions, ask the dogmatists, if not belief? In this case, there is not, strictly speaking, a formal contradiction, but the behavior of the Pyrrhonians clashes with their statements. For example, a skeptic will get out of the way of oncoming traffic. This suggests that the skeptic believes (i) that there is oncoming traffic and (ii) that it would be bad to be struck by a car.

The skeptics, however, have answers to both charges of inconsistency. They deny any logical inconsistency by arguing that the views they put forward are not their own beliefs. They employ premises in their arguments only dialectically. These premises are either hypothetical or things that their opponents believe. The skeptics themselves need not believe them in order to have a debate. I do not have to believe any of what you say in order to point out that your argument has inconsistencies or logical flaws. In response to the practical inconsistency, Pyrrhonians argue that action requires appearance but not necessarily belief. The oncoming car \textit{appears} dangerous, and so the skeptic moves out of the way quickly. It is not necessary, say the skeptics, to have any beliefs about the matter.

As you can imagine, these debates continued without any clearcut or definitive resolution. For what it is worth, I believe that the skeptics make a solid defense against the first attack, but the second attack remains troubling. Dialectical debate where one or both parties argue from premises that they do not necessarily endorse is well-known and common. And this style of debate was certainly familiar in antiquity, as we can see from the Sophists, Socrates, and Aristotle's work on rhetoric. I am unsure, however, about the skeptical response to the second attack. That attack is often put as this question: can the skeptics live their skepticism?\footnote{See, for example, \textcite{burnyeat1983a}.} That is, can they actually live an entire life without belief, based only on appearances? In my view, the jury is still out on this question. Additionally, even if it is possible, such a life would likely look nothing like ordinary life. In other words, even if the skeptics convince us that they can lead \textit{a} life, we may still wonder whether it would be a \textit{good} life, a life worth living.

\subsection*{Early Modern Skepticism}

After receiving very little attention during the medieval period, Pyrrhonian skepticism was rediscovered and reintroduced to western Europe during the Renaissance.\footnote{This section is heavily dependent on two excellent articles: \textcite{schmitt1983,popkin1993}.} In 1427, an Italian scholar brought Greek manuscripts of Sextus Empiricus from Constantinople to Italy. Between this period and 1520, Italian interest in skepticism grew, and scholars have traced an increase in attention to skepticism in France and other parts of Europe north of Italy later in the sixteenth century. The breakthrough for Pyrrhonian skepticism came in the 1560s when two Frenchmen, Henri Estienne  and Gentian Hervet, translated the works of Sextus into Latin. After this, Pyrrhonian arguments become increasingly common in early modern philosophy.

This rediscovered skepticism first appears in religious debates. Several Catholic authors used skepticism to argue against Protestant challenges. They argued roughly as follows:

\begin{enumerate}
    \item Skepticism shows that nothing can be known.
    \item Since nothing can be known, Calvinism cannot be known. Faith rather than reason justifies religion.
    \item In the face of an undecidable conflict between faiths, we should stick to tradition.
    \item Our (French) tradition is Catholicism.
    \item So we should stick to Catholicism.
\end{enumerate}

The last few steps in this argument may seem surprising since we might assume that skepticism is revolutionary and attacks the \textit{status quo}. However, Pyrrhonian skepticism has some potentially conservative features. As mentioned earlier, skeptics recommend the suspension of judgment in the face of equally weighted argument or undecidable conflict. When action is required, they often follow appearance. But in a case like this---choosing between two religious views---there is not necessarily a single appearance to follow. In such cases, the skeptics sometimes recommend following the laws and customs of one's country. In this way, their views can lead to conservative results.

Skeptical arguments were an important part of early modern religious debates, and these arguments had a great impact on the course of religious belief. Throughout the seventeenth century, Protestants challenged Catholics with skeptical arguments, and Protestants defended themselves against the attacks of other Protestants with skepticism. Richard Popkin has also argued that the atheism of the eighteenth century developed partly as a result of these battles. As he puts it, there was ``just a short step'' from the skeptical view that faith alone, and not reason, supported religious belief to a skepticism ``without any religious faith'' \parencite[19]{popkin1993}.

In addition to its role in religious arguments, Pyrrhonian skepticism was important in debates about philosophy and education. Early modern education was based on Scholasticism, an amalgam of Catholic theology and Aristotelian philosophy combined with bits of Platonism and Stoicism. After the rediscovery of the works of Sextus, philosophers began to use skeptical arguments to attack Scholasticism, often in support of new scientific discoveries. This leads in neatly to Descartes, whose education at La Flèche followed the Scholastic curriculum.

\subsection*{Descartes's use of skepticism}

Descartes employs skepticism in an innovative and surprising manner. In all the previous cases we have discussed, someone has applied skeptical arguments in order to undermine belief and knowledge. This makes sense since all skeptical modes end, as we saw above, with suspension of judgment. If you cannot make a decision---i.e., a judgment, then you do not form a belief, and you cannot know anything without true beliefs. But Descartes wants to take skeptical arguments to their absolute limit in order to strengthen and support his claims to knowledge. In the first meditation, he doubts as much as he can, and he relies on skeptical arguments to give him reasons to doubt. But, as he says in another work, he does not ``doubt only to doubt'' (\textbf{CSM} I 125; \textbf{AT} VI 29) like the skeptics. Descartes believes that if he unleashes an unrestricted skepticism on all of his beliefs, anything that survives this doubt will be true and certain. By doubting everything he can, Descartes hopes to discover that which can not be doubted. This is his strategy in the first two meditations.
% -]] Skepticism

% [[- About The Text
\section{About The Text}

The text of this edition has two main sources: volume VII of the twelve-volume edition of Charles Adam and Paul Tannery and volume II of the three-volume edition of Ferdinand Alquié. The Adam-Tannery edition remains the standard edition of Descartes, though Alquié challenges and clarifies their text in several small ways. I also consulted John Cottingham's Latin-English edition of 2013. (Cottingham, in turn, is a lightly altered version of the Adam-Tannery text.) These three editions print the same text except for minor differences in orthography and modernization. As Cottingham explains, the Alquié and the Adam-Tannery versions of the \textit{Meditations} have no substantive differences in meaning \parencite[xxxii, footnote 5]{cottingham2013}.

Although Descartes wrote highly readable neo-Latin, I alter his text in several ways to make it more comfortable for modern readers of Latin. First I replace  \textit{ſ}, the long s, with the more familiar \textit{s} character. Second, I replace \textit{j} with \textit{i}. Last, I replace \textit{\&} with \textit{et}. None of these changes affect the meaning of what Descartes wrote, but they should make the text much more familiar to contemporary readers of Latin.

This text presents the \textit{Meditations} with two different types of marginal numbering. The numbers in the outer right margin of the text refer to the pages of volume VII of the Adam-Tannery edition. You can use these numbers when referring to other editions or translations in any language. Almost all editions include the Adam-Tannery numeration. Since these numbers are in all editions, cross-edition and cross-linguistic reference becomes much simpler than it would be otherwise. The paragraph divisions in this edition follow Adam-Tannery, but the line numbers do not necessarily correspond to any other text of the \textit{Meditations}. The numbers on the left hand margin are line numbers for individual paragraphs in this edition only. These line numbers support internal cross-references in the commentary. For example, §4.10 in a note refers to paragraph 4 line 10. Within the commentary, all cross-references refer to the meditation they appear in unless stated otherwise.
% -]] About The Text
