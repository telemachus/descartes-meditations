% [[- Chapter title
\chapter{Introduction}
% -]] Chapter title

% [[- Descartes's life
\section*{Life}

René Descartes was born on March 31, 1596 to Jeanne and Joachim Descartes. His birthplace was La Haye, a small town near Tours in west-central France. (In honor of the philosopher, the town is now officially named Descartes.) Scholars disagree about the precise social status of Descartes's family, but we know that they were well-educated and of significant means. There were doctors, lawyers, and civil servants on both sides of Descartes's family, and an inheritance allowed the frugal philosopher to live nearly his whole life without any need for outside income. Joachim Descartes was a magistrate and lawyer for the Parliment at Brittany, where he lived for part of every year. Jeanne Descartes passed away during childbirth when Descartes was just over a year old. He and his siblings, an older brother Pierre and an older sister Jeanne, grew up with their maternal grandmother in La Haye.

Descartes received an excellent education. He attended the Jesuit-run Collège Royal Henry-IV de La Flèche for eight years. The exact dates are uncertain, but the current consensus is that he started in 1607, when he was 10, and left in 1615. He received oustanding training in Latin, rhetoric, and classical literature. Descartes also studied logic, mathematics, science, and philosophy. As rigorous as this education was, it had some significant weaknesses in the eyes of the adult Descartes. Mathematics at La Flèche focused too much on applied problems, and the science curriculum still relied on Aristotle. After finishing at La Flèche, Descartes went to the University of Poitiers, where he received a Baccalaureat and a degree in Law.

Unlike his father and older brother, Descartes did not choose the legal profession. Instead in 1618, Descartes left for Holland and he joined a volunteer army. It is not clear what his duties were, and the suggestions of biographers are very varied. Some believe that he was a soldier, others that he worked as a military engineer, and yet others that Descartes was a student at a military academy who saw no active service as a soldier or engineer. In later years, Descartes himself says only that he wanted to travel and see more of the world.

The years 1618 and 1619 were enormously important for Descartes. In 1618, Descartes met Isaac Beeckman (1588-1637), a Dutch thinker. Beeckman helped to inspire Descartes towards what became a lifelong devotion to mathematics and science. In November of 1619, Descartes had an epiphany. While alone in winter quarters, he spent a day thinking intensely and that night he had a series of three, extraordinarily powerful dreams. From this time on, he strove to produce a universal science that would encompass the natural sciences, mathematics, and philosophy. He spent the rest of his life working on this plan in one form or another. Although his precise view of what he was doing changed over the next thirty years, the underlying goals and principles of the plan remained remarkably consistent.

Although he returned to France more than once, Descartes lived most of his adult life elsewhere in Europe. In 1628, Descartes moved to the Netherlands. He moved around a great deal, but lived in various places there for most of the next twenty years. After a brief return to France in 1647, Descartes moved to Sweden in 1649 at the invitation of Queen Christina. He caught pneumonia in February of 1650 and died on February 11 of that year.
% -]] Descartes's life

% [[- Descartes's two projects
\section*{Science and Philosophy}

Descartes is best known now as a philosopher with a serious sideline in mathematics, but he thought of himself as a scientist as well. Throughout his life, he devoted much of his effort to questions in the natural sciences. Before discussing the \textit{Meditations} specifically, I want to give an overview of Descartes's career and describe how his different projects fit together.\footnote{I rely heavily on \cite{gaukroger2011} in this section.}

After meeting Isaac Beeckman in 1618, Descartes initially devoted himself to mathematics. He worked intermittently from 1619 to 1628 on a book he eventually left unfinished: his \textit{Regulae ad directionem ingenii} (\textit{Rules for the direction of the mind}). In this work, Descartes attempted to describe a universal method of inquiry based on mathematics. The work was inspired by his conviction that researchers could establish proofs in math that were entirely certain. Although he eventually realized that he could not use the method he described to represent all mathematical knowledge, much less all knowledge as a whole, he retained a lifelong interest in method and the pursuit of certainty. He also continued to use mathematics as a exemplary discipline.

Beginning in 1629, Descartes conceived of an even grander project, which again he never finished. He intended to write a three part science: the first part \textit{Le Monde} (\textit{The World}) on inanimate nature; second \textit{L'Homme} (\textit{Man}) on the nature of living creatures, especially humans; and a third part on thought and the mind. Descartes wrote quite a lot of \textit{Le Monde} and \textit{L'Homme}, but he never began the third part. In addition, he didn't publish the first two sections in their original forms for a reason I'll explain soon.

Without going into too much detail, the key fact about this project is how well it fits into the scientific revolution of the seventeenth century. Like Galileo and others, Descartes argued for a heliocentric view of the universe. Like Boyle and many others, Descartes explained the workings of inanimate nature by means of atomism, a complex theory of microscopic particles in motion. Like Harvey and others, Descartes studied the circulatory and nervous systems of animals and humans; they treated the bodies of living creatures like machines, subject to the same laws of matter as the rest of the universe. Descartes clashed with other thinkers over countless details, but in a larger sense, he fits into his times perfectly.

In 1633 the Catholic Church condemned Galileo for supporting heliocentrism, and this led Descartes to supress his scientific work temporarily. Descartes knew that he was liable to censure or condemnation as well since he advanced essentially the same controversial views as Galileo. They both followed Copernicus and argued for heliocentrism; they both rejected the view of an unmoving Earth at the center of the universe. Throughout his life, Descartes avoided quarrels with the church wherever he could. At the same time, he hoped that his new science might supplant the traditional curriculum based on Aristotle in Catholic teaching. We can see him therefore as trying to work from within to change the church's teaching rather than challenging them openly.

At the urging of friends, Descartes publishes his first mature work in 1637, \textit{Discours de la méthode} (\textit{Discourse on the method}). Descartes returns to his interests in methodology and certainty, but the \textit{Discourse} is innovative in two important ways. First, although Descartes still aims at certainty, he attempts to reach it in a new manner. Descartes uses skepticism, paradoxically, to achieve certainty. In a nutshell, he applies skeptical doubt to \textit{everything} he believes. Whatever survives this doubt, cannot be doubted and is thus certain. The second innovation is formal. Instead of a traditional essay or set of theses followed by supporting arguments, the \textit{Discourse} is essentially an autobiography. Descartes uses his own life story as a ``fable'' (\textbf{AT} VI, 4) in order to make his abstract arguments more concrete and engaging. We will see both of these features, the use of skepticism and formal innovation, again in the \textit{Meditations}.

Descartes also uses this opportunity to repurpose some of his work on \textit{The World} and \textit{Man}. He reorganizes three large chunks of material from those works, removes any connection to heliocentrism, and publishes them together with the \textit{Discourse} as ``essays that exemplify the method''. The idea being that these detailed scientific essays provide examples of what you can accomplish if you follow Descartes's method. At the same time, this allows Descartes to put forth some of his work in science without running the risk of condemnation by the church.

In 1641 Descartes publishes what will prove to be his best known work, \textit{Meditationes de prima philosophia} (\textit{Metaphysical Meditations}). This work expands on the \textit{Discourse} since it also offers an account of certain knowledge reached via skepticism, and the \textit{Meditations} also provides arguments for the existence of god and the nature of the human mind and thought. (We can see this as the previously unwritten third part of the grand project Descartes began in 1619.)

Descartes spent the remaining nine years of his life largely consolidating his views, responding to critics, and offering clarifications where needed. In 1644 Descartes published \textit{Principia Philosophiae} (\textit{Principles of Philosophy}). Descartes hoped that this work could become a textbook, and in it he brought together the epistemology and metaphysics of his \textit{Discourse} and \textit{Meditations} with his earlier scientific work from \textit{The World} and \textit{Man}. In 1649, Descartes released the last work published in his lifetime, \textit{Les Passions de l'âme} (\textit{The Passions of the Soul}). This work was sparked by a long period of correspondance between Descartes and Princess Elizabeth of Bohemia. In response to her probing questions and criticisms, Descartes attempted to clarify the connection between mind and body in his system. He focused on human emotions since these are an especially obvious and important case of interaction between mind and body: what you think affects how you feel, and how you feel affects what you think. (Infamously, Descartes isolates a specific \textit{physical location} for the interaction of mind and body: the pineal gland.)

After his death, many of Descartes's unpublished works were released as well as letters. In addition, his executors discovered an otherwise unknown manuscript \textit{La Recherche de la Verité par la lumière naturell} (\textit{The Search for Truth by the means of Natural Light}---`natural light' is a phrase Descartes often uses to mean native human intellect). This work is very brief---only around fifteen pages survived---and the original French manuscript was soon lost again. However, it is unique: Descartes used the form here of a quasi-Platonic dialogue in order to tackle familiar questions about certainty and knowledge from his \textit{Discourse} and \textit{Meditations}.

It should be clear how much more varied Descartes's writing was than we might initially think. He wrote on subjects as distinct to us as atomic physics, the refraction of light, the mechanics of telescopic lenses, weather, animal and human physiology, the nature of knowledge, the existence and nature of god, the freedom of the human will, human thought and emotion, and more. He also wrote elegantly in both French and Latin, using French when he wanted to reach a wider and more varied audience and Latin when he was concerned primarily with scholars and the Church. (He allowed several translations to appear in his lifetime of his French works into Latin and his Latin works into French, in order to reach the largest possible overall audience.) Descartes also devoted significant effort to formal innovations as a writer, frequently varying the style and approach of his works and adapting different genres, again in an attempt to engage and expand his readership.
% -]] Descartes's two projects

% [[- The Meditations
\section*{The Meditations}

\subsection*{Goals}

Descartes sets out his goals very clearly in the subtitle to the second edition of the \textit{Meditations}.\footnote{The first edition had no subtitle.}

\begin{quote}
    In quibus Dei existentia et animae a corpore distinctio demonstrantur

    In which the existence of god and the distinction of soul from body are proven
\end{quote}

Descartes offers a proof of god's existence in the third meditation (as well as a secondary proof in the fifth meditation), and he proves and briefly explains the distinction between soul and body in the sixth meditation. We may well wonder why he is not more direct. What are the other meditations for?

Descartes takes an indirect route because he wants to provide arguments that don't rely on anyone's authority, not even his own. It was extremely common in Descartes's time to base arguments on appeals to the authority of ancient philosophers, especially Aristotle, or the Catholic church.\footnote{\citet[4--7]{cottingham1986} and \citet[4--5]{garber1998} discuss the importance of authority in the seventeenth century. They also both consider why and how Descartes rejected authority.} Descartes believed, however, that each person should work out these arguments for themselves.(By rejecting traditional sources of authority in favor of personal introspection, Descartes seems very in tune with the Reformation spirit, despite his personal Catholicism.)

Descartes begins the \textit{Meditations} by clearing the ground entirely. He rejects all previous beliefs and lets skeptical doubts extend as widely as possible. He continues doubting until he reaches a single truth that he considers impossible to doubt: \textit{I think therefore I exist}.\footnote{Descartes doesn't put it exactly this way in the \textit{Meditations}, but this is the traditional formula, and it will do for now.} This argument, which is traditionally called the \textit{cogito}, requires no external authority to support it. Each person can verify its validity for themselves.

Descartes builds upon the \textit{cogito} in order to develop his arguments for the existence of god and the distinction between the soul and body. He argues that the \textit{cogito} shows that the mind is better known than the body. By reflecting on innate ideas, Descartes reaches his idea of god; this leads him to his first proof that god exists. Once he has established both that he exists and that god exists, Descartes considers more generally how to distinguish true from false beliefs and also to explain how a benevolent god allows humans to believe falsely. Finally, Descartes argues that the existence and goodness of god helps to guarantee the reality of the external world and our knowledge of it. Along the way, he offers further arguments that god exists and distinguishes between mental life and our physical nature.

\subsection*{Form}

The form of Descartes's \textit{Meditations} is unusual and requires some explanation. Instead of writing a treatise, essays, or even a dialogue---all standard forms of philosophical writing in the 1600s---Descartes chose to present his ideas in what was traditionally a religious form. Contemporary readers would have associated the title and the first person voice of Descartes's \textit{Meditations} with books of spiritual guidance. So we need to consider why Descartes made this unusual choice and what it means for us.

Contemporary readers would have expected specific things from a book of meditations, and we should be on the lookout for these features. First, a book of meditations was expected to offer practical guidance and spiritual experiences rather than information or factual knowledge. Readers went to meditations in search of training in new practices and a path to new experiences. By spending time with a book of meditations, the reader took on new habits and new behavior; the reader felt new things and behaved in new ways. The goal was not, in the first instance, knowledge but transformation. Second, when a reader sees `I' in a book of meditations, they don't associate the first person with the author but with \textit{themselves}. Since a book of meditations is meant to be a lived experience, each reader is meant to take that position for themselves.

We find both of these features in Descartes's \textit{Meditations}. In the preface, Descartes says that he wants only readers who are prepared `to meditate seriously with me and to draw their mind away from the senses and at the same time away from all prejudices' (\textbf{AT} VII.9). Descartes demands readers who will be active, change their habits and give up their previous life. In addition, although many readers associate the speaking `I' with the author Descartes, we should not do so. Descartes appears to have modeled the meditative experiences of the narrator on some of his own experiences, but (i) he has altered the situation in many ways from that of his own life and (ii) he ascribes to the narrating `I' several views that we know the actual Descartes did not hold in 1641 and very likely had not held in 1619 when he first conceived of his grand project.

Finally, although Descartes adopts a great deal from religious meditations, his \textit{Meditations} are still a work of philosophy. Descartes certainly does want to change what his readers \textit{believe} as well as what they feel and how they behave. By placing philosophical content in a meditative form, Descartes has created something new. What the \textit{Meditations} offers its readers is training and argument that will lead us to new cognitive practices, new ways of thinking, and new ways of understanding the world.
% -]] The Meditations

% [[- Skepticism
\section*{Skepticism}

Philosophical skepticism comes in an enormous variety of specific forms, so it's important to say something about what I mean here by `skepticism'. Although I will discuss the ancient and early modern history of skepticism, I'm less concerned with any one school or set of arguments than I am with a broad tendency or cluster of views. In a nutshell, any view counts as skeptical in what follows if it argues for or implies (i) that knowledge is impossible or (ii) that although knowledge is possible, nobody---or nearly nobody---possesses any. This may seem overbroad, but I will explain at the conclusion of this section why it's a valuable way to frame the problem of skepticism before reading Descartes.

\subsection*{Ancient Skepticism}

Various forms of skepticism appear throughout ancient philosophy. Among Presocratic philosophers, the earliest philosophers in ancient Greece, we already find skeptical views advanced by Xenophanes and Democritus. There were also skeptical thinkers in the time of Plato and Aristotle, and in post-Aristotelian or Hellenistic times, skepticism thrived in two separate schools. Many of the earlier skeptical thinkers are difficult to discuss because only fragments of their work survives, and many of the later thinkers left behind highly technical writings that are difficult to discuss briefly. Therefore, I will begin with Socrates. Socrates himself wrote nothing, but we have several accounts of what he was like both as a person and as a philosopher. Although it can be difficult to sort out truth from fiction, I think we can make at least general statements about how Socrates contributed to the history of skepticism.

Socrates preferred to work out his views in dialogue, often in an adversarial manner. That is, Socrates preferred to talk with people rather than to instruct them on his views, and he often chose to engage in argument rather than work cooperatively to solve problems. He frequently employed a question and answer style that has come to be known as `Socratic elenchus'. (The Greek word `elenchus' (\textgreek{ἔλεγχος}) means `a trial, a test; an inquiry, an examination; a disproval, a refutation'.)

A typical elenchus works as follows:

\begin{enumerate}
    \item Socrates asks someone for the definition of an important term in common use. Usually the terms are from ethics, e.g. `justice' or `courage', but sometimes the terms are not ethical.
    \item The interlocutor gives Socrates a definition.
    \item Socrates then asks the interlocutor several other questions whose connection to the original definition is not obvious.
    \item Eventually Socrates shows that when the original definition is combined with the answers to the secondary questions, a contradiction occurs.
    \item Socrates tells the interlocutor that they must either withdraw their original definition or their secondary answers. Theoretically, either withdrawal would solve the contradiction, but in practice the interlocutors always withdraw their definition. (This is probably because Socrates makes sure that the secondary questions he asks have very clear answers that would be difficult to deny.)
\end{enumerate}

This process---definition, secondary questions, contradiction---is often repeated several times, but the eventual result is almost always the same thing: the interlocutor becomes frustrated and completely confused. At this point, the interlocutor generally says that he thought he knew the definition in question, and indeed that he still thinks he does, but that he cannot give a satisfactory answer to Socrates. The conversations usually end with this frustrating disappointment.

The skeptical implications of Socratic elenchus should be clear, and Socrates himself readily drew the obvious conclusion. He believed that he lacked all knowledge of the matters he talked about, and he said that he had never met anyone who had such knowledge. Socrates desperately wanted to find this knowledge, but if we take him at his word, he never succeeded. We should be clear that Socrates did not call himself a skeptic, and his position is not explicitly skeptical. That is, he does not say that knowledge is impossible. However, the practical effect of Socratic elenchus was profoundly demoralizing and extremely suggestive of skeptical conclusions.

Later Greek and Roman skeptics made deliberate use of many of Socrates's arguments and methods, in particular his elenctic method. Like Socrates they preferred to ask questions and undermine the views of their opponents rather than offer any views of their own. If pressed, they would say that they were `still seeking' knowledge. This is the origin of the term `skeptic' since the Greek root `skept-' (\textgreek{σκεπτ}-) indicates inquiry, investigation, and examination.

Before leaving the ancient world, I want to point out that even explicitly anti-skeptical philosophers from antiquity indirectly fed into skepticism. Philosophers like Parmenides and Plato argued that the world we are familiar with, a world of change and perception, was deeply false or unreal. They claimed that there was a more truthful and more real world beyond the senses, and that knowledge only applied to this extra-sensory, unchanging world. Philosophers like Plato (again) and the Stoics also placed such immensely heavy demands on `knowledge' that they were forced to admit that few, if any, people had actually ever known anything. In all these ways, these philosophers can be seen as undermining ordinary claims to knowledge, even though they vehemently disagreed with contemporary skeptics.

The ancient philosopher who offered the most explicit support for a kind of commonsense anti-skeptical position was Aristotle. He argued that all knowledge began ultimately with ordinary sensory information. Over time people build simple concepts and eventually complex scientific theories on the basis of ordinary perceptual experiences. The information we take in via our senses is truthful under normal conditions, and the theories we build on top of this information can yield deeper scientific knowledge. As we will see, Descartes will reject any defense of knowledge that relies on the senses.

\subsection*{Early Modern Skepticism}


% -]] Skepticism

% [[- About The Text
\section*{About The Text}

TODO: Brief discussion of the texts I've compared to produce this text.

TODO: The numbers in the margin of the text refer to the pagination in the standard edition of Adam and Tannery.

TODO: Include brief discussion of how I've Classicized the text. (I removed accents, switched \textit{j}- back to \textit{i}-, and replaced \& with \textit{et}.)

% -]] About The Text
