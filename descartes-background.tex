% [[- Chapter title
\chapter{Background Material}
% -]] Chapter title

% [[- Plato's Theaetetus
\section*{Plato's \textit{Theaetetus} and the dream argument}

When Descartes uses dreaming as an argument in favor of global scepticism, he appears to have the following passage from Plato's \textit{Theaetetus} in mind. Socrates asks Theaetetus ``What is knowledge?'', and Theaetetus offers as a first definition that knowledge is perception. Socrates connects this to Protagorean relativism and Heracleitean flux, and in the selection below, he presses Theaetetus to consider some possible objections to his definition.

\begin{quote}
    \begin{greek}
        {\textbf{ΣΩΚΡΑΤΗΣ.} Ταῦτα δή, ὦ Θεαίτητε, ἆρ᾽ ἡδέα δοκεῖ σοι εἶναι, καὶ γεύοιο ἂν αὐτῶν ὡς ἀρεσκόντων;

        \textbf{ΘΕΑΙΤΗΤΟΣ.} Οὐκ οἶδα ἔγωγε, ὦ Σώκρατες· καὶ γὰρ οὐδὲ περὶ σοῦ δύναμαι κατανοῆσαι πότερα δοκοῦντά σοι λέγεις αὐτὰ ἢ ἐμοῦ ἀποπειρᾷ.

        \textbf{ΣΩ.} Οὐ μνημονεύεις, ὦ φίλε, ὅτι ἐγὼ μὲν οὔτ᾽ οἶδα οὔτε ποιοῦμαι τῶν τοιούτων οὐδὲν ἐμόν, ἀλλ᾽ εἰμὶ αὐτῶν ἄγονος, σὲ δὲ μαιεύομαι καὶ τούτου ἕνεκα ἐπᾴδω τε καὶ παρατίθημι ἑκάστων τῶν σοφῶν ἀπογεύσασθαι, ἕως ἂν εἰς φῶς τὸ σὸν δόγμα συνεξαγάγω· ἐξαχθέντος δὲ τότ᾽ ἤδη σκέψομαι εἴτ᾽ ἀνεμιαῖον εἴτε γόνιμον ἀναφανήσεται. ἀλλὰ θαρρῶν καὶ καρτερῶν εὖ καὶ ἀνδρείως ἀποκρίνου ἃ ἂν φαίνηταί σοι περὶ ὧν ἂν ἐρωτῶ.

        \textbf{ΘΕΑΙ.} Ἐρώτα δή.

        \textbf{ΣΩ.} Λέγε τοίνυν πάλιν εἴ σοι ἀρέσκει τὸ μή τι εἶναι ἀλλὰ γίγνεσθαι ἀεὶ ἀγαθὸν καὶ καλὸν καὶ πάντα ἃ ἄρτι διῇμεν.

        \textbf{ΘΕΑΙ.} Ἀλλ᾽ ἔμοιγε, ἐπειδὴ σοῦ ἀκούω οὕτω διεξιόντος, θαυμασίως φαίνεται ὡς ἔχειν λόγον καὶ ὑποληπτέον ᾗπερ διελήλυθας.

        \textbf{ΣΩ.} Μὴ τοίνυν ἀπολίπωμεν ὅσον ἐλλεῖπον αὐτοῦ. λείπεται δὲ ἐνυπνίων τε πέρι καὶ νόσων τῶν τε ἄλλων καὶ μανίας, ὅσα τε παρακούειν ἢ παρορᾶν ἤ τι ἄλλο παραισθάνεσθαι λέγεται. οἶσθα γάρ που ὅτι ἐν πᾶσι τούτοις ὁμολογουμένως ἐλέγχεσθαι δοκεῖ ὃν ἄρτι διῇμεν λόγον, ὡς παντὸς μᾶλλον ἡμῖν ψευδεῖς αἰσθήσεις ἐν αὐτοῖς γιγνομένας, καὶ πολλοῦ δεῖ τὰ φαινόμενα ἑκάστῳ ταῦτα καὶ εἶναι, ἀλλὰ πᾶν τοὐναντίον οὐδὲν ὧν φαίνεται εἶναι.

        \textbf{ΘΕΑΙ.} Ἀληθέστατα λέγεις, ὦ Σώκρατες.

        \textbf{ΣΩ.} Τίς δὴ οὖν, ὦ παῖ, λείπεται λόγος τῷ τὴν αἴσθησιν ἐπιστήμην τιθεμένῳ καὶ τὰ φαινόμενα ἑκάστῳ ταῦτα καὶ εἶναι τούτῳ ᾧ φαίνεται;

        \textbf{ΘΕΑΙ.} Ἐγὼ μέν, ὦ Σώκρατες, ὀκνῶ εἰπεῖν ὅτι οὐκ ἔχω τί λέγω, διότι μοι νυνδὴ ἐπέπληξας εἰπόντι αὐτό. ἐπεὶ ὡς ἀληθῶς γε οὐκ ἂν δυναίμην ἀμφισβητῆσαι ὡς οἱ μαινόμενοι ἢ οἱ ὀνειρώττοντες οὐ ψευδῆ δοξάζουσιν, ὅταν οἱ μὲν θεοὶ αὐτῶν οἴωνται εἶναι, οἱ δὲ πτηνοί τε καὶ ὡς πετόμενοι ἐν τῷ ὕπνῳ διανοῶνται.

        \textbf{ΣΩ.} Ἆρ᾽ οὖν οὐδὲ τὸ τοιόνδε ἀμφισβήτημα ἐννοεῖς περὶ αὐτῶν, μάλιστα δὲ περὶ τοῦ ὄναρ τε καὶ ὕπαρ;

        \textbf{ΘΕΑΙ.} Τὸ ποῖον;

        \textbf{ΣΩ.} Ὃ πολλάκις σε οἶμαι ἀκηκοέναι ἐρωτώντων, τί ἄν τις ἔχοι τεκμήριον ἀποδεῖξαι, εἴ τις ἔροιτο νῦν οὕτως ἐν τῷ παρόντι πότερον καθεύδομεν καὶ πάντα ἃ διανοούμεθα ὀνειρώττομεν, ἢ ἐγρηγόραμέν τε καὶ ὕπαρ ἀλλήλοις διαλεγόμεθα.

        \textbf{ΘΕΑΙ.} Καὶ μήν, ὦ Σώκρατες, ἄπορόν γε ὅτῳ χρὴ ἐπιδεῖξαι τεκμηρίῳ· πάντα γὰρ ὥσπερ ἀντίστροφα τὰ αὐτὰ παρακολουθεῖ. ἅ τε γὰρ νυνὶ διειλέγμεθα οὐδὲν κωλύει καὶ ἐν ὕπνῳ δοκεῖν ἀλλήλοις διαλέγεσθαι· καὶ ὅταν δὴ ὄναρ ὀνείρατα δοκῶμεν διηγεῖσθαι, ἄτοπος ἡ ὁμοιότης τούτων ἐκείνοις.

        \textbf{ΣΩ.} Ὁρᾷς οὖν ὅτι τό γε ἀμφισβητῆσαι οὐ χαλεπόν, ὅτε καὶ πότερόν ἐστιν ὕπαρ ἢ ὄναρ ἀμφισβητεῖται, καὶ δὴ ἴσου ὄντος τοῦ χρόνου ὃν καθεύδομεν ᾧ ἐγρηγόραμεν, ἐν ἑκατέρῳ διαμάχεται ἡμῶν ἡ ψυχὴ τὰ ἀεὶ παρόντα δόγματα παντὸς μᾶλλον εἶναι ἀληθῆ, ὥστε ἴσον μὲν χρόνον τάδε φαμὲν ὄντα εἶναι, ἴσον δὲ ἐκεῖνα, καὶ ὁμοίως ἐφ᾽ ἑκατέροις διισχυριζόμεθα.

        \textbf{ΘΕΑΙ.} Παντάπασι μὲν οὖν.

        \textbf{ΣΩ.} Οὐκοῦν καὶ περὶ νόσων τε καὶ μανιῶν ὁ αὐτὸς λόγος, πλὴν τοῦ χρόνου ὅτι οὐχὶ ἴσος;

        \textbf{ΘΕΑΙ.} Ὀρθῶς.

        \textbf{ΣΩ.} Τί οὖν; πλήθει χρόνου καὶ ὀλιγότητι τὸ ἀληθὲς ὁρισθήσεται;

        \textbf{ΘΕΑΙ.} Γελοῖον μεντἂν εἴη πολλαχῇ.

        \textbf{ΣΩ.} Ἀλλά τι ἄλλο ἔχεις σαφὲς ἐνδείξασθαι ὁποῖα τούτων τῶν δοξασμάτων ἀληθῆ;

        \textbf{ΘΕΑΙ.} Οὔ μοι δοκῶ.} (Plato's \textit{Theaetetus} 157c--158e4\footnote{The text follows \cite{plato1995}.})
    \end{greek}
\end{quote}

\begin{quote}
\textbf{Socrates.} Well, Theaetetus, does this look to you a tempting meal and could you take a bite of the delicious stuff?

\textbf{Theaetetus.} I really don't know, Socrates. I can't even quite see what you're getting at---whether the things you're saying are what you think yourself, or whether you're just trying me out.

\textbf{Soc.} You're forgetting, my friend. I don't know anything about this kind of thing myself, and I don't claim any of it as my own. I'm barren of theories; my business is to attend you in your labor. So I chant incantations over you and offer you little tidbits from each of the wise until I succeed in assisting you at bringing your own belief forth into the light. When it's been born, I'll consider whether it's fertile or a wind-egg. But you must have courage and patience; answer boldly whatever appears to you to be true about the things I ask you.

\textbf{Theaet.} All right, go on with your questions.

\textbf{Soc.} Tell me again, then, whether you like the suggestion that good and beautiful and all the things we were just speaking of cannot be said to `be' anything, but are always `coming to be'.

\textbf{Theaet.} Well, as far as I'm concerned, while I'm listening to your exposition of it, it seems to me an extraordinarily reasonable view; and I feel that the way you've set out the matter must be accepted.

\textbf{Soc.} In that case, we better not pass over any point where our theory is still incomplete. What we've not yet discussed is the question of dreams, and of insanity and other diseases; also what's called mishearing or misseeing or other cases of misperception. You realize, I suppose, that it would be generally agreed that all these cases appear to provide a refutation of the theory we've just expounded. For in these conditions, we surely have false perceptions. Here it's far from being true that all things which appear to an individual also are. On the contrary, no one of the things which appear to the person really is.

\textbf{Theaet.} That's very true, Socrates.

\textbf{Soc.} Well then, my child, what argument is left for the person who maintains that knowledge is perception and that what appears to anyone also is, for the person to whom it appears to be?

\textbf{Theaet.} Well, Socrates, I'm reluctant to tell you that I don't know what to say, since I've just got into trouble with you for that. But I really wouldn't know how to dispute the suggestion that a madman believes what is false when they think that they're a god; or a dreamer when they believe they have wings are are flying in their sleep.

\textbf{Soc.} But there's a point here which \textit{is} a matter of dispute, especially as regards dreams and real life---don't you see?

\textbf{Theaet.} What do you mean?

\textbf{Soc.} There's a question you must often have heard people ask---the question what evidence we could offer if we were asked whether in the present instance, at this moment, we are asleep and dreaming all our thoughts, or awake and talking to each other in real life.

\textbf{Theaet.} Yes, Socrates, it certainly is difficult to find the evidence we need here. The two states seem to correspond in all their characteristics. There's nothing to prevent us from thinking when we're asleep that we're having the very same discussion that we've just had. And when we dream that we're telling the story of a dream, there's an extraordinary likeness between the two experiences.

\textbf{Soc.} You see, then, it's not difficult to find matter for dispute, when it's disputed even whether this is real life or a dream. Indeed we may say that, as our periods of sleeping and waking are of equal length, and as in each period the soul contends that the beliefs of the moment are pre-eminently true, the result is that for half our lives we assert the reality of the one set of objects and for half that of the other set. And we make our assertions with equal conviction in both cases.

\textbf{Theaet.} That's certainly so.

\textbf{Soc.} And doesn't the same argument apply in the cases of disease and madness, except that the periods of time are not equal?

\textbf{Theaet.} Yes, that's right.

\textbf{Soc.} Well now, are we going to fix the limits of truth by the clock?

\textbf{Theaet.} That would be a very funny thing to do.

\textbf{Soc.} But can you produce some other clear indication to show which of these beliefs are true?

\textbf{Theaet.} I don't think I can. (lightly adapted translation by M.J. Levett)
\end{quote}

At this point in the dialogue, Socrates is not arguing for or against any particular thesis. He's elicited a definition of knowledge from Theaetetus---that knowledge is perception---and Socrates is still in the process of fleshing out the implications of that definition. Socrates raises the problems of dreams, disease, and insanity as part of that broader review of Theaetetus' definition.

We can readily see, however, why this passage might appeal to Descartes. Descartes wants to introduce general or broad reasons for sceptical doubt, and Socrates alludes to just such a broad scope for the dreaming argument in particular: ``[I]n these conditions, we surely have false perceptions. Here's it's far from being true that all things which appear to someone also are. On the contrary, none of the things which appear to someone also are.'' Dreams offer a convenient example of a situation in which \textit{everything} we think appears false or unreal.

This isn't to say that everything in dreams is equally surreal. I might dream that I'm doing something perfectly ordinary. For example, washing dishes. But of course I'm \textit{not} actually washing dishes because I'm asleep. I only think that I'm washing dishes.

% -]] Plato's Theaetetus
