% [[- Chapter title
\chapter{Meditatio Secunda}
% -]] Chapter title

% [[- Meditatio secunda

% [[- Introduction
In the second meditation, Descartes finds a certain truth: I exist as long as I am thinking. (Descartes doesn't use the phrase \textit{cogito ergo sum} in the meditations, though he does elsewhere.) As he unravels the significance of this truth, Descartes examines the nature of thought and what it means to be a creature that thinks. He argues that the essence of thought is perhaps different from what we expect and that our human nature is also different from what everyday beliefs suggest.

Descartes also answers the objection that physical things are better and more easily known than ourselves. To many people it may seem that our knowledge of everyday items---such as tables, chairs, rocks, trees, and so forth---is more immediate, clearer, and more important than knowledge of the self. Descartes offers a two-part reply to this. First, he argues that it is more difficult to know ordinary physical objects than we think. Second, he argues that the knowledge of such objects that we have comes from thought and not sensation. Thus, he concludes our nature as thinking things is more essential to us and easier for us to grasp than common sense suggests.

\clearpage
% -]] Introduction

% [[- An Archimedean point
\clearpage
\begin{center}
    \beginnumbering
    \numberlinefalse
    \pstart
    \textit{De natura mentis humanae: quod ipsa sit notior quam corpus}
    \pend
    \endnumbering
\end{center}

\beginnumbering
\pstart
\begin{latin}
    \textenglish{\textbf{1.}} In tantas dubitationes hesterna meditatione coniectus sum ut nequeam amplius earum oblivisci nec videam tamen qua ratione solvendae sint. Sed, tanquam in profundum gurgitem ex improviso delapsus, ita turbatus sum ut nec possim in imo pedem figere nec enatare ad summum. Enitar tamen et tentabo rursus eandem viam quam heri fueram ingressus removendo scilicet illud omne quod vel minimum dubitationis admittit nihilo secius quam si omnino falsum esse comperissem; pergamque porro donec aliquid certi vel, si nihil aliud, saltem hoc ipsum pro certo---nihil esse certi---cognoscam. Nihil nisi punctum petebat Archimedes quod esset firmum et immobile ut integram terram loco dimoveret. Magna quoque speranda sunt si vel minimum quid invenero quod certum sit et inconcussum.
\end{latin}
\pend
\endnumbering

\beginnumbering
\pstart
\begin{latin}
    \textenglish{\textbf{2.}} Suppono igitur omnia quae video falsa esse. Credo nihil unquam extitisse eorum quae mendax memoria repraesentat. Nullos plane habeo sensus. Corpus, figura, extensio, motus, locusque sunt chimerae. Quid igitur erit verum? Fortassis hoc unum: nihil esse certi.
\end{latin}
\pend
\endnumbering

\prenotes

\textbf{§1.} Yesterday was hard. We wanted to doubt as much as possible, but we may have succeeded far too well. We may end up proving that \textit{everything} can be doubted, in which case the only certain thing will be that nothing is certain. Still, we should keep looking: if we can find even one certain truth, we can use it as a foundation for so much more knowledge.

\lemc{1 hesterna} although Descartes expressed hope that readers would spend as long as needed---weeks or months, if necessary---on the first meditation, the meditations as a whole follow the dramatic convention that each meditation is one day for the narrator.

\lemc{2 qua ratione} not simply an alternative for \textit{quo modo}. The trouble that Descartes has caused himself must be solved by reason, not in any other way.

\lemc{2--3 Sed\dots summum} We might try to connect each part of this analogy to a different outcome. If Descartes discovers that there is nothing certain, he will have reached the bottom of the whirlpool of doubt. If, on the other hand, he discovers a certainty, he escapes doubt and swims to the surface. However, even if we are not so precise, the analogy conveys how utterly overpowered Descartes feels during his exericse in radical doubt.

\lemc{3 turbatus sum} a form like this is ambiguous. The default interpretation is that it is a perfect passive: `I was troubled'. In this case, we take the participle and the form of \textit{esse} together as one compound verb form. But in some cases the words should be taken separately: a form of \textit{esse} and a participle serving as an attributive adjective. On this interpretation, the two words here are in the present tense: `I am troubled'. Many perfect passive participles show this ambiguity between their use in compound passive forms and their use as adjectives. The second reading makes better sense in this passage: take \textit{turbatus} as an adjective rather than as half of a compound verb.

\lemc{3--4 nec\dots nec} despite the position of the first \textit{nec}, these two join the infinitives \textit{figere} and \textit{enatare} rather than two main verbs.

\lemc{5 fueram ingressus} = \textit{eram ingressus}, the pluperfect tense. The pluperfect passive in classical Latin is normally the imperfect of \textit{esse} and a perfect passive participle, but there are scattered examples like this: a pluperfect of \textit{esse} and the perfect passive participle. The tendency towards these shifted pluperfects increases in later Latin. See H-S §179 and, more generally, \citet[§298]{väänänen1981}.

\lemc{5 removendo} ablative of means of a gerund. Descartes will try again `by setting aside' anything which may be uncertain.

\lemc{6 nihilo secius} probably a litotes. That is `not at all other than if' means `exactly as if'.

\lemc{7 aliquid certi} see the note on I §8.6.

\lemc{7 saltem hoc ipsum} the demonstrative \textit{hic, haec, hoc} is often used to point forward. In this case, it anticipates \textit{nihil esse certi}. Descartes will keep striving for certainty, even if the only certain thing he discovers is that there is no certainty.

\lemc{8 punctum petebat Archimedes} Archimedes was an important scientist and mathematician (c. 287-212 BCE, born in Syracuse). Descartes alludes to his work with levers: Archimedes famously said, `Give me a place to stand, and I will move the Earth'. In the same way that a lever magnifies a person's strength, Descartes believes that the discovery of a single certainty will indirectly increase his knowledge far more than the one certainty itself. If Descartes is right, a single certainty can provide the foundation for a vast edifice of scientific knowledge.

\textbf{§2.} To recap our program of doubt: we will doubt all the evidence of our senses, we will not believe our memories, we will even reject the most general categories of thought. Perhaps the only truth left will be that there is no certainty.

\lem{1 unquam} = \textit{umquam}

\lemc{3 chimerae} In Homer's \textit{Iliad} the chimera was lion in the front, snake in the back and goat in the middle. It also breathed fire. To say that something is `a chimera' is to say that it is mere fantasy.

\lemc{4 nihil esse certi} implied indirect statement depending on the understood \textit{erit verum}.

% -]] An Archimedean point

% [[- Ego sum, ego existo
\clearpage

\beginnumbering
\pstart
\begin{latin}
    \textenglish{\textbf{3.}} Sed unde scio nihil esse diversum ab iis omnibus quae iam iam recensui de quo ne minima quidem occasio sit dubitandi? Nunquid est aliquis Deus, vel quocunque nomine illum vocem, qui mihi has ipsas cogitationes immittit? Quare vero hoc putem cum forsan ipsemet illarum author esse possim? Nunquid ergo saltem ego aliquid sum? Sed iam negavi me habere ullos sensus et ullum corpus. Haereo tamen; nam quid \at{25} inde? Sumne ita corpori sensibusque alligatus ut sine illis esse non possim? Sed mihi persuasi nihil plane esse in mundo: nullum coelum, nullam terram, nullas mentes, nulla corpora; nonne igitur etiam me non esse? Imo certe ego eram si quid mihi persuasi. Sed est deceptor nescio quis, summe potens, summe callidus, qui de industria me semper fallit. Haud dubie igitur ego etiam sum si me fallit. Et fallat quantum potest: nunquam tamen efficiet ut nihil sim quamdiu me aliquid esse cogitabo. Adeo ut, omnibus satis superque pensitatis, denique statuendum sit hoc pronuntiatum `Ego sum, ego existo' quoties a me profertur vel mente concipitur necessario esse verum.
\end{latin}
\pend
\endnumbering

\prenotes

\textbf{§3.} How can we even be sure that there is nothing certain? Perhaps we simply haven't found it yet. Whatever source we imagine for our thoughts---god, a demon, or ourselves---we must at least be something insofar as we're thinking. It doesn't matter that everything we normally think about ourselves and the world is false. Even if that's so, we \textit{are} thinking it. Go back to the evil demon idea: let him trick us all he likes, we must therefore exist for him to trick. As long as we are thinking, we aren't nothing. We exist.

\lem{2 Nunquid} = \textit{Numquid}, a word with no one exact translation into English. In classical Latin it introduces questions that (i) involve emotion, (ii) express incredulity, or (iii) are rhetorical. (These categories can overlap.) Usually questions with \textit{numquid} expect a negative answer. In the meditations, however, Descartes uses \textit{nunquid} in contexts where it is clear he expects an affirmative answer. In this case, Descartes is suggesting the existence of god as perhaps something that provides no possibility for doubt.

\lemc{4 putem} a deliberative subjunctive `Why should I think this?' The existence of god is not certain since Descartes may be responsible for his own thoughts.

\lemc{4 Nunquid ergo\dots ego aliquid sum} if there is no god and Descartes himself is responsible for his thoughts, then surely he must exist.

\lemc{6 nam quid inde?} the verb implied is something like `follows' or `happens'. Descartes is confused: he's admitted that he may have no body at all, and he's not sure what follows from that. In particular, he's not sure that it means that he might not exist, which is the question before him.

\lemc{8 me non esse} resupply \textit{mihi persuasi} from earlier in the sentence to govern this indirect statement.

\lemc{8 Imo certe} this is the turning point from doubt towards certainty. These two words emphatically reject the last question, and the rest of the sentence justifies Descartes conviction that he must exist.

\lemc{9 si quid mihi persuasi} in English we would say `if I persuaded myself of anything' or `about anything', but in Latin the `something' that a person comes to believe is the accusative direct object of \textit{persuadeo}. Normally, however, we don't notice this because the object of the verb is usually a clause: an indirect statement or an indirect command.

\lemc{11 fallat} an independent subjunctive expressing a command.

\lemc{11--12 quamdiu\dots cogitabo} on first reading this clause may not stand out, but it will turn out to matter a great deal. Descartes will suggest that if he stops thinking entirely, he ceases to exist. So he means it when he says that he will always be something `as long as' he thinks.

\lemc{12 Adeo ut} these words introduce what serves as the main clause of the sentence. They are functionally equivalent here to \textit{igitur} or \textit{quare}.

\lemc{12 satis superque} Descartes has thought about these matters `enough and more than enough'. I.e., very extensively.

\lemc{12--13 statuendum sit} the subject of this verb is the indirect statement \textit{hoc pronuntiatum\dots necessario esse verum}.

\lemc{13 hoc} anticipates \textit{`Ego sum, ego existo'}. Those words are `this statement'.

\lemc{necessario} the necessary truth is `that I exist whenever I say so or think I do' not simply `that I exist'. Descartes is not committed to the view that he must exist.
% -]] Ego sum, ego existo

% [[- What is the ego that exists?
\clearpage

\beginnumbering
\pstart
\begin{latin}
    \textenglish{\textbf{4.}} Nondum vero satis intelligo quisnam sim ego ille qui iam necessario sum; deincepsque cavendum est ne forte quid aliud imprudenter assumam in locum mei sicque aberrem etiam in ea cognitione quam omnium certissimam evidentissimamque esse contendo. Quare iam denuo meditabor quidnam me olim esse crediderim priusquam in has cogitationes incidissem; ex quo deinde subducam quidquid allatis rationibus vel minimum potuit infirmari ut ita tandem praecise remaneat illud tantum quod certum est et inconcussum.
\end{latin}
\pend
\endnumbering

\beginnumbering
\pstart
\begin{latin}
    \textenglish{\textbf{5.}} Quidnam igitur antehac me esse putavi? Hominem scilicet. Sed quid est homo? Dicamne animal rationale? Non, quia postea quaerendum foret quidnam animal sit et quid rationale, atque ita ex una quaestione in plures difficilioresque delaberer; nec iam mihi tantum otii est ut illo velim inter istiusmodi subtilitates abuti. Sed hic potius attendam quid sponte \at{26} et natura duce cogitationi meae antehac occurrebat quoties quid essem considerabam.
\end{latin}
\pend
\endnumbering

\prenotes

\textbf{§4.} We have our one certainty: we exist. But how much substance does this certainty have? What do we know about the `we' that exists? What are we? If we are not careful, we will quickly go wrong just as we've found a tiny bit of certainty. To prevent that, let's use the same method we just employed: we will consider what we think about ourselves, and we'll reject whatever has any doubt. What's left will be certain and unshakeable.

\lem{1 intelligo} = \textit{intellego}.

\lem{2 cavendum est} introduces a compound fear clause containing two verbs joined by \textit{que}: \textit{assumam} (2) and \textit{aberrem} (3).

\lemc{3 omnium certissimam evidentissimam} the partitive genitive is common with superlatives.

\lemc{crediderim} perfect subjunctive in an indirect question following \textit{meditabor}.

\lemc{5--6 allatis rationibus} ablative of means with \textit{infirmari} (6).

\lemc{6 vel} adverbial rather than a conjunction. It means `even'.

\lemc{6 minimum} an adverbial use of the accusative, meaning `in the smallest degree' or `the least bit'.

\lemc{6 potuit} the tense is somewhat odd, and we would expect a future perfect in Latin. The point is that Descartes will remove whatever will prove to be doubtful once he has brought forward arguments. The removal will happen in the future, and so \textit{subducam} is future. At that time, Descartes will \textit{already} have proven that doubt is possible. Hence, he uses a perfect tense for \textit{potuit}, even though this is not strictly speaking the correct tense. It hasn't happened yet, but it will have happened at an imagined moment in the future.

\lemc{6 tandem praecise} despite the imprecise word order, you should take \textit{tandem} with \textit{remaneat} and \textit{praecise} with \textit{illud}.

\lemc{7 tantum} not `so big' or `so great' but `only' or `just'. This is a common idiomatic use of \textit{tantum}, and you must take care to distinguish it from the other meanings of the word.

\textbf{§5.} What have we believed up until now about ourselves? We believe that we are people, that people have bodies and minds, and that bodies and minds are responsible for distinct features of what it is to be a person.

\lemc{1 Hominem scilicet} supply \textit{me esse putavi} to make this fragment a complete thought.

\lemc{2 Dicam} a deliberative subjunctive.

\lemc{2 animal rationale} this is a traditional definition of `human', derived ultimately from Aristotle. Descartes rejects it since (i) it would drag him into arguments about definitions and (ii) he doesn't think that it possesses the kind of certainty he hopes to find. He only states (i) explicitly.

\lemc{2 Non} that is, `I should not say this'.

\lemc{2 foret} an alternative form of \textit{esset}, the imperfect subjunctive of \textit{esse}. The imperfect subjunctive is used in an implied present contrary-to-fact conditional sentence: `I should not say this because (if I were to say it), I would have to pursue tedious questions about definitions'.

\lemc{3 plures difficiliores} Latin frequently uses `and' with the adjective \textit{multus, multa, multum} in a way that is unidiomatic in English. E.g., `many and difficult problems' rather than `many difficult problems'. This is the same thing, but \textit{plures} is comparative.

\lemc{4 mihi} dative of possession with \textit{tantum otii est}.

\lemc{4 illo} ablative object of the verb \textit{abuti} (5).

\lemc{5 natura duce} an ablative absolute. Classical Latin has no present participle for \textit{esse}, and so the verb `being' is implied in an ablative absolute consisting of two nouns, like this one. (Less often you'll see one noun and one adjective. E.g. \textit{Aenea pio}: `with Aeneas being dutiful' or `since Aeneas is dutiful'.) `With nature being leader' is equivalent to something like `under the guidance of nature' or, even less literally, `naturally'.

\lemc{5 cogitationi meae} dative with the compound verb \textit{occurrebat} (6).
% -]] What is the ego that exists?

% [[- What is the ego (cont.)?
\clearpage

\beginnumbering
\pstart
\setline{7}
\begin{latin}
    \textenglish{\textbf{5. (cont.)}} Nempe occurrebat primo me habere vultum, manus, brachia, totamque hanc membrorum machinam qualis etiam in cadavere cernitur et quam corporis nomine designabam. Occurrebat praeterea me nutriri, incedere, sentire, et cogitare: quas quidem actiones ad animam referebam. Sed quid esset haec anima, vel non advertebam, vel exiguum nescio quid imaginabar, instar venti, vel ignis, vel aetheris, quod crassioribus mei partibus esset infusum. De corpore vero ne dubitabam quidem, sed distincte me nosse arbitrabar eius naturam, quam si forte, qualem mente concipiebam, describere tentassem, sic explicuissem: per `corpus' intelligo illud omne quod aptum est figura aliqua terminari, loco circumscribi, spatium sic replere ut ex eo aliud omne corpus excludat; tactu, visu, auditu, gustu, vel odoratu percipi, necnon moveri pluribus modis---non quidem a seipso, sed ab alio quopiam a quo tangatur. Namque habere vim seipsum movendi, item sentiendi, vel cogitandi, nullo pacto ad naturam corporis pertinere iudicabam; quinimo mirabar potius tales facultates in quibusdam corporibus reperiri.
\end{latin}
\pend
\endnumbering

\prenotes

\lemc{8 membrorum machinam} a striking phrase that implicitly puts the reader in the right frame of mind to think of a human body as something separate from the person whose body it is. (The genitive is likely of material (\textbf{AG} §344).)

\lemc{8 qualis etiam in cadavere cernitur} this phrase also suggests the gap between a person and that person's body.

\lemc{9 nutriri} we might think of feeding as something more bodily than mental, but there is a tradition stretching back at least as far as Aristotle that argues otherwise. The idea is this: anything which is alive possesses what Aristotle would call a ψυχή and Descartes calls an \textit{anima}: a soul. Even plants, as living creatures, possess souls. Their souls, however, are responsible only for the most basic functions of life: nutrition, growth, and reproduction. (This verb appears in classical Latin both in deponent and non-deponent forms.)

\lemc{10 Sed quid esset\dots} Descartes rarely thought about the nature of the soul before. When he did, he went no further than commonplace analogies of the soul as a kind of wind, flame or air spread throughout his limbs.

\lemc{12--13 ne dubitabam quidem} remember that \textit{ne X quidem} means `not even X'.

\lemc{13--14 qualem mente concipiebam} depends on \textit{describere} (14). That is, `to describe what sort of thing I thought (that it was)'.

\lem{14 tentassem} = \textit{tentavissem} (\textbf{AG} §181). This process is often known as `syncopation'.

\lemc{15 aptum est} the infinitives \textit{terminari} (15), \textit{circumscribi} (15), \textit{replere} (16), \textit{percipi} (17), and \textit{moveri} (17) all depend on this phrase. They are explanatory or limiting infinitives: they explain qualities that befit (\textit{aptum}) a body. This sort of infinitive, often called `epexegetical', is common in Greek but rare in classical Latin. Roman poets began to use the construction in imitation of Greek, and it became more common in later Latin.

\lem{17 seipso} = \textit{se ipso}.

\lemc{18 habere vim} this infinitive depends on \textit{pertinere iudicabam} (19). Descartes did not believe (literally `judge') that it belonged to the nature of the body `to have the power'.

\lem{18 seipsum} = \textit{se ipsum}.

\lemc{18 movendi\dots sentiendi\dots cogitandi} these genitives define \textit{vim}. In English we would more likely say `the power to move, to perceive, to think', but gerunds are possible for us too.

\lemc{19 nullo pacto} an common ablative of manner meaning `in no way'. \textit{quo pacto}, sometimes written as one word, similarly means `in what way?' or simply `how?'.

\lem{19 quinimo} = \textit{quin immo}, indicating an emphatic contrast to the preceding thought. There's an important contrast between something that `belongs to the nature of body' and something that is `found in a body'. The first refers to essential qualities of bodies; the second includes non-essential qualities. Descartes didn't think that movement, perception, or thought were essential to bodies, and \textit{what's more} (\textit{quinimo}) he was surprised to find them associated with bodies at all. Throughout this paragraph Descartes admits that he was predisposed to sharply distinguish body and mind.
% -]] What is the ego (cont.)?

% -]] Meditatio secunda
