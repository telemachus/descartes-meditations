% [[- Chapter title
\chapter{Meditatio Secunda}
% -]] Chapter title

% [[- Meditatio secunda

% [[- Introduction
In the second meditation, Descartes finds a certain truth: I exist as long as I am thinking. (Descartes doesn't use the phrase \textit{cogito ergo sum} in the meditations, though he does elsewhere.) As he unravels the significance of this truth, Descartes examines the nature of thought and what it means to be a creature that thinks. He argues that the essence of thought is perhaps different from what we expect and that our human nature is also different from what everyday beliefs suggest.

Descartes also answers the objection that physical things are better and more easily known than ourselves. To many people it may seem that our knowledge of everyday items---such as tables, chairs, rocks, trees, and so forth---is more immediate, clearer, and more important than knowledge of the self. Descartes offers a two-part reply to this. First, he argues that it is more difficult to know ordinary physical objects than we think. Second, he argues that the knowledge of such objects that we have comes from thought and not sensation. Thus, he concludes our nature as thinking things is more essential to us and easier for us to grasp than common sense suggests.

\clearpage
% -]] Introduction

% [[- An Archimedean point
\clearpage
\begin{center}
    \beginnumbering
    \numberlinefalse
    \pstart
    \textit{De natura mentis humanae: quod ipsa sit notior quam corpus}
    \pend
    \endnumbering
\end{center}

\beginnumbering
\pstart
\begin{latin}
    \textenglish{\textbf{1.}} In tantas dubitationes hesterna meditatione coniectus sum, ut nequeam amplius earum oblivisci, nec videam tamen qua ratione solvendae sint; sed, tanquam in profundum gurgitem ex improviso delapsus, ita turbatus sum, ut nec possim in imo pedem figere, nec enatare ad summum. Enitar tamen et tentabo rursus eandem viam quam heri fueram ingressus, removendo scilicet illud omne quod vel minimum dubitationis admittit, nihilo secius quam si omnino falsum esse comperissem; pergamque porro donec aliquid certi, vel, si nihil aliud, saltem hoc ipsum pro certo, nihil esse certi, cognoscam. Nihil nisi punctum petebat Archimedes, quod esset firmum et immobile, ut integram terram loco dimoveret; magna quoque speranda sunt, si vel minimum quid invenero quod certum sit et inconcussum.
\end{latin}
\pend
\endnumbering

\beginnumbering
\pstart
\begin{latin}
    \textenglish{\textbf{2.}} Suppono igitur omnia quae video falsa esse; credo nihil unquam extitisse eorum quae mendax memoria repraesentat; nullos plane habeo sensus; corpus, figura, extensio, motus, locusque sunt chimerae. Quid igitur erit verum? Fortassis hoc unum, nihil esse certi.
\end{latin}
\pend
\endnumbering

\prenotes

\textbf{§1.} Yesterday was hard. We wanted to doubt as much as possible, but we may have succeeded far too well. We may end up proving that \textit{everything} can be doubted, in which case the only certain thing will be that nothing is certain. Still, we should keep looking: if we can find even one certain truth, we can use it as a foundation for so much more knowledge.

\lemc{1 hesterna} although Descartes expressed hope that readers would spend as long as needed---weeks or months, if necessary---on the first meditation, the meditations as a whole follow the dramatic convention that each meditation is one day for the narrator.

\lemc{2 qua ratione} not simply an alternative for \textit{quo modo}. The trouble that Descartes has caused himself must be solved by reason, not in any other way.

\lemc{2--3 Sed\dots summum} We might try to connect each part of this analogy to a different outcome. If Descartes discovers that there is nothing certain, he will have reached the bottom of the whirlpool of doubt. If, on the other hand, he discovers a certainty, he escapes doubt and swims to the surface. However, even if we are not so precise, the analogy conveys how utterly overpowered Descartes feels during his exericse in radical doubt.

\lemc{3 turbatus sum} a form like this is ambiguous. The default interpretation is that it is a perfect passive: ``I was troubled''. In this case, we take the participle and the form of \textit{esse} together as one compound verb form. But in some cases the words should be taken separately: a form of \textit{esse} and a participle serving as an attributive adjective. On this interpretation, the two words here are in the present tense: ``I am troubled''. Many perfect passive participles show this ambiguity between their use in compound passive forms and their use as adjectives. The second reading makes better sense in this passage: take \textit{turbatus} as an adjective rather than as half of a compound verb.

\lemc{3--4 nec\dots nec} despite the position of the first \textit{nec}, these two join the infinitives \textit{figere} and \textit{enatare} rather than two main verbs.

\lemc{5 fueram ingressus} = \textit{eram ingressus}, the pluperfect tense. The pluperfect passive in classical Latin is normally the imperfect of \textit{esse} and a perfect passive participle, but there are scattered examples like this: a pluperfect of \textit{esse} and the perfect passive participle. The tendency towards these shifted pluperfects increases in later Latin. See H-S §179 and, more generally, \cite[§298]{väänänen1981}.

\lemc{5 removendo} ablative of means of a gerund. Descartes will try again ``by setting aside'' anything which may be uncertain.

\lemc{6 nihilo secius} probably a litotes. That is ``not at all other than if'' means ``exactly as if''.

\lemc{7 aliquid certi} see the note on I §8.6.

\lemc{7 saltem hoc ipsum} the demonstrative \textit{hic, haec, hoc} is often used to point forward. In this case, it anticipates \textit{nihil esse certi}. Descartes will keep striving for certainty, even if the only certain thing he discovers is that there is no certainty.

\lemc{8 punctum petebat Archimedes} Archimedes was an important scientist and mathematician (c. 287-212 BCE, born in Syracuse). Descartes alludes to his work with levers: Archimedes famously said, ``Give me a place to stand, and I will move the Earth''. In the same way that a lever magnifies a person's strength, Descartes believes that the discovery of a single certainty will indirectly increase his knowledge far more than the one certainty itself. If Descartes is right, a single certainty can provide the foundation for a vast edifice of scientific knowledge.

\textbf{§2.} To recap our program of doubt: we will doubt all the evidence of our senses, we will not believe our memories, we will even reject the most general categories of thought. Perhaps the only truth left will be that there is no certainty.

\lem{1 unquam} = \textit{umquam}

\lemc{3 chimerae} In Homer's \textit{Iliad} the chimera was lion in the front, snake in the back and goat in the middle. It also breathed fire. To say that something is ``a chimera'' is to say that it is mere fantasy.

\lemc{4 nihil esse certi} implied indirect statement depending on the understood \textit{erit verum}.

% -]] An Archimedean point

% [[- Ego sum, ego existo
\clearpage

\beginnumbering
\pstart
\begin{latin}
    \textenglish{\textbf{3.}} Sed unde scio nihil esse diversum ab iis omnibus quae iam iam recensui, de quo ne minima quidem occasio sit dubitandi? Nunquid est aliquis Deus, vel quocunque nomine illum vocem, qui mihi has ipsas cogitationes immittit? Quare vero hoc putem, cum forsan ipsemet illarum author esse possim? Nunquid ergo saltem ego aliquid sum? Sed iam negavi me habere ullos sensus, et ullum corpus. Haereo tamen; nam quid \at{25} inde? Sumne ita corpori sensibusque alligatus, ut sine illis esse non possim? Sed mihi persuasi nihil plane esse in mundo, nullum coelum, nullam terram, nullas mentes, nulla corpora; nonne igitur etiam me non esse? Imo certe ego eram, si quid mihi persuasi. Sed est deceptor nescio quis, summe potens, summe callidus, qui de industria me semper fallit. Haud dubie igitur ego etiam sum, si me fallit; et fallat quantum potest, nunquam tamen efficiet, ut nihil sim quamdiu me aliquid esse cogitabo. Adeo ut, omnibus satis superque pensitatis, denique statuendum sit hoc pronuntiatum \textit{Ego sum, ego existo}, quoties a me profertur, vel mente concipitur, necessario esse verum.
\end{latin}
\pend
\endnumbering

\prenotes

\textbf{§3.} How can we even be sure that there is nothing certain? Perhaps we simply haven't found it yet. Whatever source we imagine for our thoughts---god, a demon, or ourselves---we must at least be something insofar as we're thinking. It doesn't matter that everything we normally think about ourselves and the world is false. Even if that's so, we \textit{are} thinking it. Go back to the evil demon idea: let him trick us all he likes, we must therefore exist for him to trick. As long as we are thinking, we aren't nothing. We exist.

\lem{2 Nunquid} = \textit{Numquid}, a word with no one exact translation into English. In classical Latin it introduces questions that (i) involve emotion, (ii) express incredulity, or (iii) are rhetorical. (These categories can overlap.) Usually questions with \textit{numquid} expect a negative answer. In the meditations, however, Descartes uses \textit{nunquid} in contexts where it is clear he expects an affirmative answer. In this case, Descartes is suggesting the existence of god as perhaps something that provides no possibility for doubt.

\lemc{4 putem} a deliberative subjunctive ``Why should I think this?'' The existence of god is not certain since Descartes may be responsible for his own thoughts.

\lemc{4 Nunquid ergo\dots ego aliquid sum} if there is no god and Descartes himself is responsible for his thoughts, then surely he must exist.

\lemc{6 nam quid inde?} the verb implied is something like ``follows'' or ``happens''. Descartes is confused: he's admitted that he may have no body at all, and he's not sure what follows from that. In particular, he's not sure that it means that he might not exist, which is the question before him.

\lemc{8 me non esse} resupply \textit{mihi persuasi} from earlier in the sentence to govern this indirect statement.

\lemc{8 Imo certe} this is the turning point from doubt towards certainty. These two words emphatically reject the last question, and the rest of the sentence justifies Descartes conviction that he must exist.

\lemc{9 si quid mihi persuasi} in English we would say ``if I persuaded myself of anything'' or ``about anything'', but in Latin the ``something'' that a person comes to believe is the accusative direct object of \textit{persuadeo}. Normally, however, we don't notice this because the object of the verb is usually a clause: an indirect statement or an indirect command.

\lemc{11 fallat} an independent subjunctive expressing a command.

\lemc{11--12 quamdiu\dots cogitabo} on first reading this clause may not stand out, but it will turn out to matter a great deal. Descartes will suggest that if he stops thinking entirely, he ceases to exist. So he means it when he says that he will always be something ``as long as'' he thinks.

\lemc{12 Adeo ut} these words introduce what serves as the main clause of the sentence. They are functionally equivalent here to \textit{igitur} or \textit{quare}.

\lemc{12 satis superque} Descartes has thought about these matters ``enough and more than enough''. I.e., very extensively.

\lemc{12--13 statuendum sit} the subject of this verb is the indirect statement \textit{hoc pronuntiatum\dots necessario esse verum}.

\lemc{13 hoc} anticipates \textit{``Ego sum, ego existo''}. Those words are ``this statement''.

\lemc{necessario} the necessary truth is ``that I exist whenever I say so or think I do'' not simply ``that I exist''. Descartes is not committed to the view that he must exist.
% -]] Ego sum, ego existo

% [[- What is the ego that exists?
\clearpage

\beginnumbering
\pstart
\begin{latin}
    \textenglish{\textbf{4.}} Nondum vero satis intelligo, quisnam sim ego ille, qui iam necessario sum; deincepsque cavendum est ne forte quid aliud imprudenter assumam in locum mei, sicque aberrem etiam in ea cognitione, quam omnium certissimam evidentissimamque esse contendo. Quare iam denuo meditabor quidnam me olim esse crediderim, priusquam in has cogitationes incidissem; ex quo deinde subducam quidquid allatis rationibus vel minimum potuit infirmari, ut ita tandem praecise remaneat illud tantum quod certum est et inconcussum.
\end{latin}
\pend
\endnumbering

\beginnumbering
\pstart
\begin{latin}
    \textenglish{\textbf{5.}} Quidnam igitur antehac me esse putavi? Hominem scilicet. Sed quid est homo? Dicamne animal rationale? Non, quia postea quaerendum foret quidnam animal sit, et quid rationale, atque ita ex una quaestione in plures difficilioresque delaberer; nec iam mihi tantum otii est, ut illo velim inter istiusmodi subtilitates abuti. Sed hic potius attendam, quid sponte \at{26} et natura duce cogitationi meae antehac occurrebat, quoties quid essem considerabam.
\end{latin}
\pend
\endnumbering

\prenotes

\textbf{§4.} We have our one certainty: we exist. But how much substance does this certainty have? What do we know about the ``we'' that exists? What are we? If we are not careful, we will quickly go wrong just as we've found a tiny bit of certainty. To prevent that, let's use the same method we just employed: we will consider what we think about ourselves, and we'll reject whatever has any doubt. What's left will be certain and unshakeable.

\lem{1 intelligo} = \textit{intellego}.

\lem{2 cavendum est} introduces a compound fear clause containing two verbs joined by \textit{que}: \textit{assumam} (2) and \textit{aberrem} (3).

\lemc{3 omnium certissimam evidentissimam} the partitive genitive is common with superlatives.

\lemc{crediderim} perfect subjunctive in an indirect question following \textit{meditabor}.

\lemc{5--6 allatis rationibus} ablative of means with \textit{infirmari} (6).

\lemc{6 vel} adverbial rather than a conjunction. It means ``even''.

\lemc{6 minimum} an adverbial use of the accusative, meaning ``in the smallest degree'' or ``the least bit''.

\lemc{6 potuit} the tense is somewhat odd, and we would expect a future perfect in Latin. The point is that Descartes will remove whatever will prove to be doubtful once he has brought forward arguments. The removal will happen in the future, and so \textit{subducam} is future. At that time, Descartes will \textit{already} have proven that doubt is possible. Hence, he uses a perfect tense for \textit{potuit}, even though this is not strictly speaking the correct tense. It hasn't happened yet, but it will have happened at an imagined moment in the future.

\lemc{6 tandem praecise} despite the imprecise word order, you should take \textit{tandem} with \textit{remaneat} and \textit{praecise} with \textit{illud}.

\lemc{7 tantum} not ``so big'' or ``so great'' but ``only'' or ``just''. This is a common idiomatic use of \textit{tantum}, and you must take care to distinguish it from the other meanings of the word.

\textbf{§5.} What have we believed up until now about ourselves? We believe that we are people, that people have bodies and minds, and that bodies and minds are responsible for distinct features of what it is to be a person.

\lemc{1 Hominem scilicet} supply \textit{me esse putavi} to make this fragment a complete thought.

\lemc{2 Dicam} a deliberative subjunctive.

\lemc{2 animal rationale} this is a traditional definition of ``human'', derived ultimately from Aristotle. Descartes rejects it since (i) it would drag him into arguments about definitions and (ii) he doesn't think that it possesses the kind of certainty he hopes to find. He only states (i) explicitly.

\lemc{2 Non} that is, ``I should not say this''.

\lemc{2 foret} an alternative form of \textit{esset}, the imperfect subjunctive of \textit{esse}. The imperfect subjunctive is used in an implied present contrary-to-fact conditional sentence: ``I should not say this because (if I were to say it), I would have to pursue tedious questions about definitions''.

\lemc{3 plures difficiliores} Latin frequently uses ``and'' with the adjective \textit{multus, multa, multum} in a way that is unidiomatic in English. E.g., ``many and difficult problems'' rather than ``many difficult problems''. This is the same thing, but \textit{plures} is comparative.

\lemc{4 mihi} dative of possession with \textit{tantum otii est}.

\lemc{4 illo} ablative object of the verb \textit{abuti} (5).

\lemc{5 natura duce} an ablative absolute. Classical Latin has no present participle for \textit{esse}, and so the verb ``being'' is implied in an ablative absolute consisting of two nouns, like this one. (Less often you'll see one noun and one adjective. E.g. \textit{Aenea pio}: ``with Aeneas being dutiful'' or ``since Aeneas is dutiful''.) ``With nature being leader'' is equivalent to something like ``under the guidance of nature'' or, even less literally, ``naturally''.

\lemc{5 cogitationi meae} dative with the compound verb \textit{occurrebat} (6).
% -]] What is the ego that exists?

% [[- What is the ego (cont.)?
\clearpage

\beginnumbering
\pstart
\setline{7}
\begin{latin}
    \textenglish{\textbf{5. (cont.)}} Nempe occurrebat primo, me habere vultum, manus, brachia, totamque hanc membrorum machinam, qualis etiam in cadavere cernitur, et quam corporis nomine designabam. Occurrebat praeterea me nutriri, incedere, sentire, et cogitare: quas quidem actiones ad animam referebam. Sed quid esset haec anima, vel non advertebam, vel exiguum nescio quid imaginabar, instar venti, vel ignis, vel aetheris, quod crassioribus mei partibus esset infusum. De corpore vero ne dubitabam quidem, sed distincte me nosse arbitrabar eius naturam, quam si forte, qualem mente concipiebam, describere tentassem, sic explicuissem: per corpus intelligo illud omne quod aptum est figura aliqua terminari, loco circumscribi, spatium sic replere, ut ex eo aliud omne corpus excludat; tactu, visu, auditu, gustu, vel odoratu percipi, necnon moveri pluribus modis, non quidem a seipso, sed ab alio quopiam a quo tangatur. Namque habere vim seipsum movendi, item sentiendi, vel cogitandi, nullo pacto ad naturam corporis pertinere iudicabam; quinimo mirabar potius tales facultates in quibusdam corporibus reperiri.
\end{latin}
\pend
\endnumbering

\prenotes

\lemc{8 membrorum machinam} a striking phrase that implicitly puts the reader in the right frame of mind to think of a human body as something separate from the person whose body it is. (The genitive is likely of material (\textbf{AG} §344).)

\lemc{8 qualis etiam in cadavere cernitur} this phrase also suggests the gap between a person and that person's body.

\lemc{9 nutriri} we might think of feeding as something more bodily than mental, but there is a tradition stretching back at least as far as Aristotle that argues otherwise. The idea is this: anything which is alive possesses what Aristotle would call a ψυχή and Descartes calls an \textit{anima}: a soul. Even plants, as living creatures, possess souls. Their souls, however, are responsible only for the most basic functions of life: nutrition, growth, and reproduction. (This verb appears in classical Latin both in deponent and non-deponent forms.)

\lemc{10 Sed quid esset\dots} Descartes rarely thought about the nature of the soul before. When he did, he went no further than commonplace analogies of the soul as a kind of wind, flame or air spread throughout his limbs.

\lemc{12--13 ne dubitabam quidem} remember that \textit{ne X quidem} means ``not even X''.

\lemc{13--14 qualem mente concipiebam} depends on \textit{describere} (14). That is, ``to describe what sort of thing I thought (that it was)''.

\lem{14 tentassem} = \textit{tentavissem} (\textbf{AG} §181). This process is often known as ``syncopation''.

\lemc{15 aptum est} the infinitives \textit{terminari} (15), \textit{circumscribi} (15), \textit{replere} (16), \textit{percipi} (17), and \textit{moveri} (17) all depend on this phrase. They are explanatory or limiting infinitives: they explain qualities that befit (\textit{aptum}) a body. This sort of infinitive, often called ``epexegetical'', is common in Greek but rare in classical Latin. Roman poets began to use the construction in imitation of Greek, and it became more common in later Latin.

\lem{17 seipso} = \textit{se ipso}.

\lemc{18 habere vim} this infinitive depends on \textit{pertinere iudicabam} (19). Descartes did not believe (literally ``judge'') that it belonged to the nature of the body ``to have the power''.

\lem{18 seipsum} = \textit{se ipsum}.

\lemc{18 movendi\dots sentiendi\dots cogitandi} these genitives define \textit{vim}. In English we would more likely say ``the power to move, to perceive, to think'', but gerunds are possible for us too.

\lemc{19 nullo pacto} an common ablative of manner meaning ``in no way''. \textit{quo pacto}, sometimes written as one word, similarly means ``in what way?'' or simply ``how?''.

\lem{19 quinimo} = \textit{quin immo}, indicating an emphatic contrast to the preceding thought. There's an important contrast between something that ``belongs to the nature of body'' and something that is ``found in a body''. The first refers to essential qualities of bodies; the second includes non-essential qualities. Descartes didn't think that movement, perception, or thought were essential to bodies, and \textit{what's more} (\textit{quinimo}) he was surprised to find them associated with bodies at all. Throughout this paragraph Descartes admits that he was predisposed to sharply distinguish body and mind.
% -]] What is the ego (cont.)?

% [[- Sum res cogitans
\clearpage

\beginnumbering
\pstart
\begin{latin}
    \textenglish{\textbf{6.}} Quid autem nunc, ubi suppono deceptorem aliquem potentissimum, et, si fas est dicere, malignum, data opera in omnibus, quantum potuit, me delusisse? Possumne affirmare me habere vel minimum quid ex iis omnibus, quae iam dixi ad naturam corporis perti\at{27}nere? Attendo, cogito, revolvo, nihil occurrit; fatigor eadem frustra repetere. Quid vero ex iis quae animae tribuebam? Nutriri vel incedere? Quandoquidem iam corpus non habeo, haec quoque nihil sunt nisi figmenta. Sentire? Nempe etiam hoc non fit sine corpore, et permulta sentire visus sum in somnis quae deinde animadverti me non sensisse. Cogitare? Hic invenio: cogitatio est; haec sola a me divelli nequit. Ego sum, ego existo; certum est. Quandiu autem? Nempe quandiu cogito; nam forte etiam fieri posset, si cessarem ab omni cogitatione, ut illico totus esse desinerem. Nihil nunc admitto nisi quod necessario sit verum; sum igitur praecise tantum res cogitans, id est, mens, sive animus, sive intellectus, sive ratio, voces mihi prius significationis ignotae. Sum autem res vera, et vere existens; sed qualis res? Dixi, cogitans.
\end{latin}
\pend
\endnumbering

\prenotes

\textbf{§6.} All of that is what we used to think, but what should we think now that we are imagining that an evil demon might be deceiving us as much as possible? In that case, we must reject everything that relies on an external world that can be doubted. Hence, we cannot be mind and body, but mind alone. And even when we focus on our minds, we must not allow anything that would require a body. For example, movement or perception. What remains is thought: this alone is certain. Therefore, we can define ourselves as essentially thinking.

\lemc{1 nunc} the previous paragraph began with \textit{antehac}, and the contrast is important. After reviewing what he used to believe, Descartes goes on to consider what he should believe now---after his sceptical investigations in the first meditation.

\lemc{1--2 si fas est dicere} in classical Latin, this was a formula of religious caution. Descartes uses it here in the same way, but in a Christian context.

\lemc{2 data opera} literally an ablative absolute ``with work having been given'', but idiomatically equivalent to ``intentionally, deliberately, on purpose''. (The same phrase also appears as \textit{dedita opera} and \textit{consulta opera}.)

\lemc{5-8 Nutriri\dots incedere\dots Sentire\dots Cogitare} The easiest way to take these infinitives is that they function as nominatives in sentences with no verb. (Remember that the gerund does not appear in the nominative, but infinitives often serve as subjects.) As in English, the grammar may seem tricky, but it's perfectly clear what Descartes means. He asks ``What should I think about the things I used to attribute to my soul?'' and then he fills in what some of these things were: ``Feeding or walking?\dots Perceiving?\dots Thinking?'' He could have used nouns instead of infinitives, and in fact he follows \textit{cogitare} with \textit{cogitatio}.

\lemc{6 iam} logical (``at this point in the argument'') rather than temporal (``now'' or ``already'').

\lemc{8 Hic invenio} supply \textit{aliquid} or a similar object for \textit{invenio}.

\lem{9 Quandiu} = \textit{Quamdiu}. Supply \textit{existo} from the previous sentence.

\lemc{10 posset} the subject of this verb is the substantive clause \textit{ut\dots desinerem} (10). In English, ``It could happen that\dots''.

\lemc{11 totus} here, as often, an adjective in the nominative in Latin is equivalent to an adverb in English: ``as a whole'', ``entirely'', or ``altogether''.

\lemc{12--13 mens\dots ratio} Descartes offers various synonyms for ``mind'' here. In some contexts, these Latin words can have different connotations, but Descartes lists them simply to indicate that when he says that he is a \textit{res cogitans}, he means the meaning shared by all of these words.

\lemc{13 voces mihi prius significationis ignotae} take \textit{mihi} with \textit{ignotae}, and take \textit{significationis ignotae} as a descriptive genitive with \textit{voces}. This aside is important: we use words such as ``mind'' or ``intellect'' all the time, but if Descartes is correct, we have not genuinely understood what they mean. In what follows, Descartes will argue for a radical revision of the concepts mental and physical.
% -]] Sum res cogitans

% [[- Call the mind away from the senses
\clearpage

\beginnumbering
\pstart
\begin{latin}
    \textenglish{\textbf{7.}} Quid praeterea? Imaginabor: non sum compages illa membrorum, quae corpus humanum appellatur; non sum etiam tenuis aliquis aër istis membris infusus, non ventus, non ignis, non vapor, non halitus, non quidquid mihi fingo: supposui enim ista nihil esse. Manet positio: nihilominus tamen ego aliquid sum. Fortassis vero contingit, ut haec ipsa, quae suppono nihil esse, quia mihi sunt ignota, tamen in rei veritate non differant ab eo me quem novi? Nescio, de hac re iam non disputo; de iis tantum quae mihi nota sunt iudicium ferre possum. Novi me existere; quaero quis sim ego ille quem novi. Certissimum est huius sic praecise sumpti notitiam non pendere ab iis quae exi\at{28}stere nondum novi; non igitur ab iis ullis, quae imaginatione effingo. Atque hoc verbum, \textit{effingo}, admonet me erroris mei: nam fingerem revera, si quid me esse imaginarer, quia nihil aliud est imaginari quam rei corporeae figuram, seu imaginem, contemplari. Iam autem certo scio me esse, simulque fieri posse ut omnes istae imagines, et generaliter quaecunque ad corporis naturam referuntur, nihil sint praeter insomnia. Quibus animadversis, non minus ineptire videor, dicendo: imaginabor, ut distinctius agnoscam quisnam sim, quam si dicerem: iam quidem sum experrectus, videoque nonnihil veri, sed quia nondum video satis evidenter, data opera obdormiam, ut hoc ipsum mihi somnia verius evidentiusque repraesentent. Itaque cognosco nihil eorum quae possum imaginationis ope comprehendere, ad hanc quam de me habeo notitiam pertinere, mentemque ab illis diligentissime esse avocandam, ut suam ipsa naturam quam distinctissime percipiat.
\end{latin}
\pend
\endnumbering

\prenotes

\textbf{§7.} What more can we say about what we are? We agree to do away with talk about the human body, and we will also ignore metaphorical notions that the mind is like wind or air inside our limbs. But if we wish to stick to certainties, what can we say about the mind? We should be careful not to use imagination since imagination relies too much on images drawn from vision---and we have already agreed that the senses cannot be relied upon. Instead we should lead the mind away from such things in order to understand the nature of thinking as clearly as possible.

\lemc{2 istis membris infusus} in the active voice, one construction of this verb is \textit{infundere aliquid alicui}. That is, ``to pour something onto, over, or into something''. In the passive, the original accusative becomes the subject, but the dative is retained---as in this phrase \textit{istis membris}.

\lemc{4--6 Fortassis\dots disputo} a very important point. It may turn out to be the case that, from the point of view of physics, our actual minds are like wind or gas or fire. Descartes cannot know one way or the other what science will discover about this. For now he is not concerned about these questions one way or another: since all facts about the physical world are taken to be uncertain, he chooses not to pass judgment on these issues. Instead, he will focus completely on what he can know.

\lemc{8--9 Certissimum est\dots notitiam non pendere} the phrase \textit{notitiam non pendere} functions as the subject of \textit{est}, and \textit{certissimum} is the predicate: ``That the knowledge (of this matter) does not depend (on things I don't know is most certain''. Whenever an infinitive phrase has a subject, the subject will be accusative. (We're most familiar with this in indirect statement, but indirect statement is far from the only case where there is an accusative subject of an infinitive in Latin.) The predicate \textit{certissimum} is neuter singular since Latin treats the entire noun phrase as a singular thing---just as in English we say ``It \textit{is} clear that Descartes writes elegant Latin''.

\lemc{8 huius sic praecise sumpti notitiam} the genitive is objective with \textit{notitiam}, and \textit{sic praecise} modify the participle \textit{sumpti}. It might help to expand \textit{sic praecise sumpti} into a full clause in English: the knowledge of this, ``if it is taken up in this precise manner'' or ``when it is taken up precisely in this way''.

\lemc{9 non igitur ab iis ullis} supply \textit{eam pendere}, which is implied by the previous clause.

\lemc{10--14} Descartes argues here that imagination is not a reliable way to proceed because it essentially involves vision and perception, and he has argued that these can no longer be trusted.

\lemc{11 nihil aliud est imaginari quam} the infinitive \textit{imaginari} is the subject and \textit{nihil aliud} is a predicate complement: ``to imagine is nothing other than\dots''.

\lemc{14--18} The analogy is perhaps unexpected but forceful: according to Descartes using imagination to understand himself is as foolish as going to sleep in order to see something more clearly.

\lem{18 cognosco} introduces two indirect statements: (i) \textit{nihil eorum\dots pertinere} (18--19) and (ii) \textit{mentem\dots esse avocandum} (19--20). The structure is a little difficult to see at first because each indirect statement has other clauses that depend on it.

\lemc{20 suam ipsa} an opportunity for a useful reminder. \textit{suus, sua, suum} is a reflexive adjective, sometimes used for emphasis; \textit{ipse, ipsa, ipsum} is an intensifier but never reflexive. It's easy to forget this becuase in English both are often represented by ``self''. It's also easy to get confused because they often appear together in Latin for emphasis, as in this case. In this case, we must call the mind away from bodily things so that the mind itself (\textit{ipsa}) can perceive its very own (\textit{suam}) nature as clearly as possible.
% -]] Call the mind away from the senses

% [[- What is the nature of thinking?
\clearpage

\beginnumbering
\pstart
\begin{latin}
    \textenglish{\textbf{8.}} Sed quid igitur sum? Res cogitans. Quid est hoc? Nempe dubitans, intelligens, affirmans, negans, volens, nolens, imaginans quoque, et sentiens.
\end{latin}
\pend
\endnumbering

\beginnumbering
\pstart
\begin{latin}
    \textenglish{\textbf{9.}} Non pauca sane haec sunt, si cuncta ad me pertineant. Sed quidni pertinerent? Nonne ego ipse sum qui iam dubito fere de omnibus, qui nonnihil tamen intelligo, qui hoc unum verum esse affirmo, nego caetera, cupio plura nosse, nolo decipi, multa vel invitus imaginor, multa etiam tanquam a sensibus venientia animadverto? Quid est horum, quamvis semper dor\at{29}miam, quamvis etiam is qui me creavit, quantum in se est, me deludat, quod non aeque verum sit ac me esse? Quid est quod a mea cogitatione distinguatur? Quid est quod a me ipso separatum dici possit? Nam quod ego sim qui dubitem, qui intelligam, qui velim, tam manifestum est, ut nihil occurrat per quod evidentius explicetur. Sed vero etiam ego idem sum qui imaginor; nam quamvis forte, ut supposui, nulla prorsus res imaginata vera sit, vis tamen ipsa imaginandi revera existit, et cogitationis meae partem facit. Idem denique ego sum qui sentio, sive qui res corporeas tanquam per sensus animadverto: videlicet iam lucem video, strepitum audio, calorem sentio. Falsa haec sunt, dormio enim. At certe videre videor, audire, calescere. Hoc falsum esse non potest; hoc est proprie quod in me sentire appellatur; atque hoc praecise sic sumptum nihil aliud est quam cogitare.
\end{latin}
\pend
\endnumbering

\prenotes

\textbf{§8.} As thinkers we doubt, understand, affirm, deny, want, refuse, imagine, and perceive.

\textbf{§9.} These several types of thought make us what we are inasmuch as we are essentially thinkers. They are all as true as our existence no matter whether we are asleep---no matter whether an evil all-powerful spirit tricks us. In the case of doubt, understanding, and wanting, this is clear. It might be less clear in the case of imagination and perception, but even if we imagine or perceive something false, it is nevertheless always true that we really \textit{are} imagining or perceiving.

\lemc{2 fere} take this with \textit{omnibus} rather than \textit{dubito}. (As a general rule, adverbs in Latin modify something to their right not their left. There are exceptions, but it's a good rule of thumb.)

\lem{3 caetera} = \textit{cētera}, the \textit{ae} representing the original long ``e'' sound.

\lemc{5--6 quamvis\dots deludat} i.e., ``whether I'm sleeping or not, whether an evil demon is deceiving me or not''.

\lemc{6 aeque verum ac} English would say ``as true as'' or ``just as true as''. Latin often uses \textit{atque} following a comparison. (\textit{atque} and \textit{ac} are two forms of the same word, just as \textit{neque} and \textit{nec}.)

\lemc{6 me esse} ``that I am''. I.e., in this context, ``my existence''. Descartes rhetorically asks ``Which of these types of thought is not just as true as that I exist?''

\lemc{7 distinguatur} subjunctive in a relative clause of characteristic. That is, not simply ``which is distinguished'' but ``which could be distinguished''. The construction of \textit{possit} in the next sentence is the same.

\lemc{8 quod ego sim} the \textit{quod} serves to nominalize (i.e., turn into a noun) this clause. In English, we can translate such a \textit{quod} as ``that'' or ``the fact that''. This clause is the subject of the main verb \textit{est} and \textit{tam manifestum} is the predicate. It's a little difficult to see this initially because the \textit{ego} is described by the three adjectival clauses \textit{qui dubitem, qui intelligam, qui velim}. First, here's a rather literal translation: ``For (the fact) that I who doubt, who understand, who want, exist is so clear that\dots''. A somewhat more natural English version is this: ``It is so clear that I exist---I who doubt, who understand, who want---that\dots''.

\lemc{9--16} This section introduces an idea that modern philosophers call the ``incorrigibility of the mental''. Roughly: we cannot be wrong about our own mental experiences. For example, if I think I'm in pain, then I am in pain. Nobody can tell me otherwise, and I can't be wrong. It is controversial, however, whether mental experiences (or even a subset of them) are incorrigible. To consider pain again, we might describe phantom limb pain, cases where someone experiences pain in a limb that has been amputated, as false pain. However this may be, Descartes offers special defense of the incorrigibility of imagination and perception because he sees that these two types of mental life might seem very fallible. For example, the shirt that looks blue to me might actually be green since I'm colorblind. However, Descartes argues, I can't be wrong about how it looks to me. Similarly, I might imagine something impossible, but it remains true that I'm imagining it as I do. So in a Cartesian spirit, we might say that a person with a phantom limb pain is not wrong \textit{about being in pain}.

\lemc{9 ut supposui} that is, as Descartes has repeatedly agreed: the entire external world of physical objects might be false.

\lemc{9 nulla prorsus} the adverb \textit{prorsus} is complicated. Its meanings include ``forward, straight ahead; without interruption; right through to the end''. But in addition, it is often used for emphasis of a word or phrase, particularly negatives as here. In such cases, you can translate as ``at all, whatsoever''.

\lemc{14 dormio} there is an implicit \textit{ut supposui} here just as in the previous sentence.

\lemc{14 videre videor, audire, calescere} even if the real world is not as it seems to perception, we still have these perceptual experiences.

\lemc{15 sentire} used as a noun to mean ``perception''.

\lemc{16 cogitare} used as a noun to mean ``thought''.
% -]] What is the nature of thinking?

% [[- Physical things seem easier to know than the mind
\clearpage

\beginnumbering
\pstart
\begin{latin}
    \textenglish{\textbf{10.}} Ex quibus equidem aliquanto melius incipio nosse quisnam sim; sed adhuc tamen videtur, nec possum abstinere quin putem, res corporeas, quarum imagines cogitatione formantur, et quas ipsi sensus explorant, multo distinctius agnosci quam istud nescio quid mei, quod sub imaginationem non venit: quanquam profecto sit mirum, res quas animadverto esse dubias, ignotas, a me alienas, distinctius quam quod verum est, quod cognitum, quam denique me ipsum, a me comprehendi. Sed video quid sit: gaudet aberrare mens mea, necdum se patitur intra veritatis limites cohiberi. Esto igitur, et adhuc semel laxissimas habe\at{30}nas ei permittamus, ut, illis paulo post opportune reductis, facilius se regi patiatur.
\end{latin}
\pend
\endnumbering

\prenotes

\textbf{§10.} Although we seem to have the beginnings of an understanding of what we are, it still seems that we understand physical things far more easily than we understand the mind. Because it is not physical, we cannot sense the mind at all, nor can we imagine it very easily. And yet this is paradoxical because we've already agreed (i) that bodily things are doubtful and distinct from our true selves (our minds) and (ii) that our minds are what we are. Hence, we're saying that something doubtful is intuitively easier to understand than our own true nature. In order to investigate further, let's consider more precisely what it means to understand something physical.

\lemc{1--4 sed adhuc\dots venit} the basic elements of this sentence are the main verb \textit{videtur} and its subject, the accusative and infinitive \textit{res corporeas multo distinctius agnosci}. The sentence has several additional clauses that make this basic structure more difficult to see. First, there's a parenthetical coordinate clause after \textit{videtur}. This clause will be described in more detail below. Skip it initially. Second, \textit{res corporeas} is described by two parenthetical relative clauses: \textit{quarum imagines cogitatione formantur} and \textit{quas ipsi sensus explorant}. Finally, because the subject of \textit{videtur} contains a comparison (\textit{multo distinctius}), it is followed by a \textit{quam} clause filling out the ``than'' part of the comparison. (The \textit{quam} clause is also modified by a further relative clause.) When faced with a complex sentence like this, the best strategy is to skim first looking for the largest significant pieces (generally clauses with subjects and verbs). Then scan more closely and find the elements in the main clause; put that together and then add the other clauses bit by bit. (Do I want to give this general advice?)

\lemc{2 quin putem} a common use of \textit{quin} is to indtroduce a clause with a subjunctive after a negated verb of doubt, prevention, refusal or the like. The precise translation of the \textit{quin} clause will vary. In a sentence like this, English idiom prefers something like ``from thinking'' rather than the more literal ``(but) that I think''. (See \textbf{AG} §557-559 for various uses of \textit{quin}. The use in this sentence is covered in §558.

\lemc{3 quod sub imaginationem non venit} We have no way to imagine our minds, according to Descartes, except through uninformative analogies such as ``like a wind spread throughout the body''. This is because, for Descartes, the mental, essentially has no physical nature, and imagination is based on sensory perception of physical objects. See §7.11--12.

\lemc{4--6 Quanquam\dots comprehendi} The word order is difficult. What is \textit{mirum} is the indirect statement \textit{res (which I know to be\dots) distinctius a me comprehendi quam (things I know much better)}.

\lem{4 Quanquam} = \textit{Quamquam}.

\lemc{5--6 quam quod verum est, quod cognitum, quam denique me ipsum} there is a switch in syntax in the final item here. The first two clauses are  of the form ``than (that) which is Y''. Reordered and with an implicit \textit{id} added for clarity: \textit{quam (id) quod est verum} and \textit{quam (id) quod est cognitum}. The last clause, \textit{quam denique me ipsum}, however, has no verb and no relative. The phrase \textit{me ipsum} is accusative because \textit{quam} is normally followed by the same case as the word or phrase being compared to, and here that word is accusative: \textit{res}.

\lemc{8 Esto} 3rd person singular, future, active, imperative of \textit{esse}. The future imperative is rare: often used in legal contexts or poetry. Here it is used of granting something for the sake of argument that the speaker does not necessarily believe: ``So be it'', ``Grant that this is so''. It anticipates the following jussive subjunctive \textit{permittamus}, and in English we would likely just say ``Fine: let us\dots''.

\lem{8 laxissimas habe|nas ei permittamus} the metaphor presents the mind as like an unruly horse. We will let out its reins, to make the horse easier to control later, presumably after its exhausted itself with a good run.

\lem{8 ei} refers back to \textit{mens} (7).

\lemc{9 illis} refers back to \textit{habenas} (8).
% -]] Physical things seem easier to know than the mind

% [[- This wax
\clearpage

\beginnumbering
\pstart
\begin{latin}
    \textenglish{\textbf{11.}} Consideremus res illas quae vulgo putantur omnium distinctissime comprehendi: corpora scilicet, quae tangimus, quae videmus; non quidem corpora in communi, generales enim istae perceptiones aliquanto magis confusae esse solent, sed unum in particulari. Sumamus, exempli causa, hanc ceram: nuperrime ex favis fuit educta; nondum amisit omnem saporem sui mellis; nonnihil retinet odoris florum ex quibus collecta est; eius color, figura, magnitudo manifesta sunt; dura est, frigida est, facile tangitur, ac, si articulo ferias, emittet sonum; omnia denique illi adsunt quae requiri videntur, ut corpus aliquod possit quam distinctissime cognosci. Sed ecce, dum loquor, igni admovetur: saporis reliquiae purgantur, odor expirat, color mutatur, figura tollitur, crescit magnitudo, fit liquida, fit calida, vix tangi potest, nec iam, si pulses, emittet sonum. Remanetne adhuc eadem cera? Remanere fatendum est; nemo negat, nemo aliter putat. Quid erat igitur in ea quod tam distincte comprehendebatur? Certe nihil eorum quae sensibus attingebam; nam quaecunque sub gustum, vel odoratum, vel visum, vel tactum, vel auditum veniebant, mutata iam sunt: remanet cera.
\end{latin}
\pend
\endnumbering

\prenotes

\textbf{§11.} We commonly think that we understand physical things very well and that our knowledge of them comes from the senses. To consider this knowledge, let's look at one specific thing: this piece of wax. It seems that we can say a great deal about it with confidence: we can smell a little of the flowery odor that remains from the honey it came from; its color, shape, and size are clear; it is firm, cool, easily handled, and if someone strikes it with a finger, it makes a distinct sound. All of this information suggests that we are right to think that physical thinks are understood very clearly. If we hold the wax close to a flame, however, everything changes. The smell disappears, the color and shape change, it grows in size, but becomes liquid and warm---difficult to touch. And if someone strikes it now, it doesn't make a sound. Although we all agree that this is still \textit{the same wax}, all of the qualities we thought belonged to that wax are gone. So what was it that we knew so clearly?

\lemc{2--3 non quidem corpora in communi} ordinarily we assume that we know all sorts of things about specific physical objects, but we don't necessarily think we know anything about physical objects in general. Our beliefs are not necessarily theoretical or abstract at all.

\lemc{4 hanc ceram} one of the more famous examples in philosophy. As often, \textit{hanc} indicates a gesture: it is as if the speaker holds up a piece of wax before the readers' eyes, saying ``\textit{this} wax''.

\lemc{sui mellis} i.e., the honey that it came from.

\lemc{6 color, figura, magnitudo} asyndeton, the omission of expected connectives. Like English, Latin usually adds a word for ``and'' before the last item in a series. Asyndeton is often used for emphasis.

\lemc{6 color, figura, magnitudo manifesta sunt} when one predicate adjective describes multiple nouns of different genders, what gender should the adjective be? The rule of thumb is this: (i) if the nouns refer to living things, the adjective will usually be masculine; (ii) if the nouns refer to non-living things, the adjective will usually be neuter. Descartes follows this rule here: \textit{manifesta} is neuter nominative plural.

\lemc{6 facile tangitur} it may be not obvious why Descartes tells us that the wax is easily touched or handled. He is anticipating how hard to touch the wax will become when it grows hot.

\lemc{7 articulo} possibly ``knuckle'', but more likely ``finger''.

\lemc{10 figura tollitur} the shape of the wax is ``destroyed'' or ``lost'', as it melts. The molten wax still has some shape, but not the one it had previously.

\lemc{11 Remanetne adhuc eadem cera?} Once it has lost all of the qualities by which we thought we understood it, is it even the same wax?

\lemc{Quid\dots comprehendebatur?} because it is so commonplace for physical objects to change---even radically change, as with this wax---it can be easy to miss how much bite this question has. Ask yourself seriously: How do I decide that this puddle of warm liquid is \textit{the same thing} as the solid, cool piece of wax that I was holding a moment ago? What is the basis for that sameness? Where is it? Can I perceive it? If not, what does that tell me about my original judgment?

\lemc{14--15 vel\dots vel\dots vel\dots vel} emphatic polysyndeton, the use of more connectives than necessary. Latin, like English, usually only uses a single ``or'' at the end of a series.

\lemc{15 remanet cera} adversative asyndeton, the omission of adversative conjunctions. E.g., ``I see it; I don't believe it'' rather than ``I see it, but I don't believe it''.
% -]] This wax

% [[- Physical items are known by the mind
\clearpage

\beginnumbering
\pstart
\begin{latin}
    \textenglish{\textbf{12.}} Fortassis illud erat quod nunc cogito: nempe ceram ipsam non quidem fuisse istam dulcedinem mellis, nec florum fragrantiam, nec istam albedinem, nec figuram, nec sonum, sed corpus quod mihi apparebat paulo ante modis istis conspicuum, nunc diversis. Quid est autem hoc praecise quod sic imaginor? Attenda\at{31}mus, et, remotis iis quae ad ceram non pertinent, videamus quid supersit: nempe nihil aliud quam extensum quid, flexibile, mutabile. Quid vero est hoc flexibile, mutabile? An quod imaginor, hanc ceram ex figura rotunda in quadratam, vel ex hac in triangularem verti posse? Nullo modo; nam innumerabilium eiusmodi mutationum capacem eam esse comprehendo, nec possum tamen innumerabiles imaginando percurrere; nec igitur comprehensio haec ab imaginandi facultate perficitur. Quid extensum? Nunquid etiam ipsa eius extensio est ignota? Nam in cera liquescente fit maior, maior in ferventi, maiorque rursus, si calor augeatur; nec recte iudicarem quid sit cera, nisi putarem hanc etiam plures secundum extensionem varietates admittere, quam fuerim unquam imaginando complexus. Superest igitur ut concedam, me nequidem imaginari quid sit haec cera, sed sola mente percipere; dico hanc in particulari, de cera enim in communi clarius est. Quaenam vero est haec cera, quae non nisi mente percipitur? Nempe eadem quam video, quam tango, quam imaginor, eadem denique quam ab initio esse arbitrabar. Atqui, quod notandum est, eius perceptio non visio, non tactio, non imaginatio est, nec unquam fuit, quamvis prius ita videretur, sed solius mentis inspectio, quae vel imperfecta esse potest et confusa, ut prius erat, vel clara et distincta, ut nunc est, prout minus vel magis ad illa ex quibus constat attendo.
\end{latin}
\pend
\endnumbering

\prenotes

\textbf{§12.} To return to our question, what was the clear perception that we had of the wax? If it was none of the sensible qualities (which we've already agreed can change while the wax remains the same), then what's left? Perhaps only that we perceived the wax as a body: a variable extended thing. And how, in turn, do we grasp such a variable extended thing? Not with imagination, but with our minds. It's true that the physical item we grasp with our minds is the same thing that we perceive with our senses. But---and this is the key point---any true understanding we possess comes from our minds rather than our senses.

\lemc{1 illud erat quod nunc cogito} Descartes returns to the final question in §11 (\textit{Quid erat igitur in ea quod tam distincte comprehendebatur?}), which he never answered. The demonstrative \textit{illud} points forward to the possible answer, namely the indirect statement that follows \textit{cogito}.

\lemc{2 istam} dismissive as often.

\lemc{3 corpus} all along the wax was not the varying appearances but the underlying physical thing, which is capable of taking on these different appearances.

\lemc{3 Quid est autem hoc praecise quod sic imaginor?} The answer \textit{corpus} needs further explanation, as Descartes is well aware.

\lemc{3--4 Attendamus\dots videamus} subjunctive expressing commands.

\lemc{6 extensum quid, flexibile, mutabile} Descartes defines \textit{corpus} as ``something extended, flexible, changeable''.

\lemc{7 An quod imaginor} in the previous section, Descartes argued that bodies are not known by perception (despite appearances), and now he argues that they are also not known by imagination. That will leave thought as the only possible choice.

\lem{8--13} Descartes argues as follows. (i) If imagination were responsible for our knowledge of bodies, (ii) it would have to work through an infinite sequence of possible qualities for each body. (iii) But it is impossible (for people) to finish an infinite sequence. (iv) Therefore, imagination is not responsible for our knowledge of bodies. The form of this argument is valid: that is, if the premises are true, then the conclusion follows and is true. Logicians call arguments of this form \textit{modus tollens (tollendo)}---the final \textit{tollendo} is often left out for brevity. The name means ``the method of removing (by removing)''. Put schematically, arguments of this type run as follows: (i) if p, (ii) then q. (iii) But not q. (iv) Therefore, not p. (See \cite[§23]{weston2009} for examples and discussion.)

\lemc{8--9 innumerabilium eiusmodi mutationum} take \textit{innumerabilium mutationum} as an objective genitive (with \textit{capacem}) and \textit{eiusmodi} as a descriptive genitive qualifying \textit{mutationum}.

\lemc{11--12} the dimensions of the wax grow larger as it grows hotter and melts, spreading out over more area.

\lemc{14 fuerim\dots complexus} perfect subunctive expressing potential. Descartes argues that there are more possible sizes for the wax to take on than he ever could have imagined.

\lemc{15 sola mente} that is, through thought rather than perception or imagination.

\lemc{16 hanc} supply \textit{ceram}.

\lemc{16 cera\dots in communi} it's more obvious that we can't learn about ``wax in general'' without throught since any general idea requires a great deal of abstraction.

\lem{16--22} The wax he thinks about and the wax he perceives are \textit{the same wax}, but Descartes insists nevertheless that we only truly grasp it via thought, not perception.

\lemc{22 constat} the implicit subject is \textit{cera}.
% -]] Physical items are known by the mind

% [[- Objections based on ordinary language
\clearpage

\beginnumbering
\pstart
\begin{latin}
    \textenglish{\textbf{13.}} Miror vero interim quam prona sit mea mens in errores; nam quamvis haec apud me tacitus et sine\at{32} voce considerem, haereo tamen in verbis ipsis, et fere decipior ab ipso usu loquendi. Dicimus enim nos videre ceram ipsammet, si adsit, non ex colore vel figura eam adesse iudicare. Unde concluderem statim: ceram ergo visione oculi, non solius mentis inspectione, cognosci; nisi iam forte respexissem ex fenestra homines in platea transeuntes, quos etiam ipsos non minus usitate quam ceram dico me videre. Quid autem video praeter pileos et vestes, sub quibus latere possent automata? Sed iudico homines esse. Atque ita id quod putabam me videre oculis, sola iudicandi facultate, quae in mente mea est, comprehendo.
\end{latin}
\pend
\endnumbering

\beginnumbering
\pstart
\begin{latin}
    \textenglish{\textbf{14.}} Sed pudeat supra vulgus sapere cupientem, ex formis loquendi quas vulgus invenit dubitationem quaesivisse; pergamusque deinceps, attendendo utrum ego perfectius evidentiusque percipiebam quid esset cera, cum primum aspexi, credidique me illam ipso sensu externo, vel saltem sensu communi, ut vocant, id est potentia imaginatrice, cognoscere? an vero potius nunc, postquam diligentius investigavi tum quid ea sit, tum quomodo cognoscatur? Certe hac de re dubitare esset ineptum; nam quid fuit in prima perceptione distinctum? Quid quod non a quovis animali haberi posse videretur? At vero cum ceram ab externis formis distinguo, et tanquam vestibus detractis nudam considero, sic illam revera, quamvis adhuc error in iudicio meo esse possit, non possum tamen sine humana mente percipere.
\end{latin}
\pend
\endnumbering

\prenotes

\textbf{§13.} We've established that our grasp of even physical things is primarily intellectual, but it's easy to slip back into errors. The way we speak can easily lead us astray. In normal contexts, we say ``I see the wax itself'' and never ``I judge that the wax is present from its color and shape''. However, ordinary forms of speech can be deceiving.

\lemc{2 apud me} i.e., ``in my own head''.

\lemc{3 ipso usu loquendi} the mere use of language is potentially enough to confuse us.

\lem{3 ipsammet} = \textit{ipsam} and -\textit{met}. The intensifying suffix -\textit{met} goes primarily on the personal pronouns in classical Latin, but there are examples with \textit{ipse, ipsa, ipsum} in antiquity. This use became far more common in later Latin. (Fun fact: French \textit{même} and Spanish \textit{mismo} both derive from what linguists call ``rebracketing''. Phrases like \textit{egomet ipse} were interpreted by speakers in such a way that the suffix -\textit{met} became a prefix \textit{met}- and forms such as \textit{*metipsimus} were formed. (The asterisk indicates that this is a hypothesized form.) Over time, this became \textit{meïsme}, then \textit{mesme}, and eventually \textit{même}. There was a similar process for \textit{mismo} in Spanish.)

\lemc{4--5 concluderem\dots respexissem} a mixed contrary-to-fact conditional sentence. The protasis (\textit{respexissem}) is past contrary-to-fact, and the apodosis (\textit{concluderem}) is present contrary-to-fact. This type of mix is fairly common.

\lemc{6 quos\dots ipsos} take \textit{quos ipsos} as the direct object of the indirect statement \textit{dico me videre} (7), and then deal with the rest of the clause.

\lemc{7--8 sub quibus latere possent automata} Descartes doesn't need this to be true---or even likely. His point is that ordinary language cannot be taken entirely literally. Hence, we must be careful if we base theories on ``what we say''.

\textbf{§14.} If our goal is more than ordinary wisdom, we shouldn't start with ordinary speech. Instead, let's consider whether we understood the wax better through perception and imagination or through thought. Whatever grasp we had through perception is no better than what any other animal can grasp. Only the human mind can gain a deeper knowledge of things, beyond the superficial information of the senses.

\lemc{1 pudeat} the subjunctive indicates that it ``should be shameful'' (a matter of obligation or propriety) rather than that it ``is shameful'' (a matter of fact).

\lemc{1--2} the impersonal verb \textit{pudet} takes an accusative of the person who feels the shame. Here the accusative is \textit{cupientem}, and the participle is used in a generic sense: ``(a person) desiring'' means ``someone who desires''. \textit{supra vulgus sapere} depend on \textit{cupientem}; hence ``someone who wants to be wise beyond ordinary people''. \textit{quaesivisse dubitationem ex formis loquendi quas vulgus invenit} is what the person should feel shame about.

\lemc{4 ut vocant} the vague ``they'' refers to scholastic philosophers, who used the term \textit{sensus communis} to represent Aristotle's κοινὴ αἴσθησις.

\lemc{6 tum\dots tum} i.e. ``both\dots and'' or ``not only\dots but also''.

\lemc{9 tanquam vestibus detractis} the clothing of an object is its sensory perception; seeing it nude is grasping it through understanding. Descartes relies on the implicit notion that external coverings can hide the true nature of a thing.
% -]] Objections based on ordinary language

% [[- The mind is known without the senses too
\clearpage

\beginnumbering
\pstart
\begin{latin}
    \textenglish{\textbf{15.}} \at{33}Quid autem dicam de hac ipsa mente, sive de me ipso? Nihildum enim aliud admitto in me esse praeter mentem. Quid, inquam, ego qui hanc ceram videor tam distincte percipere? Nunquid me ipsum non tantum multo verius, multo certius, sed etiam multo distinctius evidentiusque, cognosco? Nam, si iudico ceram existere, ex eo quod hanc videam, certe multo evidentius efficitur me ipsum etiam existere, ex eo ipso quod hanc videam. Fieri enim potest ut hoc quod video non vere sit cera; fieri potest ut ne quidem oculos habeam, quibus quidquam videatur; sed fieri plane non potest cum videam sive (quod iam non distinguo) cum cogitem me videre, ut ego ipse cogitans non aliquid sim. Simili ratione, si iudico ceram esse, ex eo quod hanc tangam, idem rursus efficietur, videlicet me esse. Si ex eo quod imaginer, vel quavis alia ex causa, idem plane. Sed et hoc ipsum quod de cera animadverto, ad reliqua omnia, quae sunt extra me posita, licet applicare.
\end{latin}
\pend
\endnumbering

\prenotes

\textbf{§15.} We've shown that physical objects (i) can be understood by the mind and (ii) that they are better understood by the mind than the senses or imagination, but what about the mind's awareness of itself? Does this require the senses? No. In fact, not only do we not need the senses in order to grasp the mind, but every act of perception serves indirectly (i) to prove that we exist and (ii) to clarify our self-understanding.

\lemc{1 de hac ipsa mente} now that he has settled the questions about his knowledge of physical things, Descartes returns to the mind, which is his primary interest.

\lemc{1 sive de me ipso} as Descartes reminds us in the next sentence, the only thing he knows for sure about himself at this point is that he is a thinking thing. Hence, he considers talk about his mind the same as talk about himself.

\lemc{2 Quid\dots ego} supply \textit{est}. That is, ``What is the I''?

\lemc{3--4 verius\dots certiius\dots distinctius evidentiusque} we must supply ``than (I understand) physical things''.

\lemc{5 ex eo quod} literally, ``from this fact that''. More colloquially in English ``because''. (Note that you can almost always replace ``because of the fact that'' with simply ``because'' in English.)

\lemc{5 efficitur} normally the subject would be a substantive clause of result (\textit{ut} or \textit{ne} with a subjunctive), but here Descartes uses an accusative and infinitive (\textit{me ipsum etiam existere}) as the subject. Compare the next sentence where \textit{Fieri potest} is followed by an \textit{ut} clause.

\lemc{6 Fieri\dots potest} literally ``it can happen'', but idiomatically ``perhaps it is the case'' or ``perhaps it happens''.

\lemc{7 quibus} ablative of means referring to \textit{oculos}.

\lemc{7 videatur} although the passive of \textit{video} frequently means ``seem'' or ``appear'' in Descartes, it has the more literal meaning here of ``be seen''. The verb is subjunctive in a relative clause of purpose: perhaps I don't even have eyes by which anything might be seen.

\lem{8 sive} joins \textit{cum videam} and \textit{cum cogitem me videre}, ``when I see or when I think that I see''.

\lemc{8 quod non distinguo} this phrase interrupts the parallel \textit{cum} clauses joined by \textit{sive} so that Descartes can make clear that for the purpose of this argument, he doesn't distinguish between ``seeing'' and ``thinking that he sees''. \textit{quod} is neuter singular referring to the general idea represented by the second \textit{cum} clause. In English, you can translate this \textit{quod} as ``what'' or ``something that''.

\lemc{9--11} We can substitute touch, any of the senses, or imagination for sight, and we will get the same result. If Descartes has some mental grasp  of the wax, he must exist.

\lemc{11 idem plane} supply \textit{efficietur}.

\lemc{11 et} adverbial ``even'' or ``also'', not the conjunction ``and''.

\lemc{11 hoc ipsum} the verb \textit{licet} is impersonal, so \textit{hoc ipsum} is not its subject. Take \textit{hoc ipsum} as the direct object of \textit{applicare}.

\lemc{13--17} It may help to know the thrust of this sentence before translating. Descartes reminds us that (i) the understanding of the wax became clearer when it was grasped not only by the senses but also by the mind. And on the basis of this he argues that (ii) this shows how much better he knows his mind than physical things because any understanding that we gain of physical things (like (i) in this case) also improves his understanding of his mind. The connection between (i) and (ii) is not stated, but the implicit idea seems to be that anytime we gain understanding of something external to us, we indirectly gain self-understanding as well because in all cases it is the mind that does the true understanding. This argument seems like a sleight of hand to me. Do you think that there is a more charitable reading of it available?
% -]] The mind is known without the senses too

% [[- Mind known w/o senses (cont.) and conclusion of meditation
\clearpage

\beginnumbering
\setline{13}
\pstart
\begin{latin}
    \textenglish{\textbf{15. (cont.)}} Porro autem, si magis distincta visa sit cerae perceptio, postquam mihi, non ex solo visu vel tactu, sed pluribus ex causis innotuit, quanto distinctius me ipsum a me nunc cognosci fatendum est, quandoquidem nullae rationes vel ad cerae, vel ad cuiuspiam alterius corporis perceptionem possint iuvare, quin eaedem omnes mentis meae naturam melius probent! Sed et alia insuper tam multa sunt in ipsa mente, ex quibus eius notitia distinctior reddi potest, ut ea, quae ex corpore ad illam emanant, vix numeranda videantur.
\end{latin}
\pend
\endnumbering

\beginnumbering
\pstart
\begin{latin}
    \textenglish{\textbf{16.}} Atque ecce tandem sponte sum reversus eo quo\at{34} volebam; nam cum mihi nunc notum sit ipsamet corpora, non proprie a sensibus, vel ab imaginandi facultate, sed a solo intellectu percipi, nec ex eo percipi quod tangantur aut videantur, sed tantum ex eo quod intelligantur aperte cognosco nihil facilius aut evidentius mea mente posse a me percipi. Sed quia tam cito deponi veteris opinionis consuetudo non potest, placet hic consistere ut altius haec nova cognitio memoriae meae diuturnitate meditationis infigatur.
\end{latin}
\pend
\endnumbering

\prenotes

\lemc{13 perceptio} not used to mean ``purely sensory perception'', but ``understanding'' more generally. That is, it includes intellectual perception as well as physical perception. (The same goes for \textit{perceptionem} on line 17 below.)

\lemc{14 innotuit} < \textit{innotesco}. Be careful: this verb has active forms, but a passive-sounding meaning in English.

\lemc{15--16 vel\dots vel} ``either\dots or'', balancing the two \textit{ad} phrases.

\lemc{15--16 ad cerae} supply \textit{perceptionem} as the object of \textit{ad} from the second \textit{ad} phrase on the next line.

\lemc{16 quin} roughly equivalent \textit{quae non} introducing a negative clause of result following \textit{nullae rationes} (15). However, given the phrase \textit{eaedem omnes}, it will sound better to translate \textit{quin} as ``without'' in this sentence. See \textbf{AG} §559 for this use of \textit{quin}.

\lemc{17--19} Immediately after arguing that sensory perception also improves the understanding of his mind as well, Descartes dismisses sensory perception as ``barely worth considering'' (\textit{vix numeranda}) in comparison with the intellect's own resources for self-knowledge.

\lemc{18 quibus} the antecedent is \textit{multa}.

\lemc{18 eius} this pronoun refers to \textit{mente}.

\lemc{18 notitia distinctior reddi potest} take \textit{notitia} as the subject of \textit{potest} and \textit{distinctior} as a predicate adjective following \textit{reddi}.

\textbf{§16.} We've proven what we set out to prove: the mind is more easily and more clearly known than anything else. We know even physical things with the mind rather than with the senses, as we have shown. However, since it is hard to give up longstanding habits of thought, let's pause here so that we can let this new insight sink in more deeply.

\lemc{1 sponte} ``spontaneously'' in the sense that Descartes has not steered the argument towards its conclusion; he simply followed where the argument went. Very much the same meaning as one use of English ``naturally'': without artifice or force.

\lemc{eo quo} both words are adverbs. As often, Latin expresses both halves of a correlative pair (literally ``(to) there where'') while English would only express one half: ``where''.

\lemc{2--5 Nam\dots percipi} this long sentence is difficult primarily because each part of it has several subdivisions. At the highest level, it is a causal \textit{cum} clause (\textit{cum mihi nunc notum sit}) and a main clause (\textit{aperte cognosco}). Inside the \textit{cum} clause, what is \textit{nunc notum} is the complex indirect statement \textit{ipsamet corpora non\dots percipi nec\dots percipi}. The indirect statement itself has two main pieces: ``bodies themselves\dots are not perceived and\dots are not perceived''. In the first half, bodies themselves are not perceived \textit{a sensibus vel ab imaginandi facultate sed a solo intellectu}. In the second half, bodies themselves are not perceived \textit{ex eo quod tangantur aut videantur sed tantum ex eo quod intelligantur}. Finally, in the main clause, Descartes says, ``I understand clearly'' that \textit{nihil posse percipi facilius aut evidentius a me} than \textit{mea mente}.

\lemc{2 notum sit} neuter singular since the subject is the long indirect statement that follows. An indirect statement, when modified by an adjective, is treated like a neuter singular.

\lemc{2--3 a\dots ab\dots a} the ablative of instrument nearly never takes a preposition, but here it does.

\lemc{5 mea mente} ablative of comparison with \textit{facilius aut evidentius}.
% -]]  Mind known w/o senses (cont.) and conclusion of meditation

% -]] Meditatio secunda
