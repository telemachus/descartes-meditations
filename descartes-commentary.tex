% [[- Chapter title
\chapter*{Commentary}
% -]] Chapter title

% [[- Meditatio I
\section*{Meditatio I}

% [[- Arrival of spring
\subsection*{Arrival of spring I 1--I 8}

Horace begins by describing the arrival of spring in two stanzas.  In the first he shows the changes in weather and the effect of those changes on the human world.  In the second stanza he depicts spring's effect at the divine level.

\comment{1}{4}

The first stanza is a single, carefully-constructed sentence.  Horace uses an elaborate ascending tricolon to organize the opening: \textbf{(1) Solvitur \dots (2) trahuntque \dots (3) ac (a) neque \dots gaudet \dots (b) nec \dots albicant}.  In addition, the stanza is shaped around a chiasmus: the first and fourth lines focus on the weather and the natural world; the second and third depict human actions and reactions to the change in weather. 

\comment{1}{1}

\lem{Solvitur}: When the weather grows warmer, the snow and ice melts;\footnote{\textit{C} I 9.5 \textit{dis\textbf{solve} frigus}} this in turn frees the earth.\footnote{This poem line 10: \textit{terrae \dots solutae}}  Horace fuses and shifts these two ideas by saying that the winter is melted.\indent\lem{acris}: Ennius and Lucretius also apply \textit{acer, acris, acre - sharp, piercing, penetrating} to \textit{hiems}. See also Horace's own \textit{gelu \dots acuto} (\textit{C} I 9.3--4).\indent\lem{veris et Favoni}: Nearly a hendiadys: \textit{spring which the west wind announces}.  (The genitives are perhaps subjective with the verbal idea in \textit{vice}.)\indent\lem{Favoni}: Greeks and Romans often say that spring begins in early February when the west wind begins to blow again.\footnote{Pliny the Elder \textit{Naturalis Historia} (hereafter \textit{NH}) 2.122 and 18.337}  Roman poets often use the Greek name for the west wind, \textit{Zephyrus}, rather than the native Latin \textit{Favonius}.  This is the first of several distinctly Roman touches to the poem.

\comment{2}{2}

\lem{machinae}: Horace has in mind winches that would help roll drydocked ships back into the sea when the sailing season began.

\comment{3}{5}

\lem{iam}: Horace's repetition of \textit{iam} here may echo a spring poem of Catullus that includes notable repetition of the word.  \textit{iam} appears on lines 1, 2, 7, and 8 of Catullus 46.

\comment{3}{3}

\lem{aut arator igni}: Supply \textit{gaudet} from the previous clause.  \textit{igni} is ablative, not dative.  The verb \textit{gaudere} regularly takes an ablative of what leads the subject to rejoice.

\comment{4}{4}

\lem{albicant}: The verb \textit{albicāre} is a poetic and rare alternative to the more prosaic \textit{albēre}.

\comment{5}{5}

\lem{Cytherea}: Cythera, an island in the Aegean sea, was one of the places Venus was supposed to have been born.\indent\lem{Venus} is particularly associated with the arrival and power of spring.\footnote{Lucretius \textit{De rerum natura} (hereafter \textit{DRN}) I 10--16 and V 737--740}

\comment{6}{6}

\lem{decentes}: The participle of \textit{decēre} does not appear as an adjective before Horace, and he may have been the first to use it this way.  (Given how much poetry is lost, however, it is impossible to be certain.)  The adjective becomes common in later poets and prose writers as a high-style and elegant word, combining the ideas of beauty and appropriateness.

\comment{7}{8}

\lem{dum gravis \dots officinas}: Horace switches from images of beauty and dancing to Vulcan working underground in the ``shops of the Cyclopes".  The transition is associative: first, Vulcan is the husband of Venus; second, lightning storms were common in the spring.  Thus, Vulcan and the Cyclopes had a lot of work to do since they supplied Jupiter with his supply of lightning.\footnote{Lucretius \textit{DRN} VI 357--358 connects spring and lightning.}

\comment{7}{7}

\lem{Cyclopum}: In Homer's \textit{Odyssey} the Cyclopes are shepherds, but as early as Hesiod's \textit{Theogony} (circa 7th century BCE), they are also represented as blacksmiths who forge lightning for Zeus.\footnote{\textit{Theogony} 139--141}

\comment{8}{8}

\lem{visit} is from \textit{viso, visere} not \textit{video, videre}.\indent\lem{officinas} is a prosaic word, and this makes the transition from the dancing Nymphs and Graces to the working Cyclopes that much more emphatic.
% -]] Arrival of spring

% -]] Meditatio I
