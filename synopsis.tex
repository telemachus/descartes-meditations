% [[- Chapter title
\chapter{Synopsis}
\markboth{Meditationes de prima philosophia}{Synopsis}
% -]] Chapter title

% [[- Synopsis

% [[- Introduction
Speaking in his own voice as author, Descartes uses the brief synopsis to tell readers what to expect, but also what not to expect, from the six meditations. Each meditation receives one paragraph. In addition to summary of the material in the meditations, the synopsis also informs us about some of Descartes implicit goals and concerns. For example, Descartes says more than once in the synopsis that he intends to ``lead readers away from the senses.'' This is not something that the inquirer ever mentions explicitly in the meditations themselves. If we keep this goal in mind, however, it can help us to understand several of the inquirer's explicit arguments better. The synopsis offers several insights such as this, and these insights make it well worth reading.
\clearpage
% -]] Introduction

% [[- 1st meditation
\begin{center}
    \beginnumbering
    \numberlinefalse
    \pstart
    \textit{Synopsis sex sequentium Meditationum}\ledsidenote{12}
    \pend
    \endnumbering
\end{center}

\beginnumbering
\pstart
\textbf{1.} \begin{latin}In prima, causae exponuntur propter quas de rebus omnibus, praesertim materialibus, possumus dubitare; quandiu scilicet non habemus alia scientiarum fundamenta, quam ea quae antehac habuimus. Etsi autem istius tantae dubitationis utilitas prima fronte non appareat, est tamen in eo maxima quod ab omnibus praeiudiciis nos liberet, viamque facillimam sternat ad mentem a sensibus abducendam; ac denique efficiat, ut de iis, quae postea vera esse comperiemus, non amplius dubitare possimus.\end{latin}
\pend
\endnumbering

\prenotes

\textbf{§1.} The first meditation sets forth reasons to doubt everything---at least until we have firmer foundations for science than we have now. This doubt will provide several benefits: (i) it will free us from preconceived opinions, (ii) the doubt will lead us away from overreliance on the senses, and (iii) whatever truths we establish after this initial doubt will be secure against future doubt.

\lemc{1 In prima} supply \textit{meditatione}.

\lemc{1--2 praesertim materialibus} supply \textit{rebus}. Although Descartes plans to doubt ``everything,'' his particular focus will be beliefs based on the evidence of the senses. Thus, we are able to doubt ``material things especially.''

\lemc{2 quandiu scilicet non habemus} if we take the punctuation before this clause seriously, we can supply something like \textit{de his rebus dubitare possumus} before \textit{quandiu}. However, a simpler approach is to translate as if the semicolon is equivalent to a modern comma or emdash. That is, ``Reasons are set out why we can doubt all things---at least as long as we don't have better foundations for science.''

\lem{2 quandiu} = \textit{quamdiu}.

\lemc{2--3 scientiarum fundamenta} Descartes uses the Aristotelian phrase ``first philosophy'' in the title of this work. That already indicates his interest in the most basic and essential truths. At this point, readers may not know precisely what he means by the ``foundations of the sciences,'' but Descartes directs readers' attention to his focus very explicitly.

\lemc{4 prima fronte} an idiomatic expression meaning ``at first glance'' or ``on first appearance.''

\lemc{4 in eo\dots quod} the \textit{eo} anticipates the following \textit{quod} clause. The greatest usefulness of the doubt lies ``in this, namely that.'' The \textit{quod} serves only to nominalize the clause it introduces. You can generally translate as ``that'' or ``the fact that.'' See \textbf{AG} §572 and \textbf{NLS} §241 for more examples and discussion. Note, however, that in later Latin the verb in such \textit{quod} clauses can be subjunctive without that necessarily implying any unreality or conditionality. Sometimes, as here in Descartes, the subjunctive merely indicates that the clause is subordinate.

\lemc{5 praeiudiciis} Descartes considers preconceived opinions to be a significant source of error in our lives (\textbf{CSM} I 218; \textbf{AT} VIII 37). Etienne Gilson, commenting on a passage from the \textit{Discourse on the method} (\textbf{CSM} I 120; \textbf{AT} VI 18), describes preconceived opinion (\textit{la prévention} in French) as ``the persistence in our thought of unreflective judgments that we made in our childhood and that guide us now as if we had proven them'' (\textcite[199]{gilson2005}; for a fuller discussion see \textcite[199]{gilson1987}).

\lemc{5--6 ad mentem a sensibus abducendam} a gerundive with \textit{ad} expressing purpose. Take the phrase with \textit{viam} rather than \textit{sternat}: the method of doubt will create ``the easiest road to lead our minds away from the senses.'' Descartes reveals the anti-empirical tendency of his method far more bluntly here than he does in the text of the first meditation.

In several Platonic dialogues (e.g., in the \textit{Phaedo}, \textit{Symposium}, and \textit{Republic}) we find arguments that sensation cannot yield knowledge and that we gain knowledge only through abstract and pure reason. Although scholars do not agree about how well Descartes knew the actual texts of Plato,\footnote{For example, see \textcite[556, especially note 11]{grene1999}.} we don't need to see a specific allusion here. Plato's criticism of the senses and praise of reason became a philosophical commonplace. In addition, the Neoplatonic influence on Christianity meant that the related contrast between body and soul became a spiritual commonplace as well. All the more fitting, then, that Descartes's philosophical meditations try to lead the readers to shift attention away from body and the senses and towards the soul and reason.

\lemc{6--7 ac denique efficiat\dots possimus} the doubts that we find at the beginning of the \textit{Meditations} will somehow free us from future doubts. Descartes does not make clear how this is true, but it is an essential feature of his method: he employs skepticism in order to free us from skeptical doubts.

% -]] 1st meditation

% [[- 2nd meditation
\clearpage

\beginnumbering
\pstart
\textbf{2.} \begin{latin}In secunda, mens quae, propria libertate utens, supponit ea omnia non existere de quorum existentia vel minimum potest dubitare, animadvertit fieri non posse quin ipsa interim existat. Quod etiam summae est utilitatis, quoniam hoc pacto facile distinguit quaenam ad se, hoc est, ad naturam intellectualem, et quaenam ad corpus pertineant. Sed quia forte nonnulli rationes de animae immortalitate illo in loco expectabunt, eos hic monendos \at{13} puto me conatum esse nihil scribere quod non accurate demonstrarem; ideoque non alium ordinem sequi potuisse, quam illum qui est apud Geometras usitatus, ut nempe omnia praemitterem ex quibus quaesita propositio dependet, antequam de ipsa quidquam concluderem. Primum autem et praecipuum quod praerequiritur ad cognoscendam animae immortalitatem, esse ut quam maxime perspicuum de ea conceptum, et ab omni conceptu corporis plane distinctum, formemus; quod ibi factum est.\end{latin}
\pend
\endnumbering

\prenotes

\textbf{§2.} The second meditation establishes a certainty: the mind that doubts must exist. This single certainty allows us to investigate more clearly the difference between mind and body. However, Descartes warns readers here that he will not argue for the immortality of the soul in the second meditation. (Note also that Descartes appears to assume the identity of ``mind'' (Latin \textit{mens}) and ``soul'' (Latin \textit{anima)}). Throughout this paragraph he moves back and forth between the two terms without appearing to distinguish between them at all.)  Descartes follows a method that forbids arguing for a proposition, such as the immortality of the soul, until establishing all the premises required to support the proposition. Descartes will not be in a position to do so until the sixth meditation. Nevertheless, the second meditation is an important step towards a proof of the soul's immortality since we gain a clear conception of the soul and its distinction from the body in this meditation.

\lemc{1--2 mens\dots animadvertit} The multiple subordinate clauses can make the structure of this sentence difficult to see initially. The subject is \textit{mens} and the main verb \textit{animadvertit}: ``the mind (which supposes\dots) notices.''

\lemc{2 vel} an adverb meaning ``even'' rather than the conjunction meaning ``or.''

\lemc{2--3 animadvertit fieri non posse} there is no subject accusative for this indirect statement. Supply ``this'' or ``that'' referring back to the previous clauses. The ``mind notices that this (i.e., for it to doubt everything that is not absolutely certain) cannot happen.''

\lem{3 quin} = ``without'' or ``unless.''

\lemc{3 interim} is elliptical. The mind must exist ``in the meantime,'' that is, ``during the time that it is doubting.''

\lem{3 Quod} is a connective relative. Translate as if it were ``Et hoc'' or ``Et illud.'' The antecedent is \textit{that the mind must exist while it doubts} or \textit{the certainty concerning the existence of the mind}.

\lem{3 summae\dots utilitatis} is a genitive of description in the predicate part of the sentence: ``And this fact is of the greatest utility.''

\lemc{4 distinguit} the subject of the verb is still \textit{mens} from the previous sentence.

\lem{4 hoc est} is equivalent in meaning to \textit{id est}. Translate ``that is'' or ``namely.''

\lem{5 forte} means ``perhaps'' rather than ``by chance'' here.

\lem{5 rationes} means ``arguments'' here.

\lemc{6--7 eos hic monendos puto me conatum esse nihil scribere} the phrase \textit{eos hic monendos} depends on \textit{puto}, \textit{me conatum esse} depends on \textit{eos hic monendos}, and finally \textit{nihil scribere} depends on \textit{me conatum esse}. Since readers might expect a discussion of the soul's immortality, Descartes explains ``I think that they should be warned here that I have tried to write nothing (except what is certain).''

Everything in the rest of this rather long paragraph implicitly depends on \textit{eos hic monendos}. That is why, despite the punctuation indicating several new sentences, the remaining main verbs in this paragraph are in the infinitive with accusative subjects.

\lemc{7 quod\dots demonstrarem}  a relative clause of characteristic. Descartes attempted to write nothing ``of the kind which'' he could not prove completely.

\lemc{7--9 ideoque\dots concluderem} Descartes follows the method of ``geometers'' insofar as he doesn't draw any conclusion until after he has fully explained the premises the conclusion depends on. This method requires Descartes to lay a good deal of groundwork before he can prove that the soul is immortal.

\lemc{ut\dots praemitterem} this substantive clause is in apposition to, and explains, \textit{non aliam ordinem} from line 7.

\lemc{10--12} In order to understand the immortality of the soul, we must first have an absolutely clear understanding of the soul itself. The second meditation focuses on this prerequisite.

\lemc{10--12 Primum\dots praecipuum\dots esse ut\dots formemus} the main verb is infinitive in indirect statement following \textit{eos hic monendos} above; the accusative subject of \textit{esse} here is the substantive \textit{ut} clause; and \textit{primum} and \textit{praecipuum} are predicate accusative, describing the \textit{ut} clause. In English, ``(People should be warned that) it is first and foremost that we develop (a clear conception of the soul).'' More idiomatically, ``First and foremost, we should develop (a clear conception of the soul).''

\clearpage

\beginnumbering
\pstart
\setline{13}
\begin{latin}
    \textenglish{\textbf{2. (cont.)}} Praeterea vero requiri etiam ut sciamus ea omnia quae clare et distincte intelligimus, eo ipso modo quo illa intelligimus, esse vera: quod ante quartam Meditationem probari non potuit; et habendum esse distinctum naturae corporeae conceptum, qui partim in ipsa secunda, partim etiam in quinta et sexta formatur; atque ex his debere concludi ea omnia quae clare et distincte concipiuntur ut substantiae diversae, sicuti concipiuntur mens et corpus, esse revera substantias realiter a se mutuo distinctas; hocque in sexta concludi. Idemque etiam in ipsa confirmari ex eo quod nullum corpus nisi divisibile intelligamus, contra autem nullam mentem nisi indivisibilem: neque enim possumus ullius mentis mediam partem concipere, ut possumus cuiuslibet \edtext{quantamvis}{\Afootnote{\nth{1} ed. quantumvis}} exigui corporis; adeo ut eorum naturae non modo diversae, sed etiam quodammodo contrariae agnoscantur. Non autem ulterius ea de re in hoc scripto me egisse; tum quia haec sufficiunt ad ostendendum ex corporis corruptione mentis interitum non sequi, atque sic ad alterius vitae spem mortalibus faciendam; tum etiam quia praemissae, ex quibus ipsa mentis immortalitas concludi potest, ex totius Physicae explicatione dependent:
\end{latin}
\pend
\endnumbering

\prenotes

\lemc{13--15 Praeterea\dots non potuit} a second requirement, before proving the immortality of the soul, is that we know that everything that we understand clearly and distinctly is true just as we understand it; we won't fulfill this requirement until the fourth meditation.

\lemc{13 requiri} indirect statement dependent on \textit{eos hic monendos} above.

\lemc{13 ut sciamus} a substantive clause dependent on \textit{requiri}.

\lemc{13--15 ea omnia\dots esse vera} indirect statement depending on \textit{sciamus}.

\lemc{15--17 et habendum\dots formatur} a third requirement is that we have a distinct concept of body; this is developed in the second, fourth, and sixth meditation.

\lemc{17--19} once these preliminary matters are understood, we can establish that when things are clearly and distinctly conceived of as different substances, we can be sure that they are in fact different. This is the case for mind and body, and we demonstrate their difference in the sixth meditation.

\lemc{17 ex his} the demonstrative picks up and summarizes the previous three requirements: (i) a clear conception of soul, (ii) proof that whatever we understand clearly and distinctly is just as we understand it, (iii) a distinct conception of body. Once we have these three, we can establish that mind and body are different.

\lemc{18--19 revera\dots realiter} it's not clear to me whether \textit{revera} and \textit{realiter} make two different points, or if the repetition is simply for emphasis.

\lem{19--23} provide a further argument: body is essentially divisible, while mind is essentially indivisible; thus, not only are body and mind different, they are even contrary.

\lem{21 mediam partem} = ``a half.''

\lemc{22 possumus cuiuslibet quantamvis exigui corporis} although we cannot conceive of a divided mind, we can conceive of any possible fraction (literally \textit{quantamvis} is ``any amount you like'') of any tiny body at all (\textit{cuiuslibet exigui corporis}). Supply the complementary infinitive \textit{concipere} from the previous clause.

\lemc{24--27} Descartes has not discussed this matter any further in the \textit{Meditations} for two reasons. First, he feels he has said enough to give us hope for an afterlife. Second, he believes that a full proof of the immortality of the soul would require a full treatment of physics.

\lemc{24 ulterius ea de re\dots me egisse} we are still in indirect statement dependent on \textit{eos hic monendos}. \textit{de aliqua re agere} is an idiom meaning ``to discuss something'' or ``to write about something.''

\lem{24--26 tum\dots tum etiam} = ``on the one hand\dots on the other hand'' or ``not only\dots but also.'' 

\lem{24 haec} refers to what Descares has written in this book, which is enough (\textit{sufficiunt}) for his present purposes.

\lemc{24--25 ad ostendendum} the gerund \textit{ostendendum} governs the following indirect statement \textit{ex corporis corruptione mentis interitum non sequi}.

\lemc{26 spem mortalibus faciendam} take \textit{mortalibus} as indirect object rather than dative of agent. The arguments seen so far provide sufficient reason ``to give mortals hope of another life.''

\lemc{26--27 ex\dots dependent} we would say ``depend on,'' but the Latin idiom is ``depend from.''

\clearpage

\beginnumbering
\pstart
\setline{28}
\begin{latin}
    \textenglish{\textbf{2. (cont.)}}  primo \at{14} ut sciatur omnes omnino substantias, sive res quae a Deo creari debent ut existant, ex natura sua esse incorruptibiles, nec posse unquam desinere esse, nisi ab eodem Deo concursum suum iis denegante ad nihilum reducantur; ac deinde ut advertatur corpus quidem in genere sumptum esse substantiam, ideoque nunquam etiam perire. Sed corpus humanum, quatenus a reliquis differt corporibus, non nisi ex certa membrorum configuratione aliisque eiusmodi accidentibus esse conflatum; mentem vero humanam non ita ex ullis accidentibus constare, sed puram esse substantiam: etsi enim omnia eius accidentia mutentur, ut quod alias res intelligat, alias velit, alias sentiat, etc., non idcirco ipsa mens alia evadit; humanum autem corpus aliud fit ex hoc solo quod figura quarumdam eius partium mutetur: ex quibus sequitur corpus quidem perfacile interire, mentem autem ex natura sua esse immortalem.
\end{latin}
\pend
\endnumbering

\prenotes

\lemc{28--31 primo\dots ac deinde} Descartes gives two reasons that a complete knowledge of physics is necessary to establish the soul's immortality. Each of the two reasons is given as a purpose clause: first, \textit{ut sciatur}, and next \textit{ut advertatur}.

\lemc{28 sciatur} an impersonal passive with an indirect statement (\textit{omnes\dots substantias\dots esse incorruptibiles}) as its subject.

\lem{28 sive} means ``or rather'' here (\textbf{OLD} 9b). When used this way, the word introduces a correction or clarification. This clause explains more precisely what Descartes means by \textit{substantias}. The idea seems to be that only God is responsible for creating basic substances (e.g., trees) though other things can be built using materials from these basic substances (e.g., tables and chairs).

\lem{29 debent} means ``must,'' as often in Descartes.

\lemc{29 ut existant} a purpose clause following \textit{a Deo creari debent}.

\lemc{30 ab eodem Deo concursum suum iis denegante} the first part of this phrase (\textit{ab eodem Deo}) is an ablative of personal agent; the participle at the end (\textit{denegante}) governs the rest of the phrase (\textit{concursum suum iis}). Latin word order like this is very common and helpful; the noun and participle show readers where the phrase begins and ends without punctuation. (\textbf{NB}: it is only a rule of thumb, not a hard and fast rule, that when a present participle ends in -\textit{e} it is an ablative absolute. In many cases, such as this one, that will not be the case.)

\lemc{30 concursum} the root meaning of this noun is the action or result of running or coming together. As such, it can mean ``a gathering (of people or things),'' ``a combination, union,'' or ``a clash, an attack.'' By a later development, the word also came to mean ``an agreement, concurrence, or assent.'' Descartes uses it here in this last sense in a rather technical way: according to one view of things, everything other than God exists only insofar as God grants his continued ``assent'' to their existence. As such, things can only be destroyed when god ``denies his concurrence'' to them.

\lemc{30 ad nihilum reducantur} a roundabout and emphatic way of describing destruction.

\lemc{31 advertatur} this verb, like \textit{sciatur} above, is impersonal passive with two indirect statements as its subject (\textit{corpus\dots esse substantiam} and \textit{nunquam etiam perire}).

\lemc{31 in genere sumptum} a clarification of the larger claim. Body, \textit{taken in general}, is a substance. The idea is that physical matter, \textit{as such}, is a substance and thus cannot be destroyed except through an act of God. But specific compounds made out of matter can be broken down and thus ``destroyed'' in an everyday sense of the word.

\lemc{33-39 Sed corpus humanum\dots esse immortalem} the human body is merely an assemblage of various limbs in a certain arrangment; the mind on the other hand is a true substance. Thus, the mind is immortal while the human body is perishable.

\lemc{34 accidentibus} this is a technical term in Aristotle and later philosophy. An ``accident'' is a non-essential property of a substance. A substance can remain what it is even if its non-essential properties, its accidents, change. So, for example, a person's hair color is such a non-essential property. According to Descartes, the human body differs from other bodies only in its accidents, not in its essential properties.

\lemc{36 ut quod} the \textit{ut} means ``as'' or ``for example.'' The \textit{quod} serves to nominalize the following clauses. Taken with the previous clause, that gives something like this: ``even if all <the mind's> accidents should change, for example <imagine> that it understands different things, wants different things, perceives different things, etc., the mind itself does not become something different for that reason.''

\lemc{36--37 ipsa mens alia evadit} take \textit{ipsa mens} as subject and \textit{alia} as a predicate nominative following \textit{evadit}, which functions here as a linking verb meaning ``turn out'' or ``become.''

\lemc{37 ex hoc solo quod} literally, ``from this thing alone, <namely> that;'' more idiomatically, ``merely because.'' As often \textit{hoc} points forward to \textit{quod}, which introduces a substantive clause.

\lemc{38 sequitur} the verb is impersonal, and the subjects are the two indirect statements \textit{corpus\dots perfacile interire} and \textit{mentem\dots esse immortalem}.
% -]] 2nd meditation

% [[- 3rd meditation
\clearpage

\beginnumbering
\pstart
\textbf{3.} \begin{latin}In tertia Meditatione, meum praecipuum argumentum ad probandum Dei existentiam satis fuse, ut mihi videtur, explicui. Verumtamen, quia, ut Lectorum animos quam maxime a sensibus abducerem, nullis ibi comparationibus a rebus corporeis petitis volui uti, multae fortasse obscuritates remanserunt, sed quae, ut spero, postea in responsionibus ad obiectiones plane tollentur; ut, inter caeteras, quomodo idea entis summe perfecti, quae in nobis est, tantum habeat realitatis obiectivae, ut non possit non esse a causa summe perfecta, quod ibi illustratur comparatione machinae valde perfectae, cuius idea est in mente alicuius artificis; ut enim artificium obiectivum huius ideae debet habere aliquam causam, nempe scientiam huius artificis, vel alicuius alterius a quo illam accepit, ita \at{15} idea Dei, quae in nobis est, non potest non habere Deum ipsum pro causa.\end{latin}
\pend
\endnumbering

\prenotes

\textbf{§3.} The third meditation attempts to prove that God exists. The proof involves some difficulties that the replies to objections will address. In particular, Descartes acknowledges that a lack of concrete physical examples may make his account unclear. However, Descartes is willing to give up some clarity because his priority is ``to draw readers' minds as far away as possible from the senses.''

\lemc{1--2 ad probandum Dei existentiam} a gerund taking a direct object is rare in classical Latin prose. Classical prose authors usually replace a gerund plus a direct object with a gerundive phrase; they put both the gerundive and the original direct object into the case of the original gerund. In this case, Caesar or Cicero probably would have written \textit{ad probandam Dei existentiam}. However, the meaning is essentially the same either way, and there are exceptions even in Roman authors.

\lemc{2--4} The purpose clause (\textit{ut\dots abducerem}) interrupts the causal clause (\textit{quia nullis ibi comparationibus\dots volui uti}) that it depends on. Translate in this order: \textit{quia} clause, purpose clause, main clause.

\lem{4 uti} is the present passive infinitive of \textit{utor}, not the alternative form of the adverb and conjunction \textit{ut}. (The infinitive \textit{uti} governs \textit{nullis\dots comparationibus\dots petitis}.)

\lem{5 ut} introduces an example of an objection.

\lem{5 caeteras} = \textit{cēteras} (with \textit{obiectiones} understood).

\lemc{6 quomodo\dots habeat} the objection comes in the form of a question; the subjunctive \textit{habeat} is probably potential: ``how could an idea of a supremely perfect being have\dots ?''

\lemc{6--7 obiectivae} this is a technical term, and Descartes discusses it further in the meditation itself.

\lemc{7 non possit non esse} the double negation makes a strong positive claim. If something ``is not able not to be,'' then it ``must be.'' (See the following lines where \textit{non potest non habere} corresponds to \textit{debet habere}.)

\lem{7 quod} is a connective relative. Descartes uses the neuter to refer to the entire question of the objection he has just raised.

\lem{7 ibi} picks up \textit{postea in responsibus ad obiectiones}. Descartes uses the example of a machine and its creator in his replies to the first set of objections (\textbf{CSM} II 75; \textbf{AT} VII 103).

\lemc{9--10 nempe\dots accepit} the cause of a machine is the knowledge of the person who makes it. Descartes adds an afterthought that the knowledge may originate in someone other than the builder, who then receives the knowledge at second hand. (The afterthought doesn't change the main point of the analogy, though it may initially confuse readers.)

\lemc{10 accepit} the understood subject of this verb is the \textit{artifex} of the previous clause. Descartes now imagines the builder receiving knowledge of how to build from some other party.
% -]] 3rd meditation

% [[- 4th meditation
\clearpage

\beginnumbering
\pstart
\textbf{4.} \begin{latin}In quarta, probatur ea omnia quae clare et distincte percipimus, esse vera, simulque in quo ratio falsitatis consistat explicatur: quae necessario sciri debent tam ad praecedentia firmanda, quam ad reliqua intelligenda. (Sed ibi interim est advertendum nullo modo agi de peccato, vel errore qui committitur in persecutione boni et mali, sed de eo tantum qui contingit in diiudicatione veri et falsi. Nec ea spectari quae ad fidem pertinent, vel ad vitam agendam, sed tantum speculativas et solius luminis naturalis ope cognitas veritates.)\end{latin}
\pend
\endnumbering

\prenotes

\textbf{§4.} The fourth meditation addresses questions about truth and falsity. The meditation is transitional: it solidifies earlier discussions and paves the way for future topics. Descartes warns us here that the fourth meditation only discusses truth and falsity from an intellectual or epistemological point of view; there is no discussion of truth or falsity in the context of sin or morality. 

\lem{1 probatur} is impersonal, and its subject is the clause \textit{ea omnia esse vera}.

\lem{2 ratio} with the genitive \textit{falsitatis} means something like ``nature'' or ``manner.''

\lem{3 ibi} refers forward to the meditation itself and is best translated with with \textit{agi}.

\lem{3 interim} has adversative or concessive force here: ``all the while'' or ``at the same time'' (\textbf{OLD} 3).

\lem{3--4 est advertendum} is impersonal with \textit{agi} as what ``must be noted.''

\lemc{4 agi de peccato} the passive infinitive is impersonal. More idiomatically in English: ``there is no treatment of sin'' or ``there is no discussion of sin.''

\lem{4 persecutione} means ``pursuit'' here not ``persecution.''

\lem{5 sed de eo} is short for \textit{sed agi de eo errore}.

\lemc{5--6 Nec ea spectari} the accusative and infinitive construction depends on \textit{advertendum est}, which must be resupplied from the previous sentence. Translate as if this: \textit{Et advertendum est ea non spectari}. Remember that \textit{nec} amounts \textit{et nōn}, but you will often need to separate the two elements and apply the negative force later in the clause.

\lem{6--7 speculativas\dots veritates} is the object of an understood \textit{ad}.
% -]] 4th meditation

% [[- 5th meditation
\clearpage

\beginnumbering
\pstart
\textbf{5.} \begin{latin}In quinta, praeterquam quod natura corporea in genere sumpta explicatur, nova etiam ratione Dei existentia demonstratur: sed in qua rursus nonnullae forte occurrent difficultates, quae postea in responsione ad obiectiones resolventur: ac denique ostenditur quo pacto verum sit, ipsarum Geometricarum demonstrationum certitudinem a cognitione Dei pendere.\end{latin}
\pend
\endnumbering

\prenotes

\textbf{§5.} The fifth meditation is a bit of a grab bag. It contains the following: (i) an explanation of the nature of corporeal objects, (ii) a new argument for the existence of god, and (iii) an explanation for how even geometrical proofs depend on knowledge of god. As in the synopsis of the third meditation, Descartes acknowledges that the new argument for god's existence may involve some difficulties that will only be resolved in the response to objections.

\lem{1 praeterquam quod} = ``except for the fact that,'' ``except that,'' ``other than that,'' or ``besides that.'' The \textit{quod} serves, as often, to nominalize the clause that follows (\textbf{AG} §572; \textbf{NLS} §241). 

\lem{1 natura corporea in genere sumpta} = ``corporeal nature taken in general.'' Descartes cares only about the qualities that physical objects have \textit{as such}. He isn't interested here in the specific qualities of this or that type of physical thing.

\lemc{2 in qua} supply \textit{ratione}. We would say ``during this proof.''

\lem{4--5} The subject of \textit{ostenditur} is the indirect question \textit{quo pacto verum sit}; the subject of the indirect question is the accusative plus infinitive phrase \textit{certitudinem\dots pendere}.

\lemc{4--5 ipsarum Geometricarum demonstrationum certitudinem} we might assume that certainty in a geometrical proof is self-contained, but, according to Descartes, unless we possess knowledge of god, \textit{all our other beliefs} lack certainty.

% -]] 5th meditation

% [[- 6th meditation
\clearpage

\beginnumbering
\pstart
\textbf{6.} \begin{latin}In sexta denique, intellectio ab imaginatione secernitur; distinctionum signa describuntur; mentem realiter a corpore distingui probatur; eandem nihilominus tam arcte illi esse coniunctam, ut unum quid cum ipsa componat, ostenditur; omnes errores qui a sensibus oriri solent recensentur; modi quibus vitari possint exponuntur; et denique rationes omnes ex quibus rerum materialium existentia possit concludi, afferuntur: non quod eas valde utiles esse putarim ad probandum id ipsum quod \at{16} probant, nempe revera esse aliquem mundum, et homines habere corpora, et similia, de quibus nemo unquam sanae mentis serio dubitavit; sed quia, illas considerando, agnoscitur non esse tam firmas nec tam perspicuas quam sunt eae, per quas in mentis nostrae et Dei cognitionem devenimus; adeo ut hae sint omnium certissimae et evidentissimae quae ab humano ingenio sciri possint. Cuius unius rei probationem in his Meditationibus mihi pro scopo proposui. Nec idcirco hic recenseo varias illas quaestiones de quibus etiam in ipsis ex occasione tractatur.\end{latin}
\pend
\endnumbering

\prenotes

\textbf{§6.} The sixth meditation closes out the work by returning to questions from the first and second meditations: the difference between mind and body, the mistakes that come from sense perception, and the doubts that these mistakes create. The meditation argues that we can establish the reality of the world around us, despite the doubts of the early meditations, but that our knowledge of the mind and god are nevertheless far more certain.

\lemc{1--2 signa} translate here as ``criteria.'' This is a common philosophical meaning of \textit{signum}.

\lem{2 probatur} has \textit{mentem\dots distingui} as its subject.

\lemc{2--3} although Descartes insists on the distinction of mind from body, he also insists that they form a ``single something'' (\textit{unum quid}) because of their close connection. Notice that the mind remains the subject of \textit{esse coniunctam} and \textit{componat}. Descartes does not believe that mind and body are \textit{equal} partners, even if they work very closely together in a living person.

\lem{3 arcte} is an alternative form of \textit{arte}.

\lem{3 ostenditur} has \textit{eandem (mentem)\dots esse coniunctam} as its subject.

\lem{9 illas} supply \textit{rationes}. The demonstrative \textit{illas} is the direct object of the gerund \textit{considerando}, but we also need to use it as the subject of \textit{non esse tam firmas nec tam perspicuas}. (That infinitive phrase is, in turn, the subject of \textit{agnoscitur}.)

\lem{10 in\dots devenimus} we would say that we ``arrive at'' or ``reach'' the knowledge of something.

\lem{12--14} the most important thing Descartes set out to do in these meditations is to prove that we know our own minds and god better than we know anything else; as a result, Descartes does not address every minor topic in this synopsis; he will discuss such topics in the meditations themselves, as appropriate.

\lemc{13 ipsis} supply \textit{Meditationes}.

\lem{tractatur} is impersonal. We might say ``concerning which things (\textit{de quibus})\dots there is discussion.''

% -]] 6th meditation

% -]] Synopsis
