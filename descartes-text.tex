% [[- Chapter title
\chapter*{Text}
\def\ind{%
    \hspace{2em}%
}
% -]] Chapter title

% [[- Meditatio I
\section*{Meditatio I}

[1] Animadverti iam ante aliquot annos quam multa, ineunte aetate, falsa pro veris admiserim, et quam dubia sint quaecunque istis postea superextruxi, ac proinde funditus omnia semel in vita esse evertenda, atque a primis fundamentis denuo inchoandum, si quid aliquando firmum et mansurum cupiam in scientiis stabilire; sed ingens opus esse videbatur, eamque aetatem expectabam, quae foret tam matura, ut capessendis disciplinis aptior nulla sequeretur. Quare tamdiu cunctatus sum ut deinceps essem in culpa, si quod temporis superest ad agendum, deliberando consumerem. Opportune igitur hodie mentem curis\at{18} omnibus exsolvi, securum mihi otium procuravi, solus secedo, serio tandem et libere generali huic mearum opinionum eversioni vacabo.

[2] Ad hoc autem non erit necesse, ut omnes esse falsas ostendam, quod nunquam fortassis assequi possem; sed quia iam ratio persuadet, non minus accurate ab iis quae non plane certa sunt atque indubitata, quam ab aperte falsis assensionem esse cohibendam, satis erit ad omnes reiiciendas, si aliquam rationem dubitandi in unaquaque reperero. Nec ideo etiam singulae erunt percurrendae, quod operis esset infiniti; sed quia, suffossis fundamentis, quidquid iis superaedificatum est sponte collabitur, aggrediar statim ipsa principia, quibus illud omne quod olim credidi nitebatur.

[3] Nempe quidquid hactenus ut maxime verum admisi, vel a sensibus, vel per sensus accepi; hos autem interdum fallere deprehendi, ac prudentiae est nunquam illis plane confidere qui nos vel semel deceperunt.

[4] Sed forte, quamvis interdum sensus circa minuta quaedam et remotiora nos fallant, pleraque tamen alia sunt de quibus dubitari plane non potest, quamvis ab iisdem hauriantur: ut iam me hic esse, fovo assidere, hyemali toga esse indutum, chartam istam manibus contrectare, et similia. Manus vero has ipsas, totumque hoc corpus meum esse, qua ratione posset negari? nisi me forte comparem nescio quibus insanis,\at{19} quorum cerebella tam contumax vapor ex atra bile labefactat, ut constanter asseverent vel se esse reges, cum sunt pauperrimi, vel purpura indutos, cum sunt nudi, vel caput habere fictile, vel se totos esse cucurbitas, vel ex vitro conflatos; sed amentes sunt isti, nec minus ipse demens viderer, si quod ab iis exemplum ad me transferrem.

[5] Praeclare sane, tanquam non sim homo qui soleam noctu dormire, et eadem omnia in somnis pati, vel etiam interdum minus verisimilia, quam quae isti vigilantes. Quam frequenter vero usitata ista, me hic esse, toga vestiri, foco assidere, quies nocturna persuadet, cum tamen positis vestibus iaceo inter strata! Atqui nunc certe vigilantibus oculis intueor hanc chartam, non sopitum est hoc caput quod commoveo, manum istam prudens et sciens extendo et sentio; non tam distincta contingerent dormienti. Quasi scilicet non recorder a similibus etiam cogitationibus me alias in somnis fuisse delusum; quae dum cogito attentius, tam plane video nunquam certis indiciis vigiliam a somno posse distingui, ut obstupescam, et fere hic ipse stupor mihi opinionem somni confirmet.

[6] Age ergo somniemus, nec particularia ista vera sint, nos oculos aperire, caput movere, manus extendere, nec forte etiam nos habere tales manus, nec tale totum corpus; tamen profecto fatendum est visa per quietem esse veluti quasdam pictas imagines, quae non nisi ad similitudinem rerum verarum fingi potuerunt; ideoque saltem generalia haec, oculos, caput, manus, totumque corpus, res quasdam non imaginarias, sed veras existere. Nam sane pictores ipsi, ne tum qui\at{20}dem, cum Sirenas et Satyriscos maxime inusitatis formis fingere student, naturas omni ex parte novas iis possunt assignare, sed tantummodo diversorum animalium membra permiscent; vel si forte aliquid excogitent adeo novum, ut nihil omnino ei simile fuerit visum, atque ita plane fictitium sit et falsum, certe tamen ad minimum veri colores esse debent, ex quibus illud componant. Nec dispari ratione, quamvis etiam generalia haec, oculi, caput, manus, et similia, imaginaria esse possent, necessario tamen saltem alia quaedam adhuc magis simplicia et universalia vera esse fatendum est, ex quibus tanquam coloribus veris omnes istae, seu verae, seu falsae, quae in cogitatione nostra sunt, rerum imagines effinguntur.

[7] Cuius generis esse videntur natura corporea in communi, eiusque extensio; item figura rerum extensarum; item quantitas, sive earumdem magnitudo et numerus; item locus in quo existant, tempusque per quod durent, et similia.

[8] Quapropter ex his forsan non male concludemus Physicam, Astronomiam, Medicinam, disciplinasque alias omnes, quae a rerum compositarum consideratione dependent, dubias quidem esse; atqui Arithmeticam, Geometriam, aliasque eiusmodi, quae nonnisi de simplicissimis et maxime generalibus rebus tractant, atque utrum eae sint in rerum natura necne, parum curant, aliquid certi atque indubitati continere. Nam sive vigilem, sive dormiam, duo et tria simul iuncta sunt quinque, quadratumque non plura habet latera quam quatuor; nec fieri posse videtur ut tam perspicuae veritates in suspicionem falsitatis incurrant.\at{21}

[9] Verumtamen infixa quaedam est meae menti vetus opinio, Deum esse qui potest omnia, et a quo talis, qualis existo, sum creatus. Unde autem scio illum non fecisse ut nulla plane sit terra, nullum coelum, nulla res extensa, nulla figura, nulla magnitudo, nullus locus, et tamen haec omnia non aliter quam nunc mihi videantur existere? Imo etiam, quemadmodum iudico interdum alios errare circa ea quae se perfectissime scire arbitrantur, ita ego ut fallar quoties duo et tria simul addo, vel numero quadrati latera, vel si quid aliud facilius fingi potest? At forte noluit Deus ita me decipi, dicitur enim summe bonus; sed si hoc eius bonitati repugnaret, talem me creasse ut semper fallar, ab eadem etiam videretur esse alienum permittere ut interdum fallar; quod ultimum tamen non potest dici.

[10] Essent vero fortasse nonnulli qui tam potentem aliquem Deum mallent negare, quam res alias omnes credere esse incertas. Sed iis non repugnemus, totumque hoc de Deo demus esse fictitium; at seu fato, seu casu, seu continuata rerum serie, seu quovis alio modo me ad id quod sum pervenisse supponant; quoniam falli et errare imperfectio quaedam esse videtur, quo minus potentem originis meae authorem assignabunt, eo probabilius erit me tam imperfectum esse ut semper fallar. Quibus sane argumentis non habeo quod respondeam, sed tandem cogor fateri nihil esse ex iis quae olim vera putabam, de quo non liceat dubitare, idque non per inconsiderantiam vel levitatem, sed propter validas et meditatas rationes; ideoque etiam ab iisdem, non minus quam ab aperte falsis,\at{22} accurate deinceps assensionem esse cohibendam, si quid certi velim invenire.

[11] Sed nondum sufficit haec advertisse, curandum est ut recorder; assidue enim recurrunt consuetae opiniones, occupantque credulitatem meam tanquam longo usu et familiaritatis iure sibi devinctam, fere etiam me invito; nec unquam iis assentiri et confidere desuescam, quamdiu tales esse supponam quales sunt revera, nempe aliquo quidem modo dubias, ut iam iam ostensum est, sed nihilominus valde probabiles, et quas multo magis rationi consentaneum sit credere quam negare. Quapropter, ut opinor, non male agam, si, voluntate plane in contrarium versa, me ipsum fallam, illasque aliquandiu omnino falsas imaginariasque esse fingam, donec tandem, velut aequatis utrimque praeiudiciorum ponderibus, nulla amplius prava consuetudo iudicium meum a recta rerum perceptione detorqueat. Etenim scio nihil inde periculi vel erroris interim sequuturum, et me plus aequo diffidentiae indulgere non posse, quandoquidem nunc non rebus agendis, sed cognoscendis tantum incumbo.

[12] Supponam igitur non optimum Deum, fontem veritatis, sed genium aliquem malignum, eundemque summe potentem et callidum, omnem suam industriam in eo posuisse, ut me falleret: putabo coelum, aërem, terram, colores, figuras, sonos, cunctaque externa nihil aliud esse quam ludificationes somniorum, quibus insidias credulitati meae tetendit: considerabo\at{23} meipsum tanquam manus non habentem, non oculos, non carnem, non sanguinem, non aliquem sensum, sed haec omnia me habere falso opinantem: manebo obstinate in hac meditatione defixus, atque ita, siquidem non in potestate mea sit aliquid veri cognoscere, at certe hoc quod in me est, ne falsis assentiar, nec mihi quidquam iste deceptor, quantumvis potens, quantumvis callidus, possit imponere, obfirmata mente cavebo. Sed laboriosum est hoc institutum, et desidia quaedam ad consuetudinem vitae me reducit. Nec aliter quam captivus, qui forte imaginaria libertate fruebatur in somnis, cum postea suspicari incipit se dormire, timet excitari, blandisque illusionibus lente connivet: sic sponte relabor in veteres opiniones, vereorque expergisci, ne placidae quieti laboriosa vigilia succedens, non in aliqua luce, sed inter inextricabiles iam motarum difficultatum tenebras, in posterum sit degenda.
% -]] Meditatio I
