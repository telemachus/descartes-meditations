% [[- LaTeX prelude
\documentclass[12pt,letterpaper]{book}

\usepackage[no-math]{fontspec}
\setmainfont{Baskerville}

\usepackage{polyglossia}
\setdefaultlanguage{english}
\setotherlanguage[variant=medieval]{latin}
\setotherlanguage[variant=ancient]{greek}
\newfontfamily\greekfont[Script=Greek]{Times New Roman}

\usepackage{natbib}
\usepackage{bibentry}
\nobibliography*
\bibpunct{(}{)}{;}{a}{,}{,}
\usepackage{enumitem}
\setlist{noitemsep}
\usepackage[super]{nth}
\usepackage[linktoc=all]{hyperref}

\usepackage[noend,nofamiliar,noledgroup,series={A}]{reledmac}
\Xlemmaseparator[A]{:}

% Insert AT numbers and a marker for where the division occurs
\newcommand{\at}[1]{%
    |\ledsidenote{{#1}}%
}

% Wrapper for textual notes. Use as follows:
% \var{word or phrase}{comment}
\newcommand{\var}[2]{%
    {#1}\footnote{{#1} : {#2}}%
}
% Second wrapper for textual variants. Use as follows:
% \vvar{word or phrase}{source}{alternative}{source}
\newcommand{\vvar}[4]{%
    {#1}\footnote{{#1} \textbf{{#2}} : {#3} \textbf{{#4}}}%
}
% A wrapper for introducing new items into the commentary
\newcommand{\lemc}[1]{\textbf{{#1}:}}
\newcommand{\lem}[1]{\textbf{{#1}}}

% Hackish way to make abbreviations look the way I want them to
\newcommand{\abbrlabel}[1]{\makebox[3cm][l]{\textbf{#1}}}
\newenvironment{abbreviations}{%
    \begin{list}{}{\renewcommand{\makelabel}{\abbrlabel}%
    \setlength{\labelwidth}{3cm}\setlength{\leftmargin}{\labelwidth+\labelsep}%
    \setlength{\itemsep}{0pt}}}{\end{list}%
}

% Clear space under a chunk of text before printing notes.
\newcommand{\prenotes}{%
    \bigskip
    \bigskip
    \footnoterule
    \bigskip
}

\begin{hyphenrules}{latin}
    \hyphenation{pro-inde su-per-ex-tru-xi}
\end{hyphenrules}

\begin{hyphenrules}{english}
    \hyphenation{Des-cartes pos-sent}
\end{hyphenrules}
% -]]

% [[- Document-
\begin{document}

\nocite{*}

% [[- Title page
\begin{titlepage}

\begin{center}

\huge \textit{Meditationes de prima philosophia}

\huge René Descartes 

\vskip2in

\large \copyright Peter Aronoff \the\year

(See LICENSE for details)

\vskip1in

\textbf{NB}: This work is in progress and likely to change.

\vskip2in

\newpage

\end{center}

\end{titlepage}
% -]] Title page


\frontmatter
    % [[- TOC
\setcounter{tocdepth}{1} % Don't show subsections in the TOC
\tableofcontents
% -]] TOC

    % [[- Chapter title
\chapter{Abbreviations}
% -]] Chapter title

% [[- Abbreviations
The works of Descartes are abbreviated as follows: TODO

\begin{itemize}
    \item[AT] \bibentry{adamtannery1913}
    \item[CB] \bibentry{cottingham1976}
    \item[EB] \bibentry{adam1975}
    \item[NLG] \bibentry{mahoney2001}
    \item[OLD] \bibentry{old1982}
\end{itemize}
% -]] Abbreviations


\mainmatter
    % [[- Chapter title
\chapter{Introduction}
% -]] Chapter title

% [[- Descartes's life
\section*{Life}

René Descartes was born on March 31, 1596 to Jeanne and Joachim Descartes. His birthplace was La Haye, a small town near Tours in west-central France. (In honor of the philosopher, the town is now officially named Descartes.) Scholars disagree about the precise social status of Descartes's family, but we know that they were well-educated and of significant means. There were doctors, lawyers, and civil servants on both sides of Descartes's family, and an inheritance allowed the frugal philosopher to live nearly his whole life without any need for outside income. Joachim Descartes was a magistrate and lawyer for the Parliment at Brittany, where he lived for part of every year. Jeanne Descartes passed away during childbirth when Descartes was just over a year old. He and his siblings, an older brother Pierre and an older sister Jeanne, grew up with their maternal grandmother in La Haye.

Descartes received an excellent education. He attended the Jesuit-run Collège Royal Henry-IV de La Flèche for eight years. The exact dates are uncertain, but the current consensus is that he started in 1607, when he was 10, and left in 1615. He received oustanding training in Latin, rhetoric, and classical literature. Descartes also studied logic, mathematics, science, and philosophy. After finishing at La Flèche, Descartes went to the University of Poitiers, where he received a Baccalaureat and a degree in Law.

Unlike his father and older brother, Descartes did not choose the legal profession. Instead in 1618, Descartes left for Holland and he joined a volunteer army. It is not clear what his duties were, and the suggestions of biographers are very varied. Some believe that he was a soldier, others that he worked as a military engineer, and yet others that Descartes was a student at a military academy and saw no active service as a soldier or engineer. In later years, Descartes himself says only that he wanted to travel and see more of the world.

The years 1618 and 1619 were enormously important for Descartes. In 1618, Descartes met Isaac Beeckman (1588-1637), a Dutch thinker. Beeckman helped to inspire Descartes towards what became a lifelong devotion to mathematics and science. In November of 1619, Descartes had an epiphany. While alone in winter quarters, he spent a day thinking intensely and that night he had a series of three, extraordinarily powerful dreams. From this time on, he strove to produce a universal science that would encompass the natural sciences, mathematics, and philosophy. He spent the rest of his life working on this plan in one form or another. Although his precise view of what he was doing changed over the next thirty years, the underlying goals and principles of the plan remained remarkably consistent.

Although he returned to France more than once, Descartes lived most of his adult life elsewhere in Europe. In 1628, Descartes moved to the Netherlands. He moved around a great deal, but lived in various places there for most of the next twenty years. After a brief return to France in 1647, Descartes moved to Sweden in 1649 at the invitation of Queen Christina. He caught pneumonia in February of 1650 and died on February 11 of that year.
% -]] Descartes's life

% [[- Descartes's two projects
\section*{Science and Philosophy\footnote{I rely heavily on \cite{gaukroger2011} in this section.}}

Descartes is best known now as a philosopher with a serious sideline in mathematics, but he also studied a wide variety of questions in the natural sciences. In the long run of history, his work in philosophy and math have been much more influential than his scientific work. But during his own life and in the hundred years after he died, this was not clear. Before discussing the \textit{Meditations} specifically, I want to give an overview of Descartes's career and describe how his different projects fit together.

After meeting Isaac Beeckman in 1618, Descartes initially devoted himself to mathematics. He worked intermittently from 1619 to 1628 on a book he eventually left unfinished: his \textit{Regulae ad directionem ingenii} (\textit{Rules for the direction of the mind}). In this work, Descartes attempted to describe a universal method of inquiry based on mathematics. The work was inspired by his conviction that researchers could establish proofs in math that were entirely certain. Although he eventually realized that he could not use the method he described even to represent all mathematical knowledge, much less all knowledge as a whole, he retained a lifelong interest in methodology. He strove to discover the universal method that eluded him early on. The \textit{Regulae} also foreshadows Descartes's ongoing fascination with certainty and his use of mathematics as a model for other disciplines.

Beginning in 1629, Descartes conceived of an even grander project, which again he never finished. He intended to write a three-part science: the first part \textit{Le Monde} (\textit{The World}) on inanimate nature; a second \textit{L'Homme} (\textit{Man}) on the nature of living creatures, especially humans; and a third part on thought and the mind. Descartes wrote quite a lot of \textit{Le Monde} and \textit{L'Homme}, but he never began the third part. Nor did he publish the first two sections in their original forms for reasons explained below.

Without going into too much detail, the key fact about this project is how well it fits into the scientific revolution of the seventeenth century. Like Galileo Galilei, Descartes argued for a heliocentric view of the universe. Like Robert Boyle, Descartes explained the workings of inanimate nature by means of atomism, a complex theory of microscopic particles in motion. Like William Harvey, Descartes studied the circulatory and nervous systems of animals and humans; they treated the bodies of living creatures like machines, subject to the same laws of matter as the rest of the universe. Descartes clashed with other thinkers over countless details, but in a larger sense, he fits into his times perfectly.

In 1633 the Catholic Church condemned Galileo for supporting heliocentrism, and this led Descartes to supress his scientific work temporarily. Descartes knew that he was liable to censure or condemnation as well since he advanced essentially the same controversial views as Galileo. They both followed Copernicus and argued for heliocentrism; they both rejected the view of an unmoving Earth at the center of the universe. Throughout his life, Descartes avoided quarrels with the Catholic Church wherever he could. At the same time, he hoped that his new science might supplant the traditional Aristotelian curriculum at Catholic institutions like La Flèche where he studied as a young man. But Descartes preferred to work through indirect persuasion rather than challenge the Catholic Church openly.

At the urging of friends, Descartes published his first mature work in 1637, \textit{Discours de la méthode} (\textit{Discourse on the method}). In this work, Descartes returned to his interests in methodology and certainty, but the \textit{Discourse} was innovative in two important ways. First, although Descartes still aimed at certainty, he attempted to reach it in a new manner. Descartes used skepticism, paradoxically, to achieve certainty. In a nutshell, he applied skeptical doubt to \textit{everything} he believed. Anything that survived this doubt has been shown to be immune to doubt and thus certain. The second innovation was formal. Instead of a traditional essay or set of theses followed by supporting arguments, the \textit{Discourse} was essentially an autobiography. Descartes employed his own life story as a `fable' (\textbf{AT} VI 4) in order to make his abstract arguments more concrete and engaging. We will see both of these features, the use of skepticism and formal innovation, again in the \textit{Meditations}.

Descartes also used this opportunity to repurpose some of his work from \textit{The World} and \textit{Man}. He reorganized three large chunks of material from those works, removed any connection to heliocentrism, and published them together with the \textit{Discourse} as `essays that exemplify the method'. These detailed scientific essays demonstrated what you could accomplish if you followed Descartes's method. At the same time, they allowed Descartes to publish some of his scientific work without running the risk of condemnation by the Catholic Church.

In 1641 Descartes published what would prove to be his best known work, \textit{Meditationes de prima philosophia} (\textit{Metaphysical Meditations}). The \textit{Meditations} covered the same topics as the fourth section of the \textit{Discours}, but at greater length. Both works used skepticism in order to reach certainty, argued for existence of god, discussed the human soul, and argued that intellectual knowledge is prior to and more important than sensory knowledge. The \textit{Meditations} provided additional argument and explanation for each of these topics, and they also explained Descartes's theory of matter, the nature of the human mind and thought, and the relationship between a person's body and soul. (In this way, the \textit{Meditations} served as the previously unwritten third part of the grand project that Descartes began in 1619.)

Descartes spent the remaining nine years of his life largely consolidating his views, responding to critics, and offering clarifications where needed. In 1644 Descartes published \textit{Principia Philosophiae} (\textit{Principles of Philosophy}). Descartes hoped that this work could become a textbook, and in it he brought together the epistemology and metaphysics of his \textit{Discourse} and \textit{Meditations} with his earlier scientific work from \textit{The World} and \textit{Man}. In 1649, Descartes released the last work published in his lifetime, \textit{Les Passions de l'âme} (\textit{The Passions of the Soul}). This work was sparked by a long period of correspondence between Descartes and Princess Elizabeth of Bohemia. In response to her probing questions and criticisms, Descartes attempted to clarify the connection between mind and body in his system. He focused on human emotions since these are an especially obvious and important case of interaction between mind and body: what you think affects how you feel, and how you feel affects what you think. (Infamously, Descartes isolates a specific \textit{physical location} for the interaction of mind and body: the pineal gland.)

After his death, many of Descartes's previously unpublished works and letters were published. In addition, his executors discovered an entirely unknown manuscript: \textit{La Recherche de la Verité par la lumière naturelle} (\textit{The Search for Truth by Means of Natural Light}\footnote{Descartes often used the phrase `natural light' to mean the basic intellectual capability that human beings share. This capability, which he also called `reason' and `good sense' is what allows people to separate truth from falsehood and to think, properly speaking. In his opinion, animals other than humans lack this capability. Although they enjoy sensory perception and have physical desires, they are only complex machines without true intelligence or free will.}). This work was very brief---only around fifteen pages survived---and the original French manuscript was soon lost again. Descartes covered familiar questions about truth, certainty, and knowledge in this work, but it is unique in form. Descartes imitated ancient writers like Plato and Cicero in \textit{The Search}: the book is a dramatic dialogue in which three characters exchange views and argue with one another.

Although he died relatively young just shy of his \nth{54} birthday, Descartes had an exceedingly wide-ranging career as a thinker and a writer. He studied atomic physics (as it was understood in the sixteenth century, that is), the refraction of light, the mechanics of telescopic lenses, weather, animal and human physiology, the existence and nature of god, epistemology, morality and free will, philosophy of mind and the emotions, and more. He wrote elegantly in both French and Latin, using French when he wanted to reach a wider and more varied audience and Latin when he was concerned primarily with scholars and the Catholic Church. However, he also allowed translations of his French works into Latin and his Latin works into French, in order to reach the largest possible overall audience. Descartes also devoted significant effort to formal innovations as a writer, frequently varying the style and approach of his works and adapting different genres, again in an attempt to engage and expand his readership.
% -]] Descartes's two projects

% [[- The Meditations
\section*{The Meditations}

\subsection*{Goals}

Descartes sets out two goals very clearly in the subtitle to the second edition of the \textit{Meditations}.\footnote{The first edition had no subtitle.}

\begin{quote}
    In quibus Dei existentia et animae a corpore distinctio demonstrantur

    In which the existence of god and the distinction of soul from body are proven
\end{quote}
We should expect the \textit{Meditations} to provide (i) proof of the existence of god and (ii) an account of soul and body that explains how they function and how they differ.

The \textit{Meditations} lives up to the promises of the subtitle, but first-time readers are likely to still have questions about the book's organization and purpose. First, it's not immediately clear how the two goals of the subtitle are connected. What unifies a book with these two topics? Second, Descartes offers proof of god's existence in the third meditation (as well as a secondary proof in the fifth meditation), and he proves and briefly explains the distinction between soul and body in the sixth meditation. But what is the purpose of the other meditations? How are their arguments related to Descartes's primary goals?

Briefly, I would argue that in addition to his two stated goals, Descartes also had two unstated goals. First, Descartes wanted whatever proofs he advanced to be clear and certain. Therefore, Descartes began the \textit{Meditations} with a skeptical purification of his beliefs, as in his earlier \textit{Discourse}. During the first and second meditations, Descartes prepared the way for his proof of god's existence, and he also implicitly taught readers how certainty and clarity could be achieved in argument. Second, Descartes implicitly included the principles of his physics in the \textit{Meditations}. In a letter to a friend in 1641, he wrote as follows:

\begin{quote}
    I will tell you, just between the two of us, that these six \textit{Meditations} contain all the foundations of my physics. But, please, it is essential not to say anything about this; for the Aristotelians might have more trouble approving of them\footnote{I.e. Aristotelian scholars would be less likely to approve of Descartes's \textit{Meditations}, if they knew how revolutionary they really were.}. And I hope that those who read them will become accustomed to my principles without realizing it, and that they will acknowledge their truth before they realize that <my principles> destroy those of Aristotle.\footnote{Descartes to Father Mersenne, January 28, 1641. \textbf{ALQ} II 316--317.}
\end{quote}

Descartes hoped that his physics and his philosophy might supplant the Aristotelian teaching that was officially sanctioned by the Catholic Church. But he did not want to `poke the bear', so to speak. As often in his life, Descartes tried to avoid confrontation. Descartes's choice might seem underhanded, but there's a more charitable way to view it. Descartes hoped to give his new ideas the best possible opportunity to convince people. He believed that the \textit{Meditations} had a better chance of persuading unprejudiced readers. He never lied about any of his principles; he simply avoided picking a fight before he had a chance to present his views.

In any case, Descartes began the \textit{Meditations} by clearing the ground entirely. The meditator\footnote{As discussed more below, the \textit{Meditations} are not written as a standard philosophical essay. An unnamed person, whom I will refer to as `the meditator' narrates the work in the first person. We should not identify this `I' with Descartes for several reasons, as we will see.} rejects all previous beliefs and lets skeptical doubts extend as widely as possible. The meditator doubts everything until reaching a single truth that is impossible to doubt. Doubt ends when the meditator comes to this: \textit{I think therefore I exist}.\footnote{Descartes doesn't put it exactly this way in the \textit{Meditations}, but this is the traditional formula, and it will do for now.} Scholars call this argument `the \textit{cogito}' since \textit{cogito} is Latin for `I think'.

The meditator employs the \textit{cogito} in pursuit of Descartes's stated goals. A complex series of arguments carries the meditator, in the second meditation, from the \textit{cogito} to the innate idea of god and from there to a proof that god exists in the third meditation. A second complex series of arguments leads to a second proof of god's existence in the fifth meditation and, finally, in the sixth meditation to a proof that the human soul is distinct from its body.

At the same time, Descartes scattered the \textit{Meditations} with key elements of his epistemology and his physics. In the second meditation, the meditator realizes that humans understand their own minds better and more intimately than they understand physical nature. After the proof of god's existence in meditation three, the meditator spends the fourth meditation analyzing the nature of truth and falsity and explaining the role of free will in human knowledge and error. In the fifth and sixth meditations, the meditator investigates the essential nature of physical things while proving the existence of god and examining the distinction between human soul and body.

If we keep in mind both his stated and his unstated goals, we will best understand the apparent detours and digressions in Descartes's \textit{Meditations}. The work is actually quite unified and well organized, but some of the principles of unity and organization are only implicit.

\subsection*{Form}

The form of Descartes's \textit{Meditations} is unusual and requires some explanation. Instead of writing a treatise, essays, or even a dialogue---all standard forms of philosophical writing in the 1600s---Descartes chose to present his ideas in what was traditionally a religious form. Contemporary readers would have associated the title and the first person voice of Descartes's \textit{Meditations} with books of spiritual guidance. So we need to consider why Descartes made this unusual choice and what it means for us.

Contemporary readers would have expected specific things from a book of meditations, and we should be on the lookout for these features. First, a book of meditations was a kind of `How To' book. Meditation books tried to change their readers behavior as well as their minds. Second, when a reader sees `I' in a book of meditations, they don't associate the first person with the author but with \textit{themselves}. Since a book of meditations is meant to be a lived experience, each reader is meant to take the `I' position for themselves.

We find both of these features in Descartes's \textit{Meditations}. In the preface, Descartes says that he wants only readers who are prepared `to meditate seriously with me and to draw their mind away from the senses and at the same time away from all prejudices' (\textbf{AT} VII.9). Descartes demands readers who will be active, change their way of thinking, and give up previous false beliefs. In addition, although many modern readers associate the speaking `I' with the author Descartes, we should not do so. Descartes appears to have modeled the meditative experiences of the narrator on some of his own experiences, but (i) he has altered the situation in many ways from that of his own life and (ii) he ascribes to the narrating `I' several views that we know the actual Descartes did not hold in 1641 and very likely had not held in 1619 when he first conceived of his grand project.

By using the genre of meditations, Descartes frees his arguments from anyone's authority, even his own. It was extremely common in Descartes's time to base arguments on appeals to the authority of ancient philosophers, especially Aristotle, or the Catholic church.\footcites()(discuss the importance of authority in the seventeenth century. They also both consider why and how Descartes rejected authority)[4--7]{cottingham1986}[4--5]{garber1998} Descartes believed, however, that each person should work through his arguments for themselves. If we read the \textit{Meditations} in the right way, we will not simply accept (or reject) their claims because of someone's authority. Instead, we will make the arguments our own and decide for ourselves what to believe.

By placing philosophical content in a meditative form, Descartes has created something new. Today's philosophers are far more likely to write an essay than a book of meditations, but they wholeheartedly agree with Descartes that readers must make the arguments of a philosophical work their own. Descartes helped modern philosophy to break out of the rut of a traditional curriculum based on authority.
% -]] The Meditations

% [[- Skepticism
\section*{Skepticism}

\subsection*{Ancient Skepticism}

The ancient skeptics who had the most influence on early modern philosophy and Descartes belonged to the Pyrrhonian school. Our most important source for Pyrrhonian skepticism is Sextus Empiricus, a philosopher and doctor. Scholars do not know exactly when Sextus was born or died, or where he lived, but he was probably active in the \nth{2} century CE.

The word `skeptic' comes from a Greek root (\textgreek{σκεπτ}- or in English transliteration \textit{skept}-) that means `inquire' or `investigate'. And according to Sextus, the key difference between skeptics and all other philosophers is that the skeptics `are still investigating' \cite[I i 3]{annasbarnes1994}. Other philosophers believe that they have discovered the truth, but skeptics are not yet convinced.

To explain what it means for the skeptics to still be investigating, consider people's views about religion. Believers and atheists both think that they have answered the question `Does god exist?' The believer answers yes, and the atheist no. Most philosophers in antiquity were like believers: they thought that they had various true theories about the world. Some ancient philosophers were like atheists: they were prepared to declare that there was no knowledge or that knowledge was impossible. If we take Sextus seriously, however, Pyrrhonian skeptics are like agnostics. Pyrrhonians don't think that they know anything, not even \textit{that nothing can be known}. Instead, they make the more restricted assertion that thus far in their search they have not found any answers. Thus, Pyrrhonians avoid the naive inconsistency of saying, for example, `We know that we know nothing'. This also explains why they are \textit{still} investigating: they haven't yet found any conclusive answer.

Over time Pyrrhonian skeptics developed what they called `modes'. A mode is a generic argument form that a skeptic can apply by plugging in specifics as needed.\footnote{For an excellent introduction to the modes and Pyrrhonian skepticism in general, see \cite{annasbarnes1994}.} There are several different lists of modes, but the most useful for Descartes is the following list of five:

\begin{description}
    \item[Disagreement] While trying to resolve some question \textit{q}, we discover undecided conflict. Since we cannot resolve the disagreement, we suspend judgment.
    \item[Infinite regress] While trying to resolve \textit{q}, someone offers a reason \textit{r1}. But \textit{r1} itself needs support from some other reason \textit{r2}, and so on. This leads to an infinite regress. Since we cannot work through an infinite series of reasons, we suspend judgment.
    \item[Relativity] While trying to resolve \textit{q}, we discover that the matter appears one way to one person (or in one situation), but another way to another person (or in another situation). Since we cannot decide between the various appearances, we suspend judgment.
    \item[Hypothesis] While trying to resolve \textit{q}, a person will sometimes offer an assumption. The assumption is meant to provide a foundation for other arguments. But since the assumption is itself not supported, we cannot know if it is true and so we suspend judgment.
    \item[Circularity] While trying to resolve \textit{q}, we offer \textit{r} as a supporting reason. However, \textit{r} in turn requires the support of \textit{q}. Since the argument is viciously circular, we suspend judgment.
\end{description}

Notice that all of these modes end in the same way: suspension of judgment.\footnote{The Greek term for this is \textgreek{ἐποχή} or \textit{epochê} when transliterated into English.} As an example, imagine you are trying to decide whether a certain genetically modified food is healthy. You are no expert, so you read several academic studies. The studies disagree, and you have no way to choose between them: the authors of the studies are all equally qualified, none of them works for the corporation responsible for the modified food, and so on. In such a case, it seems that the only reasonable thing to do is stay neutral, not pick either side, and continue to investigate if possible. This is what skeptics mean by `suspend judgment'. The skeptics are `still investigating' since they discover that the result of \textit{all} investigations is equally weighted argument on either side of the question.\footnote{Pyrrhonians call this \textgreek{ἰσοθενία} or \textit{isothenia} when transliterated into English. Scholars and translators often use `equipollence' to represent this idea in English.}

These skeptical arguments suggest a strikingly strong conclusion: in (nearly?) all cases, we have no reason to believe \textit{anything}. In such cases, if we suspend judgment as the skeptics recommend, we will end up with no opinion on the matter. In order to show how unsettling this result is, consider the following. If we asked `Is the total number of stars in the sky odd or even?', surely most people would answer that they had no idea. There's no practical way to count or to guess. Any answer would be no better than flipping a coin. In this case, the question has no bearing on our lives, so it is easy to respond `I have no idea.' What the skeptics suggest is that we should be equally unsure of nearly \textit{all} questions in life. This should shock us.

Ancient skeptical arguments should suprise us for a second reason: they undermine belief rather than knowledge. Modern skeptical arguments attack  only knowledge. They (mostly) ignore or downplay belief. That is, the conclusion of a modern skeptical argument might be that (i) you believe \textit{x} and (ii) you even have good reason to believe \textit{x}, but (iii) you don't really know \textit{x}. But whether or not you know, you still have your beliefs, and those beliefs are possibly even well-justified. But ancient skeptics argue us into a much tighter corner. When they are done with us, we have no more reason to believe \textit{x} than \textit{not x}, and as a result we end up with no opinion on the matter at all. (To be clear, by attacking belief, ancient skeptics \textit{also} destroy knowledge since knowledge relies on true belief. But the ancient skeptics target belief first and foremost, not knowledge.) If skeptical doubt covers everything successfully, then everything becomes as unclear as whether the total number of stars in the sky is even or odd.

In response to this radical attack, non-skeptical philosophers---whom the skeptics called `dogmatists'---accused the Pyrrhonians of two kinds of inconsistency. First, the dogmatists argued that the Pyrrhonians were logically inconsistent. On the one hand, they claimed to have no beliefs, but on the other hand, they made numerous assertions. According to the dogmatists, this places skeptics in a bind: either the skeptics believe in their assertions or they don't. If they believe their own assertions, then they \textit{do} have some beliefs, and they are inconsistent. If they don't believe their own assertions, then we don't need to take their arguments seriously. After all, when we are having serious discussions, we normally dismiss things people say at random or for no particular reason. Second, the dogmatists accused Pyrrhonians of a practical inconsistency. In this case, there is not, strictly speaking, a formal inconsistency, but the behavior of the Pyrrhonians clashes with their statements. The skeptics say that they have no beliefs, but they still make choices, they still act. What motivates their actions, ask the dogmatists, if not belief? For example, a skeptic will get out of the way of oncoming traffic. Doesn't this show that the skeptic believes (i) that there is oncoming traffic and (ii) that it would be bad to be struck by a car?

The skeptics, however, have answers to both charges of inconsistency. They deny any logical inconsistency by arguing that the views they put forward are not their own beliefs. They employ premises in their arguments dialectically only. These premises are either hypothetical or things that their opponents believe. The skeptics themselves need not believe them in order to have a debate. I don't have to believe any of what you say in order to point out that your argument has inconsistencies or logical flaws. In response to the practical inconsistency, Pyrrhonians argue that action requires appearance but not necessarily belief. The oncoming car \textit{appears} dangerous, and so the skeptic moves out of the way quickly. It is not necessary, say the skeptics, to have any beliefs about the matter.

As you can imagine, these debates continued without any clearcut or definitive resolution. For what it's worth, I believe that the skeptics make a solid defense against the first attack, but the second attack remains troubling. Dialectical debate where one or both parties argue from premises that they don't necessarily endorse is well-known and common. And this style of debate was certainly familiar in antiquity, as we can see from the Sophists, Socrates, and Aristotle's work on rhetoric. I'm unsure, however, about the skeptical response to the second attack. That attack is often put as this question: can the skeptics live their skepticism? That is, can they actually live an entire life without belief, based only on appearances? In my view, the jury is still out on this question. Additionally, even if it is possible, such a life would likely look nothing like ordinary life. On this matter, therefore, I suspend judgment.

\subsection*{Early Modern Skepticism\footcites(This section is heavily dependent on two excellent articles:)(and){schmitt1983}{popkin1993}}

After being virtually unknown during the medieval period, Pyrrhonian skepticism was rediscovered and reintroduced to western Europe during the Renaissance. In 1427, an Italian scholar brought Greek manuscripts of Sextus Empiricus from Constantinople to Italy. Between this period and 1520, Italian interest in skepticism grew, and scholars have traced an increase in attention to skepticism in France and other parts of Europe north of Italy later in the sixteenth century. The breakthrough for Pyrrhonian skepticism came in the 1560s when two Frenchmen, Henri Estienne  and Gentian Hervet, translated the works of Sextus into Latin. After this, Pyrrhonian arguments become increasingly common in early modern philosophy.

Rediscovered Pyrrhonian skepticism first appears in religious debates. Several Catholic authors used skepticism to argue against Protestant challenges. They argued roughly as follows:

\begin{enumerate}
    \item Skepticism shows that nothing can be known.
    \item Since nothing can be known, Calvinism can't be known. Faith rather than reason justifies religion.
    \item In the face of an undecideable conflict between faiths, we should stick to tradition.
    \item Our (French) tradition is Catholicism.
    \item So we should stick to Catholicism.
\end{enumerate}

The last few steps in this argument may seem surprising since we might assume that skepticism would be revolutionary and attack the \textit{status quo}. However, Pyrrhonian skepticism has some potentially conservative features. As mentioned earlier, skeptics recommend the suspension of judgment in the face of equally weighted argument or undecideable conflict. When action is required, they often follow appearance. But in a case like this---choosing between two religious views---there is no clear appearance to follow. In such cases, the skeptics recommend following the laws and customs of one's country. In this sense, their views have conservative tendencies.

In any event, skeptical arguments were an important part of early modern religious debates, and these arguments had a great impact on the course of religious belief. Throughout the seventeenth century, Protestants challenged Catholics with skeptical arguments, and Protestants defended themselves against the attacks of other Protestants with skepticism. Richard Popkin has also argued that the atheism of the eighteenth century developed partly as a result of these battles. As he puts it, there was `just a short step' from the skeptical view that faith alone, and not reason, supported religious belief to a skepticism `without any religious faith' \cite[19]{popkin1993}.

In addition to its role in religious arguments, Pyrrhonian skepticism was important in debates about philosophy and education. Early modern education was based on Scholasticism, an amalgam of Catholic theology and Aristotelian philosophy combined with bits of Platonism and Stoicism. After the rediscovery of the works of Sextus, philosophers began to use skeptical arguments to attack Scholasticism, often in support of new scientific discoveries. This leads in neatly to Descartes, whose education at La Flèche followed the Scholastic curriculum.

\subsection*{Descartes's use of skepticism}

Descartes employs skepticism in an innovative and surprising manner. In all the previous cases we've discussed, someone has applied skeptical arguments in order to undermine belief and knowledge. This makes sense since all skeptical modes end, as we saw above, with suspension of judgment. If you can't make a decision---i.e., a judment, then you don't form a belief, and you can't know anything without true beliefs. But Descartes wants to take skeptical arguments to their absolute limit in order to strengthen and support his claims to knowledge. In the first meditation, he doubts as much as he can, and he relies on skeptical arguments to give him reasons to doubt. But, as he says in another work, he does not `doubt only to doubt' (\textbf{AT} VI 29) like the skeptics. Descartes believes that if he unleashes an unrestricted skepticism on all of his beliefs, anything that survives this doubt will be true and certain. By doubting everything he can, Descartes hopes to discover that which can not be doubted. This is his strategy in the first two meditations.
% -]] Skepticism

% [[- About The Text
\section*{About The Text}

The text of this edition has two main sources: volume VII of the twelve-volume edition of Charles Adam and Paul Tannery (1897--1913; my reprint is from 1983) and volume II of the three-volume edition of Ferdinand Alquié (1963--1973; I use the 2010 edition which has been corrected by Denis Moreau). The edition of Adam and Tannery remains the standard edition of Descartes, though Alquié challenges and clarifies their text in several small ways. I also consulted John Cottingham's Latin-English edition of 2013. (Cottingham, in turn, is a lightly altered version of the Adam-Tannery text.) For all intents and purposes, these three editions print the same thing in nearly all cases. Their most obvious differences concern modernization, and Cottingham writes the text Alquié and the Adam-Tannery text have no substantive differences in meaning \cite[xxxii, footnote 5]{cottingham2013}.

I should also explain the marginal numbers in this edition. The numbers in the outer right margin of the text refer to the pages of volume IX of the Adam-Tannery edition. You can use these numbers when referring to other editions or translations in any language. (Almost all editions include the Adam-Tannery numeration somehow. That's part of what it means to be a standard edition of an author like Descartes. It makes cross-edition and cross-linguistic reference much simpler.) The numbers on the left hand margin are line numbers for individual paragraphs. Each meditation is divided into paragraphs in all editions of Descartes. But for ease of reference, I've added numbers to the paragraphs and the lines within each paragraph. The paragraphs follow Adam-Tannery, but the line numbers do not necessarily correspond to any other edition. They are for internal reference only.

Although Descartes wrote in a highly readable neo-Latin that is very classical in most respects, I alter his text in several ways to make it more comfortable for modern readers of Latin. First, as Alquié and Cottingham do, I replace Descartes's use of the long s (`ſ') with the more familiar `s' character. Second, I replace Descartes's use of `j' with `i'. (Alquié and Cottingham leave `j' in the text. This is reasonable in one obvious way: it was what Descartes wrote.) I also replace Descartes's copious use of `\&' with `et'. (Cottingham also uses `et' instead of `\&'; Alquié follows Adam-Tannery and leaves `\&'.) None of these changes are especially significant, but I feel that in combination they make the text much more familiar to students who have learned Latin recently.

I make one significant type of change, about which I still have some hesitation. Descartes uses punctuation, especially the comma, in ways that are likely to confuse students. I therefore repunctuate his text. For the most part, I simply remove commas that might mislead (for example commas before indirect questions). In several cases, however, I also break very, very long sentences into smaller pieces. I work hard to make the meaning of the text more obvious from its punctuation, but if you notice anything that seems wrong or confusing, please let me know.
% -]] About The Text

    % [[- Chapter title
\chapter{Meditatio Prima}
\markboth{Meditationes de prima philosophia}{Meditatio Prima}
% -]] Chapter title

% [[- Meditatio prima

% [[- Introduction
Although her goal is knowledge, the inquirer dedicates the first meditation to doubt. Starting from everyday mistakes, such as misjudging the size of a distant object, she rapidly builds towards radical, hyperbolic doubt as she imagines an all-powerful demon whose only goal is to confuse and mislead her. At the end of the first meditation, the inquirer is so shaken that she fears she may be trapped forever in the unescapable darkness of error.

However, as Descartes himself says elsewhere, the inquirer is not like the sceptics who ``doubt only to doubt and pretend always to be undecided'' (\textbf{CSM} I 125; \textbf{AT} VI 29). The inquirer believes that she can use doubt and error strategically in order to achieve certainty and truth. Although the first meditation contains some of the most vivid writing and imagery in the work, Descartes would have been very sorry for readers to remember it best. The goal of the \textit{Meditations} is to overcome doubt rather than to dwell on it.

One tip about what follows. Before each new section of the inquirer's argument, you will find a paragraph---sometimes more than one---meant to get you thinking. These paragraphs are not exactly paraphrases or summaries. Sometimes I leave out some parts of what the inquirer says, and sometimes I include material from other places. Also the paragraphs act as though \textit{we} were thinking or talking about the issues rather than the anonymous inquirer. But they should help guide you along the main path of the arguments that the inquirer makes, as least as I see it. Good luck and enjoy!

\clearpage
% -]] Introduction

% [[- Background
\begin{center}
    \beginnumbering
    \numberlinefalse
    \pstart
    \textit{De iis quae in dubium revocari possunt}\ledsidenote{17}
    \pend
    \endnumbering
\end{center}

\beginnumbering
\pstart
\textbf{1.} \begin{latin}Animadverti iam ante aliquot annos quam multa, ineunte aetate, falsa pro veris admiserim, et quam dubia sint quaecunque istis postea superextruxi, ac proinde funditus omnia semel in vita esse evertenda, atque a primis fundamentis denuo inchoandum, si quid aliquando firmum et mansurum cupiam in scientiis stabilire; sed ingens opus esse videbatur, eamque aetatem expectabam, quae foret tam matura, ut capessendis disciplinis aptior nulla sequeretur. Quare tamdiu cunctatus sum ut deinceps essem in culpa, si quod temporis superest ad agendum, deliberando consumerem. Opportune igitur hodie mentem curis\at{18} omnibus exsolvi, securum mihi otium procuravi, solus secedo, serio tandem et libere generali huic mearum opinionum eversioni vacabo.\end{latin}
\pend
\endnumbering

\prenotes

\textbf{§1.} Although we humans are rational animals, our reason is not fully developed when we are young. In fact, by the time we begin to consciously evaluate our own beliefs, we inevitably have many false ideas that we picked up earlier in life, often unintentionally and unconsciously. If, however, we want to achieve certain knowledge, how can we allow ourselves to rely on such shaky foundations? Perhaps it is necessary to try something extreme: reject all of our beliefs and accept back only the ones that are certain.

\lem{1 Animadverti}, the main verb of the sentence, takes four objects: two indirect questions and two indirect statements. The two indirect questions are (i) \textit{quam multa\dots admiserim} and (ii) \textit{quam dubia sint}. The two indirect statements are (i) \textit{omnia\dots esse evertenda} and (ii) \textit{inchoandum <esse>}.

\lem{1 ante} usually means ``before,'' but it can also mean ``ago.'' In this use, it is followed by either an ablative or, as here, an accusative indicating how long ago.

\lemc{1 ineunte aetate} we can take this as an ablative absolute that modifies \textit{admiserim} or perhaps Descartes was thinking of Lucretius, who often uses the phrase \textit{ex ineunete aevo}. Either way, Descartes uses the phrase to focus on mistakes made in, or starting from, his youth.

\lemc{2 admiserim\dots sint} although the main verb \textit{animadverti} is secondary sequence (\textit{ante aliquot annos} guarantees this), these two verbs are primary sequence. This may be for vividness, but throughout this sentence the inquirer shifts perspective from her past realization and resolution to her present plans.

\lem{2 admiserim} assumes a particular view of belief. We imagine people reviewing impressions and then either granting them entrance (i.e., believing them) or turning them away (i.e., not believing them). This suggests that belief is voluntary---or at least somehow subject to the will. This is controversial, and the inquirer will discuss it later.

\lemc{2 quam dubia sint quaecunque\dots } the relative clause serves as the subject of \textit{sint}. I.e., the inquirer realized how doubtful are \textit{whatever things he built on top of those foundations}.

\lemc{2 quaecunque istis postea superextruxi} throughout the \textit{Meditations} the inquirer uses metaphors drawn from construction and demolition. She represents knowledge and science as buildings that can be firm and lasting or have weak foundations; they can be torn down and built back up again. (Descartes makes heavy use of this same metaphor, and the related metaphor of the house, in his \textit{Discours de la méthode} as well. See \cite[22]{curtis1984}.)

\lem{2 quaecunque} is an alternative spelling of \textit{quaecumque}, neuter accusative plural.

\lem{2 istis} is probably dative with the compound \textit{superextruxi}, and the pronoun refers back to the \textit{falsa} that the inquirer believed to be true when she was younger. The inquirer continues to use the implicit metaphor of construction.

\lemc{2 superextruxi} this verb does not appear in classical Latin, but \textit{ex(s)truo} does, and compound verbs with \textit{super}- are common in Latin.

\lemc{2 proinde} The inquirer does not attempt to justify the inference from ``I realize that I have made mistakes'' to ``I should overturn all my beliefs.'' She moves quickly at this point, assuming that the reader is willing to go along to see where the argument leads. 

\lemc{3 esse evertenda\dots inchoandum} infinitives of the passive periphrastic. This periphrastic has the force of obligation, necessity, or propriety (\textbf{AG} §194b).

\lem{5 foret} is equivalent to \textit{esset}, imperfect subjunctive in a relative clause of characteristic. Note \textit{\textbf{eam} aetatem}: ``\textit{that (kind of)} age that would.''

\lemc{6 cappesendis disciplinis} the dative depends on \textit{aptior}, and the gerundive phrase expresses purpose.

\lemc{7 quod temporis} the genitive is partitive, and the phrase serves as the object of \textit{consumerem}: ``I would be at fault if I wasted what time remains.'' There is no antecedent for \textit{quod}. This is common for indefinite antecedents (\textbf{AG} §307c). Fully spelled out the thought is \textit{si illud temporis quod ad agendum superest deliberando consumerem}.

\lem{8 curis omnibus} ablative of separation with \textit{exsolvo}.

\lem{9--10 generali huic\dots eversioni} dative with \textit{vacabo}.

\lem{10 mearum opionum} objective genitive with \textit{eversioni}.

% -]] Background

% [[- A shortcut
\clearpage

\beginnumbering
\pstart
\textbf{2.} \begin{latin}Ad hoc autem non erit necesse ut omnes esse falsas ostendam, quod nunquam fortassis assequi possem; sed quia iam ratio persuadet, non minus accurate ab iis quae non plane certa sunt atque indubitata, quam ab aperte falsis assensionem esse cohibendam, satis erit ad omnes reiiciendas, si aliquam rationem dubitandi in unaquaque reperero. Nec ideo etiam singulae erunt percurrendae, quod operis esset infiniti; sed quia, suffossis fundamentis, quidquid iis superaedificatum est sponte collabitur, aggrediar statim ipsa principia, quibus illud omne quod olim credidi nitebatur.\end{latin}
\pend
\endnumbering

\prenotes

\textbf{§2.} It would be an impossibly long task to show that each and every one of our beliefs is uncertain, but perhaps this is not necessary. Since we want absolutely secure knowledge, we should withold assent from anything uncertain just as much as from clear falsehoods. And if we undercut the beliefs at the foundation of our thinking, then any beliefs based on those foundations will collapse as well. Thus, we are quickly on our way to removing all of our beliefs.

\lem{1 hoc} refers to the overturning (\textit{eversio} §1.10) of the inquirer's beliefs from the previous paragraph. The inquirer uses the neuter rather than the feminine perhaps because he has in mind the general idea rather than the specific word.

\lemc{1 ut\dots ostendam} a substantive result clause, the subject of \textit{erit necesse}.

\lemc{1 quod} neuter like \textit{hoc} because it refers back to the entire clause \textit{ut omnes esse falsas ostendam}.

\lemc{2 possem} effectively the apodosis of a present contrary-to-fact conditional sentence, though the protasis is only implied. ``<Even if I were to try,> \textit{I would never be able\dots}.''

\lemc{3 iis} this form is a common alternative for \textit{eis}, the dative and ablative plural of \textit{is, ea, id} for all genders. Similarly \textit{ii} is an alternative for \textit{ei}, the masculine nominative plural. The dative singular \textit{ei} (all genders) has no such alternative. (These alternatives also appear often in compounds of \textit{is, ea, id}. E.g., \textit{iisdem}.)

\lemc{3--4 assensionem esse cohibendam} The technical vocabulary of ``withholding assent'' derives from ancient debates between skeptics and Stoics. In this context ``assent'' means roughly accepting the content of an appearance, forming a judgment. According to Stoic theory, perception and appearances are automatic, but belief and judgment are ``up to us.'' People naturally see, hear, and so on, but they can control what opinions they form about the world. The sun may look very small when people see it from this vast distance, but people can decide what attitude they take to this appearance. They can assent to the opinion that the sun is very small, reject it, or continue investigating. To assent is to judge an opinion to be true; to reject to judge an opinion to be false; to continue investigating is to form no judgement of truth or falsity. The last option was often described as ``suspending'' or ``withholding'' judgment or assent, and that is what the inquirer means here. This means that the inquirer is not telling us that we should consider anything uncertain and doubtful to be false. She is only saying that we should withold judgment. The Stoics argued that people can achieve certain knowledge by being very careful and thoughful about when they give assent. The skeptics used the Stoics own theory against them, arguing that no circumstances were completely certain and thus that, by the Stoics own arguments, people should always withold assent from all judgments.\footnote{Julia Annas provides an excellent introduction to these issues and the debate between Stoics and skeptics in her essay ``Stoic epistemology'' \parencite{annas1990}.} The inquirer invokes this debate because it puts readers immediately into a context where radical doubt is a real possibility. Under normal circumstances, people do not necessarily believe that they should treat everything that is uncertain or in any way doubtful as if it were clearly false. However, in the inquirer's special situation---a single-minded search for truth and absolute certainty---this rule may be justified.

\lemc{4 reiiciendas} a gerundive from \textit{reicio}. Some authors spell with a double \textit{i} to reflect the word's pronunciation: \textit{ray-yicio}.

\lemc{5 quod} like the \textit{quod} in §2.1, the neuter refers back to the action of the previous clause \textit{singulae erunt percurrendae}.

\lem{5--6 operis\dots infiniti} is a predicative genitive (\textbf{AG} §343b): ``which thing (i.e., to run through each opinion one by one) would be an infinite task.''

\lemc{6 suffossis fundamentis} an ablative absolute. In this context, translate it as the protasis of a conditional sentence: ``if the foundations are undermined.''

\lem{6 iis} is dative with \textit{superaedificatum est}.

\lemc{7 quibus} ablative with \textit{nitebatur} (\textbf{AG} §431).

% -]] A shortcut

% [[- Attack the senses first
\clearpage

\beginnumbering
\pstart
\textbf{3.} \begin{latin}Nempe quidquid hactenus ut maxime verum admisi, vel a sensibus, vel per sensus accepi; hos autem interdum fallere deprehendi, ac prudentiae est nunquam illis plane confidere qui nos vel semel deceperunt.\end{latin}
\pend
\endnumbering
\beginnumbering
\pstart
\textbf{4.} \begin{latin}Sed forte, quamvis interdum sensus circa minuta quaedam et remotiora nos fallant, pleraque tamen alia sunt de quibus dubitari plane non potest, quamvis ab iisdem hauriantur: ut iam me hic esse, foco assidere, hyemali toga esse indutum, chartam istam manibus contrectare, et similia. Manus vero has ipsas, totumque hoc corpus meum esse, qua ratione posset negari? nisi me forte comparem nescio quibus insanis,\at{19} quorum cerebella tam contumax vapor ex atra bile labefactat, ut constanter asseverent vel se esse reges, cum sunt pauperrimi, vel purpura indutos, cum sunt nudi, \edtext{vel caput habere fictile}{\Afootnote{The French translation omits this phrase.}}, vel se totos esse cucurbitas, vel ex vitro conflatos; sed amentes sunt isti, nec minus ipse demens viderer, si quod ab iis exemplum ad me transferrem.\end{latin}
\pend
\endnumbering

\prenotes

\textbf{§3.} Our most basic beliefs rely on the senses, and so we should attack the senses first. A brisk argument will do: (i) the senses are sometimes wrong and (ii) we want absolute certainty; thus, (iii) we should never trust the senses.

\lem{1 quidquid} functions as the direct object of both \textit{admisi} and \textit{accepi}: ``whatever I admitted\dots I received.''

\lemc{1--2 vel a sensibus vel per sensus} In another work (\textbf{CB} 3; \textbf{EB} 2), Descartes explains this as a distinction between information received directly from a sense (e.g., seeing a color or shape) and information received indirectly (e.g., learning about things by hearing other people speak).

\lemc{2 hos\dots fallere deprehendi} we can supply \textit{me} or \textit{nos} as the direct object of \textit{fallere}, or we can take it absolutely to mean simply ``deceive, be deceptive.''

\lemc{2 prudentiae} this genitive, which is sometimes called the genitive of characteristic or the genitive of the mark, is a type of possessive genitive (\textbf{AG} §343c). It belongs to wisdom to distrust anything that has been deceptive even once. I.e., it is wise to be suspicious in such a case.

\lem{3 vel} means ``even,'' here, and it modifies \textit{semel}.

\textbf{§4.} But perhaps this is too fast. Maybe the senses only deceive us in certain kinds of cases. Moreover, anyone who doubts \textit{all} of their sensory beliefs appears to have lost their grip on sanity altogether. Perhaps we are moving away from secure truths, rather than towards them. 

\lem{3 ut} means ``for example'' here. This use of \textit{ut} is a development of its adverbial meaning ``as, in such a manner.'' In a sentence like this, the adverb introduces specific cases which exemplify some general point. The indirect statements following \textit{ut} are all examples of the kinds of beliefs that are not easily doubted, although they come from the senses.

\lemc{4--5 Manus\dots posset negari} the two infinitive plus accusative phrases (\textit{Manus has ipsas <meas esse>} and \textit{totum hoc corpus meum esse}) are both subjects of the main verb \textit{posset}. I.e., how could it be denied that these are my hands and that this is my body?

\lemc{4--5 Manus\dots has ipsas, totum\dots hoc corpus} it was already difficult to imagine being wrong about the previous examples, but these last two are meant to be show-stoppers. The repeated \textit{has\dots hoc} implies, as often in Latin, a gesture on the part of the speaker: \textit{these} hands\dots \textit{this} body---the ones right in front of you. In a similar fashion, G.E. Moore famously argued as follows:
\begin{quote}
    I can prove now, for instance, that two human hands exist. How? By holding up my two hands, and saying, as I make a certain gesture with the right hand, ``Here is one hand'', and adding, as I make a certain gesture with the left, ``and here is another'' \parencite[165--166]{baldwin1993}.
\end{quote}
Do you think that an argument like this can settle the philosophical problem of scepticism?

\lem{5 qua ratione} literally means ``by what reasoning,'' but idiomatically it means ``how.''

\lem{5 posset} is a potential subjunctive. The imperfect tense makes this very similar to a contrary-to-fact apodosis, probably under the influence of \textit{comparem} in the next sentence.

\lemc{5--6 nescio quibus} the verb \textit{nescio} can be joined with various words to create indefinite adjectives, pronouns, and adverbs. The verbal part is treated like an indeclinable prefix. Editors disagree over whether to write these forms as one word or two, but there is no difference in meaning either way. Here \textit{nescio quibus} is an indefinite adjective, modifying \textit{insanis}: ``some insane people.''

\lemc{6 quorum cerebella\dots labefactat} Black bile is one of the four humors; the other three are blood, yellow bile, and phlegm. The balance or imbalance of these four was thought to control a person's physical and mental well-being.

\lemc{7 asseverent vel\dots vel\dots} these examples are extreme in two ways. First, the beliefs in question are wildly wrong. Second, they involve mistakes about very basic facts, things that people do not normally get wrong. Hence, this objection poses a serious threat to the inquirer's opening argument against the senses. She now needs to answer the challenge that her position is tantamount to insanity.
% -]] Attack the senses first

% [[- Dreams, part 1
\clearpage

\beginnumbering
\pstart
\textbf{5.} \begin{latin}Praeclare sane, tanquam non sim homo qui soleam noctu dormire, et eadem omnia in somnis pati, vel etiam interdum minus verisimilia, quam quae isti vigilantes. Quam frequenter vero usitata ista, me hic esse, toga vestiri, foco assidere, quies nocturna persuadet, cum tamen positis vestibus iaceo inter strata! Atqui nunc certe vigilantibus oculis intueor hanc chartam, non sopitum est hoc caput quod commoveo, manum istam prudens et sciens extendo et sentio; non tam distincta contingerent dormienti. Quasi scilicet non recorder a similibus etiam cogitationibus me alias in somnis fuisse delusum; quae dum cogito attentius, tam plane video nunquam certis indiciis vigiliam a somno posse distingui, ut obstupescam, et fere hic ipse stupor mihi opinionem somni confirmet.\end{latin}
\pend
\endnumbering

\prenotes

\textbf{§5.} And yet what about our dreams? We hope that we are reasonable and not at all like people who imagine that they are have pumpkins for heads, but do not we regularly dream of things just as outrageous as those beliefs? Pick anything you like, anything you are certain of right now. You have dreamed of exactly the same thing, or things just like it, only to realize the next morning that it was ``only a dream'', right? Pick any belief based on the senses that you like: how can you be sure that you are not merely dreaming it? For example, I see three books on a table in front of me now, but I could easily dream the exact same sight.

\lemc{1 Praeclare sane} these words respond with heavy sarcasm to the argument at the end of the previous section. There is an ellipsis where this phrase's verb would be. Something like ``you argued'' or ``that was argued'' would fill the ellipse, but simply saying ``Oh, brilliant!'' would be a more idiomatic English. An exaggerated eye-roll would help.

\lem{1 Tanquam} is an alternative spelling of \textit{tamquam}.

\lemc{1 Tanquam non sim} the subjunctive with \textit{tamquam} introduces a comparative clause. Usually the comparison is hypothetical or unreal, and so the subjunctive often appears in these clauses. E.g., you read Latin as if you were a native speaker. Note that the translation should sound counter-factual in English, even though the subjunctive in Latin is present tense rather than imperfect. (For a good discussion of the idiom and the tense, see \textbf{AG} §524, especially Note 2.)

\lem{2 quam} picks up both \textit{eadem} and \textit{minus verisimilia}. ``The same things \textit{as} those people,'' and ``things even less likely to be truth \textit{than} what those people.''

\lemc{2--3 isti vigilantes} the pronoun refers back to the \textit{insani} of the previous paragraph; the participle should be taken circumstantially as a temporal clause. I.e., ``when they are awake.'' This phrase lacks a verb. We can easily supply one (e.g., ``experience'') or leave it out in English as well.

\lemc{3--4 Quam\dots strata!} the \textit{quam} is exclamatory, modifying \textit{frequenter}: ``How often\dots !''; the subject of the sentence, \textit{quies nocturna} is a periphrasis for ``a night's sleep,'' \textit{usitata ista} is the (internal) direct object of \textit{persuadet}, and the three accusative and infinitives are indrect statements in apposition to \textit{usitata ista}. I.e., sleep persuades me of these familiar things, namely that \textit{x}, \textit{y}, and \textit{z}.

\lemc{5 vigilantibus oculis} this phrase can be taken as an ablative of means or an ablative absolute.

\lemc{6 prudens et sciens extendo et sentio} Latin idiom often uses an adjective in the nominative where English idiom would use an adverb or adverbial phrase. So here, ``I extend and feel this hand deliberately (\textit{prudens}) and with full knowledge (\textit{sciens}).''

\lemc{7 contingerent dormienti} the subjunctive here is effectively the apodosis of a present contrary-to-fact conditional sentence the participle serves as the protasis. I.e., ``if I were sleeping, I would not have such clear impressions.'' (The participle is dative with a compound verb.) 

\lemc{7 Quasi scilicet} heavily sarcastic, like ``Praeclare sane'' above.

\lemc{7--8 a similibus\dots cogitationibus} the ablative of means does not normally have a preposition. However, it is unnecessary to insist that the \textit{cogitationes} are personified (and thus ablative of personal agent) or that the phrase must be an ablative of cause. (The prepositions \textit{ab}, \textit{de}, or \textit{ex}, and \textit{in} appear with causal ablatives more often than prepositions appear with clear ablatives of means.)

\lemc{8 fuisse delusum} is a perfect passive infinitive, equivalent to \textit{esse delusum}.

\lemc{8 quae dum cogito}: translate as if \textit{et dum haec cogito}. I.e., \textit{quae} is a connective relative (\textbf{AG} §308, note f).

\lem{9 nunquam} is an alternative spelling of \textit{numquam}.

\lemc{9--10 ut obstupescam\dots confirmet} result clauses. Note the \textit{tam} in the preceding clause.

\lemc{9 fere} be careful to distinguish this adverb. Although \textit{fere} is placed early in the clause, very likely for emphasis, it modifies the main verb at the end of the clause ``this amazement itself \textit{nearly confirms} in me the belief that I am sleeping.''

\lemc{10 opinionem somni} the genitive is objective. The phrase ``thought of sleep'' refers to the hypothesis that the meditator may be sleeping. Thus the irony of the final sentence is that the intellectual vertigo (\textit{stupor}) that this argument produces, \textit{itself} (\textit{ipse}) strengthens the worry that all our thoughts may be false dreams.
% -]] Dreams, part 1

% [[- Dreams, part 2
\clearpage

\beginnumbering
\pstart
\textbf{6.} \begin{latin}Age ergo somniemus, nec particularia ista vera sint, nos oculos aperire, caput movere, manus extendere, nec forte etiam nos habere tales manus, nec tale totum corpus tamen profecto fatendum est visa per quietem esse veluti quasdam pictas imagines, quae non nisi ad similitudinem rerum verarum fingi potuerunt; ideoque saltem generalia haec, oculos, caput, manus, totumque corpus, res quasdam non imaginarias, sed veras existere. Nam sane pictores ipsi, ne tum qui\at{20}dem, cum Sirenas et Satyriscos maxime inusitatis formis fingere student, naturas omni ex parte novas iis possunt assignare, sed tantummodo diversorum animalium membra permiscent; vel si forte aliquid excogitent adeo novum, ut nihil omnino ei simile fuerit visum, atque ita plane fictitium sit et falsum, certe tamen ad minimum veri colores esse debent, ex quibus illud componant.\end{latin}
\pend
\endnumbering

\prenotes

\textbf{§6.} In fact, it gets worse. We already agreed that I might be dreaming that I am eating a meal when in fact I am sleeping in bed. But what if there are no meals or beds at all? What if the familiar objects of our world are merely mental illusions? Well, at the very least, our dreams must be based on \textit{something}. Painters must use color, even if they avoid representing any specific thing from the real world. In the same way, even if ordinary objects do not exist, something must underlie our illusions.

\lemc{1 Age} the imperative of \textit{agere} is often used idiomatically to introduce a main verb. Very often the main verb is an imperative, and \textit{age}/\textit{agite} can be translated as idiomatically in English as ``C'mon'' or   ``Alright.'' Here, the main verb is a concessive use of the jussive subjunctive, and \textit{age} might be translated idiomatically as ``Ok'' or ``Fine.'' It introduces the concession that \textit{somniemus} makes.

\lemc{1 somniemus} this subjunctive expresses a command. Idiomatically in Latin such subjunctives often introduce concessions made in the course of an argument (\textbf{AG} §440). In English, we often introduce such concessions with ``Grant that.'' Here, the inquirer concedes that the previous objection may be true, ``Fine, therefore, grant that we are dreaming.''

\lemc{1 particularia ista} these words point forward to the indirect statements that follow. Those specific things are conceded to be false, namely that \textit{x}, \textit{y}, and \textit{z}.

\lemc{3 tamen profecto} at this point, the inquirer draws a line in the sand, so to speak. Even if we grant all the preceding claims, nevertheless the following must be the case.

\lem{3 fatendum est} a passive periphrastic expressing necessity. What must be admitted is the indirect statement that follows: \textit{visa\dots esse}

\lemc{3 visa per quietem} a very elaborate way of saying ``dreams'': the things which appear during sleep.

\lem{4 ad} means roughly ``with an eye towards.''

\lemc{4 rerum verarum} in addition to meaning ``true'' and ``false,'' the words \textit{verus} and \textit{falsus} sometimes mean ``real'' and ``unreal.'' This can be confusing, but many philosophers have felt that truth and reality were in some sense the same thing and thus that falsity and unreality were also the same. (This may be very wrong, but the idea clearly has \textit{some} intuitive appeal.)

\lemc{5--6 res\dots non imaginarias, sed veras} these phrases are predicate with \textit{existere}. I.e., these very general items exist not as imaginary but as true things.

\lemc{6 Nam sane pictores ipsi\dots} the inquirer initially argues that even fantastical creatures of mythology are based in reality insofar as we can see features of everyday animals in them. But then he quickly switches gears and argues that even if painters create something entirely original, as far as visuals go, they still must use color to create their images.

\lemc{6 ne tum quidem} the phrase \textit{ne X quidem} is Latin for ``not even X.'' You must remember to adjust the word order for English idiom. Also, keep in mind that \textit{X} can be more than one word, which can make the Latin idiom harder to see.

\lemc{6--7 Sirenas et Satyriscos} two types of mythological creatures. Sirens were bird women whose singing bewitched sailors and led them to their deaths. Satyrs were part goat and part human. These examples support the inquirer's claim that many imaginary creatures are just combinations of actual things.

\lem{7 omni ex parte} modifies \textit{novas} and means ``entirely'' or ``in every respect'' (\textbf{OLD} 14d).

\lemc{9--10 ut\dots fuerit visum\dots sit} subjunctives in a result clause. The tense of \textit{fuerit sit} is surprising. It probably should be \textit{sit visum}, a perfect subjunctive, rather than a pluperfect. However, perhaps the inquirer uses the pluperfect to indicate a kind of contrary-to-fact feel in this clause. I.e., ``nothing similar could ever have been seen'' rather than ``nothing similar was ever seen.''

\lemc{10 ad minimum} just like the English idiom ``at least.''

\lemc{10 veri colores} despite the word order, \textit{colores} is subject, and \textit{veri} is predicate. I.e., the colors should be real.

\lemc{11 ex quibus illud componant} the relative clause describes the \textit{colores}. I.e., ``the colors from which.'' \textit{illud} refers back to the entirely new thing (\textit{aliquid\dots novum}) that the inquirer concedes that someone might paint. Finally, \textit{componant} is subjunctive because this entire sentence refers to a hypothesis: even if they were to imagine such an entirely new thing, the colors from which \textit{they would compose} it would have to be real. (The subjects of \textit{componant} are the hypothetical creative painters.)
% -]] Dreams, part 2

% [[- More basic truths
\clearpage

\beginnumbering
\pstart
\setline{11}
\textbf{6. (cont.)} \begin{latin}Nec dispari ratione, quamvis etiam generalia haec, oculi, caput, manus, et similia, imaginaria esse possent, necessario tamen saltem alia quaedam adhuc magis simplicia et universalia vera esse fatendum est, ex quibus tanquam coloribus veris omnes istae, seu verae seu falsae, quae in cogitatione nostra sunt, rerum imagines effinguntur.\end{latin}
\pend
\endnumbering
\beginnumbering
\pstart
\textbf{7.} \begin{latin}Cuius generis esse videntur natura corporea in communi, eiusque extensio; item figura rerum extensarum; item quantitas, sive earumdem magnitudo et numerus; item locus in quo existant, tempusque per quod durent, et similia.\end{latin}
\pend
\endnumbering

\prenotes

\lemc{11 Nec dispari ratione\dots effinguntur} this sentence is rather complex, so it will help to look at its overall structure before reading it more closely. \textit{quamvis\dots possent} is a concession: ``although these general things would be false (on the hypothesis under consideration).'' The main clause follows. \textit{necessario\dots vera esse fatendum est} means ``it must necessarily be admitted that other, more basic, things are real.'' The rest of the sentence describes these \textit{alia} further: \textit{ex quibus\dots imagines effinguntur} means ``from which all those images that are in our thinking---whether true or false---are formed.''

\lemc{11 Nec dispari ratione} a litotes that modifies the main verb \textit{fatendum est}. I.e., ``and by exactly the same argument.''

\lemc{12--13 imaginaria esse\dots vera esse} in both these cases, the adjectives are predicative after \textit{esse}. These general things would be imaginary, but it must be admitted that even more basic things must be real.

\textbf{§7.} What might these more basic underlying items be? Well, as a start, the following: corporeality, i.e., the essential physicality of objects; shape; quantity and number; place; and time. Even if all the people---even if people in general---in our thoughts (our dreams?) do not exist, they all have physical natures. These physical natures have shapes, they have sizes, there are specific numbers of them, they exist in different locations, and they exist over time.

\lemc{§7.1 Cuius generis esse videntur natura corporea\dots} this sentence is hard to get into English exactly as it is written. Strictly speaking \textit{Cuius generis} is the predicate following \textit{esse videntur}, and the phrases that follow are all the subject of \textit{videntur}. To make matters even more complicated, \textit{Cuius} is a connective relative. Perhaps something like this, ``The following things seem to be of this type: bodily nature\dots.''

\lemc{1 natura corporea\dots eiusque extensio} the \textit{que} joins together very closely \textit{natura corporea} and \textit{eius extensio}. The inquirer will argue later that the essential nature of physical objects is that they are extended things in space. Thus, we could argue that the \textit{que} is explanatory and means ``i.e.'' or ``that is.'' The pronoun \textit{eius} stands for \textit{corporis}. The noun \textit{corpus} has not been explicitly used in this sentence, but it is easily supplied from the adjective \textit{corporea}.

\lemc{2--3 quantitas, sive\dots magnitudo et numerus} the \textit{sive} clause explains \textit{quantitas}. Quantity has two important aspects: size and number.

\lemc{3 et similia} it is unclear to me whether the inquirer genuinely has other similar basic features of reality in mind or if this is simply an imprecise phrase.
% -]] More basic truths

% [[- Simple versus composite disciplines
\clearpage

\beginnumbering
\pstart
\textbf{8.} \begin{latin}Quapropter ex his forsan non male concludemus Physicam, Astronomiam, Medicinam, disciplinasque alias omnes, quae a rerum compositarum consideratione dependent, dubias quidem esse; atqui Arithmeticam, Geometriam, aliasque eiusmodi, quae nonnisi de simplicissimis et maxime generalibus rebus tractant, atque utrum eae sint in rerum natura necne, parum curant, aliquid certi atque indubitati continere. Nam sive vigilem, sive dormiam, duo et tria simul iuncta sunt quinque, quadratumque non plura habet latera quam quatuor; nec fieri posse videtur ut tam perspicuae veritates in suspicionem falsitatis incurrant.\end{latin}
\pend
\endnumbering

\prenotes

\textbf{§8.} So if we want to find absolutely certain truth, perhaps we should stick to these basic realities. Fields such as arithmetic and geometry deal in such facts. Perhaps they are safe. However, we will need to abandon many disciplines which seem fruitful, but rely on claims about higher-level objects---things we can no longer be certain about. The disciplines we must let go include astronomy and medicine.

\lem{1 ex his} refers back to the considerations in the previous paragraphs.

\lemc{1 non male} perhaps a litotes, suggesting that what follows is a very reasonable conclusion. But perhaps the inquirer deliberately downplays the strength of this argument.

\lemc{1 concludemus} a potential subjunctive with \textit{forsan}. (The \textit{non} helps to see this since an independent subjuctive expressing a command or a wish would use \textit{ne} as its negative.)

\lemc{1--2 Physicam, Astronomiam, Medicinam} we get a better sense of why the inquirer chooses these sciences from the contrast with arithmetic and geometry in the next part of the sentence. Physics, astronomy, and medicine make claims about the world (\textit{utrum eae sint in rerum natura necne}) while arithmetic and geometry are, in a significant sense, purely theoretical. (Note that early modern physics was far more practical than contemporary physics.)

\lem{4 eiusmodi} is another way of writing \textit{eius modi}, a genitive of description, ``of this sort.''

\lemc{4--5 quae\dots curant} the structure here is \textit{quae tractant atque parum curant utrum eae sint necne}. That is, \textit{tractant} and \textit{curant} are parallel verbs in the relative clause introduced by \textit{quae}, and \textit{utrum\dots} is an indirect question dependent on \textit{curant}.

\lemc{4 nonnisi} literally means ``not unless.'' We can translate it more idiomatically here as ``only.'' Latin is much more comfortable than English using such doubly negative phrases. For example, ``nonnulli,'' which is literally ``not no people,'' is often used to mean ``some people,'' ``several people,'' or ``a number of people.''

\lemc{5 utrum eae sint in rerum natura necne} first, \textit{utrum\dots necne} introduces a double indirect question: ``whether\dots or not''; second, \textit{eae} refers to the very simple and general things (\textit{rebus}) from the previous clause; third, \textit{sint} is existential here. Arithmetic and geometry do not care very much whether or not their subjects \textit{exist} in the world (\textit{rerum natura}). That is, we can study arithmetic and geometry in an abstract and theoretical manner without applying them to the real world, but this is not possible, according to the meditator, in the case of sciences such as physics and medicines.

\lemc{5--6 aliquid certi atque indubitati} the genitives are partitive, but in English we would treat them simply as adjectives modifying \textit{aliquid}. That is, ``something certain and without doubt.''

\lemc{6--7 sive vigilem sive dormiam\dots sunt\dots habet} the mood of the verbs reinforces the argument. The inquirer cannot say whether she is awake or sleeping, so she uses subjunctives (\textit{vigilem} and \textit{dormiam}) to present the two open possibilities. On the other hand, no matter whether she is asleep or awake, the facts about math and geometry are true. (At least, that is what she is arguing at this point. She will turn around and attack this position soon enough.) The inquirer uses indicatives (\textit{sunt} and \textit{habet}) to emphasize the truth of those basic facts.

\lemc{7 quadratumque non plura habet latera quam quatuor} there is a harmless imprecision here. The inquirer means to say that a square has precisely four sides not merely that it has no more than four---as if it might possibly have \textit{less} than four.

\lem{7 quatuor} is an alternative spelling of \textit{quattuor}

\lemc{8 ut tam\dots incurrant} a substantive clause of result and the subject of \textit{videtur}. Compare this English version: ``It seems impossible that such clear truths come into any suspicion of falsehood.'' What seems impossible is the ``that'' clause. While English uses ``it'' as a dummy subject, the true subject remains the noun clause introduced by ``that.'' (See \cite[218]{huddleston2005}.)
% -]] Simple versus composite disciplines

% [[- A deceptive god?
\clearpage

\beginnumbering
\pstart
\textbf{9.} \begin{latin}\at{21}Verumtamen infixa quaedam est meae menti vetus opinio, Deum esse qui potest omnia, et a quo talis, qualis existo, sum creatus. Unde autem scio illum non fecisse ut nulla plane sit terra, nullum coelum, nulla res extensa, nulla figura, nulla magnitudo, nullus locus, et tamen haec omnia non aliter quam nunc mihi videantur existere? Imo etiam, quemadmodum iudico interdum alios errare circa ea quae se perfectissime scire arbitrantur, ita ego ut fallar quoties duo et tria simul addo, vel numero quadrati latera, vel si quid aliud facilius fingi potest? At forte noluit Deus ita me decipi, dicitur enim summe bonus; sed si hoc eius bonitati repugnaret, talem me creasse ut semper fallar, ab eadem etiam videretur esse alienum permittere ut interdum fallar; quod ultimum tamen non potest dici.\end{latin}
\pend
\endnumbering

\prenotes

\textbf{§9.} It gets still worse. Up until now we have found three possible sources of secure truths: (i) categories of reality such as corporeality, quantity, space, and time; (ii) very basic disciplines such as mathematics; and (iii) very basic facts, such as that two plus two equals four or that squares have four sides. And yet: an all-powerful but deceptive god might deceive us even in these there areas.

It is important to be clear about what we are arguing here. We do not need to prove that there is such a deceptive god, nor that this is likely. All this argument requires is that the existence of such a god is logically possible and that such a god could mislead us even when we consider truths from the three categories above. If this is the case, we are back to square one, left with no candidates for certainty at all.

\lemc{1 meae menti} dative with \textit{infixa}.

\lemc{1 Deum esse} an indirect statement dependent on \textit{vetus opinio}.

\lemc{2 potest omnia} supply \textit{facere} with \textit{omnia} as its direct object.

\lemc{2 talis qualis existo sum creatus} in English it is idiomatic to express only one half of the correlative pair \textit{talis qualis}, but in Latin it is idiomatic to express both halves. I.e., god made Descarts ``such as he is'' (in English), but ``of that sort which sort he is'' (in Latin). This doubling of correlatives is the norm in Latin.

\lemc{3 fecisse ut} idiomatically means ``to have brought it about that, to have acted (such) that.'' The \textit{ut} clause is a substantive clause of result and the direct object of \textit{fecisse}.

\lem{3 coelum} is an alternative spelling of \textit{caelum}.

\lemc{5 interdum} take this with \textit{errare} rather than \textit{iudico}. The inquirer argues that people sometimes make mistakes; she is not concerned here with how often she thinks that they make mistakes.

\lemc{6 ita ego ut fallar} after \textit{ita}, which is correlative with \textit{quemadmodum} above, we must supply something like \textit{unde scio Deum non fecisse} from the previous sentence. Here is a paraphrase of the argument: even worse than the previous possibility (\textit{imo etiam}), how do I know that god does not deceive me every time (\textit{quoties}) I think about simple addition or geometry, just like cases where people make mistakes about things that they believe they understand perfectly.

\lemc{7 At forte\dots} The inquirer imagines the objection that a supremely good god would not always deceive her. In reply she argues as follows:

\begin{enumerate}
    \item Perhaps a supremely good god would not wish the inquirer always to be mistaken. (It would clash with god's goodness to want such a thing.)
    \item But the same thought suggests that a supremely good god should not allow her sometimes to be mistaken.
    \item However, it cannot be denied that the inquirer is sometimes mistaken.
\end{enumerate}

This is not a complete reply to the objection: how might you fill in the steps of the argument?

\lemc{8 dicitur\dots summe bonus} supply \textit{esse} and take \textit{summe bonus} as predicate. I.e., ``he is said (to be) supremely good.''

\lemc{8 hoc} anticipates the indirect statement that follows. Translate ``this, namely that.''

\lemc{ab} Latin idiom says ``foreign from'' while English says ``foreign to.''

\lemc{9 eadem} refers back to \textit{bonitati} in line 8.

\lemc{10 quod ultimum} the wording here is somewhat imprecise. What is the ``final thing'' that cannot be denied? Probably this: it cannot be said that god never permits the inquirer to be mistaken. (This essentially follows the earliest French translation: ``I cannot doubt that he does allow this.'')
% -]] A deceptive god?

% [[- However I came to be, I make mistakes
\clearpage

\beginnumbering
\pstart
\textbf{10.} \begin{latin}Essent vero fortasse nonnulli qui tam potentem aliquem Deum mallent negare, quam res alias omnes credere esse incertas. Sed iis non repugnemus, totumque hoc de Deo demus esse fictitium; at seu fato, seu casu, seu continuata rerum serie, seu quovis alio modo me ad id quod sum pervenisse supponant; quoniam falli et errare imperfectio quaedam esse videtur, quo minus potentem originis meae authorem assignabunt, eo probabilius erit me tam imperfectum esse ut semper fallar. Quibus sane argumentis non habeo quod respondeam, sed tandem cogor fateri nihil esse ex iis quae olim vera putabam, de quo non liceat dubitare, idque non per inconsiderantiam vel levitatem, sed propter validas et meditatas rationes; ideoque etiam ab iisdem, non minus quam ab aperte falsis,\at{22} accurate deinceps assensionem esse cohibendam, si quid certi velim invenire.\end{latin}
\pend
\endnumbering

\prenotes

\textbf{§10.} Actually our argument does not even require a deceiving god. No matter how we were created, we makes mistakes. Therefore even our most basic beliefs can be doubted. The previous argument imagined an all-powerful god who created us and made us subject to error. But the argument will not collapse if someone rejects this premise. We can respond that if a person, force, or entity less powerful than god created us, it is even more likely that we are always mistaken. Hence, the conclusion of the previous paragraph holds even if we deny the existence of an all-powerful creator god.

\lemc{1 Essent} like the apodosis of a present contrary-to-fact conditional sentence, with an implied protasis. ``<If they heard the previous argument>, perhaps there would be some people who.'' See §2.2.

\lemc{1 mallent negare quam\dots credere} the verb \textit{malo, malle} often leads to an explicit comparison with \textit{quam}. You can translate as ``prefer to x <rather> than to y'' or ``rather x than y.''

\lemc{2 iis} dative with \textit{repugnemus}. Like the simple verb \textit{pugnare}, \textit{repugnare} tends to be intransitive. The person you resist or fight against goes into the dative or is placed into a prepositional phrase.

\lemc{2--3 non repugnemus\dots demus} translators into French and English universally take these two verbs as hortatory. Although we would expect \textit{ne}, there are examples of \textit{non} with commands (\textbf{AG} §439, note 3).

\lemc{demus} acts like a verb of speech or thought here and takes an indirect statement since it means ``grant'' or ``concede.''

\lemc{3--6 at seu\dots fallar} not an obvious argument. (i) Making mistakes is a sign of imperfection. (ii) The meditator makes mistakes. Therefore (iii) the weaker an origin people attribute to her the more likely that she is always wrong. The argument relies on an unstated and unsupported assumption that a more ``powerful'' origin produces (or can produce) a more perfect creation.

\lemc{4 me ad id quod sum pervenisse} ``to have reached (to) that thing which I am'' means ``to have become what I am.''

\lemc{5--6 quo minus\dots eo probabilius} again we see find both halves of the correlation.  When you read Latin, you should note how the parts of the expression work together: ``\textit{by how much} less powerful a creator\dots \textit{by that much} more probable.'' But when you translate into English, you should simplify the phrasing.

\lemc{7 Quibus\dots argumentis} is dative indirect object with \textit{respondeam}. The relative ``quibus'' is a connective relative; translate as if the sentence began \textit{Et his} instead of \textit{Quibus}.

\lemc{7 tandem} The inquirer finally reaches the goal she set out to achieve in the first paragraph. She believes that she has shown that she has reason to doubt all of her former beliefs.

\lemc{nihil\dots ex iis} expresses a quasi-partitive idea. None of the things that the inquirer previously believed to be true was beyond the reach of doubt.

\lemc{8 id} there is no precise syntax for this pronoun in the sentence, but it is perfectly clear what the inquirer means. She has just said that that he must admit that there is nothing that he used to believe which is not open to doubt. She continues ``and this (\textit{id}) not because of lack of reflection or triviality but on account of strong and well thought-out reasons.'' We could supply ``I have concluded'' before ``this,'' but neither Latin nor English require the addition.

\lemc{8--9 non per inconsiderantiam vel levitatem sed propter validas et meditatas rationes} these words echo \textit{Discourse on the Method}, which Descartes published four years earlier. In a context similar to this paragraph, Descartes says that he strives to examine ideas ``not by weak conjectures but by clear and certain reasonings'' (\textbf{CSM} I 125; \textbf{AT} VI 29). This point appears in both works since the arguments in both works are liable to shock readers.

\lemc{10 iisdem} refers back to \textit{iis quae olim vera putabam}. I.e., everything that the inquirer formerly thought was true.

\lemc{10 non minus quam ab aperte falsis} this part of inquirer's argument is rightfully controversial. However, it is essential to remember that he does not recommend this attitude for everyday life but only as part of the pursuit of certain truth. See the end of this sentence where he adds \textit{si quid certi velim invenire}.

\lemc{10--11 assensionem esse cohibendam} depends on \textit{cogor fateri} and is parallel to \textit{nihil esse\dots dubitare}. The inquirer is forced to admit two things: (i) nothing that he formerly believed is beyond doubt and (ii) we should withhold assent from doubtful ideas no less than we reject clearly false ideas.

% -]] However I came to be, I make mistakes

% [[- I must counteract the inertia towards belief
\clearpage

\beginnumbering
\pstart
\textbf{11.} \begin{latin}Sed nondum sufficit haec advertisse, curandum est ut recorder; assidue enim recurrunt consuetae opiniones, occupantque credulitatem meam tanquam longo usu et familiaritatis iure sibi devinctam, fere etiam me invito; nec unquam iis assentiri et confidere desuescam, quamdiu tales esse supponam quales sunt revera, nempe aliquo quidem modo dubias, ut iam iam ostensum est, sed nihilominus valde probabiles, et quas multo magis rationi consentaneum sit credere quam negare. Quapropter, ut opinor, non male agam, si, voluntate plane in contrarium versa, me ipsum fallam, illasque aliquandiu omnino falsas imaginariasque esse fingam, donec tandem, velut aequatis utrimque praeiudiciorum ponderibus, nulla amplius prava consuetudo iudicium meum a recta rerum perceptione detorqueat. Etenim scio nihil inde periculi vel erroris interim sequuturum, et me plus aequo diffidentiae indulgere non posse, quandoquidem nunc non rebus agendis, sed cognoscendis tantum incumbo.\end{latin}
\pend
\endnumbering

\prenotes

\textbf{§11.} We find it easier to believe than to doubt. As a result, we need a counter-weight against the natural bias towards belief. Instead of thinking of the relevant ideas as merely uncertain, we can lie to ourselves and pretend that they are wholly false. No harm will come of this since we are concerned only with knowledge and not with action. That is, we are in no immediate danger even if we deceive ourselves in this way.

\lemc{2 ut recorder} a substantive clause of result and the object of \textit{curandum est}.

\lemc{2 recurrunt\dots occupantque} a military metaphor. Habitual beliefs rush back and seize hold of belief, like soldiers who retreat and then attack again.

\lem{2 tanquam} is an alternative spelling of \textit{tamquam}. 

\lemc{3 longo usu et familiaritatis iure sibi devinctam} the analogy shifts from the battlefield to the law court. A tendency to believe (\textit{credulitatem}) is owed (\textit{devinctam}) to those opinions because of ``long use and the law of familiarity.''

\lemc{3 sibi devinctam} the pronoun \textit{sibi} refers back to \textit{opiniones}, and \textit{devinctam} agrees with \textit{credulitatem}. Latin uses the reflextive pronoun since the reference is to the subject of the main verb. However, English would more naturally use ``to them'' not ``to themselves.'' (There are a number of situations where Latin uses the third person reflexive pronoun, but English would use a non-reflexive one. The most common is indirect statement. \textit{Caesar se vicisse dixit} is simply ``Caesar said that he had won.'' To say ``Caesar said that he himself had won'' in Latin, you would need to add a form of \textit{ipse, ipsa, ipsum}: ``Caesar se ipsum vicisse dixit.'' See note on \textit{me ipsum} below.)

\lemc{3 me invito} an ablative absolute, equivalent in meaning to an adverbial phrase like ``against my will'' or ``despite what I want.''

\lem{3 unquam} is an alternative spelling of \textit{umquam}.

\lemc{4--5 revera} a common phrase, sometimes written as two words \textit{re vera}. The phrase is an ablative of manner. Translate as ``truly'' or ``actually,'' but compare the English ``in actual fact'' for the virtual redundancy.

\lemc{5 nempe\dots quidem} work together for emphasis. \textit{nempe} modifies the entire clause, while \textit{quidem} goes closely with \textit{aliquo modo}. Notice them both in the Latin, but if translating you only need one ``certainly'' or ``surely.''

\lemc{5 iam iam} emphatic ``just this very moment,'' ``only a moment ago,'' or the like.

\lemc{6 quas} functions as the object of both \textit{credere} and \textit{negare}. It goes into the accusative, the default object case, even though the closer verb \textit{credere} is intransitive and takes a dative as object. (In addition, a dative could have confused readers since \textit{consentaneum sit} leads to the dative \textit{rationi}.)

\lemc{7--8 voluntate\dots versa} ablative absolute.

\lemc{8 me ipsum} \textit{me} is already reflexive here; hence \textit{ipsum} is emphatic. In this case the inquirer will fool herself \textit{of her own accord} rather than let habit lull her back into complancency.

\lem{8 aliquandiu} is an alternative spelling of \textit{aliquamdiu}. 

\lemc{9 velut\dots ponderibus} an analogy form balancing items in a scale. The inquirer will lean too much in one direction intentionally in order to counter-balance a natural human flaw, the tendency to believe what we have always believed.

\lemc{9 praeiudiciorum} in his \textit{Discourse on the Method} (\textbf{CSM} I 120; \textbf{AT} VI 18), Descartes warns against two key cognitive dangers: hasty judgment (\textit{la précipitation}) and prejudice (\textit{la prévention}).

\lemc{10 prava\dots recta} these two adjectives are natural opposites already in classical Latin: what is \textit{rectus} is straight, proper, good, correct while what is \textit{pravus} is crooked, bent, twisted, wrong.  In the {Discourse} Descartes frequently employs metaphors drawn from roads and travel where one's aim is to go the right way and not deviate from a proper path in order to reach one's goal.

\lem{11 sequuturum} is an alternative spelling of \textit{secuturum}. \textit{scio} introduces indirect statement, so supply \textit{esse} to form the future active infinitive.

\lemc{11-13 me\dots incumbo} this is essential for the inquirer. It is \textit{impossible} to be too cautious when pursuing truth and knowledge.
% -]] I must counteract the inertia towards belief

% [[- The evil demon
\clearpage

\beginnumbering
\pstart
\textbf{12.} \begin{latin}Supponam igitur non optimum Deum, fontem veritatis, sed genium aliquem malignum, eundemque summe potentem et callidum, omnem suam industriam in eo posuisse, ut me falleret: putabo coelum, aërem, terram, colores, figuras, sonos, cunctaque externa nihil aliud esse quam ludificationes somniorum, quibus insidias credulitati meae tetendit: considerabo\at{23} meipsum tanquam manus non habentem, non oculos, non carnem, non sanguinem, non aliquem sensum, sed haec omnia me habere falso opinantem: manebo obstinate in hac meditatione defixus, atque ita, siquidem non in potestate mea sit aliquid veri cognoscere, at certe hoc quod in me est, ne falsis assentiar, nec mihi quidquam iste deceptor, quantumvis potens, quantumvis callidus, possit imponere, obfirmata mente cavebo. Sed laboriosum est hoc institutum, et desidia quaedam ad consuetudinem vitae me reducit. Nec aliter quam captivus, qui forte imaginaria libertate fruebatur in somnis, cum postea suspicari incipit se dormire, timet excitari, blandisque illusionibus lente connivet: \vvar{sic}{\nth{2}}{hic}{\nth{1}} sponte relabor in veteres opiniones, vereorque expergisci, ne placidae quieti laboriosa vigilia succedens, non in aliqua luce, sed inter inextricabiles iam motarum difficultatum tenebras, in posterum sit degenda.\end{latin}
\pend
\endnumbering

\prenotes

\textbf{§12.} In order to counteract the force of habit, we can employ a thought experiment. We will imagine that an all-powerful evil demon exists and that the demon uses all of its efforts to deceive us. If this is so, we had better remain far more wary of ordinary belief than we would normally. This thought experiment solves an additional problem for us. We can return to the questions concerning an all-powerful god from §9 without risking censure or anger from religious readers. (For a contemporary equivalent to the evil demon, compare the thought experiment that we may be merely ``brains in a vat'' \parencite[5]{harman1973}.)

\lemc{1 Supponam} a future indicative rather than a hortatory present subjunctive, as the subsequent verbs \textit{putabo}, \textit{considerabot}, \textit{manebo}, and \textit{cavebo} show.

\lemc{3 eo} anticipates the following substantive \textit{ut} clause. The evil demon has worked so hard ``on this: that he deceive me.''

\lemc{4 quibus} ablative of means. All external things are merely dream nonsense ``by means of which'' the demon has set a trap for the inquirer.

\lemc{5 insidias\dots tetendit} a metaphor from hunting. The demon has laid out a snare to catch the inquirer's gullibility.

\lem{5 meipsum} is equivalent to \textit{me ipsum}.

\lemc{7 haec omnia me habere} an indirect statement dependent on \textit{opinantem}. The main structure is \textit{considerabo me ipsum tanquam falso opinantem}: ``I will consider myself like (someone) falsely believing \textit{that I have all these things}.'' Thus, \textit{me} is the subject and \textit{haec omnia} is the direct object of \textit{habere}.

\lemc{8 siquidem\dots cognoscere} at this point in the work, the meditator may or may not discover any certain truths.

\lemc{8--10 at\dots cavebo} the structure of this last sentence is complicated and difficult. The main verb is \textit{cavebo} and its initial object is \textit{hoc} which points forward to the relative clause \textit{quod in me est}. That gives us ``But at least I will guard against this which is in my power.'' The two clauses \textit{ne\dots assentiar} and \textit{nec\dots possit imponere} are substantive clauses in apposition to \textit{hoc}, and in sense they are the true direct objects of \textit{cavebo}. ``I will certainly take guard that I not assent\dots and that\dots is not possible.'' \textit{nec} before the second clause stands for \textit{et ne} or \textit{neve}; normally \textit{nec} replaces \textit{et non}.

\lemc{9--10 nec\dots possit imponere} as we say in English, the demon will ``not be able to put anything over on'' the inquirer. In Latin ``to impose something on someone'' (\textit{alicui aliquid imponere}) is an idiom meaning ``to deceive, to cheat, to trick.''

\lemc{9--10 quantumvis potens, quantumvis callidus} modifies the \textit{deceptor}. The deceiver will fail ``however powerful, however clever (he may be).''

\lemc{10 obfirmata mente} it requires significant effort to avoid falling into error.

\lemc{12 Nec aliter quam} litotes for ``exactly like.''

\lemc{12--14 captivus\dots connivet} The inquirer, who has worried that her whole life might be a dream, now compares herself to a slave who dreams of freedom and resists waking up. The inquirer alludes to something familiar: a dreamer can simultaneously start to wake up and try to continue dreaming for a while longer. Both options, however, look bad to the inquirer. If she remains asleep, she knowingly lets herself remain in error. But if she wakes up, she runs the risk of never finding any certain truth and spending her whole life in the darkness of doubt.

\lem{13 lente} is somewhat oxymoronic. The slave \textit{tenaciously} turns a blind eye to the falsity of his pleasant dreams. He stubbornly does something rather passive.

\lemc{14 vereor\dots ne} remember that after a word of fearing, \textit{ne} introduces a clause that the writer or speaker fears \textit{will} happen. All the negative force is lost. (Clauses expressing what the speaker or writer fears \textit{will not} happen are introduced by \textit{ut} or sometimes \textit{ne\dots non}.

\lemc{15--16 luce\dots tenebras} the imagery of a future lived in light or shadow suits a work that deliberately invokes the style of religious meditations. Note also that this first meditation ends on a rather ominous cliffhanger.
% -]] The evil demon

% -]] Meditatio prima

    % [[- Chapter title
\chapter{Meditatio Secunda}
% -]] Chapter title

% [[- Meditatio secunda

% [[- Introduction
In the second meditation, Descartes finds a certain truth: I exist as long as I am thinking. (Descartes doesn't use the phrase \textit{cogito ergo sum} in the meditations, though he does elsewhere.) As he unravels the significance of this truth, Descartes examines the nature of thought and what it means to be a creature that thinks. He argues that the essence of thought is perhaps different from what we expect and that our human nature is also different from what everyday beliefs suggest.

Descartes also answers the objection that physical things are better and more easily known than ourselves. To many people it may seem that our knowledge of everyday items---such as tables, chairs, rocks, trees, and so forth---is more immediate, clearer, and more important than knowledge of the self. Descartes offers a two-part reply to this. First, he argues that it is more difficult to know ordinary physical objects than we think. Second, he argues that the knowledge of such objects that we have comes from thought and not sensation. Thus, he concludes our nature as thinking things is more essential to us and easier for us to grasp than common sense suggests.

\clearpage
% -]] Introduction

% [[- An Archimedean point
\clearpage
\begin{center}
    \beginnumbering
    \numberlinefalse
    \pstart
    \textit{De natura mentis humanae: quod ipsa sit notior quam corpus}
    \pend
    \endnumbering
\end{center}

\beginnumbering
\pstart
\begin{latin}
    \textenglish{\textbf{1.}} In tantas dubitationes hesterna meditatione coniectus sum ut nequeam amplius earum oblivisci nec videam tamen qua ratione solvendae sint. Sed, tanquam in profundum gurgitem ex improviso delapsus, ita turbatus sum ut nec possim in imo pedem figere nec enatare ad summum. Enitar tamen et tentabo rursus eandem viam quam heri fueram ingressus removendo scilicet illud omne quod vel minimum dubitationis admittit nihilo secius quam si omnino falsum esse comperissem; pergamque porro donec aliquid certi vel, si nihil aliud, saltem hoc ipsum pro certo---nihil esse certi---cognoscam. Nihil nisi punctum petebat Archimedes quod esset firmum et immobile ut integram terram loco dimoveret. Magna quoque speranda sunt si vel minimum quid invenero quod certum sit et inconcussum.
\end{latin}
\pend
\endnumbering

\beginnumbering
\pstart
\begin{latin}
    \textenglish{\textbf{2.}} Suppono igitur omnia quae video falsa esse. Credo nihil unquam extitisse eorum quae mendax memoria repraesentat. Nullos plane habeo sensus. Corpus, figura, extensio, motus, locusque sunt chimerae. Quid igitur erit verum? Fortassis hoc unum: nihil esse certi.
\end{latin}
\pend
\endnumbering

\prenotes

\textbf{§1.} Yesterday was hard. We wanted to doubt as much as possible, but we may have succeeded far too well. We may end up proving that \textit{everything} can be doubted, in which case the only certain thing will be that nothing is certain. Still, we should keep looking: if we can find even one certain truth, we can use it as a foundation for so much more knowledge.

\lemc{1 hesterna} although Descartes expressed hope that readers would spend as long as needed---weeks or months, if necessary---on the first meditation, the meditations as a whole follow the dramatic convention that each meditation is one day for the narrator.

\lemc{2 qua ratione} not simply an alternative for \textit{quo modo}. The trouble that Descartes has caused himself must be solved by reason, not in any other way.

\lemc{2--3 Sed\dots summum} We might try to connect each part of this analogy to a different outcome. If Descartes discovers that there is nothing certain, he will have reached the bottom of the whirlpool of doubt. If, on the other hand, he discovers a certainty, he escapes doubt and swims to the surface. However, even if we are not so precise, the analogy conveys how utterly overpowered Descartes feels during his exericse in radical doubt.

\lemc{3 turbatus sum} a form like this is ambiguous. The default interpretation is that it is a perfect passive: `I was troubled'. In this case, we take the participle and the form of \textit{esse} together as one compound verb form. But in some cases the words should be taken separately: a form of \textit{esse} and a participle serving as an attributive adjective. On this interpretation, the two words here are in the present tense: `I am troubled'. Many perfect passive participles show this ambiguity between their use in compound passive forms and their use as adjectives. The second reading makes better sense in this passage: take \textit{turbatus} as an adjective rather than as half of a compound verb.

\lemc{3--4 nec\dots nec} despite the position of the first \textit{nec}, these two join the infinitives \textit{figere} and \textit{enatare} rather than two main verbs.

\lemc{5 fueram ingressus} = \textit{eram ingressus}, the pluperfect tense. The pluperfect passive in classical Latin is normally the imperfect of \textit{esse} and a perfect passive participle, but there are scattered examples like this: a pluperfect of \textit{esse} and the perfect passive participle. The tendency towards these shifted pluperfects increases in later Latin. See H-S §179 and, more generally, \citet[§298]{väänänen1981}.

\lemc{5 removendo} ablative of means of a gerund. Descartes will try again `by setting aside' anything which may be uncertain.

\lemc{6 nihilo secius} probably a litotes. That is `not at all other than if' means `exactly as if'.

\lemc{7 aliquid certi} see the note on I §8.6.

\lemc{7 saltem hoc ipsum} the demonstrative \textit{hic, haec, hoc} is often used to point forward. In this case, it anticipates \textit{nihil esse certi}. Descartes will keep striving for certainty, even if the only certain thing he discovers is that there is no certainty.

\lemc{8 punctum petebat Archimedes} Archimedes was an important scientist and mathematician (c. 287-212 BCE, born in Syracuse). Descartes alludes to his work with levers: Archimedes famously said, `Give me a place to stand, and I will move the Earth'. In the same way that a lever magnifies a person's strength, Descartes believes that the discovery of a single certainty will indirectly increase his knowledge far more than the one certainty itself. If Descartes is right, a single certainty can provide the foundation for a vast edifice of scientific knowledge.

\textbf{§2.} To recap our program of doubt: we will doubt all the evidence of our senses, we will not believe our memories, we will even reject the most general categories of thought. Perhaps the only truth left will be that there is no certainty.

\lem{1 unquam} = \textit{umquam}

\lemc{3 chimerae} In Homer's \textit{Iliad} the chimera was lion in the front, snake in the back and goat in the middle. It also breathed fire. To say that something is `a chimera' is to say that it is mere fantasy.

\lemc{4 nihil esse certi} implied indirect statement depending on the understood \textit{erit verum}.

% -]] An Archimedean point

% [[- Ego sum, ego existo
\clearpage

\beginnumbering
\pstart
\begin{latin}
    \textenglish{\textbf{3.}} Sed unde scio nihil esse diversum ab iis omnibus quae iam iam recensui de quo ne minima quidem occasio sit dubitandi? Nunquid est aliquis Deus, vel quocunque nomine illum vocem, qui mihi has ipsas cogitationes immittit? Quare vero hoc putem cum forsan ipsemet illarum author esse possim? Nunquid ergo saltem ego aliquid sum? Sed iam negavi me habere ullos sensus et ullum corpus. Haereo tamen; nam quid \at{25} inde? Sumne ita corpori sensibusque alligatus ut sine illis esse non possim? Sed mihi persuasi nihil plane esse in mundo: nullum coelum, nullam terram, nullas mentes, nulla corpora; nonne igitur etiam me non esse? Imo certe ego eram si quid mihi persuasi. Sed est deceptor nescio quis, summe potens, summe callidus, qui de industria me semper fallit. Haud dubie igitur ego etiam sum si me fallit. Et fallat quantum potest: nunquam tamen efficiet ut nihil sim quamdiu me aliquid esse cogitabo. Adeo ut, omnibus satis superque pensitatis, denique statuendum sit hoc pronuntiatum `Ego sum, ego existo' quoties a me profertur vel mente concipitur necessario esse verum.
\end{latin}
\pend
\endnumbering

\prenotes

\textbf{§3.} How can we even be sure that there is nothing certain? Perhaps we simply haven't found it yet. Whatever source we imagine for our thoughts---god, a demon, or ourselves---we must at least be something insofar as we're thinking. It doesn't matter that everything we normally think about ourselves and the world is false. Even if that's so, we \textit{are} thinking it. Go back to the evil demon idea: let him trick us all he likes, we must therefore exist for him to trick. As long as we are thinking, we aren't nothing. We exist.

\lem{2 Nunquid} = \textit{Numquid}, a word with no one exact translation into English. In classical Latin it introduces questions that (i) involve emotion, (ii) express incredulity, or (iii) are rhetorical. (These categories can overlap.) Usually questions with \textit{numquid} expect a negative answer. In the meditations, however, Descartes uses \textit{nunquid} in contexts where it is clear he expects an affirmative answer. In this case, Descartes is suggesting the existence of god as perhaps something that provides no possibility for doubt.

\lemc{4 putem} a deliberative subjunctive `Why should I think this?' The existence of god is not certain since Descartes may be responsible for his own thoughts.

\lemc{4 Nunquid ergo\dots ego aliquid sum} if there is no god and Descartes himself is responsible for his thoughts, then surely he must exist.

\lemc{6 nam quid inde?} the verb implied is something like `follows' or `happens'. Descartes is confused: he's admitted that he may have no body at all, and he's not sure what follows from that. In particular, he's not sure that it means that he might not exist, which is the question before him.

\lemc{8 me non esse} resupply \textit{mihi persuasi} from earlier in the sentence to govern this indirect statement.

\lemc{8 Imo certe} this is the turning point from doubt towards certainty. These two words emphatically reject the last question, and the rest of the sentence justifies Descartes conviction that he must exist.

\lemc{9 si quid mihi persuasi} in English we would say `if I persuaded myself of anything' or `about anything', but in Latin the `something' that a person comes to believe is the accusative direct object of \textit{persuadeo}. Normally, however, we don't notice this because the object of the verb is usually a clause: an indirect statement or an indirect command.

\lemc{11 fallat} an independent subjunctive expressing a command.

\lemc{11--12 quamdiu\dots cogitabo} on first reading this clause may not stand out, but it will turn out to matter a great deal. Descartes will suggest that if he stops thinking entirely, he ceases to exist. So he means it when he says that he will always be something `as long as' he thinks.

\lemc{12 Adeo ut} these words introduce what serves as the main clause of the sentence. They are functionally equivalent here to \textit{igitur} or \textit{quare}.

\lemc{12 satis superque} Descartes has thought about these matters `enough and more than enough'. I.e., very extensively.

\lemc{12--13 statuendum sit} the subject of this verb is the indirect statement \textit{hoc pronuntiatum\dots necessario esse verum}.

\lemc{13 hoc} anticipates \textit{`Ego sum, ego existo'}. Those words are `this statement'.

\lemc{necessario} the necessary truth is `that I exist whenever I say so or think I do' not simply `that I exist'. Descartes is not committed to the view that he must exist.
% -]] Ego sum, ego existo

% [[- What is the ego that exists?
\clearpage

\beginnumbering
\pstart
\begin{latin}
    \textenglish{\textbf{4.}} Nondum vero satis intelligo quisnam sim ego ille qui iam necessario sum; deincepsque cavendum est ne forte quid aliud imprudenter assumam in locum mei sicque aberrem etiam in ea cognitione quam omnium certissimam evidentissimamque esse contendo. Quare iam denuo meditabor quidnam me olim esse crediderim priusquam in has cogitationes incidissem; ex quo deinde subducam quidquid allatis rationibus vel minimum potuit infirmari ut ita tandem praecise remaneat illud tantum quod certum est et inconcussum.
\end{latin}
\pend
\endnumbering

\beginnumbering
\pstart
\begin{latin}
    \textenglish{\textbf{5.}} Quidnam igitur antehac me esse putavi? Hominem scilicet. Sed quid est homo? Dicamne animal rationale? Non, quia postea quaerendum foret quidnam animal sit et quid rationale, atque ita ex una quaestione in plures difficilioresque delaberer; nec iam mihi tantum otii est ut illo velim inter istiusmodi subtilitates abuti. Sed hic potius attendam quid sponte \at{26} et natura duce cogitationi meae antehac occurrebat quoties quid essem considerabam.
\end{latin}
\pend
\endnumbering

\prenotes

\textbf{§4.} We have our one certainty: we exist. But how much substance does this certainty have? What do we know about the `we' that exists? What are we? If we are not careful, we will quickly go wrong just as we've found a tiny bit of certainty. To prevent that, let's use the same method we just employed: we will consider what we think about ourselves, and we'll reject whatever has any doubt. What's left will be certain and unshakeable.

\lem{1 intelligo} = \textit{intellego}.

\lem{2 cavendum est} introduces a compound fear clause containing two verbs joined by \textit{que}: \textit{assumam} (2) and \textit{aberrem} (3).

\lemc{3 omnium certissimam evidentissimam} the partitive genitive is common with superlatives.

\lemc{crediderim} perfect subjunctive in an indirect question following \textit{meditabor}.

\lemc{5--6 allatis rationibus} ablative of means with \textit{infirmari} (6).

\lemc{6 vel} adverbial rather than a conjunction. It means `even'.

\lemc{6 minimum} an adverbial use of the accusative, meaning `in the smallest degree' or `the least bit'.

\lemc{6 potuit} the tense is somewhat odd, and we would expect a future perfect in Latin. The point is that Descartes will remove whatever will prove to be doubtful once he has brought forward arguments. The removal will happen in the future, and so \textit{subducam} is future. At that time, Descartes will \textit{already} have proven that doubt is possible. Hence, he uses a perfect tense for \textit{potuit}, even though this is not strictly speaking the correct tense. It hasn't happened yet, but it will have happened at an imagined moment in the future.

\lemc{6 tandem praecise} despite the imprecise word order, you should take \textit{tandem} with \textit{remaneat} and \textit{praecise} with \textit{illud}.

\lemc{7 tantum} not `so big' or `so great' but `only' or `just'. This is a common idiomatic use of \textit{tantum}, and you must take care to distinguish it from the other meanings of the word.

\textbf{§5.} What have we believed up until now about ourselves? We believe that we are people, that people have bodies and minds, and that bodies and minds are responsible for distinct features of what it is to be a person.

\lemc{1 Hominem scilicet} supply \textit{me esse putavi} to make this fragment a complete thought.

\lemc{2 Dicam} a deliberative subjunctive.

\lemc{2 animal rationale} this is a traditional definition of `human', derived ultimately from Aristotle. Descartes rejects it since (i) it would drag him into arguments about definitions and (ii) he doesn't think that it possesses the kind of certainty he hopes to find. He only states (i) explicitly.

\lemc{2 Non} that is, `I should not say this'.

\lemc{2 foret} an alternative form of \textit{esset}, the imperfect subjunctive of \textit{esse}. The imperfect subjunctive is used in an implied present contrary-to-fact conditional sentence: `I should not say this because (if I were to say it), I would have to pursue tedious questions about definitions'.

\lemc{3 plures difficiliores} Latin frequently uses `and' with the adjective \textit{multus, multa, multum} in a way that is unidiomatic in English. E.g., `many and difficult problems' rather than `many difficult problems'. This is the same thing, but \textit{plures} is comparative.

\lemc{4 mihi} dative of possession with \textit{tantum otii est}.

\lemc{4 illo} ablative object of the verb \textit{abuti} (5).

\lemc{5 natura duce} an ablative absolute. Classical Latin has no present participle for \textit{esse}, and so the verb `being' is implied in an ablative absolute consisting of two nouns, like this one. (Less often you'll see one noun and one adjective. E.g. \textit{Aenea pio}: `with Aeneas being dutiful' or `since Aeneas is dutiful'.) `With nature being leader' is equivalent to something like `under the guidance of nature' or, even less literally, `naturally'.

\lemc{5 cogitationi meae} dative with the compound verb \textit{occurrebat} (6).
% -]] What is the ego that exists?

% [[- What is the ego (cont.)?
\clearpage

\beginnumbering
\pstart
\setline{7}
\begin{latin}
    \textenglish{\textbf{5. (cont.)}} Nempe occurrebat primo me habere vultum, manus, brachia, totamque hanc membrorum machinam qualis etiam in cadavere cernitur et quam corporis nomine designabam. Occurrebat praeterea me nutriri, incedere, sentire, et cogitare: quas quidem actiones ad animam referebam. Sed quid esset haec anima, vel non advertebam, vel exiguum nescio quid imaginabar, instar venti, vel ignis, vel aetheris, quod crassioribus mei partibus esset infusum. De corpore vero ne dubitabam quidem, sed distincte me nosse arbitrabar eius naturam, quam si forte, qualem mente concipiebam, describere tentassem, sic explicuissem: per `corpus' intelligo illud omne quod aptum est figura aliqua terminari, loco circumscribi, spatium sic replere ut ex eo aliud omne corpus excludat; tactu, visu, auditu, gustu, vel odoratu percipi, necnon moveri pluribus modis---non quidem a seipso, sed ab alio quopiam a quo tangatur. Namque habere vim seipsum movendi, item sentiendi, vel cogitandi, nullo pacto ad naturam corporis pertinere iudicabam; quinimo mirabar potius tales facultates in quibusdam corporibus reperiri.
\end{latin}
\pend
\endnumbering

\prenotes

\lemc{8 membrorum machinam} a striking phrase that implicitly puts the reader in the right frame of mind to think of a human body as something separate from the person whose body it is. (The genitive is likely of material (\textbf{AG} §344).)

\lemc{8 qualis etiam in cadavere cernitur} this phrase also suggests the gap between a person and that person's body.

\lemc{9 nutriri} we might think of feeding as something more bodily than mental, but there is a tradition stretching back at least as far as Aristotle that argues otherwise. The idea is this: anything which is alive possesses what Aristotle would call a ψυχή and Descartes calls an \textit{anima}: a soul. Even plants, as living creatures, possess souls. Their souls, however, are responsible only for the most basic functions of life: nutrition, growth, and reproduction. (This verb appears in classical Latin both in deponent and non-deponent forms.)

\lemc{10 Sed quid esset\dots} Descartes rarely thought about the nature of the soul before. When he did, he went no further than commonplace analogies of the soul as a kind of wind, flame or air spread throughout his limbs.

\lemc{12--13 ne dubitabam quidem} remember that \textit{ne X quidem} means `not even X'.

\lemc{13--14 qualem mente concipiebam} depends on \textit{describere} (14). That is, `to describe what sort of thing I thought (that it was)'.

\lem{14 tentassem} = \textit{tentavissem} (\textbf{AG} §181). This process is often known as `syncopation'.

\lemc{15 aptum est} the infinitives \textit{terminari} (15), \textit{circumscribi} (15), \textit{replere} (16), \textit{percipi} (17), and \textit{moveri} (17) all depend on this phrase. They are explanatory or limiting infinitives: they explain qualities that befit (\textit{aptum}) a body. This sort of infinitive, often called `epexegetical', is common in Greek but rare in classical Latin. Roman poets began to use the construction in imitation of Greek, and it became more common in later Latin.

\lem{17 seipso} = \textit{se ipso}.

\lemc{18 habere vim} this infinitive depends on \textit{pertinere iudicabam} (19). Descartes did not believe (literally `judge') that it belonged to the nature of the body `to have the power'.

\lem{18 seipsum} = \textit{se ipsum}.

\lemc{18 movendi\dots sentiendi\dots cogitandi} these genitives define \textit{vim}. In English we would more likely say `the power to move, to perceive, to think', but gerunds are possible for us too.

\lemc{19 nullo pacto} an common ablative of manner meaning `in no way'. \textit{quo pacto}, sometimes written as one word, similarly means `in what way?' or simply `how?'.

\lem{19 quinimo} = \textit{quin immo}, indicating an emphatic contrast to the preceding thought. There's an important contrast between something that `belongs to the nature of body' and something that is `found in a body'. The first refers to essential qualities of bodies; the second includes non-essential qualities. Descartes didn't think that movement, perception, or thought were essential to bodies, and \textit{what's more} (\textit{quinimo}) he was surprised to find them associated with bodies at all. Throughout this paragraph Descartes admits that he was predisposed to sharply distinguish body and mind.
% -]] What is the ego (cont.)?

% [[- Sum res cogitans
\clearpage

\beginnumbering
\pstart
\begin{latin}
    \textenglish{\textbf{6.}} Quid autem nunc ubi suppono deceptorem aliquem potentissimum et (si fas est dicere) malignum data opera in omnibus, quantum potuit, me delusisse? Possumne affirmare me habere vel minimum quid ex iis omnibus quae iam dixi ad naturam corporis perti\at{27}nere? Attendo, cogito, revolvo: nihil occurrit; fatigor eadem frustra repetere. Quid vero ex iis quae animae tribuebam? Nutriri vel incedere? Quandoquidem iam corpus non habeo, haec quoque nihil sunt nisi figmenta. Sentire? Nempe etiam hoc non fit sine corpore, et permulta sentire visus sum in somnis quae deinde animadverti me non sensisse. Cogitare? Hic invenio: cogitatio est; haec sola a me divelli nequit. Ego sum, ego existo; certum est. Quandiu autem? Nempe quandiu cogito; nam forte etiam fieri posset, si cessarem ab omni cogitatione, ut illico totus esse desinerem. Nihil nunc admitto nisi quod necessario sit verum; sum igitur praecise tantum res cogitans, id est, mens, sive animus, sive intellectus, sive ratio---voces mihi prius significationis ignotae. Sum autem res vera et vere existens; sed qualis res? Dixi: cogitans.
\end{latin}
\pend
\endnumbering

\prenotes

\textbf{§6.} All of that is what we used to think, but what should we think now that we are imagining that an evil demon might be deceiving us as much as possible? In that case, we must reject everything that relies on an external world that can be doubted. Hence, we cannot be mind and body, but mind alone. And even when we focus on our minds, we must not allow anything that would require a body. For example, movement or perception. What remains is thought: this alone is certain. Therefore, we can define ourselves as essentially thinking.

\lemc{1 nunc} the previous paragraph began with \textit{antehac}, and the contrast is important. After reviewing what he used to believe, Descartes goes on to consider what he should believe now---after his sceptical investigations in the first meditation.

\lemc{1--2 si fas est dicere} in classical Latin, this was a formula of religious caution. Descartes uses it here in the same way, but in a Christian context.

\lemc{2 data opera} literally an ablative absolute `with work having been given', but idiomatically equivalent to `intentionally, deliberately, on purpose'. (The same phrase also appears as \textit{dedita opera} and \textit{consulta opera}.)

\lemc{5-8 Nutriri\dots incedere\dots Sentire\dots Cogitare} The easiest way to take these infinitives is that they function as nominatives in sentences with no verb. (Remember that the gerund does not appear in the nominative, but infinitives often serve as subjects.) As in English, the grammar may seem tricky, but it's perfectly clear what Descartes means. He asks `What should I think about the things I used to attribute to my soul?' and then he fills in what some of these things were: `Feeding or walking?\dots Perceiving?\dots Thinking?' He could have used nouns instead of infinitives, and in fact he follows \textit{cogitare} with \textit{cogitatio}.

\lemc{6 iam} logical (`at this point in the argument') rather than temporal (`now' or `already').

\lemc{8 Hic invenio} supply \textit{aliquid} or a similar object for \textit{invenio}.

\lem{9 Quandiu} = \textit{Quamdiu}. Supply \textit{existo} from the previous sentence.

\lemc{10 posset} the subject of this verb is the substantive clause \textit{ut\dots desinerem} (10). In English, `It could happen that\dots'.

\lemc{11 totus} here, as often, an adjective in the nominative in Latin is equivalent to an adverb in English: `as a whole', `entirely', or `altogether'.

\lemc{12--13 mens\dots ratio} Descartes offers various synonyms for `mind' here. In some contexts, these Latin words can have different connotations, but Descartes lists them simply to indicate that when he says that he is a \textit{res cogitans}, he means the meaning shared by all of these words.

\lemc{13 voces mihi prius significationis ignotae} take \textit{mihi} with \textit{ignotae}, and take \textit{significationis ignotae} as a descriptive genitive with \textit{voces}. This aside is important: we use words such as `mind' or `intellect' all the time, but if Descartes is correct, we have not genuinely understood what they mean. In what follows, Descartes will argue for a radical revision of the concepts mental and physical.
% -]] Sum res cogitans

% [[- Call the mind away from the senses
\clearpage

\beginnumbering
\pstart
\begin{latin}
    \textenglish{\textbf{7.}} Quid praeterea? Imaginabor: non sum compages illa membrorum quae corpus humanum appellatur; non sum etiam tenuis aliquis aër istis membris infusus, non ventus, non ignis, non vapor, non halitus, non quidquid mihi fingo: supposui enim ista nihil esse. Manet positio: nihilominus tamen ego aliquid sum. Fortassis vero contingit ut haec ipsa quae suppono nihil esse quia mihi sunt ignota, tamen in rei veritate non differant ab eo me quem novi? Nescio; de hac re iam non disputo. De iis tantum quae mihi nota sunt iudicium ferre possum. Novi me existere; quaero quis sim ego ille quem novi. Certissimum est huius sic praecise sumpti notitiam non pendere ab iis quae exi\at{28}stere nondum novi; non igitur ab iis ullis quae imaginatione effingo. Atque hoc verbum `effingo' admonet me erroris mei: nam fingerem revera si quid me esse imaginarer quia nihil aliud est imaginari quam rei corporeae figuram seu imaginem contemplari. Iam autem certo scio me esse simulque fieri posse ut omnes istae imagines, et generaliter quaecunque ad corporis naturam referuntur, nihil sint praeter insomnia. Quibus animadversis, non minus ineptire videor dicendo `imaginabor ut distinctius agnoscam quisnam sim', quam si dicerem `iam quidem sum experrectus, videoque nonnihil veri, sed quia nondum video satis evidenter, data opera obdormiam ut hoc ipsum mihi somnia verius evidentiusque repraesentent'. Itaque cognosco nihil eorum quae possum imaginationis ope comprehendere ad hanc quam de me habeo notitiam pertinere, mentemque ab illis diligentissime esse avocandam ut suam ipsa naturam quam distinctissime percipiat.
\end{latin}
\pend
\endnumbering

\prenotes

\textbf{§7.} What more can we say about what we are? We agree to do away with talk about the human body, and we will also ignore metaphorical notions that the mind is like wind or air inside our limbs. But if we wish to stick to certainties, what can we say about the mind? We should be careful not to use imagination since imagination relies too much on images drawn from vision---and we have already agreed that the senses cannot be relied upon. Instead we should lead the mind away from such things in order to understand the nature of thinking as clearly as possible.

\lemc{2 istis membris infusus} in the active voice, one construction of this verb is \textit{infundere aliquid alicui}. That is, `to pour something onto, over, or into something'. In the passive, the original accusative becomes the subject, but the dative is retained---as in this phrase \textit{istis membris}.

\lemc{4--6 Fortassis\dots disputo} a very important point. It may turn out to be the case that, from the point of view of physics, our actual minds are like wind or gas or fire. Descartes cannot know one way or the other what science will discover about this. For now he is not concerned about these questions one way or another: since all facts about the physical world are taken to be uncertain, he chooses not to pass judgment on these issues. Instead, he will focus completely on what he can know.

\lemc{8--9 Certissimum est\dots notitiam non pendere} the phrase \textit{notitiam non pendere} functions as the subject of \textit{est}, and \textit{certissimum} is the predicate: `That the knowledge (of this matter) does not depend (on things I don't know is most certain'. Whenever an infinitive phrase has a subject, the subject will be accusative. (We're most familiar with this in indirect statement, but indirect statement is far from the only case where there is an accusative subject of an infinitive in Latin.) The predicate \textit{certissimum} is neuter singular since Latin treats the entire noun phrase as a singular thing---just as in English we say `It \textit{is} clear that Descartes writes elegant Latin'.

\lemc{8 huius sic praecise sumpti notitiam} the genitive is objective with \textit{notitiam}, and \textit{sic praecise} modify the participle \textit{sumpti}. It might help to expand \textit{sic praecise sumpti} into a full clause in English: the knowledge of this, `if it is taken up in this precise manner' or `when it is taken up precisely in this way'.

\lemc{9 non igitur ab iis ullis} supply \textit{eam pendere}, which is implied by the previous clause.

\lemc{10--14} Descartes argues here that imagination is not a reliable way to proceed because it essentially involves vision and perception, and he has argued that these can no longer be trusted.

\lemc{11 nihil aliud est imaginari quam} the infinitive \textit{imaginari} is the subject and \textit{nihil aliud} is a predicate complement: `to imagine is nothing other than\dots'.

\lemc{14--18} The analogy is perhaps unexpected but forceful: according to Descartes using imagination to understand himself is as foolish as going to sleep in order to see something more clearly.

\lem{18 cognosco} introduces two indirect statements: (i) \textit{nihil eorum\dots pertinere} (18--19) and (ii) \textit{mentem\dots esse avocandum} (19--20). The structure is a little difficult to see at first because each indirect statement has other clauses that depend on it.

\lemc{20 suam ipsa} an opportunity for a useful reminder. \textit{suus, sua, suum} is a reflexive adjective, sometimes used for emphasis; \textit{ipse, ipsa, ipsum} is an intensifier but never reflexive. It's easy to forget this becuase in English both are often represented by `self'. It's also easy to get confused because they often appear together in Latin for emphasis, as in this case. In this case, we must call the mind away from bodily things so that the mind itself (\textit{ipsa}) can perceive its very own (\textit{suam}) nature as clearly as possible.
% -]] Call the mind away from the senses

% [[- What is the nature of thinking?
\clearpage

\beginnumbering
\pstart
\begin{latin}
    \textenglish{\textbf{8.}} Sed quid igitur sum? Res cogitans. Quid est hoc? Nempe dubitans, intelligens, affirmans, negans, volens, nolens, imaginans quoque, et sentiens.
\end{latin}
\pend
\endnumbering

\beginnumbering
\pstart
\begin{latin}
    \textenglish{\textbf{9.}} Non pauca sane haec sunt si cuncta ad me pertineant. Sed quidni pertinerent? Nonne ego ipse sum qui iam dubito fere de omnibus, qui nonnihil tamen intelligo, qui hoc unum verum esse affirmo, nego caetera, cupio plura nosse, nolo decipi, multa vel invitus imaginor, multa etiam tanquam a sensibus venientia animadverto? Quid est horum---quamvis semper dor\at{29}miam, quamvis etiam is qui me creavit, quantum in se est, me deludat---quod non aeque verum sit ac me esse? Quid est quod a mea cogitatione distinguatur? Quid est quod a me ipso separatum dici possit? Nam quod ego sim qui dubitem, qui intelligam, qui velim, tam manifestum est ut nihil occurrat per quod evidentius explicetur. Sed vero etiam ego idem sum qui imaginor; nam quamvis forte, ut supposui, nulla prorsus res imaginata vera sit, vis tamen ipsa imaginandi revera existit et cogitationis meae partem facit. Idem denique ego sum qui sentio sive qui res corporeas tanquam per sensus animadverto: videlicet iam lucem video, strepitum audio, calorem sentio. Falsa haec sunt, dormio enim. At certe videre videor, audire, calescere. Hoc falsum esse non potest; hoc est proprie quod in me sentire appellatur; atque hoc praecise sic sumptum nihil aliud est quam cogitare.
\end{latin}
\pend
\endnumbering

\prenotes

\textbf{§8.} As thinkers we doubt, understand, affirm, deny, want, refuse, imagine, and perceive.

\textbf{§9.} These several types of thought make us what we are inasmuch as we are essentially thinkers. They are all as true as our existence no matter whether we are asleep---no matter whether an evil all-powerful spirit tricks us. In the case of doubt, understanding, and wanting, this is clear. It might be less clear in the case of imagination and perception, but even if we imagine or perceive something false, it is nevertheless always true that we really \textit{are} imagining or perceiving.

\lemc{2 fere} take this with \textit{omnibus} rather than \textit{dubito}. (As a general rule, adverbs in Latin modify something to their right not their left. There are exceptions, but it's a good rule of thumb.)

\lem{3 caetera} = \textit{cētera}, the \textit{ae} representing the original long `e' sound.

\lemc{5--6 quamvis\dots deludat} i.e., `whether I'm sleeping or not, whether an evil demon is deceiving me or not'.

\lemc{6 aeque verum ac} English would say `as true as' or `just as true as'. Latin often uses \textit{atque} following a comparison. (\textit{atque} and \textit{ac} are two forms of the same word, just as \textit{neque} and \textit{nec}.)

\lemc{6 me esse} `that I am'. I.e., in this context, `my existence'. Descartes rhetorically asks `Which of these types of thought is not just as true as that I exist?'

\lemc{7 distinguatur} subjunctive in a relative clause of characteristic. That is, not simply `which is distinguished' but `which could be distinguished'. The construction of \textit{possit} in the next sentence is the same.

\lemc{8 quod ego sim} the \textit{quod} serves to nominalize (i.e., turn into a noun) this clause. In English, we can translate such a \textit{quod} as `that' or `the fact that'. This clause is the subject of the main verb \textit{est} and \textit{tam manifestum} is the predicate. It's a little difficult to see this initially because the \textit{ego} is described by the three adjectival clauses \textit{qui dubitem, qui intelligam, qui velim}. First, here's a rather literal translation: `For (the fact) that I who doubt, who understand, who want, exist is so clear that\dots'. A somewhat more natural English version is this: `It is so clear that I exist---I who doubt, who understand, who want---that\dots'.

\lemc{9--16} This section introduces an idea that modern philosophers call the `incorrigibility of the mental'. Roughly: we cannot be wrong about our own mental experiences. For example, if I think I'm in pain, then I am in pain. Nobody can tell me otherwise, and I can't be wrong. It is controversial, however, whether mental experiences (or even a subset of them) are incorrigible. To consider pain again, we might describe phantom limb pain, cases where someone experiences pain in a limb that has been amputated, as false pain. However this may be, Descartes offers special defense of the incorrigibility of imagination and perception because he sees that these two types of mental life might seem very fallible. For example, the shirt that looks blue to me might actually be green since I'm colorblind. However, Descartes argues, I can't be wrong about how it looks to me. Similarly, I might imagine something impossible, but it remains true that I'm imagining it as I do. So in a Cartesian spirit, we might say that a person with a phantom limb pain is not wrong \textit{about being in pain}.

\lemc{9 ut supposui} that is, as Descartes has repeatedly agreed: the entire external world of physical objects might be false.

\lemc{9 nulla prorsus} the adverb \textit{prorsus} is complicated. Its meanings include `forward, straight ahead; without interruption; right through to the end'. But in addition, it is often used for emphasis of a word or phrase, particularly negatives as here. In such cases, you can translate as `at all, whatsoever'.

\lemc{14 dormio} there is an implicit \textit{ut supposui} here just as in the previous sentence.

\lemc{14 videre videor, audire, calescere} even if the real world is not as it seems to perception, we still have these perceptual experiences.

\lemc{15 sentire} used as a noun to mean `perception'.

\lemc{16 cogitare} used as a noun to mean `thought'.
% -]] What is the nature of thinking?

% [[- Physical things seem easier to know than the mind
\clearpage

\beginnumbering
\pstart
\begin{latin}
    \textenglish{\textbf{10.}} Ex quibus equidem aliquanto melius incipio nosse quisnam sim; sed adhuc tamen videtur (nec possum abstinere quin putem) res corporeas, quarum imagines cogitatione formantur et quas ipsi sensus explorant, multo distinctius agnosci quam istud nescio quid mei, quod sub imaginationem non venit. Quanquam profecto sit mirum, res quas animadverto esse dubias, ignotas, a me alienas, distinctius quam quod verum est, quod cognitum, quam denique me ipsum, a me comprehendi. Sed video quid sit: gaudet aberrare mens mea, necdum se patitur intra veritatis limites cohiberi. Esto igitur, et adhuc semel laxissimas habe\at{30}nas ei permittamus ut, illis paulo post opportune reductis, facilius se regi patiatur.
\end{latin}
\pend
\endnumbering

\prenotes

\textbf{§10.} Although we seem to have the beginnings of an understanding of what we are, it still seems that we understand physical things far more easily than we understand the mind. Because it is not physical, we cannot sense the mind at all, nor can we imagine it very easily. And yet this is paradoxical because we've already agreed (i) that bodily things are doubtful and distinct from our true selves (our minds) and (ii) that our minds are what we are. Hence, we're saying that something doubtful is intuitively easier to understand than our own true nature. In order to investigate further, let's consider more precisely what it means to understand something physical.

\lemc{1--4 sed adhuc\dots venit} the basic elements of this sentence are the main verb \textit{videtur} and its subject, the accusative and infinitive \textit{res corporeas multo distinctius agnosci}. The sentence has several additional clauses that make this basic structure more difficult to see. First, there's a parenthetical coordinate clause after \textit{videtur}. This clause will be described in more detail below. Skip it initially. Second, \textit{res corporeas} is described by two parenthetical relative clauses: \textit{quarum imagines cogitatione formantur} and \textit{quas ipsi sensus explorant}. Finally, because the subject of \textit{videtur} contains a comparison (\textit{multo distinctius}), it is followed by a \textit{quam} clause filling out the `than' part of the comparison. (The \textit{quam} clause is also modified by a further relative clause.) When faced with a complex sentence like this, the best strategy is to skim first looking for the largest significant pieces (generally clauses with subjects and verbs). Then scan more closely and find the elements in the main clause; put that together and then add the other clauses bit by bit. (Do I want to give this general advice?)

\lemc{2 quin putem} a common use of \textit{quin} is to indtroduce a clause with a subjunctive after a negated verb of doubt, prevention, refusal or the like. The precise translation of the \textit{quin} clause will vary. In a sentence like this, English idiom prefers something like `from thinking' rather than the more literal `(but) that I think'. (See \textbf{AG} §557-559 for various uses of \textit{quin}. The use in this sentence is covered in §558.

\lemc{3 quod sub imaginationem non venit} We have no way to imagine our minds, according to Descartes, except through uninformative analogies such as `like a wind spread throughout the body'. This is because, for Descartes, the mental, essentially has no physical nature, and imagination is based on sensory perception of physical objects. See §7.11--12.

\lemc{4--6 Quanquam\dots comprehendi} The word order is difficult. What is \textit{mirum} is the indirect statement \textit{res (which I know to be\dots) distinctius a me comprehendi quam (things I know much better)}.

\lem{4 Quanquam} = \textit{Quamquam}.

\lemc{5--6 quam quod verum est, quod cognitum, quam denique me ipsum} there is a switch in syntax in the final item here. The first two clauses are  of the form `than (that) which is Y'. Reordered and with an implicit \textit{id} added for clarity: \textit{quam (id) quod est verum} and \textit{quam (id) quod est cognitum}. The last clause, \textit{quam denique me ipsum}, however, has no verb and no relative. The phrase \textit{me ipsum} is accusative because \textit{quam} is normally followed by the same case as the word or phrase being compared to, and here that word is accusative: \textit{res}.

\lemc{8 Esto} 3rd person singular, future, active, imperative of \textit{esse}. The future imperative is rare: often used in legal contexts or poetry. Here it is used of granting something for the sake of argument that the speaker does not necessarily believe: `So be it', `Grant that this is so'. It anticipates the following jussive subjunctive \textit{permittamus}, and in English we would likely just say `Fine: let us\dots'.

\lem{8 laxissimas habe|nas ei permittamus} the metaphor presents the mind as like an unruly horse. We will let out its reins, to make the horse easier to control later, presumably after its exhausted itself with a good run.

\lem{8 ei} refers back to \textit{mens} (7).

\lemc{9 illis} refers back to \textit{habenas} (8).
% -]] Physical things seem easier to know than the mind

% [[- This wax
\clearpage

\beginnumbering
\pstart
\begin{latin}
    \textenglish{\textbf{11.}} Consideremus res illas quae vulgo putantur omnium distinctissime comprehendi: corpora scilicet quae tangimus, quae videmus; non quidem corpora in communi (generales enim istae perceptiones aliquanto magis confusae esse solent) sed unum in particulari. Sumamus, exempli causa, hanc ceram: nuperrime ex favis fuit educta; nondum amisit omnem saporem sui mellis; nonnihil retinet odoris florum ex quibus collecta est; eius color, figura, magnitudo manifesta sunt; dura est, frigida est, facile tangitur, ac si articulo ferias, emittet sonum; omnia denique illi adsunt quae requiri videntur ut corpus aliquod possit quam distinctissime cognosci. Sed ecce: dum loquor, igni admovetur. Saporis reliquiae purgantur, odor expirat, color mutatur, figura tollitur, crescit magnitudo, fit liquida, fit calida, vix tangi potest, nec iam, si pulses, emittet sonum. Remanetne adhuc eadem cera? Remanere fatendum est; nemo negat, nemo aliter putat. Quid erat igitur in ea quod tam distincte comprehendebatur? Certe nihil eorum quae sensibus attingebam; nam quaecunque sub gustum vel odoratum vel visum vel tactum vel auditum veniebant, mutata iam sunt; remanet cera.
\end{latin}
\pend
\endnumbering

\prenotes

\textbf{§11.} We commonly think that we understand physical things very well and that our knowledge of them comes from the senses. To consider this knowledge, let's look at one specific thing: this piece of wax. It seems that we can say a great deal about it with confidence: we can smell a little of the flowery odor that remains from the honey it came from; its color, shape, and size are clear; it is firm, cool, easily handled, and if someone strikes it with a finger, it makes a distinct sound. All of this information suggests that we are right to think that physical thinks are understood very clearly. If we hold the wax close to a flame, however, everything changes. The smell disappears, the color and shape change, it grows in size, but becomes liquid and warm---difficult to touch. And if someone strikes it now, it doesn't make a sound. Although we all agree that this is still \textit{the same wax}, all of the qualities we thought belonged to that wax are gone. So what was it that we knew so clearly?

\lemc{2--3 non quidem corpora in communi} ordinarily we assume that we know all sorts of things about specific physical objects, but we don't necessarily think we know anything about physical objects in general. Our beliefs are not necessarily theoretical or abstract at all.

\lemc{4 hanc ceram} one of the more famous examples in philosophy. As often, \textit{hanc} indicates a gesture: it is as if the speaker holds up a piece of wax before the readers' eyes, saying `\textit{this} wax'.

\lemc{sui mellis} i.e., the honey that it came from.

\lemc{6 color, figura, magnitudo} asyndeton, the omission of expected connectives. Like English, Latin usually adds a word for `and' before the last item in a series. Asyndeton is often used for emphasis.

\lemc{6 color, figura, magnitudo manifesta sunt} when one predicate adjective describes multiple nouns of different genders, what gender should the adjective be? The rule of thumb is this: (i) if the nouns refer to living things, the adjective will usually be masculine; (ii) if the nouns refer to non-living things, the adjective will usually be neuter. Descartes follows this rule here: \textit{manifesta} is neuter nominative plural.

\lemc{6 facile tangitur} it may be not obvious why Descartes tells us that the wax is easily touched or handled. He is anticipating how hard to touch the wax will become when it grows hot.

\lemc{7 articulo} possibly `knuckle', but more likely `finger'.

\lemc{10 figura tollitur} the shape of the wax is `destroyed' or `lost', as it melts. The molten wax still has some shape, but not the one it had previously.

\lemc{11 Remanetne adhuc eadem cera?} Once it has lost all of the qualities by which we thought we understood it, is it even the same wax?

\lemc{Quid\dots comprehendebatur?} because it is so commonplace for physical objects to change---even radically change, as with this wax---it can be easy to miss how much bite this question has. Ask yourself seriously: How do I decide that this puddle of warm liquid is \textit{the same thing} as the solid, cool piece of wax that I was holding a moment ago? What is the basis for that sameness? Where is it? Can I perceive it? If not, what does that tell me about my original judgment?

\lemc{14--15 vel\dots vel\dots vel\dots vel} emphatic polysyndeton, the use of more connectives than necessary. Latin, like English, usually only uses a single `or' at the end of a series.

\lemc{15 remanet cera} adversative asyndeton, the omission of adversative conjunctions. E.g., `I see it; I don't believe it' rather than `I see it, but I don't believe it'.
% -]] This wax

% -]] Meditatio secunda

    % [[- Chapter title
\chapter*{Background Material}
% -]] Chapter title

% [[- Plato's Theaetetus
\section*{Plato's \textit{Theaetetus} and the dream argument}

When Descartes uses dreaming as an argument in favor of global scepticism, he appears to have the following passage from Plato's \textit{Theaetetus} in mind. Socrates asks Theaetetus ``What is knowledge?'', and Theaetetus offers as a first definition that knowledge is perception. Socrates connects this to Protagorean relativism and Heracleitean flux, and in the selection below, he presses Theaetetus to consider some possible objections to his definition.

\begin{quote}
{\g\textbf{ΣΩΚΡΑΤΗΣ.} Ταῦτα δή, ὦ Θεαίτητε, ἆρ᾽ ἡδέα δοκεῖ σοι εἶναι, καὶ γεύοιο ἂν αὐτῶν ὡς ἀρεσκόντων;

\textbf{θΕΑΊΤΗΤΟΣ.} Οὐκ οἶδα ἔγωγε, ὦ Σώκρατες· καὶ γὰρ οὐδὲ περὶ σοῦ δύναμαι κατανοῆσαι πότερα δοκοῦντά σοι λέγεις αὐτὰ ἢ ἐμοῦ ἀποπειρᾷ.

\textbf{ΣΩ.} Οὐ μνημονεύεις, ὦ φίλε, ὅτι ἐγὼ μὲν οὔτ᾽ οἶδα οὔτε ποιοῦμαι τῶν τοιούτων οὐδὲν ἐμόν, ἀλλ᾽ εἰμὶ αὐτῶν ἄγονος, σὲ δὲ μαιεύομαι καὶ τούτου ἕνεκα ἐπᾴδω τε καὶ παρατίθημι ἑκάστων τῶν σοφῶν ἀπογεύσασθαι, ἕως ἂν εἰς φῶς τὸ σὸν δόγμα συνεξαγάγω· ἐξαχθέντος δὲ τότ᾽ ἤδη σκέψομαι εἴτ᾽ ἀνεμιαῖον εἴτε γόνιμον ἀναφανήσεται. ἀλλὰ θαρρῶν καὶ καρτερῶν εὖ καὶ ἀνδρείως ἀποκρίνου ἃ ἂν φαίνηταί σοι περὶ ὧν ἂν ἐρωτῶ.

\textbf{ΘΕΑΙ.} Ἐρώτα δή.

\textbf{ΣΩ.} Λέγε τοίνυν πάλιν εἴ σοι ἀρέσκει τὸ μή τι εἶναι ἀλλὰ γίγνεσθαι ἀεὶ ἀγαθὸν καὶ καλὸν καὶ πάντα ἃ ἄρτι διῇμεν.

\textbf{ΘΕΑΙ.} Ἀλλ᾽ ἔμοιγε, ἐπειδὴ σοῦ ἀκούω οὕτω διεξιόντος, θαυμασίως φαίνεται ὡς ἔχειν λόγον καὶ ὑποληπτέον ᾗπερ διελήλυθας.

\textbf{ΣΩ.} Μὴ τοίνυν ἀπολίπωμεν ὅσον ἐλλεῖπον αὐτοῦ. λείπεται δὲ ἐνυπνίων τε πέρι καὶ νόσων τῶν τε ἄλλων καὶ μανίας, ὅσα τε παρακούειν ἢ παρορᾶν ἤ τι ἄλλο παραισθάνεσθαι λέγεται. οἶσθα γάρ που ὅτι ἐν πᾶσι τούτοις ὁμολογουμένως ἐλέγχεσθαι δοκεῖ ὃν ἄρτι διῇμεν λόγον, ὡς παντὸς μᾶλλον ἡμῖν ψευδεῖς αἰσθήσεις ἐν αὐτοῖς γιγνομένας, καὶ πολλοῦ δεῖ τὰ φαινόμενα ἑκάστῳ ταῦτα καὶ εἶναι, ἀλλὰ πᾶν τοὐναντίον οὐδὲν ὧν φαίνεται εἶναι.

\textbf{ΘΕΑΙ.} Ἀληθέστατα λέγεις, ὦ Σώκρατες.

\textbf{ΣΩ.} Τίς δὴ οὖν, ὦ παῖ, λείπεται λόγος τῷ τὴν αἴσθησιν ἐπιστήμην τιθεμένῳ καὶ τὰ φαινόμενα ἑκάστῳ ταῦτα καὶ εἶναι τούτῳ ᾧ φαίνεται;

\textbf{ΘΕΑΙ.} Ἐγὼ μέν, ὦ Σώκρατες, ὀκνῶ εἰπεῖν ὅτι οὐκ ἔχω τί λέγω, διότι μοι νυνδὴ ἐπέπληξας εἰπόντι αὐτό. ἐπεὶ ὡς ἀληθῶς γε οὐκ ἂν δυναίμην ἀμφισβητῆσαι ὡς οἱ μαινόμενοι ἢ οἱ ὀνειρώττοντες οὐ ψευδῆ δοξάζουσιν, ὅταν οἱ μὲν θεοὶ αὐτῶν οἴωνται εἶναι, οἱ δὲ πτηνοί τε καὶ ὡς πετόμενοι ἐν τῷ ὕπνῳ διανοῶνται.

\textbf{ΣΩ.} Ἆρ᾽ οὖν οὐδὲ τὸ τοιόνδε ἀμφισβήτημα ἐννοεῖς περὶ αὐτῶν, μάλιστα δὲ περὶ τοῦ ὄναρ τε καὶ ὕπαρ;

\textbf{ΘΕΑΙ.} Τὸ ποῖον;

\textbf{ΣΩ.} Ὃ πολλάκις σε οἶμαι ἀκηκοέναι ἐρωτώντων, τί ἄν τις ἔχοι τεκμήριον ἀποδεῖξαι, εἴ τις ἔροιτο νῦν οὕτως ἐν τῷ παρόντι πότερον καθεύδομεν καὶ πάντα ἃ διανοούμεθα ὀνειρώττομεν, ἢ ἐγρηγόραμέν τε καὶ ὕπαρ ἀλλήλοις διαλεγόμεθα.

\textbf{ΘΕΑΙ.} Καὶ μήν, ὦ Σώκρατες, ἄπορόν γε ὅτῳ χρὴ ἐπιδεῖξαι τεκμηρίῳ· πάντα γὰρ ὥσπερ ἀντίστροφα τὰ αὐτὰ παρακολουθεῖ. ἅ τε γὰρ νυνὶ διειλέγμεθα οὐδὲν κωλύει καὶ ἐν ὕπνῳ δοκεῖν ἀλλήλοις διαλέγεσθαι· καὶ ὅταν δὴ ὄναρ ὀνείρατα δοκῶμεν διηγεῖσθαι, ἄτοπος ἡ ὁμοιότης τούτων ἐκείνοις.

\textbf{ΣΩ.} Ὁρᾷς οὖν ὅτι τό γε ἀμφισβητῆσαι οὐ χαλεπόν, ὅτε καὶ πότερόν ἐστιν ὕπαρ ἢ ὄναρ ἀμφισβητεῖται, καὶ δὴ ἴσου ὄντος τοῦ χρόνου ὃν καθεύδομεν ᾧ ἐγρηγόραμεν, ἐν ἑκατέρῳ διαμάχεται ἡμῶν ἡ ψυχὴ τὰ ἀεὶ παρόντα δόγματα παντὸς μᾶλλον εἶναι ἀληθῆ, ὥστε ἴσον μὲν χρόνον τάδε φαμὲν ὄντα εἶναι, ἴσον δὲ ἐκεῖνα, καὶ ὁμοίως ἐφ᾽ ἑκατέροις διισχυριζόμεθα.

\textbf{ΘΕΑΙ.} Παντάπασι μὲν οὖν.

\textbf{ΣΩ.} Οὐκοῦν καὶ περὶ νόσων τε καὶ μανιῶν ὁ αὐτὸς λόγος, πλὴν τοῦ χρόνου ὅτι οὐχὶ ἴσος;

\textbf{ΘΕΑΙ.} Ὀρθῶς.

\textbf{ΣΩ.} Τί οὖν; πλήθει χρόνου καὶ ὀλιγότητι τὸ ἀληθὲς ὁρισθήσεται;

\textbf{ΘΕΑΙ.} Γελοῖον μεντἂν εἴη πολλαχῇ.

\textbf{ΣΩ.} Ἀλλά τι ἄλλο ἔχεις σαφὲς ἐνδείξασθαι ὁποῖα τούτων τῶν δοξασμάτων ἀληθῆ;

\textbf{ΘΕΑΙ.} Οὔ μοι δοκῶ.} (Plato's \textit{Theaetetus} 157c--158e4\footnote{The text follows \cite{plato1995}.})
\end{quote}

\begin{quote}
\textbf{Socrates.} Well, Theaetetus, does this look to you a tempting meal and could you take a bite of the delicious stuff?

\textbf{Theaetetus.} I really don't know, Socrates. I can't even quite see what you're getting at---whether the things you're saying are what you think yourself, or whether you're just trying me out.

\textbf{Soc.} You're forgetting, my friend. I don't know anything about this kind of thing myself, and I don't claim any of it as my own. I'm barren of theories; my business is to attend you in your labor. So I chant incantations over you and offer you little tidbits from each of the wise until I succeed in assisting you at bringing your own belief forth into the light. When it's been born, I'll consider whether it's fertile or a wind-egg. But you must have courage and patience; answer boldly whatever appears to you to be true about the things I ask you.

\textbf{Theaet.} All right, go on with your questions.

\textbf{Soc.} Tell me again, then, whether you like the suggestion that good and beautiful and all the things we were just speaking of cannot be said to `be' anything, but are always `coming to be'.

\textbf{Theaet.} Well, as far as I'm concerned, while I'm listening to your exposition of it, it seems to me an extraordinarily reasonable view; and I feel that the way you've set out the matter must be accepted.

\textbf{Soc.} In that case, we better not pass over any point where our theory is still incomplete. What we've not yet discussed is the question of dreams, and of insanity and other diseases; also what's called mishearing or misseeing or other cases of misperception. You realize, I suppose, that it would be generally agreed that all these cases appear to provide a refutation of the theory we've just expounded. For in these conditions, we surely have false perceptions. Here it's far from being true that all things which appear to an individual also are. On the contrary, no one of the things which appear to the person really is.

\textbf{Theaet.} That's very true, Socrates.

\textbf{Soc.} Well then, my child, what argument is left for the person who maintains that knowledge is perception and that what appears to anyone also is, for the person to whom it appears to be?

\textbf{Theaet.} Well, Socrates, I'm reluctant to tell you that I don't know what to say, since I've just got into trouble with you for that. But I really wouldn't know how to dispute the suggestion that a madman believes what is false when they think that they're a god; or a dreamer when they believe they have wings are are flying in their sleep.

\textbf{Soc.} But there's a point here which \textit{is} a matter of dispute, especially as regards dreams and real life---don't you see?

\textbf{Theaet.} What do you mean?

\textbf{Soc.} There's a question you must often have heard people ask---the question what evidence we could offer if we were asked whether in the present instance, at this moment, we are asleep and dreaming all our thoughts, or awake and talking to each other in real life.

\textbf{Theaet.} Yes, Socrates, it certainly is difficult to find the evidence we need here. The two states seem to correspond in all their characteristics. There's nothing to prevent us from thinking when we're asleep that we're having the very same discussion that we've just had. And when we dream that we're telling the story of a dream, there's an extraordinary likeness between the two experiences.

\textbf{Soc.} You see, then, it's not difficult to find matter for dispute, when it's disputed even whether this is real life or a dream. Indeed we may say that, as our periods of sleeping and waking are of equal length, and as in each period the soul contends that the beliefs of the moment are pre-eminently true, the result is that for half our lives we assert the reality of the one set of objects and for half that of the other set. And we make our assertions with equal conviction in both cases.

\textbf{Theaet.} That's certainly so.

\textbf{Soc.} And doesn't the same argument apply in the cases of disease and madness, except that the periods of time are not equal?

\textbf{Theaet.} Yes, that's right.

\textbf{Soc.} Well now, are we going to fix the limits of truth by the clock?

\textbf{Theaet.} That would be a very funny thing to do.

\textbf{Soc.} But can you produce some other clear indication to show which of these beliefs are true?

\textbf{Theaet.} I don't think I can. (lightly adapted translation by M.J. Levett)
\end{quote}

At this point in the dialogue, Socrates is not arguing for or against any particular thesis. He's elicited a definition of knowledge from Theaetetus---that knowledge is perception---and Socrates is still in the process of fleshing out the implications of that definition. Socrates raises the problems of dreams, disease, and insanity as part of that broader review of Theaetetus' definition.

We can readily see, however, why this passage might appeal to Descartes. Descartes wants to introduce general or broad reasons for sceptical doubt, and Socrates alludes to just such a broad scope for the dreaming argument in particular: ``[I]n these conditions, we surely have false perceptions. Here's it's far from being true that all things which appear to someone also are. On the contrary, none of the things which appear to someone also are.'' Dreams offer a convenient example of a situation in which \textit{everything} we think appears false or unreal.

This isn't to say that everything in dreams is equally surreal. I might dream that I'm doing something perfectly ordinary. For example, washing dishes. But of course I'm \textit{not} actually washing dishes because I'm asleep. I only think that I'm washing dishes.

% -]] Plato's Theaetetus

    % [[- Chapter title
\chapter{Vocabulary}
\markboth{Vocabulary}{Vocabulary}
% -]] Chapter title

% [[- Introduction
This chapter provides a complete vocabulary for Descartes's \textit{Meditationes de prima philosophia}. Here's a brief description of the principles I followed in putting together the vocabulary as well as some important conventions.

The list should include every word in the \textit{Meditations}, no matter how basic, but I have flagged the more common or important words. If an asterisk appears in the outer left margin of a word, that word is either (i) especially common in ancient Latin or (ii) particularly important for Descartes. In order to determine what words are especially common in ancient Latin, I relied on the ``Latin Core Vocabulary'' from the Dickinson College Commentary website \parencite{cfrancese2014}. Chrisopher Francese, the primary compiler of that list, notes that those 1,000 words cover nearly 70\% of all the forms in typical ancient Latin texts \parencite{cfrancese2013}. Any word in the DCC Latin List will have an asterisk, but I've also placed asterisks on some additional words. These aditional words fall into one of two categories: (i) cognates of words already on the DCC list or (ii) words I feel are especially common in or important for Descartes. Words with an asterisk are a great first place for students to direct their attention when learning new vocabulary.

I relied on three sources for the entries. For meanings, I used both the \textit{Oxford Latin Dictionary} and \textit{A Latin Dictionary} \parencite{old1982,lewisshort}. Since these dictionaries often leave questions about the vowel length of hidden quantities unanswered (e.g., \textit{actum} or \textit{āctum}?), I also consulted Michael Weiss's \textit{Outline of the Historical and Comparative Grammar of Latin} \parencite{weiss2011}. Within entries, commas separate synonymous meanings, and semicolons separate distinct meanings.
\clearpage
% -]] Introduction

% [[- Vocabulary
\begin{description}
    \item[ā, ab, abs] \marginnote{*}from, away from; by
    \item[aberrō (1)] wander away, get lost; wander from one's subject, digress; depart from, differ from; vary
    \item[abstineō, abstinēre, abstinuī, abstentum] keep back, keep away, keep off; hold back, restrain; abstain from food or drink, fast
    \item[abūtor, abūtī, abūsus sum]  use up, consume, exhaust; make full use of, utilize; put to a wrong use, misapply, abuse, misuse
    \item[accipiō, accipere, accēpī, acceptum] \marginnote{*}take in one's grasp, receive; take into one's possession or control; have given to one, acquire, get
    \item[accūrō (1)] give attention to, perform with care; take care (that), see to it (that); attend to, take care of
    \item[actiō, actiōnis, f.] a doing, activity, action; act, deed; proposal, measure, course of action, policy
    \item[ad] \marginnote{*}towards, to, up to; near; to (a point in time), until
    \item[addō, addere, addidī, additum] \marginnote{*}put or fit (onto), attach (to), place (along with); add onto, pile (on top of); add, attach; apply
    \item[adeō] \marginnote{*}to (such) a high degree, to (such) a great extent, (so) very, extremely; to the point or place (where); to such a pass; (after a negative) so much, so very much, all that much
    \item[adhūc] \marginnote{*}until now, as yet, up to the present time; up to that time; (in a negative phrase) so far, (as) yet; still
    \item[admittō, admittere, admīsī, admissum] (of visitors, etc.) let come, admit, give access, receive; admit, endure, tolerate; grant access (to), allow (to); permit, allow, sanction; agree to, accept, receive; let go, let loose, relsease
    \item[admoneō (2)] remind (of or that), remind (a person) about (something), put in mind of; give advice to, advise, urge, bid; prompt, admonish; caution, warn, admonish; apprise of, inform, advise (that)
    \item[admoveō, admovēre, admōvī, admōtum]  move (something) near (to), bring (something) near (to), bring (something) into contact with; move towards; move up, bring up, bring near; move, lead, guide (someone or something) towards
    \item[adsum, adesse, adfuī, adfutūrum] \marginnote{*}to be at, be present, be at hand, be here; attend, be present (at a meeting, gathering, etc.)
    \item[advertō, advertere, advertī, adversum] \marginnote{*}turn or direct towards; direct, steer, guide; give ear to, pay attention, notice, see, give heed (to); remark, ascertain, discover
    \item[aequus, aequa, aequum] \marginnote{*}flat, even, level; favorable, convenient, advantageous; equal, like, alike; fair, just, reasonable, right; impartial, fair-minded
    \item[āër, āëris, m.] \marginnote{*}air; the atmosphere, air surrounding the earth; sky, heavens
    \item[aetās, aetātis, f.] \marginnote{*}one's age, the number of years one has lived; the age (of anything); period or time of life; an age group, people of a particular age; lifetime, years, the span of one's life
    \item[aethēr, aetheris, m.] \marginnote{*}upper regions of sky, heaven, the ether; the air, the sky
    \item[afferō, afferre, attulī, allātum] \marginnote{*}bring, fetch, carry, convey; confer, bestow, add, contribute; put forward, contribute, offer, recommend; lead, conduct, bring forward
    \item[affirmō (1)] assert strongly, maintain with certainty, affirm, swear, promise; support, corroborate (a statement or fact)
    \item[aggredior, aggredi, aggressus sum] \marginnote{*}go or advance (towards), approach; assault, attack, assail; confront; set about, start on, undertake (a task, a job, etc.); attempt, proceed, begin (+ infinitive)
    \item[agnoscō, agnoscere, agnōvī, agnitum] \marginnote{*}recognize, know again, identify; acknowledge as one's own, recognize as one's own; admit to, claim; admit liability for, become responsibile for; acknowledge, appreciate
    \item[agō, agere, ēgī, āctum] \marginnote{*}drive, set in motion; ride; bring, carry, bear; force to move, drive; force, push, throw; give off, emit, send forth; do, perform, achieve, accomplish
    \item[albēdō, albēdinis, f.] white color, whiteness, white (non-classical)
    \item[aliēnus, aliēna, aliēnum] \marginnote{*}of another, belonging to another, not one's own; foreign, alien; strange, unusual, unnatural
    \item[aliquamdiū] for some time, for a considerable time; for a while
    \item[aliquandiū] see \textit{aliquamdiū}
    \item[aliquī, aliqua, aliquod] \marginnote{*}(indefinite adj.) some, some or other, a; any at all, any whatsoever, a single; a kind of, a sort of; of some (extent, degree, amount), a certain amount of, some; (pl.) a certain number of, at least some, a few; some sort of, some kind of
    \item[aliquis, aliqua, aliquid] \marginnote{*}(indefinite pronoun) some one, any one, anybody; something, anything; (pl.) some, a few, a number
    \item[aliquot] a number of, several, some
    \item[aliter] \marginnote{*}otherwise, in another way, in another manner, differently
    \item[alius, alia, aliud] \marginnote{*}another, other, a different; a further, another; (multiple forms in same case) some\dots others; (multiple forms in different cases) one\dots another; (pl.) (the) rest, (the) others
    \item[alligō (1)] tie, bind, fasten (one thing to another); secure, fasten together, tie up; bond, unite, hold together; put in chains, restrain; immobilize, pin down
    \item[alō, alere, aluī, al(i)tum] \marginnote{*}feed, nourish; support, sustain, maintain, nurture; rear, bring up, raise
    \item[alter, altera, alterum] \marginnote{*}a second, a further, one other, other; the second, the next; (with negation) either (of two); (referring to more than two) some other, another, other
    \item[āmēns, āmentis] out of one's senses, insane, demented, out of one's mind; frantic, distracted, very excited
    \item[āmittō, āmittere, āmīsī, āmissum] \marginnote{*}send away, dismiss, part with; give up, abandon; pass over, forgive; fail to catch or hold, miss, let slip; let go, lose sight of, lose track of; incur the loss of, forfeit, lose
    \item[amplus, ampla, amplum] \marginnote{*}having ample size, bulk, or extent; large, spacious, ample; impressive in size and appearance, magnificent; great, extensive, powerful, intense; distinguished, eminent, great, impressive; comprehensive, all-embracing, large, full, unrestricted
    \item[an] \marginnote{*}(introducing direct questions, often with an added notion of surprise, indignation, or strong feeling) Can it really be that\dots ? Is it really the case that\dots ?; (introducing a possible answer to a question just asked) Or\dots? Is it\dots?, but often best untranslated; (introducing continuations of multiple questions) or; (in an indirect question) whether, if; \textit{haud scio an} I am inclined to think, probably
    \item[anima, animae, f.] \marginnote{*}air, breath; life, soul
    \item[animaduertō, animadvertere, animadvertī, animadversum] \marginnote{*}direct the mind towards; pay attention to, attend to, heed; become aware of, observe, notice; judge, appraise, estimate; censure, criticize, find fault with
    \item[animal, animālis, n.] \marginnote{*}a living being, animal; (sometimes) non-human animal
    \item[animus, animī, m.] \marginnote{*}mind (as opposed to body), soul; mind, consciousness, intelligence, spirit; design, purpose, intention; will, desire; (usually plural) anger, animosity; courage, pride, spirit, morale; disposition, character
    \item[annus, annī, m.] \marginnote{*}a year, twelve months; (pl.) age
    \item[ante] \marginnote{*}\textit{adverb} in front, in front of one; before, in advance, ahead; (with difference in time expressed by ablative or accusative) before this, ago
    \item[ante] \marginnote{*}\textit{preposition with accusative} before, in front of; before (a time), by; (before in preference or rank) above, more than
    \item[antehāc] before this time, un until now, previously; in the past, before now
    \item[aperiō, aperīre, aperuī, apertum] \marginnote{*}open, open up; clear, open, repair; uncover, lay bare, reveal; make available or possible, put at one's disposal
    \item[appāreō (2)] \marginnote{*}appear, to be clear, to be evident; come to hand, turn up; come into sight, appear
    \item[appellō (1)] \marginnote{*} address, speak to, accost; appeal to (for help), call on, beseech, apply to; name, call by name
    \item[applicō (1)] bring (something) into contact with (something else), lean (something) against (another thing); fit on, attach; bring to bear (on), apply (to); assign, set; accomodate, adapt
    \item[aptus, apta, aptum] \marginnote{*}composed, fitted together; tied, fastened, bound; associated, connected; prepared, equipped, ready; handy, convenient, suitable for use, useful; efficient, good at, fitted for, able to
    \item[apud] \marginnote{*}(preposition with the accusative) at, near, in the area of, close to, next to, by, besides; among, in the presence of; at the house of, at the residence of
    \item[arbitror (1)] \marginnote{*}observe, notice, witness; judge, decide; consider, reckon; think, judge, imagine, be of the opinion
    \item[Archimēdēs, Archimēdis, m.] famous mathematician and inventor of the third century BCE
    \item[argūmentum, argūmentī, n.] proof, argument, process of reasoning; conclusion based on inference, deduction; motive, basis, reason; narrative, story
    \item[arithmētica, arithmēticae, f.] arithmetic; science or study of arithmetic
    \item[articulus, articulī, m.] a joint (of a limb, a finger, or a toe); a knuckle; limb, member, finger; a juncture, critical moment, point in time; clause, section
    \item[aspiciō, aspicere, aspexī, aspectum] \marginnote{*}catch sight of, observe, notice; look at, look upon, behold, gaze upon; examine, inspect, look over; perceive, appreciate, see (mentally); think about, consider, investigate
    \item[assēnsiō, assēnsiōnis, f.] approval, approbation, applause; assent (to the truth of an idea, statement, proposition), agreement; belief
    \item[assentior, assentīrī, assēnsus sum] agree in opinion, assent, approve; admit the truth of, agree
    \item[assequor, assequī, assecūtus sum] follow, go after; overtake, catch up to; come upon; attain (to), acquire, achieve, win; succeed in bringing about, achieve
    \item[assevērō (1)] declare, affirm, assert emphatically; proclaim
    \item[assideō, assidēre, assēdī, assessum] sit by, sit near; sit in council, be present in court; sit beside, watch over; camp near; besiege; dwell close (to), adjoin, be situated (near)
    \item[assiduē] continually, regularly, constantly
    \item[assignō (1)] allot, apportion, distribute as a portion, allocate, assign; award, confer, bestow; ascribe, impute, put down to (someone)
    \item[assūmō, assūmere, assūmpsī, assūmptum] take or use as an addition, insert, add; take possession of, make one's own, take; acquire, gain, take on, take up; adopt, take; derive, draw, borrow; take for granted, assume
    \item[astronomia, astronomiae, f.] astronomy, the science of the heavenly bodies
    \item[at] \marginnote{*}but; however, on the other hand (\textit{at} marks a stronger contrast than \textit{sed})
    \item[āter, ātra, ātrum] black, dark-colored; devoid of light, dark; discolored, stained; sordid, squalid; black and blue, bruised; unlucky, ill-omened
    \item[atque] \marginnote{*}and; and also, and what is more; and in fact, and even
    \item[atquī] but, and yet, nevertheless; all the same
    \item[attendō, attendere, attendī, attentum] pay attention, be attentive; listen carefully, pay close attention to; examine or study closely; look into carefully
    \item[attingō, attingere, attigī, attactum] touch, make contact with; lay a hand on, assault, attack; be contiguous to, be next to, adjoin; reach, stretch as far as; touch, reach, make contact with, strike; attack, afflict, affect; reach, arrive at, enter; attain
    \item[audiō (4)] \marginnote{*}hear; listen to
    \item[augeō, augēre, auxī, auctum] \marginnote{*}increase, enlarge, extend, swell, make larger; increase in value or amount; raise, cause to grow, make grow; stengthen, make stronger; help, assist; equip, furnish, provide
    \item[aut] \marginnote{*}or (usually used to introduce exclusive alternatives); \textit{aut\dots aut} either\dots or
    \item[autem] \marginnote{*}(almost always postpositive) on the other hand, while; but; moreover, also, too, furthermore; and in fact, and indeed
    \item[author, authoris, m.] (non-classical equivalent for \textit{auctor, auctōris}) person with power or authority in some situation; vendor, seller; guarantor; witness; authority, spokesperson; advocate, supporter; author, creator, source
    \item[automatum, automatī, n.] (sometimes \textit{automaton} in the singular nominative and accusative) automaton, a machine
    \item[āvocō (1)] call off, call away, summon away; take away, withdraw; turn aside, divert; dissuade, divert; distract, interrupt
    \item[bīlis, bīlis, f.]  bile, fluid secreted by the liver; anger, ill temper, spleen; madness, insanity
    \item[blandus, blanda, blandum] charming, ingratiating, attractive (influencing others by flattery, coaxing, etc.); winning, persuasive, propitiatory; alluring, seductive; insincere, fawning; gentle, tame, affectionate; calm, gentle; pleasing, sweet, soft
    \item[bonitās, bonitātis, f.] moral excellence, good behavior, excellence; kindness, benevolence, generosity; goodness, excellence, good quality
    \item[bonus, bona, bonum] \marginnote{*}good, efficient, expert; morally good, well-behaved, worthy, fine; obliging, accommodating, kind(ly), gracious, good
    \item[brāc(c)hium, brac(c)hiī, n.]  the forearm, lower arm (from elbow to hand); arm (from shoulder to hand)
    \item[cadāver, cadāveris, n.] a dead body, corpse, carcass
    \item[cadō, cadere, cecidī, cāsum] \marginnote{*}fall, fall over, fall down; descend, sink, set; droop, hang down; be overthrown, fall, be destroyed
    \item[caelum, caelī, n.] \marginnote{*} sky, heavens
    \item[calēscō, calēscere, ———, ———] grow warm, grow hot; be heated, be hot
    \item[calidus, calida, calidum] warmed, hot; warm
    \item[callidus, callida, callidum] practiced, expert, experienced; adroit, skilled; clever, ingenious; cunning, wily
    \item[calor, calōris, m.] warmth, heat; glow; fever
    \item[capax, capācis] wide, large, spacious, roomy, capacious, able to hold a lot
    \item[capessō, capessere, capessīvī/capessiī, capessītum] take hold of, grasp; snatch, seize hold of; apprehend (with the mind or senses), grasp; enter on, engage in
    \item[captīvus, captīvī, m.] prisoner (usuall of war), captive
    \item[caput, capitis, n.] \marginnote{*}head; (transferred to things) top, summit, end; (metonymically) a person or animal
    \item[caro, carnis, m.] (sometimes also \textit{carnis} in nominative singular) flesh, skin; meat
    \item[causa, causae, f.] \marginnote{*}legal case, trial; case, claim; motive, reason; cause; object, purpose
    \item[caveō, cavēre, cāvī, cautum] \marginnote{*}be on one's guard, take care, take precautions; beware, guard against, avoid
    \item[cēra, cērae, f.] wax, beeswax
    \item[cerebellum, cerebellī, n.] brain
    \item[cernō, cernere, crēvī, crētum] \marginnote{*}separate, sift; distinguish, see, discern, perceive; decide, determine; look at, examine
    \item[certō] certainly, without doubt, for a fact; firmly, surely, in a manner that can be depended on; \textit{certō scīre} know for certain, know for a fact
    \item[certus, certa, certum] \marginnote{*}fixed, settled, definite; determined, resolved, certain; a particular, a certain; indisputable, about which there is no doubt
    \item[cessō (1)] hesitate, be slow, hold back; desist, cease, stop; do nothing, be idle, delay, loiter, be remiss
    \item[cēterus, cētera, cēterum] \marginnote{*}the other, remainder, rest; (pl.) the others, rest, remaining
    \item[c(h)arta, c(h)artae, f.]  a leaf of papyrus, paper; (pl.) pages, writings
    \item[chim(a)era, chim(a)erae, f.] a mythological monster, part lion, part snake, part goat
    \item[cieō, ciēre, cīvī, citum] cause to go, move, set in motion, stir up; excite, give rise to; provoke, cause; summon, muster; call on by name, invoke; appeal to, call upon
    \item[circā] \marginnote{*}(adverb or preposition with the accusative) round about, around; round, in a circle; in the company of, with; near, close to
    \item[circumscrībō, circumscrībere, circumscrīpsī, circumscrīptum] draw a line or circle around; confine within specific limits; restrict; delimit; exclude, rule out; define, outline (in words); exclude, rule out; abridge, write in concise form
    \item[clārus, clāra, clārum] \marginnote{*}clear, bright, shining, gleaming; (of sound) loud, sonorous; clear, seeing clearly; dinstinct, unambiguous, clear, plain; well-known, notorious
    \item[coelum, coelī, n.] \marginnote{*}(post-classical form of \textit{caelum}) sky, heavens
    \item[cōgitātiō, cōgitātiōnis, f.] \marginnote{*}the act or process of thinking, reflection, thought; the outcome of thinking, an idea, thought; reflection; preoccupation, consideration
    \item[cōgitō (1)] \marginnote{*}think, think about; consider thoroughly, ponder, weigh, reflect upon, deliberate, think through
    \item[cognitiō, cognitiōnis, f.] \marginnote{*}the act of getting to know, acquiring or possession of knowledge; an idea, notion; study, investigation
    \item[cognoscō, cognoscere, cognōvī, cognitum] \marginnote{*}get to know, learn, find out; study, master, acquire knowledge of; (in perfect tenses) know (i.e. having learned, one knows)
    \item[cōgō, cōgere, coēgī, coāctum] \marginnote{*}drive together, bring together; assemble, muster; collect, summon, convene, gather; compel, force, constrain
    \item[cohibeō (2)] hold together, secure; keep secret, contain; restrain, hold back, check, stop, prevent; control, suppress; withold
    \item[collābor, collābī, collāpsus sum] slip, fall down, collapse; give way, fail; collapse, end suddenly, be lost
    \item[colligō, colligere, collēgī, collectum] \marginnote{*}gather together, collect, pick up, harvest; get hold of, get possession of; accumulate, amass, build up; hold, keep together; assemble, bring together
    \item[color, colōris, m.] \marginnote{*}color, hue, tint (of an object); a particular color
    \item[commoueō, commovēre, commōvī, commōtum] shake, agitate, move vigorously; stir up, stir, move, budge; upset, jolt, disturb; awake, wake up (someone else)
    \item[commūnis, commūne] \marginnote{*}joint, common, shared, general; neutral, impartial
    \item[compāgēs, compāgis, f] the action of joining together, binding, bond, tie; joint, suture; a composite structure, framework, assemblage
    \item[comparō (1)] \marginnote{*}place together, align; compare, match, set against; treat as equal, put in the same class with
    \item[comperiō, comperīre, comperī, compertum] find out by investigation, learn, discover, ascertain; prove, establish, verify
    \item[complector, complectī, complexus sum] embrace, hug; welcome, take up, adopt; seize, grasp, grip, cling to; comprehend, take in, understand; include, involve, associate, bring in
    \item[compōnō, compōnere, composuī, compositum] \marginnote{*}place things together, add together; pack up, store or collect together; settle in a place or position; arrange in order, dispose, organize, draw up; construct, build, put together
    \item[compositus, composita, compositum] composed of, made of; composite, blended, compound; well-arranged, well-ordered; orderly, well-disciplined, law-abiding; practiced, studied; calm, undisturbed, placid, sedate
    \item[comprehendō, comprehendere, comprehendī, comprehensum] take hold of, grip, catch hold of; take hold, take root; catch; fasten, unite, tie or join together, hold together; find, seize, discover, detect; enclose, surround, include
    \item[comprehēnsiō, comprehēnsiōnis, f.] the action of taking hold, grasping; arrest, apprehension; scope, range; apprehension, perception, mental grasp, understanding
    \item[concēdō, concēdere, concessī, concessum] \marginnote{*}go away, withdraw, retire; go over, transfer; give place, make way, yield, defer; give place to, concede, surrender, give up, hand over
    \item[concipiō, concipere, concēpī, conceptum] take in, absorb, catch; conceive, become pregnant; produce, form, bring into existence; contain, hold; grasp, conceive (in the mind), imagine, form an idea of; devise; undertake, assume
    \item[conclūdō, conclūdere, conclūsī, conclūsum] shut up, close in, imprison, enclose, confine; brind to an end, conclude, finish; infer, deduce, conclude
    \item[confīdō, confīdere, confīsus sum] put one's trust in, have confidence in; trust confidently, be sure
    \item[confirmō (1)] make firm, secure; assure, reassure, make confident, encourage; confirm, establish firmly, prove
    \item[conflō (1)] blow on, raise, start, ignite, set fire to; arouse, stir up; form, invent, concoct; make (by melting or heat), cast, weld; assemble, run up
    \item[confundō, confundere, confūdī, confūsum] pour together, mingle, mix, blend; mix up, stir up; jumble, upset, disorder, confuse; ruin, destroy; blur, obscure, disfigure; disconcert, trouble, dismay, upset (a person)
    \item[coniciō, conicere, coniēcī, coniectum] throw or put together, bring together; throw, cast, hurl; dispatch, make (a person) go; assign, insert; conjecture, guess
    \item[connīueō, connīvēre, connivī/connīxī, ———] (also in forms \textit{cōnīveō}, etc.) close (the eyes) tightly, close (eyes) in sleep; turn a blind eye (to), overlook, ignore
    \item[cōnsentāneus, consentānea, consentāneum] fitting, agreeable, appropriate, consistent; constant in principles, consistent (of a person)
    \item[cōnsīderātiō, cōnsīderātiōnis, f.] action of looking, gaze, inspection; contemplation, consideration; examination, contemplation, consideration
    \item[cōnsīderō (1)] look at closely, attentively, or carefully, inspect, examine, investigate; notice, remark; bear in mind, take into consideration; take care, bear in mind (that)
    \item[cōnsistō, cōnsistere, cōnstitī, cōnstitum] \marginnote{*}stand still, stand, halt, stop, stay in place; take a stand, post oneself; congeal, set; pause, linger, dwell on; take a position, take a stand (for fighting)
    \item[cōnspicuus, conspicua, conspicuum] clearly seen, visible, in full view; remarkable, striking, attracting attention; notable, famous, illustrious
    \item[cōnstanter] without change; regularly, steadily; firmly, resolutely, in a determined manner; faithfully, loyally, steadfastly
    \item[cōnstō, cōnstāre, cōnstitī, cōnstātum] \marginnote{*}stand together; stand still, remain constant, be steady; be composed, consist of; be dependent upon, be based upon; exist, continue, last; remain, continue, remain unchanged; be fixed or established; be manifest, be apparent, be known, be certain, be established (often impersonal in the third person singular); (of statements, views, etc.) be consistent, agree; (of sums) be correct, balance
    \item[cōnsuēscō, cōnsuēscere, cōnsuēvī, cōnsuētum]  become accustomed or used (to something), form a habit; make accustomed or used; (perfect) be in the habit of, be used to
    \item[cōnsuētūdō, cōnsuētūdinis, f.] a custom, habit, usual practice, usage; normal state or condition; convention, custom (as a source of law); linguistic usage, normal manner of speaking
    \item[cōnsūmō, cōnsūmere, cōnsūmpsī, cōnsūmptum] \marginnote{*}destroy, wear away, consume; kill; reduce, make smaller, reduce to nothing; weaken severely, wear down, exhaust; use up, expend; eat, devour; expend, use up, employ; spend (money, resources, time, effort)
    \item[contemplō (1)] look at, examine visually, gaze at; observe, notice, study; ponder, consider, contemplate
    \item[contendō, contendere, contendī, contentum] stretch, bend, draw tight; throw vigorously, hurl, shoot; strain, exert, make an effort, strive, exert oneself; go quickly, hasten, press forward
    \item[contineō, continēre, continuī, contentum] hold together, link, join; fasten, secure; keep together, sustain; maintain; retain, keep; surround, enclose, embrace; contain, include; imply, involve; comprise, consist in
    \item[contingō, contingere, contigī, contactum] \marginnote{*}come into contact with, touch; border on, be contiguous with, be in contact with, be connected with; arrive at, reach; hit, strike; extend to, reach to; attain, achieve; affect, move, touch, concern; (intransitive with dative) fall to one's lot, happen, be granted; (absolute) come about, happen
    \item[continuātus, continuāta, continuātum] uninterrupted, unbroken; consecutive; (with dative) contiguous, adjacent to
    \item[contrārius, contrāria, contrārium] opposite, on the opposite side; reverse, opposite to normal; opposed, in opposition; hostile, adverse, opposing; contrary, antithetical, opposite in kind or type; incompatible, opposite in intension
    \item[contrectō (1)] touch, handle (an object), come in contact with, feel; deal with, handle (a subject), apply oneself to
    \item[contumax, contumācis] proud and unyielding, stubborn, defiant (usually in a negative sense); wilfully disobedient, contumacious
    \item[corporeus, corporea, corporeum] \marginnote{*}physical, material, corporeal, consisting of a body or endowed with a body; related to the body, bodily
    \item[corpus, corporis, n.] \marginnote{*}the body of a living creature; body (as opposed to soul or mind); any physical object (as opposed to immaterial things)
    \item[crassus, crassa, crassum] solid, thick, dense; fat, plump, stout; coarse, rough; rude, homely (opposed to elegant or fine); insensitive, dull, stupid, crass (opposed to perceptive or intelligent)
    \item[crēdō, crēdere, crēdidī, crēditum] \marginnote{*}commit to, entrust; confide; (with dative) trust, rely on, have faith in, believe, give credence to
    \item[crēdulitās, crēdulitātis, f.] ready belief, credulity, trustfulness; rash confidence
    \item[creō (1)] \marginnote{*}procreate; bring into being, create, produce; institute, establish (an idea or custom)
    \item[crēscō, crēscere, crēvī, crētum] \marginnote{*}come into being, spring up; develop; lengthen, increase, swell, expand; progress, advance; increase in numbers, amount, or the like; multiply
    \item[cucurbita, cucurbitae, f.] a gourd, a pumpkin; a cupping glass (from the shape); a dolt, `pumpkin-head'
    \item[culpa, culpae, f.] \marginnote{*}responsibility, blame, fault; wrongdoing, offence, midsconduct; failure, neglect, error, mistake
    \item[cum] \marginnote{*}(preposition with ablative) with, together with, along with
    \item[cum] \marginnote{*}(conjunction with indicative) at that time, when; from that time, since; (conjunction with subjunctive) under the circumstances when, when; because, since; although
    \item[cunctor (1)] delay, hang back; hesitate; tarry, linger
    \item[cunctus, cuncta, cunctum] \marginnote{*}the whole of, all; total, complete
    \item[cupiō, cupere, cupīvī/cupiī, cupītum] \marginnote{*}wish for, desire, want; wish well to, favor, be well disposed to (plus dative)
    \item[cūra, cūrae, f.] \marginnote{*}anxiety, worry, care, distress; carefulness, pains, (serious) attention; solicitude, concern
    \item[cūrō (1)] \marginnote{*}watch over, look after, care for; tend, rear; treat (a sick person, an illness, a wound), tend to, cure; undertake, have done, see to it (that)
    \item[dē] \marginnote{*}(preposition with ablative) down from; away from, off; from; concerning, about; of, out of
    \item[dēbeō (2)] \marginnote{*}be under an obligation, be indebted, owe; ought, should
    \item[dēceptor, dēceptōris, m.] a betrayer; a deceiver
    \item[dēcipiō, dēcipere, dēcēpī, dēceptum] deceive, mislead, dupe; frustrate, foil, cheat; escape the notice of, elude
    \item[dēfīgō, dēfīgere, dēfīxī, dēfīxum] plant, embed, sink, bury; affix, attach; keep (one's eyes or thoughts) directed (on), fix, focus; fix (with a glance); petrify, dumbfound, render incapable of thought or movement
    \item[dēgō, dēgere, dēgī, ———] spend, pass (time, one's life); remain alive, live on, continue
    \item[deinceps] in succession, in turn; after that, after this, next, then
    \item[deinde] \marginnote{*}afterwards, then, next; from there; in the next (second, third, etc.) place
    \item[dēlābor, dēlābī, dēlāpsus] fall, drop; sink, slip down; descend, fly or glide down; flow down; fail, lose strength
    \item[dēlīberō (1)] engage in careful thought, weigh the pros and cons of; deliberate; take counsel (with someone), consult; consider carefully, think over, ponder
    \item[dēlūdō, dēlūdere, dēlūsī, dēlūsum] deceive, dupe, delude
    \item[dēmō, dēmere, dēmpsī, dēmptum] remove, take away, take off, subtract
    \item[dēnique] \marginnote{*}finally, at last, at length, in the end; lastly; in sum, in short, to sum up; in point of fact, indeed
    \item[dēnuō] anew, over again, from a fresh start; for a second time, once more, again; in turn, then again
    \item[dēpendeō, dēpendēre, dēpendī, ———] hang down (from); proceed or derive from; depend (on)
    \item[dēpōnō, dēpōnere, dēposuī, dēpositum] put down, lay down; give up, surrender, shed; let rest; deposit, lodge, give for safekeeping; abandon, resign, drop
    \item[dēprehendō, dēprehendere, dēprehendī, dēprehēnsum] intercept, seize, catch; meet suddenly or unexpectedly, come upon, surprise; discover, detect, recognize; take by surprise, overtake
    \item[dēscrībō, dēscrībere, dēscrīpsī, dēscrīptum] draw, mark out, describe, trace out; write down, transcribe, copy down; prescribe, establish
    \item[dēsidia, dēsidiae, f.] idleness, slackness, inactivity; leisure, freedom (from work or responsibility)
    \item[dēsignō (1)] mark out, trace out, draw in outline; mark; indicate, denote, designate; earmark, assign; appoint, elect; order, plan
    \item[dēsinō, dēsinere, dēs(i)ī/dēsīvī, dēsitum] \marginnote{*}leave off, desist, finish, stop, cease; come to an end, end (with or in), end
    \item[dēsuēscō] unlearn, forget about, become unaccustomed to, become unused to
    \item[dētorqueō] turn away, deflect; turn aside, divert, sway, change (often for the worse), pervert; bend out of shape, twist, distort
    \item[dētrahō, dētrahere, dētrāxī, dētrāctum] detach (by pulling), pull, draw, remove; drag away, draw off; force down, induce to come down; pull down, demolish; detach, dislodge; derive; detract (in speech), say in disparagement (of); exclude, omit, subtract, cut out
    \item[deus, deī, m.] \marginnote{*}a god, a divinity
    \item[dēvinciō, dēvincīre, dēvinxī, dēvīnctum] bind fast, tie up (tightly); fix in position, hold fast; oblige, constrain, bind (morally, emotionally, etc.); subjugate, bring into subjection; unite, bind together
    \item[dīcō, dīcere, dīxī, dictum] \marginnote{*}say, speak, utter; tell, mention, relate; affirm, declare, state, assert
    \item[differō, differre, distulī, dīlātum] \marginnote{*}carry away in different directions, scatter, disperse; confound, bewilder, distract; spread around, publish, make known; postpone, defer, put off, keep waiting; differ, be different
    \item[difficilis, difficile] \marginnote{*}hard, difficult, troublesome; intractable, obdurate, inflexible, hard to manage
    \item[difficultās, difficultātis, f.] difficulty, situation involving trouble, trouble
    \item[diffīdentia, diffīdentiae, f.] mistrust, distrust; diffidence, lack of confidence
    \item[dīligēns, dīligentis] fond (of), devoted (to); careful, attentive, diligent, scrupulous; thrifty, economical
    \item[dīmoveō, dīmovēre, dīmōvī, dīmōtum] cleave, part; cause to part, disperse; displace, remove; set aside, dismiss, divert
    \item[disciplīna, disciplīnae, f.] instruction, teaching, training, education; a branch of study, discipline; a philosophical school or sect; discipline, orderly conduct based on moral training, order (maintained in a group of people)
    \item[dispār, disparis] unequal; different (in character or kind), dissimilar
    \item[disputō (1)] argue one's case or point of view; reason out, debate, argue
    \item[distinctus, distincta, distinctum] \marginnote{*}different, distinct; definite, precise, not vague; clear, lucid
    \item[distinguō, distinguere, distīnxī, disctīnctum] separate, divide, part; mark off, distinguish, separate (from); punctuate (a process), relieve, interrupt; distinguish (from another), make or perceive a difference between; define, specify; resolve, settle
    \item[diūturnitās, diūturnitātis, f.] passage of a long period of time, lapse of time; long duration, permanence, durability, longevity
    \item[dīvellō, dīvellere, dīvellī/dīvulsī, divulsum] tear open, tear apart, tear to pieces; pull away (from), tear away (from); break up, sunder, disrupt
    \item[dīversus, dīversa, dīversum] turned, pointed, or facing in different directions; set apart, separate; distant, remote; different, differing, distinct, divergent, inconsistent
    \item[dō, dare, dedī, datum] \marginnote{*}give, offer, grant; impose, assign, appoint; concede, allow
    \item[dōnec] \marginnote{*}until, up to the time at which; as long as, while
    \item[dormiō (4)] \marginnote{*}sleep, be asleep, fall asleep
    \item[dubitātiō, dubitātiōnis, f.] uncertainty, doubt, perplexity; wavering, hesitation
    \item[dubitō (1)] \marginnote{*}be in doubt, be uncertain; doubt, waver, hesitate over
    \item[dubius, dubia, dubium] \marginnote{*}hesitatnt, undecided, wavering, faltering; uncertain, doubtful, dubious
    \item[dūcō, dūcere, dūxī, ductum] \marginnote{*}lead, conduct, draw, bring forward, guide; consider, reckon, hold, account, esteem, believe
    \item[dulcēdō, dulcēdinis, f.] sweetness; pleasantness, charm; a pleasure, something sweet or pleasant
    \item[dum] \marginnote{*}while, as long as; until
    \item[duo, duae, duo] \marginnote{*}two; a couple, a pair
    \item[dūrō (1)] make hard, harden, solidify; become hard, become solid; harden onself, steel oneself; hold out, endure, last; endure, bear with
    \item[ē, ex] \marginnote{*}out of, from within, from
    \item[ecce] \marginnote{*}behold! look! see!
    \item[ēdūcō, ēdūcere, ēdūxī, ēductum] \marginnote{*}lead or bring out or away; lead forth; draw out, extract; elicit, evoke; drain away, draw away; bring forth, conceive
    \item[efficiō, efficere, effēcī, effectum] \marginnote{*}manufacture, make, construct; compose; cause to occur, bring about; bring it about that, be the cause that; yield, produce, bear; make up, constitute; make or perform completely, finish; deduce from premises, prove, (passive) follow
    \item[effingō, effingere, effīnxī, effictum] shape, mould, fashion, form; portray, depict, represent; reproduce, copy, imitate
    \item[ēmanō (1)] pour fourth, flow (out, forth); arise, emanate; become diffused, spread (out, about, around)
    \item[ēmittō, ēmittere, ēmīsī, ēmissum] \marginnote{*}send out, send forth, dispatch; make known, publish; release, free, let go, discharge, let loose; let fly, launch, shoot; give off, emit; utter (words, sounds)
    \item[ēnatō] swim away, escape by swimming; float up or forth
    \item[enim] \marginnote{*}(usually postpositive) for; for instance, namely, that is to say; I mean, in fact
    \item[ēnītor, ēnītī, ēnīsus/ēnīxus sum] struggle up or out; take pains, strive, exert onself; give birth to, produce
    \item[eō] \marginnote{*}(adverb) for that reason, consequently, therefore; (with comparatives, often correlative to \textit{quo} or another relative) by that degree, so much, to such a degree; there, to that place
    \item[eō, īre, iī/īvī, itum] \marginnote{*}go, proceed, make one's way, move, pass
    \item[equidem] (with the first person singular or in replies about the speaker or writer) I for my part, I personally, I and no one else (often better left untranslated); (as an emphatic particle not with the first person singular) indeed, in truth
    \item[ergō] \marginnote{*}for that reason, therefore, then, so, accordingly; (in questions) in that case, then; well then, all right then; (marking a concession and objection) yes, but; ah, but
    \item[errō (1)] \marginnote{*}wander about, roam; float, flit, drift; be in doubt, be uncertain, waver; go astray, wander from the course, go wrong; think or act in error, be mistaken, be incorrect; go wrong morally
    \item[error, errōris, m.] \marginnote{*}a wandering, straying, strolling; uncertainty, doubt, perplexity; mistake, error; moral lapse
    \item[et] \marginnote{*}and; even, also; \textbf{et\dots et} both\dots and
    \item[etenim] and indeed, the fact is, for
    \item[etiam] \marginnote{*}still, yet, even now; yet again; also, in addition, as well, too; (indicating an affirmative answer to a question) indeed, yes
    \item[ēversiō, ēversiōnis, f.] the action of overturning or upsetting; expulsion, turning out; destruction
    \item[ēvertō, ēvertere, ēvertī, ēversum] overturn, turn upside down, reverse, throw back; throw down, cause to fall violently, bring down; ruin, overthrow, destroy, upset
    \item[ēvidēns, ēvidentis] perceptible; clear, obvious; open, unconcealed
    \item[ex] see \textit{ē}
    \item[excitō (1)] call out, summon forth, bring out; wake, rouse; excite, stir, agitate, rouse to vigorous physical or mental activity
    \item[exclūdō, exclūdere, exclūsī, exclūsum] shut out, keep out, exclude; cut off, remove, separate; debar, hinder, prevent; leave out, omit
    \item[excōgitō (1)] think out, contrive, devise, invent
    \item[exemplum, exemplī, n.] \marginnote{*}a sample, specimen; example, instance; precedent, pattern, model, parallel; a copy, reproduction, transcript
    \item[exeō, exīre, exiī/exīvī, exitum] \marginnote{*}go out, go forth; go away, depart, move away; withdraw, retire
    \item[exiguus, exigua, exiguum] small (in size, amount, or quantity), scanty, little; short, brief; trivial, slight, petty
    \item[expergiscor, expergiscī, experrectus sum] wake up, become awake; rouse oneself
    \item[explicō (1)] unfold, untwine, straighten; level out, smooth; disentangle, settle, solve; make clear, reveal to view, make known, give an account of
    \item[explōrō (1)] reconnoitre, inspect; inquire into, investigate; ascertain (that); ensure (that); \textit{explōrātum habēre} know for certain, be sure
    \item[ex(s)istō, ex(s)istere, ex(s)titī, ex(s)titum] appear, arise; come forward, present oneself; come into being, emerge, arise; prove to be, show oneself
    \item[ex(s)olvō, ex(s)olvere, ex(s)oluī, ex(s)olūtum] unfasten, undo, loose; open, solve; set free, release; put an end to, do away with; perform, discharge; pay; award
    \item[ex(s)pectō] wait for, await; look forward to, hope for; expect; wait to see or know
    \item[ex(s)pīrō (1)] breathe out, exhale, emit; breathe one's last, die, expire, come to an end, perish;
    \item[extendō, extendere, extendī, extentum/extēnsum] make taut, stretch; stretch out, thrust out; lie full length; prolong, extend the duration of, continue; strain, exert oneself
    \item[extēnsiō, extēnsiōnis, f.] distance covered by stretching out, span, extent
    \item[externus, externa, externum] outward, external, situated on the outside; extraneous, not intrinsic; foreign, coming from abroad, alien
    \item[extrā] \marginnote{*}(adverb or preposition with the accusative) on the outside, externally; out, outward; beyond the boundaries of, outside, out of, beyond the scope of
    \item[facilis, facile] \marginnote{*}easy, not requiring great effort; straightforward, simple; involving no difficulties, tractable, light, tolerable
    \item[faciō, facere, fēcī, factum] \marginnote{*}make, build, construct; produce, bring forth, cause to grow; write, compose, form; appoint, institute, create; do, cause, bring about
    \item[facultās, facultātis, f.] ability, power, capacity, skill; strength, support; a particular skill, faculty, talent, endowment; power, potency, property (as an ability, not a piece of land); possibility, opportunity, chance
    \item[fallō, fallere, fefellī, falsum] \marginnote{*}deceive, trick, mislead; fail to support, prove treacherous; belie, dissapoint, fail to come up to, fail to execute; escape the notice of, be unperceived by; go unnoticed; avoid, elude, escape
    \item[falsitās, falsitātis, f.] falsity, the quality of being false; a falsehood, a false statement, belief, or similar (non-classical)
    \item[familiāritās, familiāritātis, f.] close friendship, intimacy, familiarity; close relationship, kinship (of family or friends); the state of being well-known, familiarity
    \item[fateor, fatērī, fassus sum] \marginnote{*}concede, acknowledge, admit; admit guilt, confess; profess, declare, avow; assent, say yes, agree to
    \item[fatīgō (1)] tire out, weary, exhaust; work at persistently, keep at (something or someone); press hard, harass; assail, worry, plague; expend, use up, exhaust
    \item[favus, favī, m.] a honeycomb; hexagonal paving stone
    \item[fenestra, fenestrae, f.] window; any hole, opening, aperture
    \item[feriō, ferīre, ———, ———] strike, hit; flot, beat; knock
    \item[ferō, ferre, tulī, lātum] \marginnote{*}bring, carry, take, convey, lead; lift up, raise; bear, endure, sustain; allege, claim; relate, tell, report
    \item[ferveō, fervēre, ferbuī, fervitum] be intensely hot, be boiling hot; boil; be warm, be hot; be inflamed, be feverish; be active or busy, move swiftly or in an agitated manner; be eager or enthusiastic
    \item[fictilis, fictile] made of clay, earthenware, terracotta
    \item[fictītius, fictītia, fictītium]  (in classical Latin, the stem is \textit{fictīc}-) artificial, not natural; unreal, made up
    \item[figmentum, figmentī, n.] an image, something made up; fiction, invention, unreality
    \item[fīgō, fīgere, fīxī, fīctum] fix, fasten, drive, thrust in; attach, affix; post, erect, set up
    \item[figūra, figūrae, f.] a form, shape, figure, outline; posture, pose, attitude
    \item[figūrō (1)] form, fashion, shape, mould; transform; arrange; depict, represent, make a likeness of; (sometimes with \textit{sibi}) form a mental image of, imagine
    \item[fingō, fingere, fīnxī, fictum] \marginnote{*}make by shaping, mould, fasion; produce artificially, make an imitation of, counterfeit; form out of original material, create; produce (offspring); make a likeness of, represent; tidy, arrange, groom; transform; modify character or behavior of (someone or something), guide, mould, influence, train; compose (literary works); invent, coin; devise, contrive, think up; bring about, produce; (with \textit{sibi}, \textit{animo}, or similar) visualize, form a mental picture of, conjure up in the mind, imagine; suppose that, imagine, assume that; make up, invent, fabricate; feign, simulate
    \item[fīō, fīerī, factus sum] \marginnote{*}(infinitive also \textit{fierī}) take place, occur, arise, happen, come about; be done, be made, be created, be produced, be prepared
    \item[firmus, firma, firmum] strong, durable; robust, sturdy; sound, strong, in good health; firm, steady, secure
    \item[flexibilis, flexibile] pliant, flexible, yielding; adaptable; tractable, pliable, open to influence or change
    \item[flōs, flōris, m.] \marginnote{*}flower, blossom, bloom
    \item[focus, focī, m.] a fireplace, hearth; (as the symbol of life in a home) one's fireside, one's home; sacrifical hearth or altar
    \item[fōns, fontis, m.] \marginnote{*}a spring, fountain; well, source
    \item[for, fārī, fātus sum] \marginnote{*}speak, talk; say, tell; tell of, reveal (something)
    \item[forma, formae, f.] \marginnote{*}a form, contour, figure, shape; appearance, looks; beauty, good looks; a geometrical figure
    \item[formō (1)] shape, fashion; form, build; change the appearance of, transform; adapt, modify
    \item[forsan] \marginnote{*}perhaps, possibly, maybe, it may be
    \item[fortasse] \marginnote{*}perhaps, possibly, maybe, it may be
    \item[fortassis] perhaps, possibly, maybe, it may be
    \item[fortis, forte] \marginnote{*}strong, hardy, vigorous, tough; healthy, in good physical health, robust; brave, bold, resolute; (of arguments or ideas) convincing, strong, forceful; (colloquial of people and conduct) honorable, decent, worthy
    \item[frāgrantia, frāgrantiae, f.] odor, scent, a smell; fragrance, sweet smell
    \item[frequenter] in large numbers, in crowds; densely, thickly; on many occasions or at frequent intervals, repeatedly, often, frequently; commonly, generally, widely
    \item[frigidus, frigida, frigidum] cold, chilly, cool; sluggish, torpid
    \item[fruor, fruī, frūctus sum] \marginnote{*}enjoy the produce of or proceeds from, derive advantage from; avail oneself of, enjoy; be blessed with, enjoy; delight in, find enjoyable, enjoy
    \item[frustrā] \marginnote{*}to no purpose, in vain, to no avail; without reason or purpose, mistakenly, needlessly
    \item[fundāmentum, fundāmentī, n.] a foundation, substructure for a building; a basis, foundation; a fundamental necessity, something indispensible, a requirement
    \item[funditus] (\textit{adverb}) from the very bottom, from the foundations, by the roots; to the ground (in context of destruction); utterly, completely, without exception
    \item[gaudeō, gaudēre, gāvīsus sum] \marginnote{*}rejoice, be glad, be joyful
    \item[generālis, generāle] of or belonging to a kind, shared by a class, common to a group or type; of universal application, general
    \item[genius, geniī, m.] guardian spirit of a person or family; spirit, inclination, genius, inner nature
    \item[genus, generis, n.] \marginnote{*}stock, descent, origin; family line; nationality, race, nation; a generation, age; a class, type, kind, variety, group; gender
    \item[geōmetria, geōmetriae, f.] geometry
    \item[gurges, gurgitis, m.] a swirling mass of water, whirlpool, eddy; waters of a river, sea, etc., a flood
    \item[gustus, gustūs, m.] the action of tasting; the sense of taste; a flavor, taste; a small portion (of food or drink), a taste; a portion, a specimen, sample
    \item[habēna, habēnae, f.] a halter, bridle, rein
    \item[habeō, habēre, habuī, habitum] \marginnote{*}have, own, possess; hold, have in hand, have under one's control; have on, wear, carry; conduct, hold; regard, look on, treat as
    \item[hāctenus] to this point in space, so far; to this point in time, up until now; to this extent or degree, so far
    \item[haereō, haerēre, haesī, haesum] cling, adhere, stick, stick fast; be fixed, hold on tightly; be contiguous in space, join on (to); be inherent (in) or connected (with); remain in place, stay put, linger; persist, continue; be at a loss, be in difficulties, be stuck
    \item[hālitus, hālitūs, m.] exhalation, vapor; breath, the air that one breathes in
    \item[haud] \marginnote{*}not, not at all, by no means
    \item[hauriō, haurīre, hausī/hauriī, haustum/haurītum] draw up, draw out, draw, scoop up; derive, draw from; wound (in such a way as to draw blood); drink, imbibe, drain; swallow; take in, absorb; consume, devour; swallow up, engulf; use up, consume
    \item[hesternus, hesterna, hesternum] of yesterday, yesterday's
    \item[hīc] \marginnote{*}in this place, here; in the present case or circumstances, in this case, in this situation
    \item[hic, haec, hoc] \marginnote{*}this, these (of something near in place, time, or uppermost in thought); the last, the recent; the latter (as opposed to \textit{ille} meaning `the former'); the following (referring to something about to be said)
    \item[hodiē] \marginnote{*}today
    \item[homō, hominis, m.] \marginnote{*}a human being, person
    \item[hūmānus, hūmāna, hūmanum] \marginnote{*}of or belonging to a person or people in general, human; civilized, humane (as opposed to nature or wild animals); cultured, cultivated; kindly, considerate, morally worthy of humanity; merciful, indulgent
    \item[hyemālis, hyemāle] (classical \textit{hiemāl}-) of or belonging to winter, winter-; wintry, stormy
    \item[iaceō, iacēre, iacuī, iacitum] \marginnote{*}lie, lie down; rest, recline; lie dead, be killed, die
    \item[iam] \marginnote{*}already now, just now; soon now
    \item[īdem, eadem, idem] \marginnote{*}the same, identical (as something previously mentioned or under discussion)
    \item[ideō] \marginnote{*}for that reason, therefore
    \item[igitur] \marginnote{*}(almost always postpositive) in that case, then; consequently, therefore, then, so; accordingly, so then
    \item[ignis, ignis, m.] \marginnote{*}(ablative singular is normally \textit{igni} but sometimes (especially later) \textit{igne}) fire; (metaphorically) fire or glow of passion; lightning, a lightning flash; fever, high temperature
    \item[ignoscō, ignoscere, ignōvī, ignōtum] (with accusative of offence and dative of offender) to pardon, forgive, excuse; overlook, allow, indulge, make allowance
    \item[īlicō] (also \textit{illicō}) on the spot, just here (there); at that moment, at once, there and then
    \item[ille, illa, illud] \marginnote{*}that, those; the famous, the well-known (or infamous); the former (as opposed to \textit{hic} as the latter)
    \item[illūsiō, illūsiōnis, f.] a mocking, the action of ridicule; saying the opposite of what is meant, irony; (post-classical) a deception of any kind, an illusion
    \item[immittō, immittere, immīsī, immissum] cause to go, send (to a place); send (against or into); set (against); cast, direct; put in, insert, introduce; grant entry, let in, admit
    \item[immō] \marginnote{*}(used to introduce a correction of something said or implied previously) rather, more correctly, more precisely, on the contrary, no indeed
    \item[immōbilis, immōbile] immovable, fixed; unmoving, motionless; unalterable, fixed, unchanging; slow to move; imperturbable, steadfast, emotionally steady
    \item[imperfectiō, imperfectiōnis, f.] imperfection, flaw
    \item[imperfectus, imperfecta, imperfectum] unfinished, not completed; imperfect
    \item[imprōvidē] without forethought, unwarily; thoughtlessly, without thought for the future
    \item[imprūdēns, imprūdentis] having no knowledge of something, ignorant; unaware of what one is doing, unwitting; unintentional; unaware of what will happen, not foreseeing; foolish, incautious, lacking in judgment or discretion
    \item[impōnō, impōnere, imposuī, impositum] \marginnote{*}place, put, or set on or; impose, force, inflict; assign, confer; post, station
    \item[imāginor (1)] form a mental picture of, imagine; give or produce an image
    \item[imāginārius, imāgināria, imāginārium] unreal, imaginary; fictitious, pretend
    \item[imāginātiō, imāginātiōnis, f.] the action of picturing mentally, fantasy, imagining; a mental image, something imagined
    \item[imāginātrix, imāginātricis] of or related to imagination, productive of imagination, causing imagination
    \item[imāgō, imāginis, f.] a representation, picture, likeness, image; an ancestral death mask used in Roman funerals; a reflection; an echo; illusory apparition, ghost, phantom, hallucination; mental image, representation in one's mind; description, sketch; semblance, imitation; duplicate, copy, replica; model, example
    \item[imō] see \textit{immō}
    \item[in] \marginnote{*}(with accusative) into, onto; against; (with ablative) in, on
    \item[incertus, incerta, incertum] not fixed or predetermined; subject to chance, unpredictable; not specified or defined; not yet decided, uncertain; unsure, unsafe, precarious
    \item[incidō, incidere, incidī, incāsum] \marginnote{*}fall on, onto, or into; fall, settle, impinge (on); fall over, stumble against; throw onself, rush (upon); fall into the possession or power of; happen on, chance to meet; fall into, happen into (a state or situation), enter inadvertently into, slip into; arise, occur, happen
    \item[incipiō, incipere, incēpī, inceptum] \marginnote{*}take in hand, begin, start, embark on; originate, take rise
    \item[incohō (1)] begin, commence, start, initiate; establish, found
    \item[inconcussus, inconcussa, inconcussum] unshaken, firm, steady; unbroken, secure, untroubled; steadfast, unwavering
    \item[incumbō, incumbere, incubuī, incubitum] bend forward, lean over, support oneself, lean on; apply force to (by leaning on), press (on), weigh (on or upon); bear down, press one's attack (on); fall to
    \item[incurrō, incurrere, incucurrī, incursum] rush or charge (at), make an attack (on); run or rush in; encounter, meet with, meet, run into
    \item[incēdō, incēdere, incessi, incessum] advance, march, proceed; stride, strut; advance, extend; go into or onto, enter; arise, come on, befall
    \item[incōnsīderantia, incōnsīderantiae, f.] lack of reflection or forethought, recklessness; inattentiveness, absent-mindedness
    \item[indicium, indiciī, n.] information, disclosure of information;evidence, an indication, token, symbol; omen, portent, warning
    \item[indubitātus, indubitāta, indubitātum] that cannot be doubted, certain, unquestionable; not hesitated over, confident
    \item[indulgeō, indulgēre, indulsī, indultum] be indulgent or lenient (to); allow (someone) to have their way; make allowance for; look favorably on, show kindness to; indulge, give free rein to; take pleasure in, indulge in, devote oneself to (plus dative)
    \item[industria, industriae, f.] diligence, purposeful activity, industry, zeal; an example of diligence, a purposeful activity
    \item[industrius, industria, industrium] active, dilgent, zealous, assiduous
    \item[induō, induere, induī, indūtum] put on; assume, adopt; clothe, dress in (often with ablative)
    \item[ineptiō, ineptīre, ineptīvī, ———] be foolish, be silly
    \item[ineptus, inepta, ineptum] without sense of what is fitting, lacking in judgment; foolish, silly
    \item[inextrīcābilis, inextrīcābile] impossible to disentangle or sort out; pathless, inescapable; insoluble
    \item[ineō, inīre, iniī/inīvī, initum] go into, enter; enter upon, commence, begin
    \item[īnfirmō (1)] weaken (physically or mentally); lessen destroy; refute, deny; annul, invalidate
    \item[īnfundō, infundere, infūdī, infūsum] pour in or into, pour on or over; (with ablative) fill, moisten, wet; pour down, shower; cause to extend, impart; stretch out or relax
    \item[īnfīgō, infīgere, infīxī, infīxum] implant, drive in; fasten, affix, attach; set firmly in place, plant, implant, impress; fix in the mind or memory, impress, implant
    \item[īnfīnītus, infīnīta, infīnītum] not limited, infinite, endless, boundless, unlimited; not specified, indefinite; unrestricted, absolute
    \item[ingredior, ingredī, ingressus sum] \marginnote{*}go into or onto; enter upon, commence, begin; walk, proceed (on foot); advance, assail, attack
    \item[ingēns, ingentis] huge, vast, very large (in size, number, or extent); very great (in degree or intensity); notable, momentous, of great importance
    \item[initium, initiī, n.] \marginnote{*}beginning, start, commencement
    \item[innumerābilis, innumerābile] countless, that cannot be counted, limitless
    \item[innōtēscō, innōtēscere, innōtuī, ———] become known or familiar; become famous or celebrated
    \item[inquam] \marginnote{*}say (usually introducing a direct quotation)
    \item[īnsaniō (4)] be out of one's mind, be mad, be insane; behave like someone insane, rave, act crazily
    \item[īnsidiae, īnsidiārum, f.] (plural in Latin with a singular meaning in English) an ambush, a surprise attack; treachery, a plot; a snare, a trap
    \item[īnsomnium, īnsomniī, n.] wakefulness, sleeplessness (usually plural); a vision, a dream, something seen in a dream or a trance
    \item[īnspectiō, īnspectiōnis, f.] the action of looking (at or into); a visual examination, inspection; theoretical examination, inquiry, investigation
    \item[īnstar, ———, n.] (found only in nominative and accusative singular) an equivalent in measure, appearance, effect, condition, etc., an equal, a likeness;
    \item[īnstituō, īnstituere, īnstituī, īnstitūtum] set in being, organize, put into operation; put up, erect; establish, set up; appoint; institute, originate, establish; train, instruct
    \item[īnsuper] \textit{adverb and preposition with accusative and ablative} on the top of, upon, over; (with accusative) on to the top of, upon, over (implying the breaking of a boundary); (with ablative) above, over (without the breaking of a boundary); in addition to
    \item[integer, integra, integrum] \marginnote{*}untouched, untried, fresh; undecided; open-minded, unprejudiced; whole, complete, (in an) undiminished (state); undamaged, whole, healthy, sound, unbroken, unscathed, unimpaired
    \item[intellego] to come to know, see into, perceive, understand, discern, comprehend, gather
    \item[intelligō, intelligere, intellēxī, intellēctum] (present stem also \textit{intelleg}-) grasp mentally, understand, realize; understand by inference, deduce; understand the value of, appreciate; (intransitive) have or exercise powers of understand, possess intelligence
    \item[inter] \marginnote{*}(preposition with accusative) among, in the presence of, amid; among, along with; beetween; during, in the middle of, while busy with, amid
    \item[interdum] at times, from time to time, now and then; occasionally; in the meantime, for the time being
    \item[interim] \marginnote{*}meanwhile, in the meantime; for the time being, for the present; all the while, at the same time; from time to time, occasionally, sometimes
    \item[intrā] \marginnote{*}(preposition with accusative) within, inside, privately, at home; in one's own country; \textit{intrā sē} to oneself, privately; (of time) within the space of, inside, before; on this side of, without passing beyond, short of; (adverb) inside, within; to the inside, inwards
    \item[intueor, intuērī, intuitus sum] look at, watch, gaze at; be a witness of, observe; (of things) be turned towards, face, have a view of; reflect upon, consider, contemplate
    \item[inveniō, invenīre, invēnī, inventum] \marginnote{*}encounter, meet, come across; find, find out, discover; devise, contrive, plan
    \item[investīgō (1)] track down, find by following a trail; search out; search out, track down, search after
    \item[invītus, invīta, invītum] not wishing, unwilling, reluctant
    \item[inūsitātus, inūsitāta, inūsitātum] unusual, uncommon, extraordinary, very rare; unfamiliar, strange
    \item[ipse, ipsa, ipsum] \marginnote{*}self, actual, very, exact (intensive not reflexive)
    \item[is, ea, id] \marginnote{*}(pronoun) he, she, it, they; (adjective) this or that (a weaker demonstrative pronoun than \textit{hic} or \textit{ille})
    \item[iste, ista, istud] \marginnote{*}that (of yours), that (often but not always with negative connotation)
    \item[ita] \marginnote{*}thus, so; in this way, in this manner, in such a manner
    \item[itaque] \marginnote{*}and so, and thus, and accordingly; therefore, for that reason, consequently
    \item[iūdicium, iūdiciī, n.] \marginnote{*}judicial investigation, trial, legal process; sentence; exercise of judgment, judging or deciding; a judgment, a decision; assessment, appraisal
    \item[iūdicō (1)] \marginnote{*}examine judicially, judge, be a judge, pass judgment; sentence, condemn; decide, evaluate, appraise, determine
    \item[iungō, iungere, iunxī, iunctus] \marginnote{*}join together, connect, attach, fasten, yoke, harness; (metaphorically) bring together, unite, join
    \item[iūs, iūris, n.] \marginnote{*}law, authority, right; a law, a rule; a legal procedure
    \item[iuvō, iuvāre, iūvī, iūtum] \marginnote{*}help, aid, assist; further, serve, support, benefit; strengthen, improve; (impersonal with an infinitive) profit, benefit
    \item[labefaciō, labefacere, labefēcī, labefactum] weaken, shake, loosen, undermine; overthrow, ruin, destroy
    \item[labōriōsus, labōriōsa, labōriōsum] involving much effort or work, hard, difficult, laborious; hard-working, industrious, diligent; involving hardship or suffering, painful, distressing
    \item[lateō, latēre, latuī] \marginnote{*}hide, be in hiding, go into hiding; take refuge, shelter; be out of sight, be invisible; be latent, lie below the surface, be concealed, lie hidden; escape notice, be overlooked, go unobserved
    \item[latus, lateris, n.] \marginnote{*}the side (of the upper body or chest), flank; a side; an extremity or edge
    \item[laxus, laxa, laxum] wide; loose, open; spacious, roomy
    \item[lentus, lenta, lentum] pliant, flexible; tough, tenacious; sticky, viscous
    \item[levitās, levitātis, f.] lightness; lack of intensity, mildness; triviality, unreliability, fickleness, shallowness
    \item[līberē] without restriction, at will, freely; open, frankly, boldly; wantonly, shamelessly
    \item[lībertās, lībertātis, f.] \marginnote{*}freedom, independence; opportunity; frankness of speech, outspokenness; lack of restraint, impertinence, licence
    \item[licet, licēre, licuit/licitum est] \marginnote{*}it is permitted, one may
    \item[līmēs, līmitis, m.] a boundary, limit; a piece of land within a boundary; lane, path, track, road; course, route
    \item[liquēscō, liquēscere, licuī, ———] become liquid, melt, liquefy; flow or melt away, dissipate
    \item[liquidus, liquida, liquidum] flowing, fluid, liquid; limpid, clear, unclouded
    \item[locus, locī, m. (pl. both locī, m. and loca, n.] \marginnote{*}a place, spot, neighborhood, locale, region
    \item[longus, longa, longum] \marginnote{*}long (in length or time); far
    \item[loquor, loquī, locūtus sum] \marginnote{*}speak, talk, say; tell, mention
    \item[lūceō, lūcēre, lūxī, ———] be light or clear; to shine, beam, emit light; be bright, glitter; be evident, be clearly known or felt; dawn, become light
    \item[lūdificātiō, lūdificātiōnis, f.] jeering, derision, mocking, mockery, mocking game
    \item[lūx, lūcis, f.] \marginnote{*}light; daylight, light of day; brightness
    \item[māchina, māchinae, f.] machine; device, plan, contrivance
    \item[magis] \marginnote{*}more, to a greater extent
    \item[magnitūdō, magnitūdinis, f.] \marginnote{*}greatness, size, bulk, magnitude
    \item[magnus, magna, magnum] \marginnote{*}great in size or extent, big, vast; great in number or amount, much, large; (of age) old, aged; notable, famous, of great consequence or importance
    \item[malignus, maligna, malignum] ungenerous, mean, grudging; scanty, pour; ill-disposed, spiteful, unkind; harmful
    \item[mālō, mālle, māluī, ———] \marginnote{*}prefer, want more
    \item[malus, mala, malum] \marginnote{*}bad; unpleasant, nasty; wicked, evil; harmful, painful, distressing
    \item[maneō, manēre, mānsī, mānsum] \marginnote{*}stay, remain (in one place); have patience, wait; wait for; be in store for, await
    \item[manifestus, manifesta, manifestum] clear, plain, apparent; evident, manifest; exposed, caught in the act, flagrant
    \item[manus, manūs, f.] \marginnote{*}hand (of people), paw (of animals); a band, a troop, company, faction; (by metonymy) handwork, workmanship
    \item[mātūrus, mātūra, mātūrum] ripe, full-grown, adult (of plants, fruits, or animals); experienced, mature; advanced in age, aged, old; occurring at the proper time, timely
    \item[medicīna, medicīnae, f.] the art or practice of healing, medicine, the knowledge or science of medicine; a treatment, a cure, a remedy
    \item[meditātiō, meditātiōnis, f.] \marginnote{*}reflection, contemplation, pondering; the subject of reflection or thought, an idea, a thought; planning, devising; practicing, rehearsal; exercise, practice
    \item[meditor (1)] contemplate, ponder, reflect; have in mind, intend; devise, plan, think out; go over, say to oneself, work over, practice
    \item[mel, mellis, n.] honey; something sweet, pleasant, or agreeable
    \item[membrum, membrī, n.] \marginnote{*}a limb, part of the body, member; (pl.) limbs, the body (as a whole); one of the main divisions or component parts of something, a part; a clause, part of a larger unit of speech or writing
    \item[memoria, memoriae, f.] \marginnote{*}the power or faculty of remembering, memory; a memory, a recollection; repute, what is remembered about someone or something
    \item[mendax, mendācis] prone to lie, untruthful; lying, false, deceiving
    \item[mēns, mentis, f.] \marginnote{*}mind; disposition, character; purpose, design, intention; significance, meaning; frame of mind, attitude
    \item[mentior (4)] lie, tell a falsehood; make a false promise, promise falsely; give a false account, misrepresent; invent, fabricate; feign, simulate
    \item[meus, mea, meum] \marginnote{*}of me, my, mine, belonging to me, my own
    \item[minimus, minima, minimum] \marginnote{*}smallest (in size, extent, duration, etc.); least; very small, very little
    \item[minus] \marginnote{*}less, to a smaller extent or degree
    \item[minūtus, minūta, minūtum] small (in length, size, duration, etc.), tiny, meager, brief; fine, consisting of fine particles
    \item[mīror (1)] \marginnote{*}be surprised, amazed, bewildered; marvel at; hold in awe, admire, revere
    \item[mīrus, mīra, mīrum] extraordinary, remarkable, astonishing, amazing
    \item[modus, modī, m.] \marginnote{*}a measure, extent, quantity; way, manner
    \item[moveō, movēre, mōvī, mōtum] \marginnote{*}(make) move, impel, set in motion; shake, agitate; wield, ply, exercise; shift, move; oust, expel, dislodge; stir, provoke, rouse
    \item[multus, multa, multum] \marginnote{*}numerous, many; (with partitive genitive) many (of); (with a singular noun) an abundance of, much
    \item[mundus, mundī, m.] \marginnote{*}the heavens, the sky; the universe; the world, the earth
    \item[mūtābilis, mūtābile] liable to change, changeable, fluctuating, uncertain; that can be changed, alterable, mutable; (of people) fickle, unsteadfast
    \item[mūtātiō, mūtātiōnis, f.] a changing, altering, a change, alteration, mutation; give and take, exchange; substitution, replacement; alternation interchange; conversion, translation
    \item[mūtō (1)] \marginnote{*}give and receive, exchange; substitute (for), take or put (one thing) in place (of another); replace, change (one thing by another); change (in some way), make different, modify, alter
    \item[nam] \marginnote{*}(affirmative or in assent) certainly, to be sure; (explanatory) for, because (generally explaining something that was just said or implied); (introducing an example or illustration) for instance
    \item[nātiō, nātiōnis, f.] birth of a child; a people, race, nation; nationality
    \item[nātūra, nātūrae, f.] \marginnote{*}nature (very broadly understood: of a person, of an object, of the universe as a whole); character, temperment, innate ability
    \item[-ne] \marginnote{*}(introduces a direct `yes' or `no' question)
    \item[necdum] (also \textit{nequedum}; sometimes written as two words) and not yet; but not yet; \textit{necdum\dots necdum} not yet either\dots or; (adverb) not yet
    \item[necessārius, necessāria, necessārium] essential, necessary requisite, needed; compelling, exercising compulsion, unavoidable
    \item[necesse] \marginnote{*}indispensable, essential, necessary; inevitable, determined by natural law; necessarily true; forced, compulsory
    \item[necne] (in direct questions) Or not?; (in indirect questions) or not
    \item[negō (1)] \marginnote{*}say (that\dots not); deny (that), deny (something), deny (the existence of); say no, refuse, decline, withold (something); forbid
    \item[nēmō, nēminis m./f.] \marginnote{*}nobody,  no one; (following a negative, usually an emphatic positive assertion) everyone, everybody; a person of no importance, a nobody
    \item[nempe] certainly, without doubt, assuredly, of course, as everybody knows; (in answers to questions or picking up another speaker's words) Why, clearly; (conceding something) admittedly, it is true
    \item[nec/neque] \marginnote{*}and not, nor, and\dots not (for purposes of translation, you will often want to put the \textit{not} much later in the clause or sentence than the \textit{and}; \textbf{nec\dots nec} neither\dots nor
    \item[nequeō, nequīre, nequiī/nequīvī, nequitum] be unable
    \item[nequidem] not even (in classical Latin, this would be \textit{ne X quidem}, but Descartes somestimes writes \textit{nequidem X})
    \item[nesciō (4)] \marginnote{*}not to know, not know, be ignorant, be unaware of; \textit{nescio an\dots} I am inclined to think that, perhaps, probably
    \item[nihil/nīl] \marginnote{*}nothing, not anything
    \item[nihildum] (often written as two words) nothing so far
    \item[nihilōminus/nīlōminus] none the less, notwithstanding just the same; likewise, as well
    \item[nihilum/nīlum, nihilī/nīlī, n.] nothing, not anything; \textit{dē nihilō} without reason, for nothing; \textit{ad nihilum venīre} come to nothing, have no effect; (in expressions as genitive of value) \textit{nihilī pendere}, \textit{nihilī facere}, \textit{nihilī putare} (or similar) set no value on, care nothing about
    \item[nisi] \marginnote{*}if\dots not, unless, except if
    \item[niteō, nitēre, nituī, ———] be radiant, shine; sparkle, glitter, be bright with reflected light; have a healthy look (of animals, land, people); be resplendent, be well decked out (of a home or building)
    \item[nītor, nītī, nīxus sum] \marginnote{*}rest one's weight, lean (on), support oneself, lean, incline; put one's faith (in), rely (on); be based (on), rest or depend (on); strive, exert oneself, direct efforts towards a goal or purpose
    \item[nocturnus, nocturna, nocturnum] of or belonging to the night; operating by night, appropriate to night, nocturnal
    \item[nōlō, nōlle, nōluī] \marginnote{*}wish\dots not, want\dots not, not want, be unwilling; refuse, decline
    \item[nōmen, nōminis, n.] \marginnote{*}name; designation, title; term, expression; a noun; fame, reputation
    \item[nōn] \marginnote{*}not
    \item[nōndum] \marginnote{*}not yet
    \item[nōnne] \marginnote{*}(introduces a yes or no question expecting the answer yes); Surely\dots ?; Isn't it the case that\dots ?
    \item[nōnnihil/nōnnīl/nōnnihilum] (with a partitive genitive) a certain amount, some, a number of; (alone) something; not a little, not a few; (adverb) to a certain extend, in some measure
    \item[nōnnisi] (often written as two words) not unless, not except, only
    \item[nōnnullus] (often written as two words) a certain amount, not a little; some, several, a number of
    \item[noscō, noscere, nōvī, nōtum] \marginnote{*}get knowledge of, become acquainted with, come to know, learn, discern; (in perfect tenses) know (i.e. `I have learned' is equivalent to `I know' and `I had learned' is equivalent to `I knew'); examine, study, inspect; familiarize oneself with, make the acquaintance of
    \item[noster, nostra, nostrum] \marginnote{*}our, our own, ours, of us
    \item[nōtitia, nōtitiae, f.] acquaintance, practical knowledge, familiarity; knowledge, understanding, awareness, cognizance; fame, celebrity; (also in a bad sense) notoriety, infamy
    \item[notō (1)] mark, to designate with a mark, brand, stamp; censure, stigmatize (since Roman censors would put a mark beside a dishonored person's name); pick out (among a group), distinguish; single out, designate; indicate (by a sign); point out, represent, denote, indicate
    \item[nōtus, nōta, nōtum] \marginnote{*}known, familiar; widely or generally known, noted
    \item[novus, nova, nobum] \marginnote{*}new; strange, unfamiliar; surprising, unforeseen; strange, alien, subversive, seditious; \textit{rēs novae}, \textit{motus novī} revolution
    \item[nox, noctis, f.] \marginnote{*}night, evening
    \item[nūdus, nūda, nūdum] \marginnote{*}naked, bare, unclothed; uncovered, exposed
    \item[nūllus, nūlla, nūllum] \marginnote{*}not any, none, no; (with a following negative, the meaning is a strong positive) \textit{nūllus nōn} every, any, all
    \item[numerō (1)] add up, count; reckon, compute; enumerate, catalogue, specify in a list
    \item[numerus, numerī, m.] \marginnote{*}sum, total; a number; a (large or small) quantity or amount
    \item[numquam] \marginnote{*}at no time, never
    \item[numquid] (introducing questions where a negative answer is expected) Is it really possible that\dots ? Surely\dots not?; (introducing rhetorical questions where the implied answer is no) Can it be said that\dots ?; (introducing an indirect question) whether, if; (non-classical, but common in Descartes, introducing a question where agreement is expected, often a rhetorical question) Surely\dots ? Isn't it the case that\dots ?
    \item[nunc] \marginnote{*}now, at this time, at present; now, under these circumstances, in view of this; \textit{nunc\dots nunc} at one moment\dots at another moment, in some cases\dots in other cases
    \item[nunquid] see \textit{numquid}
    \item[nūper, nūpera, nūperum] not long ago, recently, just now; (in reference to past time) not long before
    \item[nūtriō (4)] suckle, feed at the breast; support with food, nourish, feed; foster, bring up, rear; support, build up, increase, encourage, look after, care for
    \item[obfirmō (1)] (often \textit{off}-) make firm against attack, secure; make firm, make inflexible; make up one's mind about or persist in
    \item[obdormiō (4)] fall asleep
    \item[oblīviscor, oblīviscī, oblītus sum] forget, forget about, forget how
    \item[obstinātus, obstināta, obstinātum] stubborn, obstinate, resolute; hardened (in vice or crime)
    \item[obstupēscō, obstupēscere, obstupuī, ———] be struck dumb, be stunned, dazed, or astounded (with any powerful emotion)
    \item[occāsiō, occāsiōnis. f] an opportunity, chance; contingency, accident
    \item[occupō (1)] \marginnote{*}take possession of, occupy; seize control of; invade, seize, cover; seize hold of, take hold of; forestall, catch first, take by surprise; be the first to act, anticipate others, get in first
    \item[occurrō, occurrere, occurrī, occursum] \marginnote{*}run to meet, hurry to meet; arrive, turn up; meet or confront (in a hostile manner), go to oppose; counteract, check; meet accidentally, run into , come upon, happen upon, become involved; (of an idea or situation) present itself, occur (to a person); occur, crop up, confront one
    \item[oculus, oculī, m.] \marginnote{*}the eye, an eye; (usually plural) operation of the eyes, sight; (usually plural) gaze, look, regard
    \item[odor, odōris, m.] smell, scent, odor; pleasant smell, fragrance; unpleasant smell, stink, stench; a whiff, hint, suggestion
    \item[odorātus, odorātūs, m.] action or sense of smelling, smell
    \item[ōlim] \marginnote{*}at that time, some time ago; once upon a time, once; formerly, of old
    \item[omnīnō] in every respect, entirely, absolutely, altogether; in all, all told, altogether; (with negatives) at all, in any way or circumstance; in general, as a whole
    \item[omnis, omne] \marginnote{*}all, the entire amount, the whole of; every one, each, every
    \item[opīniō, opīniōnis, f.] (an) opinion, belief, view; expectation; imagination (as a faculty or a mental picture); reputation
    \item[opīnor (1)] think, believe, suppose; have in mind, imagine, conceive; look for, expect; express an opinion, opine
    \item[opportūnus, opportūna, opportūnum] advantageous, convenient, favorable; timely, opportune; appropriate (to an occasion) seasonable; available when wanted, ready to hand; liable, exposed, susceptible (to something, generally something bad)
    \item[ops, opis, f.] \marginnote{*}power, ability, might; forces, troops; power, dominion, influence; (usually plural) financial resources, wealth, property; aid, assistance, help
    \item[opus, operis, n.] \marginnote{*}work, a task, an undertaking; occupation, employment; activity, working, effort, strenuous activity; an achievement, accomplishment, product of work or labor (e.g., a work of art, a finished building, a literary work, etc.)
    \item[orīgō, orīginis, f.] a beginning, commencement; source, start; descent, lineage, birth, origin
    \item[ostendō, ostendere, ostendī, ostentum/ostensum] \marginnote{*}hold out, present; show, display, reveal; exhibit (for sale, use, inspection, etc.); point out, mention; make clear, show, evince, demonstrate, disclose, make known
    \item[ōtium, ōtiī, n.] \marginnote{*}unoccupied or spare time, leisure time; leisure, freedom from work, rest, relaxation, ease; peace, tranquility, calm; inactivity, idleness; temporary cessation, respite, lull
    \item[pactum, pactī, n.] an agreement, compact; means, manner, method, grounds, consideration
    \item[pars, partis, f.] \marginnote{*}a portion, part, piece, share
    \item[particulāris, particulāre] particular; partial, of or concerning a part
    \item[parum] (as a noun) an insufficient amount, too little, not enough; as an adverb) insufficiently, too little, not enough
    \item[parvus, parva, parvum] \marginnote{*}small (in size, extent, duration, amount, quantity, etc.); insignificant, unimportant
    \item[patior, patī, passus sum] \marginnote{*}to be subjected to, undergo, experience; put up with, tolerate; endure, make do; allow, permit; admit of, allow, grant, accept
    \item[paucus, pauca, paucum] \marginnote{*}few, little
    \item[paulus, paula, paulum] \marginnote{*}little, small (in size or quantity)
    \item[pauper, pauperis] \marginnote{*}poor, not wealthy; of little worth, cheap; meager, poor, unproductive
    \item[pendeō, pendēre, pependī] hang, be suspended; hang down, hang onto; overhang; be suspended, be left incomplete or hanging; depend, hinge, be based (on), result (from)
    \item[pēnsitō (1)] weigh (in a set of scales); pay, weight out (in payment); weigh in one's mind, ponder, consider; (with \textit{cum}) weigh against, compare (with)
    \item[per] \marginnote{*}(preposition with accusative) through, across; along, over, along with; through the length of, along (some part of); all over, throughout; in the course of, during (a time); through in succession; as far as (some person or authority) is concerned; as a result of, by reason of, through
    \item[perceptiō, perceptiōnis, f.] the action of taking or taking possession; gathering; the right to gather; mental grasp, perception
    \item[percipiō, percipere, percēpī, perceptum] take, take possession of; harvest, earn, reap, acquire; catch hold of; perceive, apprhend, notice, take in or grasp with the mind
    \item[percurrō, percurrere, per(cu)currī, percursum] run, move quickly over or through; travel quickly from end to end, pass through, visit in quick succession; run over, skim over; run over (in words or thought), survey, review; run through in sequence
    \item[perficiō, perficere, perfēcī, perfectum] bring (an action or process) to its end, complete, finish; make perfect, bring to perfection; carry out, execute; bring about, achieve, effect; destroy, kill; use up, exhaust
    \item[pergō, pergere, perrexī, perrectum] \marginnote{*}proceed, move onward; go on, lead
    \item[perīculum, perīculī, n.] \marginnote{*}test, trial, proof; danger, risk; liability, responsibility for damage or loss
    \item[permisceō, permiscēre, permiscuī, permixtum] mix or blend thoroughly, mix together; bring into association, combine in a group; mix up, convound; involve, embroil; disturb, throw into confusion, confuse
    \item[permittō, permittere, permīsī, permissum] \marginnote{*}(literal meaning very rare) send through; let run free, relax, allow full scope to, give rein to; cede, relinquish, surrender; commit, entrust; leave (to another) to decide or do, refer; permit, allow, sanction
    \item[permultus, permulta, permultum] very much, very many, a great many
    \item[perspicuus, perspicua, perspicuum] transparent, pellucid; clearly visible, conspicuous; lucid, clear
    \item[persuādeō, persuādēre, persuāsī, persuāsum] \marginnote{*}persuade, prevail upon, convince
    \item[pertineō, pertinēre, pertinuī, ———] \marginnote{*}extend, reach, stretch; be aimed (at), be directed (towards); tend, be conducive (to); relate or pertain (to), have to do (with); be a concern (to), concern, be the business (of)
    \item[perveniō, pervenīre, pervēnī, perventum] \marginnote{*}come (to), get (to), arrive (at); get through (to the end or conclusion); extend or reach (to); pass into the hands of, become the property of
    \item[pēs, pedis, m.] \marginnote{*}foot (of a person); foot (as a measure of distance); foot (in poetry)
    \item[petō, petere, petīvī/petiī/petī, petitum] \marginnote{*}make for, go towards, direct one's course to; go for, go after, attack; seek and bring, fetch, procure; seek, aim at, strive for, strive after, pursue
    \item[physica, physicōrum, n. pl.] natural science, the study of physical nature, physics (though in a much broader sense than ours)
    \item[pictor, pictōris, m.] a painter
    \item[pīleus, pīleī, m.] (also with the stem \textit{pill}-) hat
    \item[pingō, pingere, pīnxī, pīctum] paint, tint, adorn with colors or colored designs; paint (a picture); paint or draw a picture of, depict; decorate, embellish
    \item[placidus, placida, placidum] kindly, indulgent; tame, quiet, friendly; peaceful, calm, free from stress
    \item[plācō (1)] conciliate, placate; reconcile (someone with someone else or something); calm, soothe, make calm
    \item[plānus, plāna, plānum] even, level, flat, plane; simple, plain; obvious, clear, manifest
    \item[platēa, platēae, f.] a street
    \item[plērusque, plēraque, plērumque] \marginnote{*}greater part of, greater number of, most of; (plural) a great number, very many; (plural substantive) most people, the majority
    \item[pondus, ponderis, n.] \marginnote{*}weight, heaviness; a heavy object, a mass; a burden or load (literal or figurative); importance, value, weight, influence
    \item[pōnō, pōnere, posuī, positum] \marginnote{*}place, set, put; station, post; put in position, set up, pitch (of a military camp); build, construct, found; put down, lay down
    \item[porrō] straight on, forward, onward, ahead; further off, beyond; hereafter, later; further, more (referring to continuing action); in turn; on top of that, next, besides; furthermore, again, moreover
    \item[positiō, positiōnis, f.] the action of placing; planting (of crops); layout; disposition, lie (of land); position, situation, mental position, attitude; condition, state
    \item[possum, posse, potuī, ———] \marginnote{*}be able (to), be capable (of), can; be possible
    \item[post] \marginnote{*}\textit{adverb and preposition with accusative} (adverb) behind, back; at a later time, afterwards; (with accusative) behind; beyond, over; after; second to, of less value or importance than, after 
    \item[posteā] \marginnote{*}subsequently, afterwards, thereafter; hereafter, in future, later; next (in time); next (in value, importance, rank, etc.)
    \item[posterus, postera, posterum] \marginnote{*}future, later; next, following; (plural substantive) descendants, posterity, future generations (one's own or in general)
    \item[postquam] \marginnote{*}after, when; ever since, from the time that, since
    \item[potēns, potentis] \marginnote{*}endowered, provided (with); having got possession (of); having or exercising power (over); capable, powerful, influential
    \item[potestās, potestātis, f.] \marginnote{*}power, (possession of) control or command; position of power, office, magistracy; jurisdiction, authority; opportunity, chance, right
    \item[potis, pote] \marginnote{*}able, capable; liable; possible
    \item[praecīdō, praecīdere, praecīdī, praecīsum] shorten, cut back, cut or break the front off; cut down or sever (a part from the main body of something or someone); cut short, bring to a sudden end, break off; deprive of
    \item[praeclārus, praeclāra, praeclārum] very clear, brilliant, bright, radiant; splendid, magnificent, grand; outstanding (in achievement or reputation), brilliant, glorious
    \item[praeiūdicium, praeiūdiciī, n.] previous  legal judgement or ruling, prejudgement; a precedent (in legal or other contexts); a preconception, prejudice, presumption
    \item[praeter] \marginnote{*}\textit{preposition with accusative} passing, past, across; beyond, surpassing, exceeding; to a greater degree than; out of line with, contrary to, at variance with; in addition to, as well as, besides; other than, with the exception of, except, but, save
    \item[praetereā] \marginnote{*}in addition (to that), besides, as well; moreover, furthermore
    \item[prāvus, prāva, prāvum] crooked, not straight, awry; twisted, distorted, misshapen, deformed; corrupt, debased, evil; (in a weaker sense) wrong-headed, misguided; defective, faulty, wrong; incompetent, bad
    \item[prīmus, prīma, prīmum] \marginnote{*}the first, first, earliest; furthest in front, leading; furthest out, uttermost, extreme; primary, fundamental
    \item[prīncipium, prīncipiī, n.] \marginnote{*}action or fact of beginning, starting, founding; origin; first part (of anything), head, source; original position, starting point; beginning, opening; (guiding) principle, basis, premiss, starting point
    \item[prior, prius] \marginnote{*}former, previous, earlier, prior, preceding; in front, leading; immediately preceding, the last; in anticipation or advance of someone else, earlier, first
    \item[prō] \marginnote{*}\textit{preposition with ablative} before, in front of; on behalf of, in the interests of, as representitive of; in favor of, on the side of; in place of, instead of; in proportion to, according to; in relation to, considering, with regard to; in view of, having regard to, to judge from
    \item[probābilis, probābile] commendable, acceptable; plausible, credible; probable, likely
    \item[probō (1)] \marginnote{*}approve, commend; think well of, esteem; give assent to, authorize, santion; examine, test, put to the test (in order to approve or not); win approval for; demonstrate, prove, show to be real or true
    \item[prōcūrō (1)] look after, attend to; administer, have charge of; expiate, avert by sacrifices; (intransitive) perform sacrifices (in order to avert something)
    \item[prōferō, prōferre, prōtulī, prōlātum] bring forth, bring out; (reflexive or passive as intransitive) come forth, emerge; show, display; bring into existence, put forth; utter, pronounce; make known, publish, disclose; carry or move forwards (reflexive or passive as intransitive) advance, come forward; extend, prolong; postpone, defer
    \item[prōficiō, prōficere, prōfēcī, prōfectum] make headway, gain results, be successful; do good, help; advance, gain ground; increase (in size or extent), rise (of prices); progress, develop, improve
    \item[profundus, profunda, profundum] extnding a long way down, (very) deep, bottomless; deep, thick, profound; insatiable, having an immense capacity; secret, abstruse, mysterious, remote from general knowledge; absorbing, intense, profound
    \item[proinde] according (as), in proportion (as); in a corresponding manner or degree, accordingly; in the same way or degree (as); equally, similarly, likewise; so then, accordingly
    \item[prōnūntiō (1)] proclaim, announce, state publicly; pronounce (a decision, a verdict, etc.); affirm, declare; state as a fact, tell, relate; assert; utter, speak, express; recite, declaim
    \item[prōnus, prōna, prōnum] leaning or bending forward, tilted forward, bending down; lying prone, prone; sloping, having a downward incline; inclined (to, towards), disposed (to), liable (to)
    \item[proprius, propria, proprium] \marginnote{*}one's own, personal, private; peculiar (to someone, something), particular (to one person or thing), special, specific; characteristic, personal; lasting, permanent, continuous
    \item[propter] \marginnote{*}\textit{adverb and preposition with accusative} (adverb) near, close by, close at hand; (with accusative) near, close to; in view of, because of, as a result of, on account of, thanks to; for the sake of, out of consideration for
    \item[prōrsus] forward, straight ahead; without interruption, straight; right through to the end; thoroughly, in every respect, altogether, quite, absolutely; (usually after a connective) more than that, even, indeed; (as a sentence connective) indeed, in fact; all in all
    \item[prout] according as, in proportion as; in so far as, inasmuch as, to the extent that
    \item[prūdēns, prūdentis] well aware of what one does or of the consequences of one's action, acting deliberately, open-eyed; aware (of), having foreknowledge (of); knowing (that), well aware (that); exercising foresight, prudent, discreet; characterized by prudence or good sense; having good understanding, clever, having a good practical understanding or skill (in)
    \item[prūdentia, prūdentiae, f.] practical understanding or wisdom, shrewdness, good sense; proficiency (in some field), practical grasp; foreknowledge
    \item[pudeō (2)] (often impersonal in the \nth{3} person singular) fill with shame, make ashamed; (with a personal subject) feel shame, be ashamed
    \item[pulsō (1)] strike (with repeated blows), beat, assault, hit; knock; push, impel, drive; send on one's way, send away, dispel
    \item[pūnctum, pūnctī, n.]  "small hole
    \item[pungō, pungere, pupugī/pepugī, pūnctum] prick, puncture, sting, jab, poke; trouble, vex, disturb
    \item[purgō (1)] free from impurities or dirt, clean, purify; clear a space (of any obstructions, pests, or unwanted people); remove the outer covering, husk, or shell (of something); free from troubles, anxieties, etc.; absolve, exonerate, free (someone from charges); apologize (i.e., purge onesel of an offense)
    \item[purpura, purpurae, f.] shellfish yielding a purple dye; purple dye; purple-dyed cloth (associated with senators, knights, and royalty); purple color
    \item[putō (1)] \marginnote{*}clean, make clean; go over in the mind, ponder; estimate, assess; consider (to be), regard (as), deem; think, suppose, believe (that)
    \item[quadrātum, quadrātī, n.] a (geometric) square; a square object, something square
    \item[quaerō, quaerere, quaes(i)ī/quaesīvī, quaesītum] \marginnote{*}try to find, search for, hunt for, seek, look for; ask to see, ask for; try to obtain, strive for, seek; require, demand, need; try to bring about, aim at, intend, try (to); seek to know about, inquire about, inquire into, examine, consider; ask a question; hold a judicial inquiry into, investigate, try a case
    \item[quaestiō, quaestiōnis, f.] the act of searching; examination, interrogation; judicial investigation, inquiry; investigation, research, inquiry; subject of discussion or dispute, problem, question, issue
    \item[quālis, quāle] \marginnote{*}of what sort, kind, or nature; what kind of a, what sort of a
    \item[quamdiū] (interrogative) For how long?; (exclamatory) How long! What a long time!; (relative) for what length of time, (as) long as, until, during
    \item[quamquam] \marginnote{*}(introducing a subordinate concessive clause) however much, although; (introducing a main clause) admittedly, to be sure
    \item[quandōquidem] inasmuch as, seeing that, since
    \item[quantitās, quantitātis, f.] magnitude, multitude, quantity, degree, size, amount
    \item[quantus, quanta, quantum] \marginnote{*}(interrogative) of what size, amount, quantity, degree, importance? how much? how many? how big?; (relative) of what size, degree, amount, number, importance
    \item[quapropter] Interrog., for what, wherefore, why
    \item[quārē] \marginnote{*}(interrogative) in what way? how?, why? for what reason?; (relative) by which means, whereby, because of which, for which reason, why; for the reason that, because; (introducing a new sentence) therefore, hence, for this reason
    \item[quasi] \marginnote{*}as if, just as, as though; as for example, say; (qualifying expressions of number, measurement, or the like) as good as, practically, more or less
    \item[quattuor] \marginnote{*}four
    \item[-que] \marginnote{*}and
    \item[quemadmodum] \marginnote{*}(interrogative) in what way? how?; (relative) in the manner in which, in which manner, just as, as
    \item[queō, quīre, quīvī/quiī, quitum] be able, can
    \item[quī, quae, quod] \marginnote{*}who, which, that
    \item[quia] \marginnote{*}because
    \item[quīcumque, quaecumque, quodcumque] \marginnote{*}\textit{also -cunque} whoever, whatever; anyone whatever, anything whatever
    \item[quīdam, quaedam, quoddam] a certain, some; a certain number, some number; a certain amount, some amount; (substantive neuter) \textit{quiddam}
    \item[quidem] \marginnote{*}certainly, indeed; if nothing else, at any rate (particularizing and emphasizing a preceding word or phrase; often found with personal pronouns, especially \textit{ego}
    \item[quiēs, quiētis, f.] (the rest of) sleep; rest, repose, relaxation; idleness, inactivity, inaction; absence of noise, stillness; a calm, peaceful state (of mind or of the world)
    \item[quiēscō, quiēscere, quiēvī, quiētum] \marginnote{*}(rest in) sleep, fall asleep; rest, relax (from work, pain, etc.), take a rest; take no action, do nothing, be still, stand by; say nothing, be quiet; be peaceful, make no disturbance, be calm
    \item[quīn] \marginnote{*}(interrogative) why not?; (as an adverbial particle) indeed, in fact; (with \textit{etiam} or \textit{et}) yes, and\dots, and furthermore; (conjunction) so as to prevent, so that\dots not
    \item[quīnque] \marginnote{*}five
    \item[quis, quid] \marginnote{*}who? which (one)? what?
    \item[quisnam, quaenam, quidnam] who (in fact)?, what (in fact)?, who (ever)?, what (ever)? (strengthened form of \textit{quis, quid})
    \item[quispiam, quaepiam, quodpiam] one or other, an unspecified, some; a particular (but unspecified), a certain, some
    \item[quisquam, quicquam/quidquam] \marginnote{*}any, anyone, anything
    \item[quisquis, quidquid/quicquid] \marginnote{*}anyone who, anything that; everyone who, everything that; all who, all that; whoever, whatever
    \item[quīvīs, quaevīs, quodvīs] whatever person you please, whatever thing you please; whoever, whatever; anyone, anything
    \item[quōmodō/quōmodo] (interrogative) in what way? how?; in the manner in which, as; to the extent to which, as far as
    \item[quoniam] \marginnote{*}as soon as, after; seeing that, now that; since, inasmuch as, because
    \item[quoque] \marginnote{*}in the same way, too, likewise, no less; besides, as well, also, too
    \item[quotiēns] \marginnote{*}(interrogative) how often? how many times? (relative) as often as, whenever; the number of times that, as many as the times that
    \item[ratiō, ratiōnis, f.] \marginnote{*}reckoning, calculation (the action or the result of the action); proportion, relation; reasoning, reckoning; theory; an explanation, reason (for), ground; an account; (exercise or faculty of) reason; an affair, concern, business; plan of action, policy, scheme
    \item[ratiōnālis, ratiōnāle] derived from or concerned with reason, theoretical, dialectical; possessing reason, rational
    \item[recēnseō, recēnsēre, recēnsuī, recēnsum] count, enumerate, make a review or census of
    \item[recordor (1)] call to mind, recollect, remember
    \item[recurrō, recurrere, recurrī, recursum] run back, hurry back, return; run in reverse; return, come back, recur; revert, go back (in condition); have recourse to, fall back on
    \item[reddō, reddere, reddidī, redditum] \marginnote{*}give back, restore, return (something), repay, put back; restore; throw back, reflect, echo; say in reply, answer; reproduce, repeat; pay, discharge (a debt)
    \item[redeō, redīre, rediī, reditum] \marginnote{*}come or go back, return, go back; revert, return, be restored (to a state or condition); recur, come back, return
    \item[redūcō, redūcere, redūxī, reductum] lead back, bring back,  conduct back, escort back; pull back, draw back; recall, restore, bring back (to a condition, state, situation, etc.); bring, reduce (to a state); bring down, reduce (in degree or quality)
    \item[referō, referre, retulī, relātum] \marginnote{*}bring back, carry home; bring again; move or force back, withdraw, bring or draw back; (reflexive and passive as intransitive) go back, return; revert, return (to something, to a situation, etc.); report, bring back (news, a message, or reply); record, enter, write down; ascribe, refer, put down; give back, give in return, restore, pay back, render; recall, mention, relate
    \item[regō, regere, rēxī, rēctum] \marginnote{*}keep straight, direct, guide; manage, steer, guide (the course of); direct the activities of, control, direct, govern, rule, command
    \item[rēiciō, rēicere, rēiēcī, rēiectum] throw, drive, thrust, or turn back; drive away, repulse, beat off; repel, deter, prevent; remove, send out of the way; throw away, discard, abandon; dismiss, cast aside; refuse (to accept, admit, or adopt), spurn, rebuff, reject; refer, hand over, transfer; put off, postpone
    \item[relābor, relābī, relāpsus sum] move or fall gradually back, slip or slide back; recede, ebb; relapse, revert, slip back, fall back
    \item[reliquiae, reliquiārum, f. pl.] what remains, the remnants, the remains, the rest; remaining members, survivors; vestiges, traces (what remains after an activity, action, condition, etc.)
    \item[reliquus, reliqua, reliquum] \marginnote{*}the rest of, the remaining, the other; left, remaining (in some place, condition, or state), surviving; future, further
    \item[remaneō, remanēre, remānsī, remānsum] stay or remain behind; remain in position, stay (where one is); be left, remain, continue to be, presist, endure
    \item[removeō, removēre, remōvī, remōtum] \marginnote{*}move back or away, remove; banish, do away with, remove; debar, disqualify; set aside, leave out of account
    \item[reor, rērī, ratus sum] \marginnote{*}think, imagine, suppose, deem, hold a belief or opinion
    \item[reperiō, reperīre, rep(p)erī, repertum] \marginnote{*}find by looking, discover, find by inquiry or consideration; light upon, acquire, get; discover, get to know, find; find to be, find (in a condition or situation); make up, devise, invent
    \item[repetō, repetere, repetīvī, repetiī, repetītum] \marginnote{*}return to, make for again, make one's way back (to); attack again, go for again; attack or go after in retaliation; resort again to; repeat; take steps to recover, get back, seek in return, seek to restore; demand, claim back, call back, recall
    \item[repleō, replēre, replēvī, replētum] replenish, refill, fill again, reoccupy; restore (to full number, strength, etc.); make up, supply (a deficiency); fill up, occupy the whole of; sate, stuff, satiate
    \item[repraesentō (1)] resent to view, exhibit, show, present; revive, bring back into the present; serve as teh equivalent of, represent; represent (in art), portray; represent (in thought or words), portray; make immediately available, bring on at once; pay (a sum), pay at once
    \item[repugnō (1)] offer resistance (to) fight back, defend oneself; fight or rebel (against), struggle (against doing something), strive (to prevent); object (to) quarrel (with), protest (against); be contrary to, be inconsistent (with), be hostile (to), clash (with)
    \item[requīrō, requīrere, requīsīvī/requīsiī, requīsītum] try to find, seek, look for; ask or inquire about; ask, demand; try to obtain or bring about, seek; look for, expect to find; need, stand in need of; feel the loss of, miss
    \item[rēs, reī, f.] \marginnote{*}property, wealth; goods, possession; a thing; matter, situation, affair, business, (often in this sense) a court case; fact, deed; activity, practice; purpose, object
    \item[respiciō, respicere, respexī, respectum] \marginnote{*}look back, look away (from something), look round; look back and see, notice behind one; look around for (something or someone needed), look to (for help or protection); review, look back on; turn one's thoughts or attention to; take notice of, take account of, heed; have regard for, show concern for; have reference (to), relate (to) be the concern (of)
    \item[respondeō, respondēre, respondī, respōnsum] \marginnote{*}reply, respond (by voice or in writing), answer; answer a charge, speak in defense, say in refutation, reply to an argument, offer an opposite point of view; answer a summons, present oneself; be consistent with, agree or accord (with), conform (to)
    \item[retineō, retinēre, retinuī, retentum] \marginnote{*}hold back, hold fast, detain, confine, keep from escaping, prevent from being taken away; hold back, stop, check, delay, restrain; keep hold of, grasp, cling to; continue to use, continue to have, retain, continue to observe, keep up, maintain
    \item[rēvērā] \marginnote{*}in fact, in reality, truly
    \item[revolvō, revolvere, revoluī, revolūtum] roll back, roll aside; unroll, roll back (to the start of); go back over (something in thought or speech), think or speak over; cause to move in a circular course, make revolve, turn (something) around; cause to return, bring around again; fall back again, relapse; fall back on; revert to, come back (to a topic or argument)
    \item[revertor, revertī, reversus sum] turn back, move back, return, come back; have recourse (to), fall back (on); go back, revert, return, change back; come around again, recur, be renewed; relate back
    \item[rotundus, rotunda, rotundum] round, circuluar, spherical, rounded; smooth, well-rounded (positive term for rhetorical style)
    \item[rūrsus] \marginnote{*}back, backwards; in reverse; once again, a second time; in one's turn; on top of that, in addition, besides; on the other hand, on the contrary; now again, at another moment
    \item[saltem] at least, at all events, anyhow; (in negative sentences) even, so much as
    \item[sanguis, sanguinis, m.] \marginnote{*}blood; bloodshed
    \item[sānus, sāna, sānum] \marginnote{*}physically sound, healthy; wholesome, causing health, healthy; undamaged, unimpaired; mentally sound, sane, sensible, reasonable, sober
    \item[sapiō, sapere, sapīvī/sapiī, ———] taste (like something); have a good taste; smell of; (figurative) be a sign of, give indication of; have taste or discernment; be intelligent, show good sense; know, understand; be in one's right mind, be sane
    \item[sapor, sapōris, m.] taste, flavor; distinctive quality, distinctive character; sense of taste; smell, odor
    \item[satyriscus, satyriscī] little satyr, satyr
    \item[scientia, scientiae, f.] \marginnote{*}knowledge, awareness, mental grasp; knowledge (as opposed to mere belief); understanding, expert knowledge; a particular branch of knowledge, an art, skill, or technical expertise; learning, erudition, wide knowledge
    \item[scīlicet] it is evident or clear (that), one may be sure (that); it is obvious; naturally, you may depend on it; as is apparent, evidently, certainly, no doubt; (ironic) to be sure, doubtless; I mean, of course; that is to say, namely
    \item[sciō, scīre, scīvī/sciī, scītum] \marginnote{*}know, understand, perceive, be aware; have (certain) knowledge of (as opposed to mere belief);  have knowledge of, be skilled in
    \item[sēcēdō, sēcēdere, sēcessī, sēcessum] detach oneself, withdraw, move away (often to a private place); secede (i.e., withdraw in protest or revolt), dissociate oneself; withdraw from Rome to the country (for vacation or in retirement from public life)
    \item[sēcius/sētius] to a lesser degree, less readily; \textit{nihilō sēcius/sētius} none the less, for all that, just the same
    \item[secundus, secunda, secundum] \marginnote{*}following; favorable, supporting, helpful, propitious; second
    \item[sēcūrus, sēcūra, sēcūrum] \marginnote{*}free from care, fear, anxiety, untroubled, undisturbed, tranquil; negligent, indifferent, nonchalant, careless, perfunctory (i.e., lacking proper concern or care); free from danger, safe, immune (from something negative)
    \item[sed] \marginnote{*}but; however
    \item[semel] \marginnote{*}once, a single time, the first time; only once, just once, once and for all; at any one time, once, ever; at one and the same time, simultaneously
    \item[semper] \marginnote{*}ever, always, at all times, continually, perpetually, forever
    \item[sēnsus, sēnsūs, m.] sensation, capacity to perceive by the senses; any one of the five senses; a sensation, any impression by a sense; the faculties of perception; self-awareness, consciousness, awareness (in general or of something other than oneself); judgement, the faculty of making distinctions, sensibility, perception of what is appropriate or right; a mental feeling, emotion, thought, idea; character, disposition
    \item[sentiō, sentīre, sēnsī, sēnsum] \marginnote{*}perceive by one of the senses; (less literally) perceive, feel, discern, recognize, become aware of; hold or express a given belief or opinion, think, feel, opine; mean, intend, have in mind
    \item[sēparō (1)] divide (off), separate, split up, divide up; keep separate, cut off, isolate, part; debar, exclude; separate (in thought or writing) treat as distinct, exclude from consideration
    \item[sequor, sequī, secūtus (sequutus) sum] \marginnote{*}follow, go after, go behind; pursue, chase, follow; follow in time or sequence, succeed, come after; follow from, follow logically; escort, accompany, attend
    \item[seriēs, seriēī, f] a series, a row, a line, procession, connected system; a continuous series, succession, sequence; a line of ancestors or descendants
    \item[sēriō] seriously, in earnest, soberly, not lightly or playfully
    \item[sērius, sēria, sērium] weighty, important, serious; grave or serious in appearance or character
    \item[sētius/sēcius] to a lesser degree, less readily; \textit{nihilō sētius/sēcius} none the less, for all that, just the same
    \item[seu/sīve] \marginnote{*}or if; (repeated) whether\dots or; if\dots or
    \item[sī] \marginnote{*}if, supposing that
    \item[sīc] \marginnote{*}so, thus, in this manner, just so; as follows; in this way, so\dots (that), in such a way (that)
    \item[significātiō, significātiōnis, f.] the action of giving signs or signals; action or fact of conveying information; outward sign, expression, intimation, indication; suggestion, hint; the meaning, sense (of a word, expression, work, etc.)
    \item[similis, simile] \marginnote{*}like, similar; \textit{vērī/vērō similis} resembling the truth, likely, reasonable, plausible, probable
    \item[similitūdō, similitūdinis, f.] similarity, likeness, resemblance; a similar thing, that which resembles something, a likeness; a simile, analogy, comparison
    \item[simplex, simplicis] having a single layer, fold, etc., single, onefold; simple, uncompounded, undecorated, plain; alone, standing alone, separate; free from complications, straightforward, simple; artless, ingenuous, innocent, direct, candid
    \item[simul] \marginnote{*}at the same time, together; at once, as soon as
    \item[sine] \marginnote{*}\textit{preposition with ablative} without
    \item[singulī, singulae, singula] \marginnote{*}one apiece, one to each; each one, every single; taken separately, one by one, individual
    \item[sīquidem] (as a strong conditional) if it is really possible that, if indeed; (concessive) even supposing; (adding a caveat or rider) at any rate if, always assuming that; (qualifying an assertion) if it is really the case that
    \item[sīrēna] a siren
    \item[sīve/seu] \marginnote{*}or if; (repeated) whether\dots or; if\dots or
    \item[soleō, solēre, solitus sum] \marginnote{*}be accustomed (to), make it a practice (to); be liable (to), be likely (to), be apt (to); (with neuter pronoun or after \textit{ut}) be the common practice, be the norm, be the usual case
    \item[solus, sola, solum] \marginnote{*}alone, only; lonely; only one, single, sole
    \item[solvō, solvere, solvī, solūtus] \marginnote{*}release, set free; dissolve, take apart; resolve, settle, solve
    \item[somniō (1)] dream; daydream; have idle thoughts about, have delusions about
    \item[somnium, somniī, n.] a dream, vision; idle hope, fantasy, delusion, daydream
    \item[somnus, somnī, m.] \marginnote{*}sleep; sleepiness, drowsiness; (euphemistically) (eternal) rest, death
    \item[sonus, sonī, m.] sound, noise; articulate sound, speech
    \item[sōpiō (4)] cause to sleep, put to sleep, overcome with sleep; render unconscious, knock out
    \item[spatium, spatiī, n.] \marginnote{*}ground used for athletics or horse racing, a course; a circuit or lap of a racecourse; area, space, room; surface area, extent, size; a stretch of time, period, term, duration
    \item[spērō (1)] \marginnote{*}hope (for), look forward (to); hope (that); anticipate, apprehend
    \item[sponte] voluntarily, of one's own will, without prompting, of one's own accord; deliberately, purposely; unaided, without help; spontaneously, of itself
    \item[stabiliō (4)] make firm, make steady; fix or establish firmly
    \item[statim] \marginnote{*}immediately, at once, without delay; \textit{with ē/ex} immediately after, straight from
    \item[statuō, statuere, statuī, statūtum] \marginnote{*}cause to stand, set up, set, establish; build, put up (a structure); establish, found (a city); decide, determine, judge, deem
    \item[sternō, sternere, strāvī, strātum] lay out on the ground (or other surface), spread; lay (stones, etc.) to form a pavement or the like, lay (a pavement or floor); scatter over the ground, strew; stretch in a horizontal position, stretch out (one's) limbs; bring to the ground, level, knock down, lay flat; strike down, stretch lifeless, lay low; overthrow, defeat utterly
    \item[strepitus, strepitūs, m.] a sound, a noise (inexpressive or meaningless sound or noise, not speech); clamor, uproar; din, turmoil
    \item[studeō, studēre, studuī, ———] \marginnote{*}give attention, be eager, be zealous; take pains, be  diligent, be busy with, be devoted, apply oneself; strive after, pursue, desire, wish
    \item[stupor, stupōris, m.] numbness, insensibility; bewilderment, stupefaction; dullness of apprehension, stupidity; (metonymically) a stupid person, a clod
    \item[sub] \marginnote{*}\textit{preposition with accusative and ablative} (with accusative) to a position below or underneath, under; down to, down into, down under; up to, to the edge of; just before (a time or event); until, up to; directly after, in response to, as a consequence of; (with ablative) under, below, beneath, underneath; at the foot of, below; immediately behind, next to
    \item[subdūcō, subdūcere, subdūxī, subductum] draw up, raise, lift; withdraw, draw off, lead away, take away, subtract, remove; steal
    \item[subtīlitās, subtīlitātis, f.] fineness, extreme slenderness; fineness of detail, delicate work; fineness of perception, acuteness, refinement; attentiveness to finer points, subtlety, minute thoroughness
    \item[succēdō, succēdere, successī, successum] move below, come under; come to the foot (of), come up (to), come as far (as), approach close (to); advance to a higher level, move upwards, advance up; move up into the position (of), move up (somewhere) as a replacement or relief; become successor (to someone), take over (from), succeed (to an office, etc.), take the place (of); come after; succeed, turn out well, prosper
    \item[sufficiō, sufficere, suffēcī, suffectum] supply, provide (especially as a replacement); appoint to a magistracy (in place of another or in case of vacancy); substitute (one thing for another); have sufficent strength, be equal (to), stand up (to); be sufficient (in quantity, extent, degree, etc.), suffice
    \item[suffodiō, suffodere, suffōdī, suffossum] dig or tunnel under; undermine (i.e., dig under a wall so as to weaken its foundations); pierce below
    \item[suī] \marginnote{*}(\nth{3} person reflexive pronoun) himself, herself, itself, themselves
    \item[sum, esse, fuī, futūrum] \marginnote{*}be, exist; be real, be true
    \item[summus, summa, summum] highest, topmost, uppermost; the top or summit (of); latest (in time or sequence), final; developed to the height of excellence, perfect; highest in rank, supreme, most exalted
    \item[sūmō, sūmere, sūmpsī, sūmptum] \marginnote{*}\marginnote{*}take into one's hands, take up; put on; take (food, drink, medicine, etc.); assume possession of, take; get, procure, borrow, exact, derive; take and apply (to some purpose), spend (money, time, effort); adopt as suitable for some purpose, pick, choose; adopt, make one's own (a child, a practice, an idea, etc.)
    \item[super] \marginnote{*}\textit{adverb and preposition with accusative and ablative} (adverb) over, above, in a higher position; on the surface or upper part, on top (of); in addition, besides (= \textit{insuper}); after what has been taken, left over, remaining (often elliptical for \textit{superest}); to an excess degree, more than sufficiently, too much, in excess; (with accusative) over, above, beyond; in close succession, (soon) after; in addition to, over and above, besides; beyond, more than, to a greater degree or extent than, above; (with ablative) higher than, over; on the top of, on the uppermost part of; about, concerning, on; in close succession to, on top of; in addition to, over and above
    \item[superaedificātus, superaedificāta, superaedificātum] built on top of (not in \textbf{OLD})
    \item[superextruō, superextruere, superextrūxī, superextūctum] build or pile on top of or onto (not in \textbf{OLD})
    \item[supersum, superesse, superfuī, superfutūrum] \marginnote{*}be higher (than), be on top (of); be set (over); be superior (to); be additional to the requirements or needs (of), be superfluous, be in excess (of), be beyond the capacity (of); remain, be left over, survive, remain in existence; remain (to be done, performed, handled, etc.)
    \item[suppōnō, suppōnere, supposuī, suppositum] place under or beneath, place at the foot of; apply from below; place under the authority or control (of), make subject (to); place below (in writing or speech), append; put in place of another, substitute; put forward fraudulently, put forward as (someone or something) that (they, it) are not, falsify, forge, counterfeit
    \item[suprā] \marginnote{*}\textit{preposition with ablative} on the upper side (of), on the top (of), above; earlier than; more than, exceeding, beyond; in charge of, over, in command of
    \item[suspiciō, suspiciōnis, f.] a suspicion, mistrustful feeling; a slight idea, inkling; a faint indication, suggestion, trace
    \item[suspicor (1)] form an idea of, guess, imagine, infer; suspect, have an inkling of (something wrong); suspect, believe (a person) guilty (of something), be suspicious of, mistrust
    \item[suus, sua, suum] \marginnote{*}of oneself, belonging to oneself; (emphatic) one's very own, belonging to someone (and no other), particularly associated or characteristic of one; his own, her own, their own; his, her, its, their
    \item[taceō (2)] \marginnote{*}be silent, not speak, say nothing; say nothing about, omit mention of, pass over in silence (with \textit{dē} or accusative direct object)
    \item[tactiō, tactiōnis, f.] a touching, the act of touch
    \item[tactus, tactūs, m.] action or fact of physical contact, touch; sense of touch; tactile qualities (i.e. the touch or feel of something); contact, influence
    \item[tālis, tāle] \marginnote{*}of such a character, kind, or type; of such an exceptional (for good or bad) sort, such (a)
    \item[tam] \marginnote{*}to such a degree, to such an extent, to that extent, so, so much
    \item[tamdiū] (for) so long, all this time
    \item[tamen] \marginnote{*}all the same, nevertheless, yet, just the same, in spite of what has been said
    \item[tamquam] \marginnote{*}in the same way, to the same degree, just as; a kind of, quasi- (when applying a term to something improperly); (with conditional clause) just as (if); as for example; (with subjunctive) in the same way as if, as though; (with subjunctive) as though (introducing a hypothesis or something contrary-to-fact); (indicating a circumstance as the basis for some action) on the ground that
    \item[tandem] \marginnote{*}(for emphasis, expressing a strong sense of protest or impatience) really, I ask you, after all; after some time, at last, at length, finally
    \item[tangō, tangere, tetigī, tāctum] \marginnote{*}touch; be immediately next to, border on; arrive at, reach; (of feelings, etc.) touch, affect, affect with emotion; make slight mention of, touch on; (colloquial) deprive fraudulently, steal by cheating
    \item[tantum] \marginnote{*}to such an extent or degree; for such a time, for such a distance; only, just, merely
    \item[tantummodo] \marginnote{*}only, merely, just
    \item[tantus, tanta, tantum] \marginnote{*}so big, so great (in size, importance, degree, etc.), so much; (plural) so many, so vast a number of
    \item[temptō/tentō (1)] see \textit{tentō}
    \item[tempus, temporis, n.] \marginnote{*}(moment or period of) time; the time, the date (for something), (appointed) time; season, any recurrent period or phase; (usually plural) a period in history, times; proper or due time; favorable time, opportunity; circumstances (existing at a particular time), moment, occasion
    \item[tendō, tendere, tetendī, tēnsum/tentum] \marginnote{*}extend outwards or upwars, stretch or hold out, offer; direct, aim; stretch out, extend (in time or space), spread out; pitch camp; exert strain on, pull tight; direct (one's steps, course, etc.), proceed; reach (to or as far as); progress, move on (to another stage, condition, etc.); (intransitive) press on, insist; (intransitive) make an effort, exert oneself, strain; (with infinitive) strive, aim (to do)
    \item[tenebrae, tenebrārum, f.] \marginnote{*}darkness; (figurative) mental darkness, ignorance, lack of knowledge or understanding; obscurity, concealment, a condition where something is unknown or unobserved
    \item[tentō/temptō (1)] \marginnote{*}handle, touch, feel; test, seek to discover the state of; test, try out, attempt, try (to do); examine, try to find out about; make an attempt on, try to get possession of
    \item[tenuis, tenue] thin, slender, narrow, fine, fine-meshed; (of various substances) watery, rarefied, insubstantial; pure, clear, fine
    \item[terminō (1)] mark boundaries of; form a boundary of, border; define, delimit, determine the limits of; limit, restrict; fix or lay down a limit; bring to a close or end, conclude; settle, decide
    \item[terra, terrae, f.] \marginnote{*}earth; land, ground, soil
    \item[timeō (2)] \marginnote{*}fear, be afraid, be fearful, be apprehensive, be afraid of, dread, apprehend
    \item[toga, togae, f.] a covering, clothing; Roman outer garment, toga
    \item[tollō, tollere, sustulī, sublātum] \marginnote{*}pick up, raise, lift, hoist; climb up, ascend; take (on to a ship or vehicle), pick up, take on board; raise the spirits or morale of, hearten, rouse; pick up and remove, take away, carry off; carry away, reap; take, steal; take out, remove, exclude; get rid of, remove, eliminate, destroy, kill, do away with, eliminate
    \item[tōtus, tōta, tōtum] \marginnote{*}the whole of, all, complete, every part of, throughout the whole, all over the; free from defect or damage, unimpaired, entire
    \item[tractō (1)] keep on pulling or dragging, drag about; handle, work (with the hands), manipulate, treat manually; have dealings with, have to do with, deal with, treat (in some manner); (reflexive) conduct oneself; manage, handle, employ (affairs, means, resources); carry out, practice, perform; examine, consider; (intransitive) deliberate, carry on a discussion; deal with, discuss, treat (a theme, subject, idea)
    \item[trānseō, trānsīre, trānsīvī/trānsiī, trānsitum] \marginnote{*}come or go across, cross over; move on; transfer allegiance, go over; change one's nature, appearance, etc., be transformed; proceed, be in transit, pass through; go through, run through; go past, pass by; overtake, pass, pass beyond, go farther than
    \item[trānsferō, trānsferre, trānstulī, trānslātum] bear across, carry or bring over; change the location of, transfer, transpose, shift, transplant; transfer (something) from one person (place, etc.) to another, transfer control or possession of; translate; bring (someone, something) over to (something new); change, transform
    \item[trēs, tria] \marginnote{*}three
    \item[triangulāris, triangulāre] triangular
    \item[tribuō, tribuere, tribūtum] share out, divide, apportion; grant, bestow, award; allocate, devote, apply; attribute or ascribe (something to someone), impute, attribute (to something), ascribe (to a cause); place value (on), pay regard (to); give credit, pay respect (to)
    \item[tum/tunc] \marginnote{*}then, at that time, moment, date; next, after that; in addition, moreover, besides
    \item[turbō (1)] \marginnote{*}(intransitive) act turbulently, riot, revolt; (transitive) agitate, stir up, disturb
    \item[ubī/ubi] \marginnote{*}in what place? where?; where; when
    \item[ūllus, ūlla, ūllum] \marginnote{*}any, any at all; anyone, anything
    \item[ultimus, ultima, ultimum] \marginnote{*}most distant, farthest away, endmost; remotest in time, earliest; latest in time, final, ultimate, last (in sequence); final, critical, decisive
    \item[umquam] \marginnote{*}at any time, ever; at all
    \item[unde] \marginnote{*}from what place? where\dots from? whence? from which place, whence, from which (point, situation, source, etc.), from whom
    \item[ūniversālis, ūniversāle] having general application, universal, applying to all
    \item[ūnus, ūna, ūnum] \marginnote{*}one, a single
    \item[ūnusquisque, ūnaquaeque, ūnumquidque/ūnumquicque/ūnumquodque] each one, every single one
    \item[ūsitātus, ūsitāta, ūsitātum] familiar, everyday, commonly used or practiced
    \item[ut] \marginnote{*}as; when; how; (introducing various subordinate clauses)
    \item[ūter, ūtra, ūtrum] which (of two)?
    \item[ūtor, ūtī, ūsus sum] \marginnote{*}use, make use of, put to use, employ; manage, handle, control; exercise, engage in, practice; (with adverb or predicate adjective) put to (such-and-such a) use; experience, undergo, enjoy
    \item[ūtrimque] from or on both sides or ends
    \item[vacō (1)] be vacant, empty, or unfilled; be without occupants or an owner; be destitute or devoid (of), be free (from); be left free for, be available to; be unengaged, have leisure, be free, have time to spare
    \item[validus, valida, validum] \marginnote{*}strong, powerful, robust, sturdy; thriving, flourishing; fit, in sound health; solid, substantial; powerful, vehement, intense
    \item[vapor, vapōris, m.] steam, exhalation, vapor; heat, warmth
    \item[varietās, varietātis, f.] difference, diversity, variety
    \item[vel] \marginnote{*}(non-exclusive) or; \textit{vel\dots vel} either\dots or
    \item[velut] \marginnote{*}as for example, for instance; in the same way that, just as, just like; as it were, so to speak; as if, as though; (giving a justification) as being
    \item[veniō, venīre, vēnī, ventum] \marginnote{*}come, approach, arrive
    \item[verbum, verbī, n.] \marginnote{*}a word; (sometimes specifically) a verb
    \item[vereor, verērī, veritus sum] \marginnote{*}fear, be afraid of; show reverence or respect for, be in awe of
    \item[vērisimilis, vērisimile] seeming true, appearing true, consistent with the truth, like the truth
    \item[veritās, veritātis, f.] truth; reality; the state of being real or actual
    \item[vertō, vertēre, vertī, versum] \marginnote{*}(cause to) turn, spin; depend on, turn on, hinge on; overturn, knock down, ruin; turn around, invert, reverse, transpose; (cause to) turn the other way; (cause to) turn tail, make flee, put to flight; turn or change the position or direction of (something)
    \item[vērumtamen] but even so, still, nevertheless
    \item[vērus, vēra, vērum] \marginnote{*}real, genuine, actual; true; proper
    \item[vestiō (4)] cover with a garment, provide with clothing, dress, clothe, cover
    \item[vestis, vestis, f.] an item of clothing, garment, piece of clothing; clothes (collectively); garments, clothing
    \item[vetō, vetāre, vetuī, vetitum] \marginnote{*}forbid, prohibit, hinder, prevent
    \item[vetus, veteris] \marginnote{*}old, having lived a long time; belonging to the past, old-time
    \item[via, viae, f.] \marginnote{*}road, track, path (made for travel); passage, channel, duct, course; a journey, march, fact or instance of traveling
    \item[vidēlicet] it is plain to see, it is clear (that); evidently, plainly; (ironic) of course, no doubt, obviously; that is to say, that is
    \item[videō, vidēre, vīdī, vīsum] \marginnote{*}see; notice, observe; be a witness of; meet, see (people, events, etc.); appreciate, perceive, note with understanding
    \item[vigil, vigilis, m.] a sentry, guard, person who keeps watch
    \item[vigil, vigilis] awake, watchful, wakeful; alert, vigilant, paying watchful attention
    \item[vigilia, vigiliae, f.] wakefulness, sleeplessness, lying awake; action or fact of keeping watch, a patrol, a guard; watchful attention, vigilance
    \item[vigilō] stay awake; be watchful, be alert
    \item[vīs, vis, f.] \marginnote{*}force, violence; compulsion, constraint; power, influence, strength
    \item[vīsiō, vīsiōnis, f.] the act or sense of seeing, vision; an appearance, sight; mental image
    \item[vīsus, vīsūs, m.] faculty or power of seeing, sight, vision; action of seeing, glance, gaze, sight; that which his seen, a sight
    \item[vīta, vītae, f.] \marginnote{*}life; a way of life, a mode of life
    \item[vītō (1)] \marginnote{*}move out of the way, avoid, dodge, keep out of the way of, keep clear of; steer clear of, shun, avoid
    \item[vitrum, vitrī, n.] glass; something made of glass
    \item[vix] \marginnote{*}with difficulty; hardly, scarcely, barely
    \item[vocō (1)] \marginnote{*}call, summon, invite; invoke, call upon; call together, convoke; designate, call by name, call (something)
    \item[volō, velle, voluī, ———] \marginnote{*}wish, want, desire; be willing, be prepared (to); be about (to), be on the point (of)
    \item[voluntās, voluntātis, f.] \marginnote{*}will, volition; one's will or wish, what one wants to happen; readiness (to do or agree to someting), willingness, approval; choice, option (as opposed to compulsion); deliberate choice; intention, inclination; disposition; favorable disposition, goodwill, sympathy
    \item[vox, vōcis, f.] \marginnote{*}(the human) voice; a sound (produced by voice), utterance; sound (in general); (singular or plural) spoken utterance, words
    \item[vulgō] \marginnote{*}publicly; in the usual way, according to the general rule or practice, generally, commonly; commonly, habitually, regularly; all together, en masse; far and wide, all over the place, at wide
    \item[vulgus, vulgī, n. (m.)] \marginnote{*}the common people, general public; a multitude, crowd (often derogatory); a flock or group of animals
    \item[vultus, vultūs, m.] \marginnote{*}facial expression, look, countenance; face, front of the head; a surface, a face (of an object); one's gaze, one's view; appearance of a face, looks, features; (of a physical object) outward appearance, face; (of an abstract thing) aspect, appearance
\end{description}
% -]] Vocabulary


\backmatter
    \bibliographystyle{apa}
    \bibliography{descartes}

\end{document}
% -]]
