% [[- LaTeX prelude
\documentclass[12pt,letterpaper]{book}

\usepackage[no-math]{fontspec}
\setmainfont{Baskerville}

\usepackage[nolocalmarks]{polyglossia}
\setdefaultlanguage{english}
\setotherlanguage[variant=medieval]{latin}
\setotherlanguage[variant=ancient]{greek}
\newfontfamily\greekfont[Script=Greek]{Times New Roman}
\uccode`\u=`\U

\usepackage{emptypage}
\usepackage{etoolbox}
\usepackage{geometry}
\usepackage{marginnote}

\makeatletter
\long\def\@mn@@@marginnote[#1]#2[#3]{%
  \begingroup
    \ifmmode\mn@strut\let\@tempa\mn@vadjust\else
      \if@inlabel\leavevmode\fi
      \ifhmode\mn@strut\let\@tempa\mn@vadjust\else\let\@tempa\mn@vlap\fi
    \fi
    \@tempa{%
      \vbox to\z@{%
        \vss
        \@mn@margintest
        \if@reversemargin\if@tempswa
            \@tempswafalse
          \else
            \@tempswatrue
        \fi\fi

          \llap{%
            \vbox to\z@{\kern\marginnotevadjust\kern #3
              \vbox to\z@{%
                \hsize\marginparwidth
                \linewidth\hsize
                \kern-\parskip
                \mn@parboxrestore
                \marginfont\raggedleftmarginnote\strut\hspace{\z@}%
                \ignorespaces#1\endgraf
                \vss
              }%
              \vss
            }%
            \if@mn@verbose
              \PackageInfo{marginnote}{xpos seems to be \@mn@currxpos}%
            \fi
            \begingroup
              \ifx\@mn@currxpos\relax\else\ifx\@mn@currpos\@empty\else
                  \kern\@mn@currxpos
              \fi\fi
              \ifx\@mn@currpage\relax
                \let\@mn@currpage\@ne
              \fi
              \if@twoside\ifodd\@mn@currpage\relax
                  \kern-\oddsidemargin
                \else
                  \kern-\evensidemargin
                \fi
              \else
                \kern-\oddsidemargin
              \fi
              \kern-1in
            \endgroup
            \kern\marginparsep
          }%
      }%
    }%
  \endgroup
}
\makeatother

\usepackage{csquotes}
\usepackage[style=windycity,citetracker=context,backend=biber,%
        hyperref=false]{biblatex}
\addbibresource{descartes.bib}

\usepackage{enumitem}
\setlist{noitemsep}
\usepackage[super]{nth}

\usepackage[noend,nofamiliar,noledgroup,series={A}]{reledmac}
\Xlemmaseparator[A]{:}

% Insert AT numbers and a marker for where the division occurs
\newcommand{\at}[1]{%
    |\ledsidenote{{#1}}%
}

% Wrapper for textual notes. Use as follows:
% \var{word or phrase}{comment}
\newcommand{\var}[2]{%
    {#1}\footnote{{#1} : {#2}}%
}
% Second wrapper for textual variants. Use as follows:
% \vvar{word or phrase}{source}{alternative}{source}
\newcommand{\vvar}[4]{%
    {#1}\footnote{{#1} \textbf{{#2}} : {#3} \textbf{{#4}}}%
}
% A wrapper for introducing new items into the commentary
\newcommand{\lemc}[1]{\textbf{{#1}:}}
\newcommand{\lem}[1]{\textbf{{#1}}}

% Clear space under a chunk of text before printing notes.
\newcommand{\prenotes}{%
    \bigskip
    \bigskip
    \footnoterule
    \bigskip
}

\begin{hyphenrules}{latin}
    \hyphenation{pro-inde su-per-ex-tru-xi}
\end{hyphenrules}

\begin{hyphenrules}{english}
    \hyphenation{Des-cartes pos-sent}
\end{hyphenrules}

\raggedbottom

\addtocontents{toc}{\setcounter{tocdepth}{0}}
\renewcommand\thesection{}
\renewcommand\thesubsection{}
\makeatletter
\def\@seccntformat#1{\csname #1ignore\expandafter\endcsname\csname the#1\endcsname\quad}
\let\sectionignore\@gobbletwo
\let\latex@numberline\numberline
\def\numberline#1{\if\relax#1\relax\else\latex@numberline{#1}\fi}
\makeatother

\usepackage{fancyhdr}
\pagestyle{fancy}
\fancyhf{}
\renewcommand{\headrulewidth}{0pt}
\fancyhead[RO, LE]{\thepage}
\fancyhead[CE]{\leftmark}
\fancyhead[CO]{\rightmark}
\renewcommand{\chaptermark}[1]{\markboth{{\textenglish{\MakeUppercase{#1}}}}{}}
\renewcommand{\sectionmark}[1]{\markright{\MakeUppercase{#1}}{}}

\usepackage[hidelinks,hyperfootnotes=false,linktoc=all]{hyperref}
% -]] Latex prelude

% [[- Document
\begin{document}

\nocite{*}

% [[- Title page
\begin{titlepage}

\begin{center}

\huge \textit{Meditationes de prima philosophia}

\huge René Descartes 

\vskip2in

\large \copyright Peter Aronoff \the\year

(See LICENSE for details)

\vskip1in

\textbf{NB}: This work is in progress and likely to change.

\vskip2in

\newpage

\end{center}

\end{titlepage}
% -]] Title page


\frontmatter
    \include{descartes-toc}
    % [[- Chapter title
\chapter{Abbreviations}
% -]] Chapter title

% [[- Abbreviations
\begin{itemize}
    \item[ALQ] \fullcite{alquie2010}
    \item[AT] \fullcite{at}
    \item[CB] \fullcite{cottingham1976}
    \item[CSM] \fullcite{csm}
    \item[CSMK] \fullcite{csmk}
    \item[EB] \fullcite{adam1975}
    \item[NLG] \fullcite{mahoney2001}
    \item[OLD] \fullcite{old1982}
\end{itemize}
% -]] Abbreviations

% [[- Works of Descartes
References to the writings of Descartes in this book work as follows. CSM and CSMK refer to the three-volume English translation of the philosophical works and letters of Descartes. AT refers to the twelve-volume edition of the original French and Latin works and letters of Descartes. When I refer to any of Descartes's work, I will always give a double reference: first to the English and then to the original edition. So, for example, the citation ``(\textbf{CSM} I 192; \textbf{AT} VI 1)'' points to the first page of \textit{Discourse on the method} both in English and French.
% -]] Works of Descartes



\mainmatter
    % [[- Chapter title
\chapter*{Introduction}
% -]] Chapter title

% [[- Descartes' life
\section*{Descartes' Life}

TODO
% -]] Descartes' life

% [[- About The Text
\section*{About The Text}

I made the text for this edition by comparing the editions of TODO

TODO: Include brief discussion of how I've Classicized the text. (No accents, no letter j-, replaced \& with \textit{et}.)

% -]] About The Text

    % [[- Chapter title
\chapter{Synopsis}
\markboth{Meditationes de prima philosophia}{Synopsis}
% -]] Chapter title

% [[- Synopsis

% [[- Introduction
Speaking in his own voice as author, Descartes uses the brief synopsis to tell readers what to expect, but also what not to expect, from the six meditations. Each meditation receives one paragraph. In addition to summary of the material in the meditations, the synopsis also informs us about some of Descartes implicit goals and concerns. For example, Descartes says more than once in the synopsis that he intends to ``lead readers away from the senses.'' This is not something that the inquirer ever mentions explicitly in the meditations themselves. If we keep this goal in mind, however, it can help us to understand several of the inquirer's explicit arguments better. The synopsis offers several insights such as this, and these insights make it well worth reading.
\clearpage
% -]] Introduction

% [[- 1st meditation
\begin{center}
    \beginnumbering
    \numberlinefalse
    \pstart
    \textit{Synopsis sex sequentium Meditationum}\ledsidenote{12}
    \pend
    \endnumbering
\end{center}

\beginnumbering
\pstart
\textbf{1.} \begin{latin}In prima, causae exponuntur propter quas de rebus omnibus, praesertim materialibus, possumus dubitare; quandiu scilicet non habemus alia scientiarum fundamenta, quam ea quae antehac habuimus. Etsi autem istius tantae dubitationis utilitas prima fronte non appareat, est tamen in eo maxima quod ab omnibus praeiudiciis nos liberet, viamque facillimam sternat ad mentem a sensibus abducendam; ac denique efficiat, ut de iis, quae postea vera esse comperiemus, non amplius dubitare possimus.\end{latin}
\pend
\endnumbering

\prenotes

\textbf{§1.} The first meditation sets forth reasons to doubt everything---at least until we have firmer foundations for science than we have now. This doubt will provide several benefits: (i) it will free us from preconceived opinions, (ii) the doubt will lead us away from overreliance on the senses, and (iii) whatever truths we establish after this initial doubt will be secure against future doubt.

\lemc{1 In prima} supply \textit{meditatione}.

\lemc{1--2 praesertim materialibus} although Descartes plans to doubt ``everything'', his particular focus will be beliefs based on the evidence of the senses. Thus, we will be able to doubt ``material things especially.''

\lemc{2 quandiu scilicet non habemus} if we take the punctuation before this clause seriously, we can supply something like \textit{de his rebus dubitare possumus} before \textit{quandiu}. However, an simpler approach is to translate as if the semicolon is equivalent to a modern comma or emdash. That is, ``Reasons are set out why we can doubt all things---at least as long as we don't have better foundations for science.''

\lem{2 quandiu} = \textit{quamdiu}.

\lemc{2--3 scientiarum fundamenta} Descartes uses the Aristotelian phrase ``first philosophy'' in the title of this work. That already indicates his interest in the most basic and essential truths. At this point, readers may not know precisely what he means by the ``foundations of the sciences,'' but Descartes directs readers' attention to his focus very explicitly.

\lemc{4 prima fronte} an idiomatic expression meaning ``at first glance'' or ``on first appearance.''

\lemc{4 in eo\dots quod} the \textit{eo} anticipates the following \textit{quod} clause. The greatest usefulness of the doubt lies ``in this, namely that.'' The \textit{quod} serves only to nominalize the clause it introduces. You can generally translate as ``that'' or ``the fact that.'' See \textbf{AG} §572 and \textbf{NLS} §241 for more examples and discussion. Note, however, that in later Latin the verb in such \textit{quod} clauses can be subjunctive without that necessarily implying any unreality or conditionality. Sometimes, as here in Descartes, the subjunctive merely indicates that the clause is subordinate.

\lemc{5 praeiudiciis} Descartes considers preconceived opinions to be a significant source of error in our lives (\textbf{CSM} I 218; \textbf{AT} VIII 37). Etienne Gilson, commenting on a passage from the \textit{Discourse on the method} (\textbf{CSM} I 120; \textbf{AT} VI 18), describes preconceived opinion (\textit{la prévention} in French) as ``the persistence in our thought of unreflective judgments that we made in our childhood and that guide us now as if we had proven them'' (\textcite[199]{gilson2005}; for a fuller discussion see \textcite[199]{gilson1987}).

\lemc{5--6 ad mentem a sensibus abducendam} a gerundive with \textit{ad} expressing purpose. Take the phrase with \textit{viam} rather than \textit{sternat}: the method of doubt will create ``the easiest road to lead our minds away from the senses.'' Descartes reveals the anti-empirical tendency of his method far more bluntly here than he does in the text of the first meditation.

In several Platonic dialogues (e.g., in the \textit{Phaedo}, \textit{Symposium}, and \textit{Republic}) we find arguments that sensation cannot yield knowledge and that we gain knowledge only through abstract and pure reason. Although scholars do not agree about how well Descartes knew the actual texts of Plato,\footnote{For example, see \textcite[556, especially note 11]{grene1999}.} we don't need to see a specific allusion here. Plato's criticism of the senses and praise of reason became a philosophical commonplace. In addition, the Neoplatonic influence on Christianity meant that the related contrast between body and soul became a spiritual commonplace as well. All the more fitting, then, that Descartes's philosophical meditations try to lead the readers to shift attention away from body and the senses and towards the soul and reason.

\lemc{6--7 ac denique efficiat\dots possimus} the doubts that we find at the beginning of the \textit{Meditations} will somehow free us from future doubts. Descartes does not make clear how this is true, but it is an essential feature of his method: he employs skepticism in order to free us from skeptical doubts.

% -]] 1st meditation

% [[- 2nd meditation
\clearpage

\beginnumbering
\pstart
\textbf{2.} \begin{latin}In secunda, mens quae, propria libertate utens, supponit ea omnia non existere de quorum existentia vel minimum potest dubitare, animadvertit fieri non posse quin ipsa interim existat. Quod etiam summae est utilitatis, quoniam hoc pacto facile distinguit quaenam ad se, hoc est, ad naturam intellectualem, et quaenam ad corpus pertineant. Sed quia forte nonnulli rationes de animae immortalitate illo in loco expectabunt, eos hic monendos \at{13} puto me conatum esse nihil scribere quod non accurate demonstrarem; ideoque non alium ordinem sequi potuisse, quam illum qui est apud Geometras usitatus, ut nempe omnia praemitterem ex quibus quaesita propositio dependet, antequam de ipsa quidquam concluderem. Primum autem et praecipuum quod praerequiritur ad cognoscendam animae immortalitatem, esse ut quam maxime perspicuum de ea conceptum, et ab omni conceptu corporis plane distinctum, formemus; quod ibi factum est.\end{latin}
\pend
\endnumbering

\prenotes

\textbf{§2.} The second meditation establishes a certainty: the mind that doubts must exist. This single certainty allows us to investigate more clearly the difference between mind and body. However, Descartes warns readers here that he will not argue for the immortality of the soul in the second meditation. (Note also that Descartes appears to assume the identity of ``mind'' (Latin \textit{mens}) and ``soul'' (Latin \textit{anima)}). Throughout this paragraph he moves back and forth between the two terms without appearing to distinguish between them at all.)  Descartes follows a method that forbids arguing for a proposition, such as the immortality of the soul, until establishing all the premises required to support the proposition. Descartes will not be in a position to do so until the sixth meditation. Nevertheless, the second meditation is an important step towards a proof of the soul's immortality since we gain a clear conception of the soul and its distinction from the body in this meditation.

\lemc{1--2 mens\dots animadvertit} The multiple subordinate clauses can make the structure of this sentence difficult to see initially. The subject is \textit{mens} and the main verb \textit{animadvertit}: ``the mind (which supposes\dots) notices.''

\lemc{2 vel} an adverb meaning ``even'' rather than the conjunction meaning ``or.''

\lemc{2--3 animadvertit fieri non posse} there is no subject accusative for this indirect statement. Supply ``this'' or ``that'' referring back to the previous clauses. The ``mind notices that this (i.e., for it to doubt everything that is not absolutely certain) cannot happen.''

\lem{3 quin} = ``without'' or ``unless.''

\lemc{3 interim} is elliptical. The mind must exist ``in the meantime,'' that is, ``during the time that it is doubting.''

\lem{3 Quod} is a connective relative. Translate as if it were ``Et hoc'' or ``Et illud.'' The antecedent is \textit{that the mind must exist while it doubts} or \textit{the certainty concerning the existence of the mind}.

\lem{3 summae\dots utilitatis} is a genitive of description in the predicate part of the sentence: ``And this fact is of the greatest utility.''

\lemc{4 distinguit} the subject of the verb is still \textit{mens} from the previous sentence.

\lem{4 hoc est} is equivalent in meaning to \textit{id est}. Translate ``that is'' or ``namely.''

\lem{5 forte} means ``perhaps'' rather than ``by chance'' here.

\lem{5 rationes} means ``arguments'' here.

\lemc{6--7 eos hic monendos puto me conatum esse nihil scribere} the phrase \textit{eos hic monendos} depends on \textit{puto}, \textit{me conatum esse} depends on \textit{eos hic monendos}, and finally \textit{nihil scribere} depends on \textit{me conatum esse}. Since readers might expect a discussion of the soul's immortality, Descartes explains ``I think that they should be warned here that I have tried to write nothing (except what is certain).''

Everything in the rest of this rather long paragraph implicitly depends on \textit{eos hic monendos}. That is why, despite the punctuation indicating several new sentences, the remaining main verbs in this paragraph are in the infinitive with accusative subjects.

\lemc{7 quod\dots demonstrarem}  a relative clause of characteristic. Descartes attempted to write nothing ``of the kind which'' he could not prove completely.

\lemc{7--9 ideoque\dots concluderem} Descartes follows the method of ``geometers'' insofar as he doesn't draw any conclusion until after he has fully explained the premises the conclusion depends on. This method requires Descartes to lay a good deal of groundwork before he can prove that the soul is immortal.

\lemc{ut\dots praemitterem} this substantive clause is in apposition to, and explains, \textit{non aliam ordinem} from line 7.

\lemc{10--12} In order to understand the immortality of the soul, we must first have an absolutely clear understanding of the soul itself. The second meditation focuses on this prerequisite.

\lemc{10--12 Primum\dots praecipuum\dots esse ut\dots formemus} the main verb is infinitive in indirect statement following \textit{eos hic monendos} above; the accusative subject of \textit{esse} here is the substantive \textit{ut} clause; and \textit{primum} and \textit{praecipuum} are predicate accusative, describing the \textit{ut} clause. In English, ``(People should be warned that) it is first and foremost that we develop (a clear conception of the soul).'' More idiomatically, ``First and foremost, we should develop (a clear conception of the soul).''

\clearpage

\beginnumbering
\pstart
\setline{13}
\begin{latin}
    \textenglish{\textbf{2. (cont.)}} Praeterea vero requiri etiam ut sciamus ea omnia quae clare et distincte intelligimus, eo ipso modo quo illa intelligimus, esse vera: quod ante quartam Meditationem probari non potuit; et habendum esse distinctum naturae corporeae conceptum, qui partim in ipsa secunda, partim etiam in quinta et sexta formatur; atque ex his debere concludi ea omnia quae clare et distincte concipiuntur ut substantiae diversae, sicuti concipiuntur mens et corpus, esse revera substantias realiter a se mutuo distinctas; hocque in sexta concludi. Idemque etiam in ipsa confirmari ex eo quod nullum corpus nisi divisibile intelligamus, contra autem nullam mentem nisi indivisibilem: neque enim possumus ullius mentis mediam partem concipere, ut possumus cuiuslibet \edtext{quantamvis}{\Afootnote{\nth{1} ed. quantumvis}} exigui corporis; adeo ut eorum naturae non modo diversae, sed etiam quodammodo contrariae agnoscantur. Non autem ulterius ea de re in hoc scripto me egisse; tum quia haec sufficiunt ad ostendendum ex corporis corruptione mentis interitum non sequi, atque sic ad alterius vitae spem mortalibus faciendam; tum etiam quia praemissae, ex quibus ipsa mentis immortalitas concludi potest, ex totius Physicae explicatione dependent:
\end{latin}
\pend
\endnumbering

\prenotes

\lemc{13--15 Praeterea\dots non potuit} a second requirement, before proving the immortality of the soul, is that we know that everything that we understand clearly and distinctly is true just as we understand it; we won't fulfill this requirement until the fourth meditation.

\lemc{13 requiri} indirect statement dependent on \textit{eos hic monendos} above.

\lemc{13 ut sciamus} a substantive clause dependent on \textit{requiri}.

\lemc{13--15 ea omnia\dots esse vera} indirect statement depending on \textit{sciamus}.

\lemc{15--17 et habendum\dots formatur} a third requirement is that we have a distinct concept of body; this is developed in the second, fourth, and sixth meditation.

\lemc{17--19} once these preliminary matters are understood, we can establish that when things are clearly and distinctly conceived of as different substances, we can be sure that they are in fact different. This is the case for mind and body, and we demonstrate their difference in the sixth meditation.

\lemc{17 ex his} the demonstrative picks up and summarizes the previous three requirements: (i) a clear conception of soul, (ii) proof that whatever we understand clearly and distinctly is just as we understand it, (iii) a distinct conception of body. Once we have these three, we can establish that mind and body are different.

\lemc{18--19 revera\dots realiter} it's not clear to me whether \textit{revera} and \textit{realiter} make two different points, or if the repetition is simply for emphasis.

\lem{19--23} provide a further argument: body is essentially divisible, while mind is essentially indivisible; thus, not only are body and mind different, they are even contrary.

\lem{21 mediam partem} = ``a half.''

\lemc{22 possumus cuiuslibet quantamvis exigui corporis} although we cannot conceive of a divided mind, we can conceive of any possible fraction (literally \textit{quantamvis} is ``any amount you like'') of any tiny body at all (\textit{cuiuslibet exigui corporis}). Supply the complementary infinitive \textit{concipere} from the previous clause.

\lemc{24--27} Descartes has not discussed this matter any further in the \textit{Meditations} for two reasons. First, he feels he has said enough to give us hope for an afterlife. Second, he believes that a full proof of the immortality of the soul would require a full treatment of physics.

\lemc{24 ulterius ea de re\dots me egisse} we are still in indirect statement dependent on \textit{eos hic monendos}. \textit{de aliqua re agere} is an idiom meaning ``to discuss something'' or ``to write about something.''

\lem{24--26 tum\dots tum etiam} = ``on the one hand\dots on the other hand'' or ``not only\dots but also.'' 

\lem{24 haec} refers to what Descares has written in this book, which is enough (\textit{sufficiunt}) for his present purposes.

\lemc{24--25 ad ostendendum} the gerund \textit{ostendendum} governs the following indirect statement \textit{ex corporis corruptione mentis interitum non sequi}.

\lemc{26 spem mortalibus faciendam} take \textit{mortalibus} as indirect object rather than dative of agent. The arguments seen so far provide sufficient reason ``to give mortals hope of another life.''

\lemc{26--27 ex\dots dependent} we would say ``depend on,'' but the Latin idiom is ``depend from.''

\clearpage

\beginnumbering
\pstart
\setline{28}
\begin{latin}
    \textenglish{\textbf{2. (cont.)}}  primo \at{14} ut sciatur omnes omnino substantias, sive res quae a Deo creari debent ut existant, ex natura sua esse incorruptibiles, nec posse unquam desinere esse, nisi ab eodem Deo concursum suum iis denegante ad nihilum reducantur; ac deinde ut advertatur corpus quidem in genere sumptum esse substantiam, ideoque nunquam etiam perire. Sed corpus humanum, quatenus a reliquis differt corporibus, non nisi ex certa membrorum configuratione aliisque eiusmodi accidentibus esse conflatum; mentem vero humanam non ita ex ullis accidentibus constare, sed puram esse substantiam: etsi enim omnia eius accidentia mutentur, ut quod alias res intelligat, alias velit, alias sentiat, etc., non idcirco ipsa mens alia evadit; humanum autem corpus aliud fit ex hoc solo quod figura quarumdam eius partium mutetur: ex quibus sequitur corpus quidem perfacile interire, mentem autem ex natura sua esse immortalem.
\end{latin}
\pend
\endnumbering

\prenotes

\lemc{28--31 primo\dots ac deinde} Descartes gives two reasons that a complete knowledge of physics is necessary to establish the soul's immortality. Each of the two reasons is given as a purpose clause: first, \textit{ut sciatur}, and next \textit{ut advertatur}.

\lemc{28 sciatur} an impersonal passive with an indirect statement (\textit{omnes\dots substantias\dots esse incorruptibiles}) as its subject.

\lem{28 sive} means ``or rather'' here (\textbf{OLD} 9b). When used this way, the word introduces a correction or clarification. This clause explains more precisely what Descartes means by \textit{substantias}. The idea seems to be that only God is responsible for creating basic substances (e.g., trees) though other things can be built using materials from these basic substances (e.g., tables and chairs).

\lem{29 debent} means ``must,'' as often in Descartes.

\lemc{29 ut existant} a purpose clause following \textit{a Deo creari debent}.

\lemc{30 ab eodem Deo concursum suum iis denegante} the first part of this phrase (\textit{ab eodem Deo}) is an ablative of personal agent; the participle at the end (\textit{denegante}) governs the rest of the phrase (\textit{concursum suum iis}). Latin word order like this is very common and helpful; the noun and participle show readers where the phrase begins and ends without punctuation. (\textbf{NB}: it is only a rule of thumb, not a hard and fast rule, that when a present participle ends in -\textit{e} it is an ablative absolute. In many cases, such as this one, that will not be the case.)

\lemc{30 concursum} the root meaning of this noun is the action or result of running or coming together. As such, it can mean ``a gathering (of people or things),'' ``a combination, union,'' or ``a clash, an attack.'' By a later development, the word also came to mean ``an agreement, concurrence, or assent.'' Descartes uses it here in this last sense in a rather technical way: according to one view of things, everything other than God exists only insofar as God grants his continued ``assent'' to their existence. As such, things can only be destroyed when god ``denies his concurrence'' to them.

\lemc{30 ad nihilum reducantur} a roundabout and emphatic way of describing destruction.

\lemc{31 advertatur} this verb, like \textit{sciatur} above, is impersonal passive with two indirect statements as its subject (\textit{corpus\dots esse substantiam} and \textit{nunquam etiam perire}).

\lemc{31 in genere sumptum} a clarification of the larger claim. Body, \textit{taken in general}, is a substance. The idea is that physical matter, \textit{as such}, is a substance and thus cannot be destroyed except through an act of God. But specific compounds made out of matter can be broken down and thus ``destroyed'' in an everyday sense of the word.

\lemc{33-39 Sed corpus humanum\dots esse immortalem} the human body is merely an assemblage of various limbs in a certain arrangment; the mind on the other hand is a true substance. Thus, the mind is immortal while the human body is perishable.

\lemc{34 accidentibus} this is a technical term in Aristotle and later philosophy. An ``accident'' is a non-essential property of a substance. A substance can remain what it is even if its non-essential properties, its accidents, change. So, for example, a person's hair color is such a non-essential property. According to Descartes, the human body differs from other bodies only in its accidents, not in its essential properties.

\lemc{36 ut quod} the \textit{ut} means ``as'' or ``for example.'' The \textit{quod} serves to nominalize the following clauses. Taken with the previous clause, that gives something like this: ``even if all <the mind's> accidents should change, for example <imagine> that it understands different things, wants different things, perceives different things, etc., the mind itself does not become something different for that reason.''

\lemc{36--37 ipsa mens alia evadit} take \textit{ipsa mens} as subject and \textit{alia} as a predicate nominative following \textit{evadit}, which functions here as a linking verb meaning ``turn out'' or ``become.''

\lemc{37 ex hoc solo quod} literally, ``from this thing alone, <namely> that;'' more idiomatically, ``merely because.'' As often \textit{hoc} points forward to \textit{quod}, which introduces a substantive clause.

\lemc{38 sequitur} the verb is impersonal, and the subjects are the two indirect statements \textit{corpus\dots perfacile interire} and \textit{mentem\dots esse immortalem}.
% -]] 2nd meditation

% [[- 3rd meditation
\clearpage

\beginnumbering
\pstart
\textbf{3.} \begin{latin}In tertia Meditatione, meum praecipuum argumentum ad probandum Dei existentiam satis fuse, ut mihi videtur, explicui. Verumtamen, quia, ut Lectorum animos quam maxime a sensibus abducerem, nullis ibi comparationibus a rebus corporeis petitis volui uti, multae fortasse obscuritates remanserunt, sed quae, ut spero, postea in responsionibus ad obiectiones plane tollentur; ut, inter caeteras, quomodo idea entis summe perfecti, quae in nobis est, tantum habeat realitatis obiectivae, ut non possit non esse a causa summe perfecta, quod ibi illustratur comparatione machinae valde perfectae, cuius idea est in mente alicuius artificis; ut enim artificium obiectivum huius ideae debet habere aliquam causam, nempe scientiam huius artificis, vel alicuius alterius a quo illam accepit, ita \at{15} idea Dei, quae in nobis est, non potest non habere Deum ipsum pro causa.\end{latin}
\pend
\endnumbering

\prenotes

\textbf{§3.} The third meditation attempts to prove that God exists. The proof involves some difficulties that the replies to objections will address. In particular, Descartes acknowledges that a lack of concrete physical examples may make his account unclear. However, Descartes is willing to give up some clarity because his priority is ``to draw readers' minds as far away as possible from the senses.''

\lemc{1--2 ad probandum Dei existentiam} a gerund taking a direct object is rare in classical Latin prose. Classical prose authors usually replace a gerund plus a direct object with a gerundive phrase; they put both the gerundive and the original direct object into the case of the original gerund. In this case, Caesar or Cicero probably would have written \textit{ad probandam Dei existentiam}. However, the meaning is essentially the same either way, and there are exceptions even in Roman authors.

\lemc{2--4} The purpose clause (\textit{ut\dots abducerem}) interrupts the causal clause (\textit{quia nullis ibi comparationibus\dots volui uti}) that it depends on. Translate in this order: \textit{quia} clause, purpose clause, main clause.

\lem{4 uti} is the present passive infinitive of \textit{utor}, not the alternative form of the adverb and conjunction \textit{ut}. (The infinitive \textit{uti} governs \textit{nullis\dots comparationibus\dots petitis}.)

\lem{5 ut} introduces an example of an objection.

\lem{5 caeteras} = \textit{cēteras} (with \textit{obiectiones} understood).

\lemc{6 quomodo\dots habeat} the objection comes in the form of a question; the subjunctive \textit{habeat} is probably potential: ``how could an idea of a supremely perfect being have\dots ?''

\lemc{6--7 obiectivae} this is a technical term, and Descartes discusses it further in the meditation itself.

\lemc{7 non possit non esse} the double negation makes a strong positive claim. If something ``is not able not to be,'' then it ``must be.'' (See the following lines where \textit{non potest non habere} corresponds to \textit{debet habere}.)

\lem{7 quod} is a connective relative. Descartes uses the neuter to refer to the entire question of the objection he has just raised.

\lem{7 ibi} picks up \textit{postea in responsibus ad obiectiones}. Descartes uses the example of a machine and its creator in his replies to the first set of objections (\textbf{CSM} II 75; \textbf{AT} VII 103).

\lemc{9--10 nempe\dots accepit} the cause of a machine is the knowledge of the person who makes it. Descartes adds an afterthought that the knowledge may originate in someone other than the builder, who then receives the knowledge at second hand. (The afterthought doesn't change the main point of the analogy, though it may initially confuse readers.)

\lemc{10 accepit} the understood subject of this verb is the \textit{artifex} of the previous clause. Descartes now imagines the builder receiving knowledge of how to build from some other party.
% -]] 3rd meditation

% [[- 4th meditation
\clearpage

\beginnumbering
\pstart
\textbf{4.} \begin{latin}In quarta, probatur ea omnia quae clare et distincte percipimus, esse vera, simulque in quo ratio falsitatis consistat explicatur: quae necessario sciri debent tam ad praecedentia firmanda, quam ad reliqua intelligenda. (Sed ibi interim est advertendum nullo modo agi de peccato, vel errore qui committitur in persecutione boni et mali, sed de eo tantum qui contingit in diiudicatione veri et falsi. Nec ea spectari quae ad fidem pertinent, vel ad vitam agendam, sed tantum speculativas et solius luminis naturalis ope cognitas veritates.)\end{latin}
\pend
\endnumbering

\prenotes

\textbf{§4.} The fourth meditation addresses questions about truth and falsity. The meditation is transitional: it solidifies earlier discussions and paves the way for future topics. Descartes warns us here that the fourth meditation only discusses truth and falsity from an intellectual or epistemological point of view; there is no discussion of truth or falsity in the context of sin or morality. 

\lem{1 probatur} is impersonal, and its subject is the clause \textit{ea omnia esse vera}.

\lem{2 ratio} with the genitive \textit{falsitatis} means something like ``nature'' or ``manner.''

\lem{3 ibi} refers forward to the meditation itself and is best translated with with \textit{agi}.

\lem{3 interim} has adversative or concessive force here: ``all the while'' or ``at the same time'' (\textbf{OLD} 3).

\lem{3--4 est advertendum} is impersonal with \textit{agi} as what ``must be noted.''

\lemc{4 agi de peccato} the passive infinitive is impersonal. More idiomatically in English: ``there is no treatment of sin'' or ``there is no discussion of sin.''

\lem{4 persecutione} means ``pursuit'' here not ``persecution.''

\lem{5 sed de eo} is short for \textit{sed agi de eo errore}.

\lemc{5--6 Nec ea spectari} the accusative and infinitive construction depends on \textit{advertendum est}, which must be resupplied from the previous sentence. Translate as if this: \textit{Et advertendum est ea non spectari}. Remember that \textit{nec} amounts \textit{et nōn}, but you will often need to separate the two elements and apply the negative force later in the clause.

\lem{6--7 speculativas\dots veritates} is the object of an understood \textit{ad}.
% -]] 4th meditation

% [[- 5th meditation
\clearpage

\beginnumbering
\pstart
\textbf{5.} \begin{latin}In quinta, praeterquam quod natura corporea in genere sumpta explicatur, nova etiam ratione Dei existentia demonstratur: sed in qua rursus nonnullae forte occurrent difficultates, quae postea in responsione ad obiectiones resolventur: ac denique ostenditur quo pacto verum sit, ipsarum Geometricarum demonstrationum certitudinem a cognitione Dei pendere.\end{latin}
\pend
\endnumbering

\prenotes

\textbf{§5.} The fifth meditation is a bit of a grab bag. It contains the following: (i) an explanation of the nature of corporeal objects, (ii) a new argument for the existence of god, and (iii) an explanation for how even geometrical proofs depend on knowledge of god. As in the synopsis of the third meditation, Descartes acknowledges that the new argument for god's existence may involve some difficulties that will only be resolved in the response to objections.

\lem{1 praeterquam quod} = ``except for the fact that,'' ``except that,'' ``other than that,'' or ``besides that.'' The \textit{quod} serves, as often, to nominalize the clause that follows (\textbf{AG} §572; \textbf{NLS} §241). 

\lem{1 natura corporea in genere sumpta} = ``corporeal nature taken in general.'' Descartes cares only about the qualities that physical objects have \textit{as such}. He isn't interested here in the specific qualities of this or that type of physical thing.

\lemc{2 in qua} supply \textit{ratione}. We would say ``during this proof.''

\lem{4--5} The subject of \textit{ostenditur} is the indirect question \textit{quo pacto verum sit}; the subject of the indirect question is the accusative plus infinitive phrase \textit{certitudinem\dots pendere}.

\lemc{4--5 ipsarum Geometricarum demonstrationum certitudinem} we might assume that certainty in a geometrical proof is self-contained, but, according to Descartes, unless we possess knowledge of god, \textit{all our other beliefs} lack certainty.

% -]] 5th meditation

% [[- 6th meditation
\clearpage

\beginnumbering
\pstart
\textbf{6.} \begin{latin}In sexta denique, intellectio ab imaginatione secernitur; distinctionum signa describuntur; mentem realiter a corpore distingui probatur; eandem nihilominus tam arcte illi esse coniunctam, ut unum quid cum ipsa componat, ostenditur; omnes errores qui a sensibus oriri solent recensentur; modi quibus vitari possint exponuntur; et denique rationes omnes ex quibus rerum materialium existentia possit concludi, afferuntur: non quod eas valde utiles esse putarim ad probandum id ipsum quod \at{16} probant, nempe revera esse aliquem mundum, et homines habere corpora, et similia, de quibus nemo unquam sanae mentis serio dubitavit; sed quia, illas considerando, agnoscitur non esse tam firmas nec tam perspicuas quam sunt eae, per quas in mentis nostrae et Dei cognitionem devenimus; adeo ut hae sint omnium certissimae et evidentissimae quae ab humano ingenio sciri possint. Cuius unius rei probationem in his Meditationibus mihi pro scopo proposui. Nec idcirco hic recenseo varias illas quaestiones de quibus etiam in ipsis ex occasione tractatur.\end{latin}
\pend
\endnumbering

\prenotes

\textbf{§6.} The sixth meditation closes out the work by returning to questions from the first and second meditations: the difference between mind and body, the mistakes that come from sense perception, and the doubts that these mistakes create. The meditation argues that we can establish the reality of the world around us, despite the doubts of the early meditations, but that our knowledge of the mind and god are nevertheless far more certain.

\lemc{1--2 signa} translate here as ``criteria.'' This is a common philosophical meaning of \textit{signum}.

\lem{2 probatur} has \textit{mentem\dots distingui} as its subject.

\lemc{2--3} although Descartes insists on the distinction of mind from body, he also insists that they form a ``single something'' (\textit{unum quid}) because of their close connection. Notice that the mind remains the subject of \textit{esse coniunctam} and \textit{componat}. Descartes does not believe that mind and body are \textit{equal} partners, even if they work very closely together in a living person.

\lem{3 arcte} is an alternative form of \textit{arte}.

\lem{3 ostenditur} has \textit{eandem (mentem)\dots esse coniunctam} as its subject.

\lem{9 illas} supply \textit{rationes}. The demonstrative \textit{illas} is the direct object of the gerund \textit{considerando}, but we also need to use it as the subject of \textit{non esse tam firmas nec tam perspicuas}. (That infinitive phrase is, in turn, the subject of \textit{agnoscitur}.)

\lem{10 in\dots devenimus} we would say that we ``arrive at'' or ``reach'' the knowledge of something.

\lem{12--14} the most important thing Descartes set out to do in these meditations is to prove that we know our own minds and god better than we know anything else; as a result, Descartes does not address every minor topic in this synopsis; he will discuss such topics in the meditations themselves, as appropriate.

\lemc{13 ipsis} supply \textit{Meditationes}.

\lem{tractatur} is impersonal. We might say ``concerning which things (\textit{de quibus})\dots there is discussion.''

% -]] 6th meditation

% -]] Synopsis

    % [[- Chapter title
\chapter{Meditatio Prima}
% -]] Chapter title

% [[- Meditatio prima

% [[- Introduction
Although his goal is knowledge, Descartes dedicates the first meditation to doubt. Starting from everyday mistakes, such as misjudging the size of a distant object, he rapidly builds towards radical, hyperbolic doubt as he imagines an all-powerful demon whose only goal is to confuse and mislead him. At the end of the first meditation, Descartes is so shaken that he fears he may be trapped in the unescapable darkness of error.

However, as Descartes himself says elsewhere, he is not like the sceptics who ``doubt only to doubt and pretend always to be undecided'' (AT VI 29). Descartes believes that he can use doubt and error strategically in order to achieve certainty and truth. Although the first meditation contains some of the most vivid writing and imagery in the work, Descartes would have been very sorry for readers to remember it best. His goal is to overcome doubt rather than to dwell on it.
% -]] Introduction

% [[- Background
\clearpage
\begin{center}
    \beginnumbering
    \numberlinefalse
    \pstart
    \textit{De iis quae in dubium revocari possunt}\ledsidenote{17}
    \pend
    \endnumbering
\end{center}

\beginnumbering
\pstart
\begin{latin}
    \textenglish{\textbf{1.}} Animadverti iam ante aliquot annos quam multa, ineunte aetate, falsa pro veris admiserim et quam dubia sint quaecunque istis postea superextruxi, ac proinde funditus omnia semel in vita esse evertenda atque a primis fundamentis denuo inchoandum, si quid aliquando firmum et mansurum cupiam in scientiis stabilire; sed ingens opus esse videbatur, eamque aetatem expectabam quae foret tam matura ut capessendis disciplinis aptior nulla sequeretur. Quare tamdiu cunctatus sum ut deinceps essem in culpa, si quod temporis superest ad agendum, deliberando consumerem. Opportune igitur hodie mentem curis\ledsidenote{18} omnibus exsolvi, securum mihi otium procuravi, solus secedo, serio tandem et libere generali huic mearum opinionum eversioni vacabo.
\end{latin}
\pend
\endnumbering

\prenotes

Descartes starts with what can be doubted, but not because he wants to shake our beliefs. He hopes to strengthen our understanding by clearing away anything that is not solid and reliable. After we remove everything that \textit{can} be doubted, what remains \textit{cannot} be doubted. In this way, Descartes intends to use doubt in order to reach certainty.

\lem{Animadverti}, the main verb of the sentence, takes four objects: two indirect questions and two indirect statements. The two indirect questions are (i) \textit{quam multa\dots admiserim} and (ii) \textit{quam dubia sint}. The two indirect statements are (i) \textit{omnia\dots esse evertenda} and (ii) \textit{inchoandum <esse>}.

\lem{ante} usually means `before', but it can also mean `ago'. In this use, it is followed by either an ablative or, as here, an accusative indicating how long ago.

\lemc{quam multa\dots quam dubia} as an interrogative adverb, \textit{quam} means `how'.

\lem{ineunte aetate} is an ablative absolute that modifies \textit{admiserim}. Remember that an ablative absolute often stands in place of a clause with temporal, causal, concessive, conditional, or coordinate force (\textbf{AG} §420). In this case, a temporal clause makes the best sense.

\lemc{admiserim\dots sint} although the main verb \textit{animadverti} is secondary sequence (\textit{ante aliquot annos} guarantees this), these two verbs are primary sequence. This may be for vividness, but throughout this sentence Descartes shifts his perspective from his past realization and resolution to his present plans. It reads very naturally, but the grammar is a bit hard to pin down.

\lem{admiserim} assumes a particular view of belief. A person is thought of as reviewing impressions and then either granting them entrance (i.e., believing them) or turning them away (i.e., not believing them). This suggests, though it does not necessarily require, that belief is voluntary, or at least subject to the will in some way. All of this is controversial, and Descartes will discuss it later in the work.

\lemc{quam dubia sint quaecunque\dots } the relative clause serves as the subject of \textit{sint}. That is, Descartes realized how doubtful \textit{whatever he built on top of those foundations} is.

\lemc{quaecunque istis postea superextruxi} throughout the \textit{Meditations} Descartes uses metaphors drawn from construction and demolition. He represents knowledge and science as buildings that can be firm and lasting or have weak foundations; they can be torn down and built back up again. (Descartes makes heavy use of this metaphor, and the related metaphor of the house, in his \textit{Discours de la méthode} as well. See \citet[22]{curtis1984}.)

\lem{quaecunque} = \textit{quaecumque}.

\lem{istis} is probably dative with the compound \textit{superextruxi}, and the pronoun refers back to the \textit{falsa} that Descartes believed to be true when he was younger. Descartes continues to use the implicit metaphor of construction.

\lemc{superextruxi} this verb does not appear in classical Latin, but \textit{ex(s)truo} does, and compound verbs with \textit{super}- are common in Latin.

\lemc{esse evertenda\dots inchoandum} infinitives of the passive periphrastic. This periphrastic has the force of obligation, necessity, or propriety (\textbf{AG} §194b).

\lemc{proinde} Descartes doesn't attempt to justify the inference from ``I'm aware I've made mistakes'' to ``I should overturn all my beliefs''. He moves quickly at this point, assuming that the reader is willing to go along to see where the argument leads. 

\lem{foret} = \textit{esset}, imperfect subjunctive in a relative clause of characteristic. Note \textit{\textbf{eam} aetatem}: `\textit{that (kind of)} age that would\dots'.

\lemc{cappesendis disciplinis} the dative depends on \textit{aptior}, and the gerundive phrase expresses purpose.

\lemc{quod temporis} the genitive is partitive, and the phrase serves as the object of \textit{consumerem}: ``I would be at fault if I wasted what time remains\dots''. There's no antecedent for \textit{quod}. This is common for indefinite antecedents (\textbf{AG} §307c). Fully spelled out the thought is \textit{si illud temporis quod ad agendum superest deliberando consumerem}.

\lem{curis omnibus} ablative of separation with \textit{exsolvo}.

\lem{generali huic\dots eversioni} dative with \textit{vacabo}.

\lem{mearum opionum} objective genitive with \textit{eversioni}.

% -]] Background

% [[- A shortcut
\clearpage

\beginnumbering
\pstart
\begin{latin}
    \textenglish{\textbf{2.}} Ad hoc autem non erit necesse ut omnes esse falsas ostendam---quod nunquam fortassis assequi possem; sed quia iam ratio persuadet non minus accurate ab iis quae non plane certa sunt atque indubitata quam ab aperte falsis assensionem esse cohibendam, satis erit ad omnes reiiciendas si aliquam rationem dubitandi in unaquaque reperero. Nec ideo etiam singulae erunt percurrendae---quod operis esset infiniti; sed quia, suffossis fundamentis, quidquid iis superaedificatum est sponte collabitur, aggrediar statim ipsa principia quibus illud omne quod olim credidi nitebatur.
\end{latin}
\pend
\endnumbering

\prenotes

Descartes wishes to test all of his beliefs, but he would rather not have to examine each of them. In order to speed things us, he argues that (i) assent should be withheld from anything uncertain just as much as from clear falsehoods and (ii) if he can undercut the beliefs at the foundation of his thinking, then any beliefs based on those foundations will collapse as well. In these two ways Descartes hopes to make his task easier.

\lem{hoc} refers to the overturning (\textit{eversio} §1.10) of Descartes' beliefs from the previous paragraph. Descartes uses the neuter rather than the feminine perhaps because he has in mind the general idea rather than the specific word.

\lemc{ut\dots ostendam} a substantive result clause, the subject of \textit{erit necesse}.

\lemc{quod} neuter like \textit{hoc} because it refers back to the entire clause \textit{ut omnes esse falsas ostendam}.

\lemc{possem} effectively the apodosis of a present contrary-to-fact conditional sentence, though the protasis is only implied. ``<Even if I were to try,> \textit{I would never be able\dots}''.

\lemc{assensionem esse cohibendam} The technical vocabulary of ``withholding assent'' derives from ancient debates between Sceptics and Stoics. Under normal circumstances, people don't necessarily believe that they should treat anything not overwhelmingly certain and indubitable as if it were clearly false. However, in Descartes' special situation---a single-minded search for truth and absolute certainty---this rule may be justified.

\lemc{iis} this form is a common alternative for \textit{eis}, the dative and ablative plural of \textit{is, ea, id} for all genders. Similarly \textit{ii} is an alternative for \textit{ei}, the masculine nominative plural. The dative singular \textit{ei} (all genders) has no such alternative. (These alternatives also appear often in compounds of \textit{is, ea, id}. E.g., \textit{iisdem}.

\lemc{quod} like the \textit{quod} in §2.11, the neuter refers back to the action of the previous clause \textit{singulae erunt percurrendae}.

\lem{operis\dots infiniti} is a predicative genitive (\textbf{AG} §343b): ``which thing (i.e., to run through each opinion one by one) would be an infinite task''.

\lemc{suffossis fundamentis} given the context, this ablative absolute is best taken as equivalent to a conditional clause.

\lem{iis} is dative with \textit{superaedificatum est}.

\lemc{quibus} ablative with \textit{nitebatur} (\textbf{AG} §431).

% -]] A shortcut

% [[- Attack the senses first
\clearpage

\beginnumbering
\pstart
\begin{latin}
    \textenglish{\textbf{3.}} Nempe quidquid hactenus ut maxime verum admisi, vel a sensibus vel per sensus accepi; hos autem interdum fallere deprehendi, ac prudentiae est nunquam illis plane confidere qui nos vel semel deceperunt.
\end{latin}
\pend
\endnumbering

\beginnumbering
\pstart
\begin{latin}
    \textenglish{\textbf{4.}} Sed forte, quamvis interdum sensus circa minuta quaedam et remotiora nos fallant, pleraque tamen alia sunt de quibus dubitari plane non potest, quamvis ab iisdem hauriantur: ut iam me hic esse, fovo assidere, hyemali toga esse indutum, chartam istam manibus contrectare, et similia. Manus vero has ipsas, totumque hoc corpus meum esse, qua ratione posset negari? nisi me forte comparem nescio quibus insanis,\ledsidenote{19} quorum cerebella tam contumax vapor ex atra bile labefactat ut constanter asseverent vel se esse reges, cum sunt pauperrimi, vel purpura indutos, cum sunt nudi, \edtext{vel caput habere fictile}{\Afootnote{The French translation omits this phrase.}}, vel se totos esse cucurbitas, vel ex vitro conflatos; sed amentes sunt isti, nec minus ipse demens viderer, si quod ab iis exemplum ad me transferrem.
\end{latin}
\pend
\endnumbering

\prenotes

Descartes proposes a shortcut to the kind of radical scepticism he wants to reach: undermine the senses, which are at the base of all our beliefs, and everything based on the senses will be undermined at the same time. His argument against the senses is brisk: they sometimes deceive, and therefore they can never be trusted. However, Descartes then raises an objection that even if \textit{some} sense-based beliefs can deceive us, others are so basic that they can never be doubted. Indeed to doubt certain very basic beliefs that derive from the senses would be tantamount to giving up reason altogether---or so claims the objection.

\lem{quidquid} functions as the direct object of both \textit{admisi} and \textit{accepi}: ``whatever I admitted\dots I received''.

\lemc{vel a sensibus vel per sensus} In another work (\textbf{CB} §1), Descartes explains this as a distinction between information received directly from a sense (e.g., seeing a color or shape) and information received indirectly (e.g., learning about things by hearing other people speak).

\lemc{hos\dots fallere deprehendi} we can supply \textit{me} or \textit{nos} as the direct object of \textit{fallere}, or we can take it absolutely to mean simply `deceive, be deceptive'.

\lemc{prudentiae} this genitive, which is sometimes called the genitive of characteristic or the genitive of the mark, is a type of possessive genitive (\textbf{AG} §343c). It belongs to wisdom to distrust anything that has been deceptive even once. That is, it's wise to be suspicious in such a case.

\lem{vel} = `even', modifying \textit{semel}.

\lem{ut} = `for example, e.g.'. This use of \textit{ut} is a development of its adverbial meaning `as, in such a manner'. In a sentence like this, the adverb introduces specific cases which exemplify some general point. The indirect statements following \textit{ut} are all examples of the kinds of beliefs that are not easily doubted, although they come from the senses.

\lemc{Manus\dots posset negari} the two infinitive plus accusative phrases (\textit{Manus has ipsas <meas esse>} and \textit{totum hoc corpus meum esse}) are both subjects of the main verb \textit{posset}. I.e., how could it be denied that these are my hands and that this is my body?

\lemc{Manus\dots has ipsas, totum\dots hoc corpus} it was already difficult to imagine being wrong about the previous examples, but these last two are meant to be show-stoppers. The repeated \textit{has\dots hoc} implies, as often in Latin, a gesture on the part of the speaker: \textit{these} hands\dots this body---the ones right in front of you. In a similar fashion, G.E. Moore \citep[165--166]{baldwin1993} famously argued as follows:
\begin{quote}
    I can prove now, for instance, that two human hands exist. How? By holding up my two hands, and saying, as I make a certain gesture with the right hand, `Here is one hand', and adding, as I make a certain gesture with the left, `and here is another'.
\end{quote}
Do you think an argument like this can settle the philosophical problem of scepticism?

\lem{qua ratione} = literally `by what reasoning', but idiomatically it means `how'.

\lemc{nescio quibus} the verb \textit{nescio} can be joined with various words to create indefinite adjectives, pronouns, and adverbs. The verbal part is treated like an indeclinable prefix. Editors disagree over whether to write these forms as one word or two, but there's no difference in meaning either way. We can do a similar thing in English. Imagine someone telling a story, saying ``And then I-don't-know-who comes along and\dots''. Here \textit{nescio quibus} is an indefinite adjective, modifying \textit{insanis}: `some crazy people'.

\lemc{quorum cerebella\dots labefactat} Black bile is one of the four humors; the other three are blood, yellow bile, and phlegm. The balance or imbalance of these four was thought to control a person's physical and mental well-being.

\lemc{asseverent vel\dots vel\dots} these examples are extreme in two ways. First, the beliefs in question are not a little wrong, but wildly wrong. Second, they all involve mistakes about very basic facts, the kind of things that people simply don't make mistakes about, under ordinary circumstances. Hence, this objection poses a serious threat to Descartes' opening argument against the senses. He now needs to answer the challenge that his position is tantamount to insanity.
% -]] Attack the senses first


% -]] Meditatio prima

    % [[- Chapter title
\chapter{Meditatio Secunda}
% -]] Chapter title

% [[- Meditatio secunda

% [[- Introduction
In the second meditation, Descartes finds a certain truth: I exist as long as I am thinking. (Descartes doesn't use the phrase \textit{cogito ergo sum} in the meditations, though he does elsewhere.) As he unravels the significance of this truth, Descartes examines the nature of thought and what it means to be a creature that thinks. He argues that the essence of thought is perhaps different from what we expect and that our human nature is also different from what everyday beliefs suggest.

Descartes also answers the objection that physical things are better and more easily known than ourselves. To many people it may seem that our knowledge of everyday items---such as tables, chairs, rocks, trees, and so forth---is more immediate, clearer, and more important than knowledge of the self. Descartes offers a two-part reply to this. First, he argues that it is more difficult to know ordinary physical objects than we think. Second, he argues that the knowledge of such objects that we have comes from thought and not sensation. Thus, he concludes our nature as thinking things is more essential to us and easier for us to grasp than common sense suggests.

\clearpage
% -]] Introduction

% [[- An Archimedean point
\clearpage
\begin{center}
    \beginnumbering
    \numberlinefalse
    \pstart
    \textit{De natura mentis humanae: quod ipsa sit notior quam corpus}
    \pend
    \endnumbering
\end{center}

\beginnumbering
\pstart
\begin{latin}
    \textenglish{\textbf{1.}} In tantas dubitationes hesterna meditatione coniectus sum ut nequeam amplius earum oblivisci nec videam tamen qua ratione solvendae sint. Sed, tanquam in profundum gurgitem ex improviso delapsus, ita turbatus sum ut nec possim in imo pedem figere nec enatare ad summum. Enitar tamen et tentabo rursus eandem viam quam heri fueram ingressus removendo scilicet illud omne quod vel minimum dubitationis admittit nihilo secius quam si omnino falsum esse comperissem; pergamque porro donec aliquid certi vel, si nihil aliud, saltem hoc ipsum pro certo---nihil esse certi---cognoscam. Nihil nisi punctum petebat Archimedes quod esset firmum et immobile ut integram terram loco dimoveret. Magna quoque speranda sunt si vel minimum quid invenero quod certum sit et inconcussum.
\end{latin}
\pend
\endnumbering

\beginnumbering
\pstart
\begin{latin}
    \textenglish{\textbf{2.}} Suppono igitur omnia quae video falsa esse. Credo nihil unquam extitisse eorum quae mendax memoria repraesentat. Nullos plane habeo sensus. Corpus, figura, extensio, motus, locusque sunt chimerae. Quid igitur erit verum? Fortassis hoc unum: nihil esse certi.
\end{latin}
\pend
\endnumbering

\prenotes

\textbf{§1.} Yesterday was hard. We wanted to doubt as much as possible, but we may have succeeded far too well. We may end up proving that \textit{everything} can be doubted, in which case the only certain thing will be that nothing is certain. Still, we should keep looking: if we can find even one certain truth, we can use it as a foundation for so much more knowledge.

\lemc{1 hesterna} although Descartes expressed hope that readers would spend as long as needed---weeks or months, if necessary---on the first meditation, the meditations as a whole follow the dramatic convention that each meditation is one day for the narrator.

\lemc{2 qua ratione} not simply an alternative for \textit{quo modo}. The trouble that Descartes has caused himself must be solved by reason, not in any other way.

\lemc{2--3 Sed\dots summum} We might try to connect each part of this analogy to a different outcome. If Descartes discovers that there is nothing certain, he will have reached the bottom of the whirlpool of doubt. If, on the other hand, he discovers a certainty, he escapes doubt and swims to the surface. However, even if we are not so precise, the analogy conveys how utterly overpowered Descartes feels during his exericse in radical doubt.

\lemc{3 turbatus sum} a form like this is ambiguous. The default interpretation is that it is a perfect passive: `I was troubled'. In this case, we take the participle and the form of \textit{esse} together as one compound verb form. But in some cases the words should be taken separately: a form of \textit{esse} and a participle serving as an attributive adjective. On this interpretation, the two words here are in the present tense: `I am troubled'. Many perfect passive participles show this ambiguity between their use in compound passive forms and their use as adjectives. The second reading makes better sense in this passage: take \textit{turbatus} as an adjective rather than as half of a compound verb.

\lemc{3--4 nec\dots nec} despite the position of the first \textit{nec}, these two join the infinitives \textit{figere} and \textit{enatare} rather than two main verbs.

\lemc{5 fueram ingressus} = \textit{eram ingressus}, the pluperfect tense. The pluperfect passive in classical Latin is normally the imperfect of \textit{esse} and a perfect passive participle, but there are scattered examples like this: a pluperfect of \textit{esse} and the perfect passive participle. The tendency towards these shifted pluperfects increases in later Latin. See H-S §179 and, more generally, \citet[§298]{väänänen1981}.

\lemc{5 removendo} ablative of means of a gerund. Descartes will try again `by setting aside' anything which may be uncertain.

\lemc{6 nihilo secius} probably a litotes. That is `not at all other than if' means `exactly as if'.

\lemc{7 aliquid certi} see the note on I §8.6.

\lemc{7 saltem hoc ipsum} the demonstrative \textit{hic, haec, hoc} is often used to point forward. In this case, it anticipates \textit{nihil esse certi}. Descartes will keep striving for certainty, even if the only certain thing he discovers is that there is no certainty.

\lemc{8 punctum petebat Archimedes} Archimedes was an important scientist and mathematician (c. 287-212 BCE, born in Syracuse). Descartes alludes to his work with levers: Archimedes famously said, `Give me a place to stand, and I will move the Earth'. In the same way that a lever magnifies a person's strength, Descartes believes that the discovery of a single certainty will indirectly increase his knowledge far more than the one certainty itself. If Descartes is right, a single certainty can provide the foundation for a vast edifice of scientific knowledge.

\textbf{§2.} To recap our program of doubt: we will doubt all the evidence of our senses, we will not believe our memories, we will even reject the most general categories of thought. Perhaps the only truth left will be that there is no certainty.

\lem{1 unquam} = \textit{umquam}

\lemc{3 chimerae} In Homer's \textit{Iliad} the chimera was lion in the front, snake in the back and goat in the middle. It also breathed fire. To say that something is `a chimera' is to say that it is mere fantasy.

\lemc{4 nihil esse certi} implied indirect statement depending on the understood \textit{erit verum}.

% -]] An Archimedean point

% [[- Ego sum, ego existo
\clearpage

\beginnumbering
\pstart
\begin{latin}
    \textenglish{\textbf{3.}} Sed unde scio nihil esse diversum ab iis omnibus quae iam iam recensui de quo ne minima quidem occasio sit dubitandi? Nunquid est aliquis Deus, vel quocunque nomine illum vocem, qui mihi has ipsas cogitationes immittit? Quare vero hoc putem cum forsan ipsemet illarum author esse possim? Nunquid ergo saltem ego aliquid sum? Sed iam negavi me habere ullos sensus et ullum corpus. Haereo tamen; nam quid \at{25} inde? Sumne ita corpori sensibusque alligatus ut sine illis esse non possim? Sed mihi persuasi nihil plane esse in mundo: nullum coelum, nullam terram, nullas mentes, nulla corpora; nonne igitur etiam me non esse? Imo certe ego eram si quid mihi persuasi. Sed est deceptor nescio quis, summe potens, summe callidus, qui de industria me semper fallit. Haud dubie igitur ego etiam sum si me fallit. Et fallat quantum potest: nunquam tamen efficiet ut nihil sim quamdiu me aliquid esse cogitabo. Adeo ut, omnibus satis superque pensitatis, denique statuendum sit hoc pronuntiatum `Ego sum, ego existo' quoties a me profertur vel mente concipitur necessario esse verum.
\end{latin}
\pend
\endnumbering

\prenotes

\textbf{§3.} How can we even be sure that there is nothing certain? Perhaps we simply haven't found it yet. Whatever source we imagine for our thoughts---god, a demon, or ourselves---we must at least be something insofar as we're thinking. It doesn't matter that everything we normally think about ourselves and the world is false. Even if that's so, we \textit{are} thinking it. Go back to the evil demon idea: let him trick us all he likes, we must therefore exist for him to trick. As long as we are thinking, we aren't nothing. We exist.

\lem{2 Nunquid} = \textit{Numquid}, a word with no one exact translation into English. In classical Latin it introduces questions that (i) involve emotion, (ii) express incredulity, or (iii) are rhetorical. (These categories can overlap.) Usually questions with \textit{numquid} expect a negative answer. In the meditations, however, Descartes uses \textit{nunquid} in contexts where it is clear he expects an affirmative answer. In this case, Descartes is suggesting the existence of god as perhaps something that provides no possibility for doubt.

\lemc{4 putem} a deliberative subjunctive `Why should I think this?' The existence of god is not certain since Descartes may be responsible for his own thoughts.

\lemc{4 Nunquid ergo\dots ego aliquid sum} if there is no god and Descartes himself is responsible for his thoughts, then surely he must exist.

\lemc{6 nam quid inde?} the verb implied is something like `follows' or `happens'. Descartes is confused: he's admitted that he may have no body at all, and he's not sure what follows from that. In particular, he's not sure that it means that he might not exist, which is the question before him.

\lemc{8 me non esse} resupply \textit{mihi persuasi} from earlier in the sentence to govern this indirect statement.

\lemc{8 Imo certe} this is the turning point from doubt towards certainty. These two words emphatically reject the last question, and the rest of the sentence justifies Descartes conviction that he must exist.

\lemc{9 si quid mihi persuasi} in English we would say `if I persuaded myself of anything' or `about anything', but in Latin the `something' that a person comes to believe is the accusative direct object of \textit{persuadeo}. Normally, however, we don't notice this because the object of the verb is usually a clause: an indirect statement or an indirect command.

\lemc{11 fallat} an independent subjunctive expressing a command.

\lemc{11--12 quamdiu\dots cogitabo} on first reading this clause may not stand out, but it will turn out to matter a great deal. Descartes will suggest that if he stops thinking entirely, he ceases to exist. So he means it when he says that he will always be something `as long as' he thinks.

\lemc{12 Adeo ut} these words introduce what serves as the main clause of the sentence. They are functionally equivalent here to \textit{igitur} or \textit{quare}.

\lemc{12 satis superque} Descartes has thought about these matters `enough and more than enough'. I.e., very extensively.

\lemc{12--13 statuendum sit} the subject of this verb is the indirect statement \textit{hoc pronuntiatum\dots necessario esse verum}.

\lemc{13 hoc} anticipates \textit{`Ego sum, ego existo'}. Those words are `this statement'.

\lemc{necessario} the necessary truth is `that I exist whenever I say so or think I do' not simply `that I exist'. Descartes is not committed to the view that he must exist.
% -]] Ego sum, ego existo

% [[- What is the ego that exists?
\clearpage

\beginnumbering
\pstart
\begin{latin}
    \textenglish{\textbf{4.}} Nondum vero satis intelligo quisnam sim ego ille qui iam necessario sum; deincepsque cavendum est ne forte quid aliud imprudenter assumam in locum mei sicque aberrem etiam in ea cognitione quam omnium certissimam evidentissimamque esse contendo. Quare iam denuo meditabor quidnam me olim esse crediderim priusquam in has cogitationes incidissem; ex quo deinde subducam quidquid allatis rationibus vel minimum potuit infirmari ut ita tandem praecise remaneat illud tantum quod certum est et inconcussum.
\end{latin}
\pend
\endnumbering

\beginnumbering
\pstart
\begin{latin}
    \textenglish{\textbf{5.}} Quidnam igitur antehac me esse putavi? Hominem scilicet. Sed quid est homo? Dicamne animal rationale? Non, quia postea quaerendum foret quidnam animal sit et quid rationale, atque ita ex una quaestione in plures difficilioresque delaberer; nec iam mihi tantum otii est ut illo velim inter istiusmodi subtilitates abuti. Sed hic potius attendam quid sponte \at{26} et natura duce cogitationi meae antehac occurrebat quoties quid essem considerabam.
\end{latin}
\pend
\endnumbering

\prenotes

\textbf{§4.} We have our one certainty: we exist. But how much substance does this certainty have? What do we know about the `we' that exists? What are we? If we are not careful, we will quickly go wrong just as we've found a tiny bit of certainty. To prevent that, let's use the same method we just employed: we will consider what we think about ourselves, and we'll reject whatever has any doubt. What's left will be certain and unshakeable.

\lem{1 intelligo} = \textit{intellego}.

\lem{2 cavendum est} introduces a compound fear clause containing two verbs joined by \textit{que}: \textit{assumam} (2) and \textit{aberrem} (3).

\lemc{3 omnium certissimam evidentissimam} the partitive genitive is common with superlatives.

\lemc{crediderim} perfect subjunctive in an indirect question following \textit{meditabor}.

\lemc{5--6 allatis rationibus} ablative of means with \textit{infirmari} (6).

\lemc{6 vel} adverbial rather than a conjunction. It means `even'.

\lemc{6 minimum} an adverbial use of the accusative, meaning `in the smallest degree' or `the least bit'.

\lemc{6 potuit} the tense is somewhat odd, and we would expect a future perfect in Latin. The point is that Descartes will remove whatever will prove to be doubtful once he has brought forward arguments. The removal will happen in the future, and so \textit{subducam} is future. At that time, Descartes will \textit{already} have proven that doubt is possible. Hence, he uses a perfect tense for \textit{potuit}, even though this is not strictly speaking the correct tense. It hasn't happened yet, but it will have happened at an imagined moment in the future.

\lemc{6 tandem praecise} despite the imprecise word order, you should take \textit{tandem} with \textit{remaneat} and \textit{praecise} with \textit{illud}.

\lemc{7 tantum} not `so big' or `so great' but `only' or `just'. This is a common idiomatic use of \textit{tantum}, and you must take care to distinguish it from the other meanings of the word.

\textbf{§5.} What have we believed up until now about ourselves? We believe that we are people, that people have bodies and minds, and that bodies and minds are responsible for distinct features of what it is to be a person.

\lemc{1 Hominem scilicet} supply \textit{me esse putavi} to make this fragment a complete thought.

\lemc{2 Dicam} a deliberative subjunctive.

\lemc{2 animal rationale} this is a traditional definition of `human', derived ultimately from Aristotle. Descartes rejects it since (i) it would drag him into arguments about definitions and (ii) he doesn't think that it possesses the kind of certainty he hopes to find. He only states (i) explicitly.

\lemc{2 Non} that is, `I should not say this'.

\lemc{2 foret} an alternative form of \textit{esset}, the imperfect subjunctive of \textit{esse}. The imperfect subjunctive is used in an implied present contrary-to-fact conditional sentence: `I should not say this because (if I were to say it), I would have to pursue tedious questions about definitions'.

\lemc{3 plures difficiliores} Latin frequently uses `and' with the adjective \textit{multus, multa, multum} in a way that is unidiomatic in English. E.g., `many and difficult problems' rather than `many difficult problems'. This is the same thing, but \textit{plures} is comparative.

\lemc{4 mihi} dative of possession with \textit{tantum otii est}.

\lemc{4 illo} ablative object of the verb \textit{abuti} (5).

\lemc{5 natura duce} an ablative absolute. Classical Latin has no present participle for \textit{esse}, and so the verb `being' is implied in an ablative absolute consisting of two nouns, like this one. (Less often you'll see one noun and one adjective. E.g. \textit{Aenea pio}: `with Aeneas being dutiful' or `since Aeneas is dutiful'.) `With nature being leader' is equivalent to something like `under the guidance of nature' or, even less literally, `naturally'.

\lemc{5 cogitationi meae} dative with the compound verb \textit{occurrebat} (6).
% -]] What is the ego that exists?

% [[- What is the ego (cont.)?
\clearpage

\beginnumbering
\pstart
\setline{7}
\begin{latin}
    \textenglish{\textbf{5. (cont.)}} Nempe occurrebat primo me habere vultum, manus, brachia, totamque hanc membrorum machinam qualis etiam in cadavere cernitur et quam corporis nomine designabam. Occurrebat praeterea me nutriri, incedere, sentire, et cogitare: quas quidem actiones ad animam referebam. Sed quid esset haec anima, vel non advertebam, vel exiguum nescio quid imaginabar, instar venti, vel ignis, vel aetheris, quod crassioribus mei partibus esset infusum. De corpore vero ne dubitabam quidem, sed distincte me nosse arbitrabar eius naturam, quam si forte, qualem mente concipiebam, describere tentassem, sic explicuissem: per `corpus' intelligo illud omne quod aptum est figura aliqua terminari, loco circumscribi, spatium sic replere ut ex eo aliud omne corpus excludat; tactu, visu, auditu, gustu, vel odoratu percipi, necnon moveri pluribus modis---non quidem a seipso, sed ab alio quopiam a quo tangatur. Namque habere vim seipsum movendi, item sentiendi, vel cogitandi, nullo pacto ad naturam corporis pertinere iudicabam; quinimo mirabar potius tales facultates in quibusdam corporibus reperiri.
\end{latin}
\pend
\endnumbering

\prenotes

\lemc{8 membrorum machinam} a striking phrase that implicitly puts the reader in the right frame of mind to think of a human body as something separate from the person whose body it is. (The genitive is likely of material (\textbf{AG} §344).)

\lemc{8 qualis etiam in cadavere cernitur} this phrase also suggests the gap between a person and that person's body.

\lemc{9 nutriri} we might think of feeding as something more bodily than mental, but there is a tradition stretching back at least as far as Aristotle that argues otherwise. The idea is this: anything which is alive possesses what Aristotle would call a ψυχή and Descartes calls an \textit{anima}: a soul. Even plants, as living creatures, possess souls. Their souls, however, are responsible only for the most basic functions of life: nutrition, growth, and reproduction. (This verb appears in classical Latin both in deponent and non-deponent forms.)

\lemc{10 Sed quid esset\dots} Descartes rarely thought about the nature of the soul before. When he did, he went no further than commonplace analogies of the soul as a kind of wind, flame or air spread throughout his limbs.

\lemc{12--13 ne dubitabam quidem} remember that \textit{ne X quidem} means `not even X'.

\lemc{13--14 qualem mente concipiebam} depends on \textit{describere} (14). That is, `to describe what sort of thing I thought (that it was)'.

\lem{14 tentassem} = \textit{tentavissem} (\textbf{AG} §181). This process is often known as `syncopation'.

\lemc{15 aptum est} the infinitives \textit{terminari} (15), \textit{circumscribi} (15), \textit{replere} (16), \textit{percipi} (17), and \textit{moveri} (17) all depend on this phrase. They are explanatory or limiting infinitives: they explain qualities that befit (\textit{aptum}) a body. This sort of infinitive, often called `epexegetical', is common in Greek but rare in classical Latin. Roman poets began to use the construction in imitation of Greek, and it became more common in later Latin.

\lem{17 seipso} = \textit{se ipso}.

\lemc{18 habere vim} this infinitive depends on \textit{pertinere iudicabam} (19). Descartes did not believe (literally `judge') that it belonged to the nature of the body `to have the power'.

\lem{18 seipsum} = \textit{se ipsum}.

\lemc{18 movendi\dots sentiendi\dots cogitandi} these genitives define \textit{vim}. In English we would more likely say `the power to move, to perceive, to think', but gerunds are possible for us too.

\lemc{19 nullo pacto} an common ablative of manner meaning `in no way'. \textit{quo pacto}, sometimes written as one word, similarly means `in what way?' or simply `how?'.

\lem{19 quinimo} = \textit{quin immo}, indicating an emphatic contrast to the preceding thought. There's an important contrast between something that `belongs to the nature of body' and something that is `found in a body'. The first refers to essential qualities of bodies; the second includes non-essential qualities. Descartes didn't think that movement, perception, or thought were essential to bodies, and \textit{what's more} (\textit{quinimo}) he was surprised to find them associated with bodies at all. Throughout this paragraph Descartes admits that he was predisposed to sharply distinguish body and mind.
% -]] What is the ego (cont.)?

% -]] Meditatio secunda

    %% [[- Chapter title
\chapter{Background Material}
% -]] Chapter title

% [[- Plato's Theaetetus
\section*{Plato's \textit{Theaetetus} and the dream argument}

When Descartes uses dreaming as an argument in favor of global scepticism, he appears to have the following passage from Plato's \textit{Theaetetus} in mind. Socrates asks Theaetetus ``What is knowledge?'', and Theaetetus offers as a first definition that knowledge is perception. Socrates connects this to Protagorean relativism and Heracleitean flux, and in the selection below, he presses Theaetetus to consider some possible objections to his definition.

\begin{quote}
    \begin{greek}
        {\textbf{ΣΩΚΡΑΤΗΣ.} Ταῦτα δή, ὦ Θεαίτητε, ἆρ᾽ ἡδέα δοκεῖ σοι εἶναι, καὶ γεύοιο ἂν αὐτῶν ὡς ἀρεσκόντων;

        \textbf{ΘΕΑΙΤΗΤΟΣ.} Οὐκ οἶδα ἔγωγε, ὦ Σώκρατες· καὶ γὰρ οὐδὲ περὶ σοῦ δύναμαι κατανοῆσαι πότερα δοκοῦντά σοι λέγεις αὐτὰ ἢ ἐμοῦ ἀποπειρᾷ.

        \textbf{ΣΩ.} Οὐ μνημονεύεις, ὦ φίλε, ὅτι ἐγὼ μὲν οὔτ᾽ οἶδα οὔτε ποιοῦμαι τῶν τοιούτων οὐδὲν ἐμόν, ἀλλ᾽ εἰμὶ αὐτῶν ἄγονος, σὲ δὲ μαιεύομαι καὶ τούτου ἕνεκα ἐπᾴδω τε καὶ παρατίθημι ἑκάστων τῶν σοφῶν ἀπογεύσασθαι, ἕως ἂν εἰς φῶς τὸ σὸν δόγμα συνεξαγάγω· ἐξαχθέντος δὲ τότ᾽ ἤδη σκέψομαι εἴτ᾽ ἀνεμιαῖον εἴτε γόνιμον ἀναφανήσεται. ἀλλὰ θαρρῶν καὶ καρτερῶν εὖ καὶ ἀνδρείως ἀποκρίνου ἃ ἂν φαίνηταί σοι περὶ ὧν ἂν ἐρωτῶ.

        \textbf{ΘΕΑΙ.} Ἐρώτα δή.

        \textbf{ΣΩ.} Λέγε τοίνυν πάλιν εἴ σοι ἀρέσκει τὸ μή τι εἶναι ἀλλὰ γίγνεσθαι ἀεὶ ἀγαθὸν καὶ καλὸν καὶ πάντα ἃ ἄρτι διῇμεν.

        \textbf{ΘΕΑΙ.} Ἀλλ᾽ ἔμοιγε, ἐπειδὴ σοῦ ἀκούω οὕτω διεξιόντος, θαυμασίως φαίνεται ὡς ἔχειν λόγον καὶ ὑποληπτέον ᾗπερ διελήλυθας.

        \textbf{ΣΩ.} Μὴ τοίνυν ἀπολίπωμεν ὅσον ἐλλεῖπον αὐτοῦ. λείπεται δὲ ἐνυπνίων τε πέρι καὶ νόσων τῶν τε ἄλλων καὶ μανίας, ὅσα τε παρακούειν ἢ παρορᾶν ἤ τι ἄλλο παραισθάνεσθαι λέγεται. οἶσθα γάρ που ὅτι ἐν πᾶσι τούτοις ὁμολογουμένως ἐλέγχεσθαι δοκεῖ ὃν ἄρτι διῇμεν λόγον, ὡς παντὸς μᾶλλον ἡμῖν ψευδεῖς αἰσθήσεις ἐν αὐτοῖς γιγνομένας, καὶ πολλοῦ δεῖ τὰ φαινόμενα ἑκάστῳ ταῦτα καὶ εἶναι, ἀλλὰ πᾶν τοὐναντίον οὐδὲν ὧν φαίνεται εἶναι.

        \textbf{ΘΕΑΙ.} Ἀληθέστατα λέγεις, ὦ Σώκρατες.

        \textbf{ΣΩ.} Τίς δὴ οὖν, ὦ παῖ, λείπεται λόγος τῷ τὴν αἴσθησιν ἐπιστήμην τιθεμένῳ καὶ τὰ φαινόμενα ἑκάστῳ ταῦτα καὶ εἶναι τούτῳ ᾧ φαίνεται;

        \textbf{ΘΕΑΙ.} Ἐγὼ μέν, ὦ Σώκρατες, ὀκνῶ εἰπεῖν ὅτι οὐκ ἔχω τί λέγω, διότι μοι νυνδὴ ἐπέπληξας εἰπόντι αὐτό. ἐπεὶ ὡς ἀληθῶς γε οὐκ ἂν δυναίμην ἀμφισβητῆσαι ὡς οἱ μαινόμενοι ἢ οἱ ὀνειρώττοντες οὐ ψευδῆ δοξάζουσιν, ὅταν οἱ μὲν θεοὶ αὐτῶν οἴωνται εἶναι, οἱ δὲ πτηνοί τε καὶ ὡς πετόμενοι ἐν τῷ ὕπνῳ διανοῶνται.

        \textbf{ΣΩ.} Ἆρ᾽ οὖν οὐδὲ τὸ τοιόνδε ἀμφισβήτημα ἐννοεῖς περὶ αὐτῶν, μάλιστα δὲ περὶ τοῦ ὄναρ τε καὶ ὕπαρ;

        \textbf{ΘΕΑΙ.} Τὸ ποῖον;

        \textbf{ΣΩ.} Ὃ πολλάκις σε οἶμαι ἀκηκοέναι ἐρωτώντων, τί ἄν τις ἔχοι τεκμήριον ἀποδεῖξαι, εἴ τις ἔροιτο νῦν οὕτως ἐν τῷ παρόντι πότερον καθεύδομεν καὶ πάντα ἃ διανοούμεθα ὀνειρώττομεν, ἢ ἐγρηγόραμέν τε καὶ ὕπαρ ἀλλήλοις διαλεγόμεθα.

        \textbf{ΘΕΑΙ.} Καὶ μήν, ὦ Σώκρατες, ἄπορόν γε ὅτῳ χρὴ ἐπιδεῖξαι τεκμηρίῳ· πάντα γὰρ ὥσπερ ἀντίστροφα τὰ αὐτὰ παρακολουθεῖ. ἅ τε γὰρ νυνὶ διειλέγμεθα οὐδὲν κωλύει καὶ ἐν ὕπνῳ δοκεῖν ἀλλήλοις διαλέγεσθαι· καὶ ὅταν δὴ ὄναρ ὀνείρατα δοκῶμεν διηγεῖσθαι, ἄτοπος ἡ ὁμοιότης τούτων ἐκείνοις.

        \textbf{ΣΩ.} Ὁρᾷς οὖν ὅτι τό γε ἀμφισβητῆσαι οὐ χαλεπόν, ὅτε καὶ πότερόν ἐστιν ὕπαρ ἢ ὄναρ ἀμφισβητεῖται, καὶ δὴ ἴσου ὄντος τοῦ χρόνου ὃν καθεύδομεν ᾧ ἐγρηγόραμεν, ἐν ἑκατέρῳ διαμάχεται ἡμῶν ἡ ψυχὴ τὰ ἀεὶ παρόντα δόγματα παντὸς μᾶλλον εἶναι ἀληθῆ, ὥστε ἴσον μὲν χρόνον τάδε φαμὲν ὄντα εἶναι, ἴσον δὲ ἐκεῖνα, καὶ ὁμοίως ἐφ᾽ ἑκατέροις διισχυριζόμεθα.

        \textbf{ΘΕΑΙ.} Παντάπασι μὲν οὖν.

        \textbf{ΣΩ.} Οὐκοῦν καὶ περὶ νόσων τε καὶ μανιῶν ὁ αὐτὸς λόγος, πλὴν τοῦ χρόνου ὅτι οὐχὶ ἴσος;

        \textbf{ΘΕΑΙ.} Ὀρθῶς.

        \textbf{ΣΩ.} Τί οὖν; πλήθει χρόνου καὶ ὀλιγότητι τὸ ἀληθὲς ὁρισθήσεται;

        \textbf{ΘΕΑΙ.} Γελοῖον μεντἂν εἴη πολλαχῇ.

        \textbf{ΣΩ.} Ἀλλά τι ἄλλο ἔχεις σαφὲς ἐνδείξασθαι ὁποῖα τούτων τῶν δοξασμάτων ἀληθῆ;

        \textbf{ΘΕΑΙ.} Οὔ μοι δοκῶ.} (Plato's \textit{Theaetetus} 157c--158e4\footnote{The text follows \cite{plato1995}.})
    \end{greek}
\end{quote}

\begin{quote}
\textbf{Socrates.} Well, Theaetetus, does this look to you a tempting meal and could you take a bite of the delicious stuff?

\textbf{Theaetetus.} I really don't know, Socrates. I can't even quite see what you're getting at---whether the things you're saying are what you think yourself, or whether you're just trying me out.

\textbf{Soc.} You're forgetting, my friend. I don't know anything about this kind of thing myself, and I don't claim any of it as my own. I'm barren of theories; my business is to attend you in your labor. So I chant incantations over you and offer you little tidbits from each of the wise until I succeed in assisting you at bringing your own belief forth into the light. When it's been born, I'll consider whether it's fertile or a wind-egg. But you must have courage and patience; answer boldly whatever appears to you to be true about the things I ask you.

\textbf{Theaet.} All right, go on with your questions.

\textbf{Soc.} Tell me again, then, whether you like the suggestion that good and beautiful and all the things we were just speaking of cannot be said to `be' anything, but are always `coming to be'.

\textbf{Theaet.} Well, as far as I'm concerned, while I'm listening to your exposition of it, it seems to me an extraordinarily reasonable view; and I feel that the way you've set out the matter must be accepted.

\textbf{Soc.} In that case, we better not pass over any point where our theory is still incomplete. What we've not yet discussed is the question of dreams, and of insanity and other diseases; also what's called mishearing or misseeing or other cases of misperception. You realize, I suppose, that it would be generally agreed that all these cases appear to provide a refutation of the theory we've just expounded. For in these conditions, we surely have false perceptions. Here it's far from being true that all things which appear to an individual also are. On the contrary, no one of the things which appear to the person really is.

\textbf{Theaet.} That's very true, Socrates.

\textbf{Soc.} Well then, my child, what argument is left for the person who maintains that knowledge is perception and that what appears to anyone also is, for the person to whom it appears to be?

\textbf{Theaet.} Well, Socrates, I'm reluctant to tell you that I don't know what to say, since I've just got into trouble with you for that. But I really wouldn't know how to dispute the suggestion that a madman believes what is false when they think that they're a god; or a dreamer when they believe they have wings are are flying in their sleep.

\textbf{Soc.} But there's a point here which \textit{is} a matter of dispute, especially as regards dreams and real life---don't you see?

\textbf{Theaet.} What do you mean?

\textbf{Soc.} There's a question you must often have heard people ask---the question what evidence we could offer if we were asked whether in the present instance, at this moment, we are asleep and dreaming all our thoughts, or awake and talking to each other in real life.

\textbf{Theaet.} Yes, Socrates, it certainly is difficult to find the evidence we need here. The two states seem to correspond in all their characteristics. There's nothing to prevent us from thinking when we're asleep that we're having the very same discussion that we've just had. And when we dream that we're telling the story of a dream, there's an extraordinary likeness between the two experiences.

\textbf{Soc.} You see, then, it's not difficult to find matter for dispute, when it's disputed even whether this is real life or a dream. Indeed we may say that, as our periods of sleeping and waking are of equal length, and as in each period the soul contends that the beliefs of the moment are pre-eminently true, the result is that for half our lives we assert the reality of the one set of objects and for half that of the other set. And we make our assertions with equal conviction in both cases.

\textbf{Theaet.} That's certainly so.

\textbf{Soc.} And doesn't the same argument apply in the cases of disease and madness, except that the periods of time are not equal?

\textbf{Theaet.} Yes, that's right.

\textbf{Soc.} Well now, are we going to fix the limits of truth by the clock?

\textbf{Theaet.} That would be a very funny thing to do.

\textbf{Soc.} But can you produce some other clear indication to show which of these beliefs are true?

\textbf{Theaet.} I don't think I can. (lightly adapted translation by M.J. Levett)
\end{quote}

At this point in the dialogue, Socrates is not arguing for or against any particular thesis. He's elicited a definition of knowledge from Theaetetus---that knowledge is perception---and Socrates is still in the process of fleshing out the implications of that definition. Socrates raises the problems of dreams, disease, and insanity as part of that broader review of Theaetetus' definition.

We can readily see, however, why this passage might appeal to Descartes. Descartes wants to introduce general or broad reasons for sceptical doubt, and Socrates alludes to just such a broad scope for the dreaming argument in particular: ``[I]n these conditions, we surely have false perceptions. Here's it's far from being true that all things which appear to someone also are. On the contrary, none of the things which appear to someone also are.'' Dreams offer a convenient example of a situation in which \textit{everything} we think appears false or unreal.

This isn't to say that everything in dreams is equally surreal. I might dream that I'm doing something perfectly ordinary. For example, washing dishes. But of course I'm \textit{not} actually washing dishes because I'm asleep. I only think that I'm washing dishes.

% -]] Plato's Theaetetus

    % [[- Chapter title
\chapter{Vocabulary}
\markboth{\MakeUppercase{Vocabulary}}{\MakeUppercase{Vocabulary}}
% -]] Chapter title

% [[- Introduction
This chapter provides a complete vocabulary for Descartes's \textit{Meditationes de prima philosophia}. Here's a brief description of the principles I followed in putting together the vocabulary as well as some important conventions.

The list should include every word in the \textit{Meditations}, no matter how basic, but I have flagged the more common or important words. If an asterisk appears in the outer left margin of an entry, that word is either (i) especially common in ancient Latin or (ii) particularly important for Descartes. In order to determine what words are especially common in ancient Latin, I relied on the ``Latin Core Vocabulary'' from the Dickinson College Commentary website \parencite{cfrancese2014}. Chrisopher Francese, the primary compiler of that list, notes that those 1,000 words cover nearly 70\% of all the forms in typical ancient Latin texts \parencite{cfrancese2013}. Any word in the DCC Latin List will have an asterisk, but I've also placed asterisks on some additional words. These aditional words fall into one of two categories: (i) cognates of words already on the DCC list or (ii) words I feel are especially common in or important for Descartes. Words with an asterisk are a great first place for students to direct their attention when learning new vocabulary.

I relied on three sources for the entries. For meanings, I used both the \textit{Oxford Latin Dictionary} and \textit{A Latin Dictionary} \parencite{old1982,lewisshort}. Since these dictionaries often leave questions about the vowel length of hidden quantities unanswered (e.g., \textit{actum} or \textit{āctum}?), I also consulted Michael Weiss's \textit{Outline of the Historical and Comparative Grammar of Latin} \parencite{weiss2011}. Within entries, commas separate synonymous meanings, and semicolons separate distinct meanings.
\clearpage
% -]] Introduction

% [[- Vocabulary
\begin{description}
    \item[ā, ab, abs] \marginnote{*}from, away from; by
    \item[abdūcō, abdūcere, abdūxī, abductum] lead away, lead aside; take away, carry off, remove
    \item[aberrō (1)] wander away, get lost; wander from one's subject, digress; depart from, differ from; vary
    \item[abstineō, abstinēre, abstinuī, abstentum] keep back, keep away, keep off; hold back, restrain; abstain from food or drink, fast
    \item[abūtor, abūtī, abūsus sum]  use up, consume, exhaust; make full use of, utilize; put to a wrong use, misapply, abuse, misuse
    \item[accidēns, accidentis, n.] a nonessential property of something, an accident (as a Latin translation of Aristotle's technical term \textgreek{συμβεβηκός}); an accident (in an everyday sense), chance
    \item[accipiō, accipere, accēpī, acceptum] \marginnote{*}take in one's grasp, receive; take into one's possession or control; have given to one, acquire, get
    \item[accūratē] precisely, exactly; carefully, attentively
    \item[accūrō (1)] give attention to, perform with care; take care (that), see to it (that); attend to, take care of
    \item[actiō, actiōnis, f.] a doing, activity, action; act, deed; proposal, measure, course of action, policy
    \item[ad] \marginnote{*}towards, to, up to; near; to (a point in time), until
    \item[addō, addere, addidī, additum] \marginnote{*}put or fit (onto), attach (to), place (along with); add onto, pile (on top of); add, attach; apply
    \item[adeō] \marginnote{*}to (such) a high degree, to (such) a great extent, (so) very, extremely; to the point or place (where); to such a pass; (after a negative) so much, so very much, all that much
    \item[adhūc] \marginnote{*}until now, as yet, up to the present time; up to that time; (in a negative phrase) so far, (as) yet; still
    \item[admittō, admittere, admīsī, admissum] (of visitors, etc.) let come, admit, give access, receive; admit, endure, tolerate; grant access (to), allow (to); permit, allow, sanction; agree to, accept, receive; let go, let loose, relsease
    \item[admoneō (2)] remind (of or that), remind (a person) about (something), put in mind of; give advice to, advise, urge, bid; prompt, admonish; caution, warn, admonish; apprise of, inform, advise (that)
    \item[admoveō, admovēre, admōvī, admōtum]  move (something) near (to), bring (something) near (to), bring (something) into contact with; move towards; move up, bring up, bring near; move, lead, guide (someone or something) towards
    \item[adsum, adesse, adfuī, adfutūrum] \marginnote{*}to be at, be present, be at hand, be here; attend, be present (at a meeting, gathering, etc.)
    \item[advertō, advertere, advertī, adversum] \marginnote{*}turn or direct towards; direct, steer, guide; give ear to, pay attention, notice, see, give heed (to); remark, ascertain, discover
    \item[aequus, aequa, aequum] \marginnote{*}flat, even, level; favorable, convenient, advantageous; equal, like, alike; fair, just, reasonable, right; impartial, fair-minded
    \item[āër, āëris, m.] \marginnote{*}air; the atmosphere, air surrounding the earth; sky, heavens
    \item[aetās, aetātis, f.] \marginnote{*}one's age, the number of years one has lived; the age (of anything); period or time of life; an age group, people of a particular age; lifetime, years, the span of one's life
    \item[aethēr, aetheris, m.] \marginnote{*}upper regions of sky, heaven, the ether; the air, the sky
    \item[afferō, afferre, attulī, allātum] \marginnote{*}bring, fetch, carry, convey; confer, bestow, add, contribute; put forward, contribute, offer, recommend; lead, conduct, bring forward
    \item[affirmō (1)] assert strongly, maintain with certainty, affirm, swear, promise; support, corroborate (a statement or fact)
    \item[aggredior, aggredi, aggressus sum] \marginnote{*}go or advance (towards), approach; assault, attack, assail; confront; set about, start on, undertake (a task, a job, etc.); attempt, proceed, begin (+ infinitive)
    \item[agnoscō, agnoscere, agnōvī, agnitum] \marginnote{*}recognize, know again, identify; acknowledge as one's own, recognize as one's own; admit to, claim; admit liability for, become responsibile for; acknowledge, appreciate
    \item[agō, agere, ēgī, āctum] \marginnote{*}drive, set in motion; ride; bring, carry, bear; force to move, drive; force, push, throw; give off, emit, send forth; do, perform, achieve, accomplish
    \item[albēdō, albēdinis, f.] white color, whiteness, white (non-classical)
    \item[aliās] (adverb) at another time, some other time; at other times; elsewhere; to another place; a second time, afterwards; previously
    \item[aliēnus, aliēna, aliēnum] \marginnote{*}of another, belonging to another, not one's own; foreign, alien; strange, unusual, unnatural
    \item[aliquamdiū] for some time, for a considerable time; for a while
    \item[aliquandiū] see \textit{aliquamdiū}
    \item[aliquī, aliqua, aliquod] \marginnote{*}(indefinite adj.) some, some or other, a; any at all, any whatsoever, a single; a kind of, a sort of; of some (extent, degree, amount), a certain amount of, some; (pl.) a certain number of, at least some, a few; some sort of, some kind of
    \item[aliquis, aliqua, aliquid] \marginnote{*}(indefinite pronoun) some one, any one, anybody; something, anything; (pl.) some, a few, a number
    \item[aliquot] a number of, several, some
    \item[aliter] \marginnote{*}otherwise, in another way, in another manner, differently
    \item[alius, alia, aliud] \marginnote{*}another, other, a different; a further, another; (multiple forms in same case) some\dots others; (multiple forms in different cases) one\dots another; (pl.) (the) rest, (the) others
    \item[alligō (1)] tie, bind, fasten (one thing to another); secure, fasten together, tie up; bond, unite, hold together; put in chains, restrain; immobilize, pin down
    \item[alō, alere, aluī, al(i)tum] \marginnote{*}feed, nourish; support, sustain, maintain, nurture; rear, bring up, raise
    \item[alter, altera, alterum] \marginnote{*}a second, a further, one other, other; the second, the next; (with negation) either (of two); (referring to more than two) some other, another, other
    \item[āmēns, āmentis] out of one's senses, insane, demented, out of one's mind; frantic, distracted, very excited
    \item[āmittō, āmittere, āmīsī, āmissum] \marginnote{*}send away, dismiss, part with; give up, abandon; pass over, forgive; fail to catch or hold, miss, let slip; let go, lose sight of, lose track of; incur the loss of, forfeit, lose
    \item[amplius] more, further, besides, in addition; more, a greater amount, larger sum; (with numerals) more than: e.g., \textit{amplius vīgintī} = more than twenty
    \item[amplus, ampla, amplum] \marginnote{*}having ample size, bulk, or extent; large, spacious, ample; impressive in size and appearance, magnificent; great, extensive, powerful, intense; distinguished, eminent, great, impressive; comprehensive, all-embracing, large, full, unrestricted
    \item[an] \marginnote{*}(introducing direct questions, often with an added notion of surprise, indignation, or strong feeling) Can it really be that\dots ? Is it really the case that\dots ?; (introducing a possible answer to a question just asked) Or\dots? Is it\dots?, but often best untranslated; (introducing continuations of multiple questions) or; (in an indirect question) whether, if; \textit{haud scio an} I am inclined to think, probably
    \item[anima, animae, f.] \marginnote{*}air, breath; life, soul
    \item[animaduertō, animadvertere, animadvertī, animadversum] \marginnote{*}direct the mind towards; pay attention to, attend to, heed; become aware of, observe, notice; judge, appraise, estimate; censure, criticize, find fault with
    \item[animal, animālis, n.] \marginnote{*}a living being, animal; (sometimes) non-human animal
    \item[animus, animī, m.] \marginnote{*}mind (as opposed to body), soul; mind, consciousness, intelligence, spirit; design, purpose, intention; will, desire; (usually plural) anger, animosity; courage, pride, spirit, morale; disposition, character
    \item[annus, annī, m.] \marginnote{*}a year, twelve months; (pl.) age
    \item[ante] \marginnote{*}\textit{adverb} in front, in front of one; before, in advance, ahead; (with difference in time expressed by ablative or accusative) before this, ago
    \item[ante] \marginnote{*}\textit{preposition with accusative} before, in front of; before (a time), by; (before in preference or rank) above, more than
    \item[antehāc] before this time, un until now, previously; in the past, before now
    \item[aperiō, aperīre, aperuī, apertum] \marginnote{*}open, open up; clear, open, repair; uncover, lay bare, reveal; make available or possible, put at one's disposal
    \item[appāreō (2)] \marginnote{*}appear, to be clear, to be evident; come to hand, turn up; come into sight, appear
    \item[appellō (1)] \marginnote{*} address, speak to, accost; appeal to (for help), call on, beseech, apply to; name, call by name
    \item[applicō (1)] bring (something) into contact with (something else), lean (something) against (another thing); fit on, attach; bring to bear (on), apply (to); assign, set; accomodate, adapt
    \item[aptus, apta, aptum] \marginnote{*}composed, fitted together; tied, fastened, bound; associated, connected; prepared, equipped, ready; handy, convenient, suitable for use, useful; efficient, good at, fitted for, able to
    \item[apud] \marginnote{*}(preposition with the accusative) at, near, in the area of, close to, next to, by, besides; among, in the presence of; at the house of, at the residence of
    \item[arbitror (1)] \marginnote{*}observe, notice, witness; judge, decide; consider, reckon; think, judge, imagine, be of the opinion
    \item[Archimēdēs, Archimēdis, m.] famous mathematician and inventor of the third century BCE
    \item[arctē/artē] closely, firmly; strictly
    \item[argūmentum, argūmentī, n.] proof, argument, process of reasoning; conclusion based on inference, deduction; motive, basis, reason; narrative, story
    \item[arithmētica, arithmēticae, f.] arithmetic; science or study of arithmetic
    \item[articulus, articulī, m.] a joint (of a limb, a finger, or a toe); a knuckle; limb, member, finger; a juncture, critical moment, point in time; clause, section
    \item[artifex, artificis, m.] an artist, creator; an expert, master (of some craft or field); a craftsperson, artisan
    \item[artificium, artificiī, n.] artistry, craftsmanship; product of art or craft; craft, handicraft
    \item[aspiciō, aspicere, aspexī, aspectum] \marginnote{*}catch sight of, observe, notice; look at, look upon, behold, gaze upon; examine, inspect, look over; perceive, appreciate, see (mentally); think about, consider, investigate
    \item[assēnsiō, assēnsiōnis, f.] approval, approbation, applause; assent (to the truth of an idea, statement, proposition), agreement; belief
    \item[assentior, assentīrī, assēnsus sum] agree in opinion, assent, approve; admit the truth of, agree
    \item[assequor, assequī, assecūtus sum] follow, go after; overtake, catch up to; come upon; attain (to), acquire, achieve, win; succeed in bringing about, achieve
    \item[assevērō (1)] declare, affirm, assert emphatically; proclaim
    \item[assideō, assidēre, assēdī, assessum] sit by, sit near; sit in council, be present in court; sit beside, watch over; camp near; besiege; dwell close (to), adjoin, be situated (near)
    \item[assiduē] continually, regularly, constantly
    \item[assignō (1)] allot, apportion, distribute as a portion, allocate, assign; award, confer, bestow; ascribe, impute, put down to (someone)
    \item[assūmō, assūmere, assūmpsī, assūmptum] take or use as an addition, insert, add; take possession of, make one's own, take; acquire, gain, take on, take up; adopt, take; derive, draw, borrow; take for granted, assume
    \item[astronomia, astronomiae, f.] astronomy, the science of the heavenly bodies
    \item[at] \marginnote{*}but; however, on the other hand (\textit{at} marks a stronger contrast than \textit{sed})
    \item[āter, ātra, ātrum] black, dark-colored; devoid of light, dark; discolored, stained; sordid, squalid; black and blue, bruised; unlucky, ill-omened
    \item[atque] \marginnote{*}and; and also, and what is more; and in fact, and even
    \item[atquī] but, and yet, nevertheless; all the same
    \item[attendō, attendere, attendī, attentum] pay attention, be attentive; listen carefully, pay close attention to; examine or study closely; look into carefully
    \item[attingō, attingere, attigī, attactum] touch, make contact with; lay a hand on, assault, attack; be contiguous to, be next to, adjoin; reach, stretch as far as; touch, reach, make contact with, strike; attack, afflict, affect; reach, arrive at, enter; attain
    \item[audiō (4)] \marginnote{*}hear; listen to
    \item[augeō, augēre, auxī, auctum] \marginnote{*}increase, enlarge, extend, swell, make larger; increase in value or amount; raise, cause to grow, make grow; stengthen, make stronger; help, assist; equip, furnish, provide
    \item[aut] \marginnote{*}or (usually used to introduce exclusive alternatives); \textit{aut\dots aut} either\dots or
    \item[autem] \marginnote{*}(almost always postpositive) on the other hand, while; but; moreover, also, too, furthermore; and in fact, and indeed
    \item[author, authoris, m.] (non-classical equivalent for \textit{auctor, auctōris}) person with power or authority in some situation; vendor, seller; guarantor; witness; authority, spokesperson; advocate, supporter; author, creator, source
    \item[automatum, automatī, n.] (sometimes \textit{automaton} in the singular nominative and accusative) automaton, a machine
    \item[āvocō (1)] call off, call away, summon away; take away, withdraw; turn aside, divert; dissuade, divert; distract, interrupt
    \item[bīlis, bīlis, f.]  bile, fluid secreted by the liver; anger, ill temper, spleen; madness, insanity
    \item[blandus, blanda, blandum] charming, ingratiating, attractive (influencing others by flattery, coaxing, etc.); winning, persuasive, propitiatory; alluring, seductive; insincere, fawning; gentle, tame, affectionate; calm, gentle; pleasing, sweet, soft
    \item[bonitās, bonitātis, f.] moral excellence, good behavior, excellence; kindness, benevolence, generosity; goodness, excellence, good quality
    \item[bonus, bona, bonum] \marginnote{*}good, efficient, expert; morally good, well-behaved, worthy, fine; obliging, accommodating, kind(ly), gracious, good
    \item[brāc(c)hium, brac(c)hiī, n.]  the forearm, lower arm (from elbow to hand); arm (from shoulder to hand)
    \item[cadāver, cadāveris, n.] a dead body, corpse, carcass
    \item[cadō, cadere, cecidī, cāsum] \marginnote{*}fall, fall over, fall down; descend, sink, set; droop, hang down; be overthrown, fall, be destroyed
    \item[caelum, caelī, n.] \marginnote{*} sky, heavens
    \item[caeterus, caetera, caeterum] see \textit{cēterus, cētera, cēterum}
    \item[calēscō, calēscere, ———, ———] grow warm, grow hot; be heated, be hot
    \item[calidus, calida, calidum] warmed, hot; warm
    \item[callidus, callida, callidum] practiced, expert, experienced; adroit, skilled; clever, ingenious; cunning, wily
    \item[calor, calōris, m.] warmth, heat; glow; fever
    \item[capax, capācis] wide, large, spacious, roomy, capacious, able to hold a lot
    \item[capessō, capessere, capessīvī/capessiī, capessītum] take hold of, grasp; snatch, seize hold of; apprehend (with the mind or senses), grasp; enter on, engage in
    \item[captīvus, captīvī, m.] prisoner (usuall of war), captive
    \item[caput, capitis, n.] \marginnote{*}head; (transferred to things) top, summit, end; (metonymically) a person or animal
    \item[caro, carnis, m.] (sometimes also \textit{carnis} in nominative singular) flesh, skin; meat
    \item[causa, causae, f.] \marginnote{*}legal case, trial; case, claim; motive, reason; cause; object, purpose
    \item[caveō, cavēre, cāvī, cautum] \marginnote{*}be on one's guard, take care, take precautions; beware, guard against, avoid
    \item[cēra, cērae, f.] wax, beeswax
    \item[cerebellum, cerebellī, n.] brain
    \item[cernō, cernere, crēvī, crētum] \marginnote{*}separate, sift; distinguish, see, discern, perceive; decide, determine; look at, examine
    \item[certitūdō, certitūdinis, f.] certainty, certain knowledge, certain truth; a certificate
    \item[certō] certainly, without doubt, for a fact; firmly, surely, in a manner that can be depended on; \textit{certō scīre} know for certain, know for a fact
    \item[certus, certa, certum] \marginnote{*}fixed, settled, definite; determined, resolved, certain; a particular, a certain; indisputable, about which there is no doubt
    \item[cessō (1)] hesitate, be slow, hold back; desist, cease, stop; do nothing, be idle, delay, loiter, be remiss
    \item[cēterus, cētera, cēterum] \marginnote{*}the other, remainder, rest; (pl.) the others, rest, remaining
    \item[c(h)arta, c(h)artae, f.]  a leaf of papyrus, paper; (pl.) pages, writings
    \item[chim(a)era, chim(a)erae, f.] a mythological monster, part lion, part snake, part goat
    \item[cieō, ciēre, cīvī, citum] cause to go, move, set in motion, stir up; excite, give rise to; provoke, cause; summon, muster; call on by name, invoke; appeal to, call upon
    \item[circā] \marginnote{*}(adverb or preposition with the accusative) round about, around; round, in a circle; in the company of, with; near, close to
    \item[circumscrībō, circumscrībere, circumscrīpsī, circumscrīptum] draw a line or circle around; confine within specific limits; restrict; delimit; exclude, rule out; define, outline (in words); exclude, rule out; abridge, write in concise form
    \item[clārus, clāra, clārum] \marginnote{*}clear, bright, shining, gleaming; (of sound) loud, sonorous; clear, seeing clearly; dinstinct, unambiguous, clear, plain; well-known, notorious
    \item[coelum, coelī, n.] \marginnote{*}(post-classical form of \textit{caelum}) sky, heavens
    \item[cōgitātiō, cōgitātiōnis, f.] \marginnote{*}the act or process of thinking, reflection, thought; the outcome of thinking, an idea, thought; reflection; preoccupation, consideration
    \item[cōgitō (1)] \marginnote{*}think, think about; consider thoroughly, ponder, weigh, reflect upon, deliberate, think through
    \item[cognitiō, cognitiōnis, f.] \marginnote{*}the act of getting to know, acquiring or possession of knowledge; an idea, notion; study, investigation
    \item[cognoscō, cognoscere, cognōvī, cognitum] \marginnote{*}get to know, learn, find out; study, master, acquire knowledge of; (in perfect tenses) know (i.e. having learned, one knows)
    \item[cōgō, cōgere, coēgī, coāctum] \marginnote{*}drive together, bring together; assemble, muster; collect, summon, convene, gather; compel, force, constrain
    \item[cohibeō (2)] hold together, secure; keep secret, contain; restrain, hold back, check, stop, prevent; control, suppress; withold
    \item[collābor, collābī, collāpsus sum] slip, fall down, collapse; give way, fail; collapse, end suddenly, be lost
    \item[colligō, colligere, collēgī, collectum] \marginnote{*}gather together, collect, pick up, harvest; get hold of, get possession of; accumulate, amass, build up; hold, keep together; assemble, bring together
    \item[color, colōris, m.] \marginnote{*}color, hue, tint (of an object); a particular color
    \item[committō, committere, commīsī, commissum] join, combine, put together, unite; begin, join, commence (a battle); commit a crime, practice or perpetrate wrong or injustice
    \item[commoueō, commovēre, commōvī, commōtum] shake, agitate, move vigorously; stir up, stir, move, budge; upset, jolt, disturb; awake, wake up (someone else)
    \item[commūnis, commūne] \marginnote{*}joint, common, shared, general; neutral, impartial
    \item[compāgēs, compāgis, f] the action of joining together, binding, bond, tie; joint, suture; a composite structure, framework, assemblage
    \item[comparātiō, comparātiōnis, f.] a comparing, a comparison; an agreement, contract; a simile
    \item[comparō (1)] \marginnote{*}place together, align; compare, match, set against; treat as equal, put in the same class with
    \item[comperiō, comperīre, comperī, compertum] find out by investigation, learn, discover, ascertain; prove, establish, verify
    \item[complector, complectī, complexus sum] embrace, hug; welcome, take up, adopt; seize, grasp, grip, cling to; comprehend, take in, understand; include, involve, associate, bring in
    \item[compōnō, compōnere, composuī, compositum] \marginnote{*}place things together, add together; pack up, store or collect together; settle in a place or position; arrange in order, dispose, organize, draw up; construct, build, put together
    \item[compositus, composita, compositum] composed of, made of; composite, blended, compound; well-arranged, well-ordered; orderly, well-disciplined, law-abiding; practiced, studied; calm, undisturbed, placid, sedate
    \item[comprehendō, comprehendere, comprehendī, comprehensum] take hold of, grip, catch hold of; take hold, take root; catch; fasten, unite, tie or join together, hold together; find, seize, discover, detect; enclose, surround, include
    \item[comprehēnsiō, comprehēnsiōnis, f.] the action of taking hold, grasping; arrest, apprehension; scope, range; apprehension, perception, mental grasp, understanding
    \item[concēdō, concēdere, concessī, concessum] \marginnote{*}go away, withdraw, retire; go over, transfer; give place, make way, yield, defer; give place to, concede, surrender, give up, hand over
    \item[conceptus, conceptūs, m.] (the action of) collecting, gathering; a collection, a conflux; (the action of) taking, catching, capturing; conception, pregnancy; a fetus; (mental) conception, thought, notion, opinion; purpose, intention
    \item[concipiō, concipere, concēpī, conceptum] take in, absorb, catch; conceive, become pregnant; produce, form, bring into existence; contain, hold; grasp, conceive (in the mind), imagine, form an idea of; devise; undertake, assume
    \item[conclūdō, conclūdere, conclūsī, conclūsum] shut up, close in, imprison, enclose, confine; brind to an end, conclude, finish; infer, deduce, conclude
    \item[concursus, concursūs, m.] (the action of) running or flocking together; a concourse, assembly, mob, crowd; combination, conjunction; onset, attack, charge; concurrence, assent, agreement
    \item[confīdō, confīdere, confīsus sum] put one's trust in, have confidence in; trust confidently, be sure
    \item[configūrātiō, configūrātiōnis, f.] similarity in form; configuration, arrangement
    \item[confirmō (1)] make firm, secure; assure, reassure, make confident, encourage; confirm, establish firmly, prove
    \item[conflō (1)] blow on, raise, start, ignite, set fire to; arouse, stir up; form, invent, concoct; make (by melting or heat), cast, weld; assemble, run up
    \item[confundō, confundere, confūdī, confūsum] pour together, mingle, mix, blend; mix up, stir up; jumble, upset, disorder, confuse; ruin, destroy; blur, obscure, disfigure; disconcert, trouble, dismay, upset (a person)
    \item[coniciō, conicere, coniēcī, coniectum] throw or put together, bring together; throw, cast, hurl; dispatch, make (a person) go; assign, insert; conjecture, guess
    \item[coniungō, coniungere, coniūnxī, coniūnctum] bind together, join together, join, connect, unite, fasten together; compose (by joining)
    \item[connīueō, connīvēre, connivī/connīxī, ———] (also in forms \textit{cōnīveō}, etc.) close (the eyes) tightly, close (eyes) in sleep; turn a blind eye (to), overlook, ignore
    \item[cōnor (1)] undertake, endeavor, try, attempt, make an effort; seek, aim
    \item[cōnsentāneus, consentānea, consentāneum] fitting, agreeable, appropriate, consistent; constant in principles, consistent (of a person)
    \item[cōnsīderātiō, cōnsīderātiōnis, f.] action of looking, gaze, inspection; contemplation, consideration; examination, contemplation, consideration
    \item[cōnsīderō (1)] look at closely, attentively, or carefully, inspect, examine, investigate; notice, remark; bear in mind, take into consideration; take care, bear in mind (that)
    \item[cōnsistō, cōnsistere, cōnstitī, cōnstitum] \marginnote{*}stand still, stand, halt, stop, stay in place; take a stand, post oneself; congeal, set; pause, linger, dwell on; take a position, take a stand (for fighting)
    \item[cōnspicuus, conspicua, conspicuum] clearly seen, visible, in full view; remarkable, striking, attracting attention; notable, famous, illustrious
    \item[cōnstanter] without change; regularly, steadily; firmly, resolutely, in a determined manner; faithfully, loyally, steadfastly
    \item[cōnstō, cōnstāre, cōnstitī, cōnstātum] \marginnote{*}be composed, consist of; exist, continue, last; stop, halt; agree, accord, be consistent; stand still, be steady
    \item[cōnsuēscō, cōnsuēscere, cōnsuēvī, cōnsuētum]  become accustomed or used (to something), form a habit; make accustomed or used; (perfect) be in the habit of, be used to
    \item[cōnsuētūdō, cōnsuētūdinis, f.] a custom, habit, usual practice, usage; normal state or condition; convention, custom (as a source of law); linguistic usage, normal manner of speaking
    \item[cōnsūmō, cōnsūmere, cōnsūmpsī, cōnsūmptum] \marginnote{*}destroy, wear away, consume; kill; reduce, make smaller, reduce to nothing; weaken severely, wear down, exhaust; use up, expend; eat, devour; expend, use up, employ; spend (money, resources, time, effort)
    \item[contemplō (1)] look at, examine visually, gaze at; observe, notice, study; ponder, consider, contemplate
    \item[contendō, contendere, contendī, contentum] assert, affirm, maintain (strongly or insistently); stretch, bend, draw tight, strain
    \item[contineō, continēre, continuī, contentum] hold together, link, join; fasten, secure; keep together, sustain; maintain; retain, keep; surround, enclose, embrace; contain, include; imply, involve; comprise, consist in
    \item[contingō, contingere, contigī, contactum] \marginnote{*}be connected with, be related to, concern; come into contact with, touch; (intransitive with dative) fall to one's lot, happen, be granted; (absolute) come about, happen; hit, strike
    \item[continuātus, continuāta, continuātum] uninterrupted, unbroken; consecutive; (with dative) contiguous, adjacent to
    \item[contrārius, contrāria, contrārium] opposite, on the opposite side; reverse, opposite to normal; opposed, in opposition; hostile, adverse, opposing; contrary, antithetical, opposite in kind or type; incompatible, opposite in intension
    \item[contrectō (1)] touch, handle (an object), come in contact with, feel; deal with, handle (a subject), apply oneself to
    \item[contumax, contumācis] proud and unyielding, stubborn, defiant (usually in a negative sense); wilfully disobedient, contumacious
    \item[corporeus, corporea, corporeum] \marginnote{*}physical, material, corporeal, consisting of a body or endowed with a body; related to the body, bodily
    \item[corpus, corporis, n.] \marginnote{*}the body of a living creature; body (as opposed to soul or mind); any physical object (as opposed to immaterial things)
    \item[corruptiō, corruptiōnis, f.] (the action of) corrupting, spoiling; corrupt condition, corruption; disintegration, decay, rot (of specific acts of corruption) bribery, seduction, infection or disease
    \item[crassus, crassa, crassum] solid, thick, dense; fat, plump, stout; coarse, rough; rude, homely (opposed to elegant or fine); insensitive, dull, stupid, crass (opposed to perceptive or intelligent)
    \item[crēdō, crēdere, crēdidī, crēditum] \marginnote{*}commit to, entrust; confide; (with dative) trust, rely on, have faith in, believe, give credence to
    \item[crēdulitās, crēdulitātis, f.] ready belief, credulity, trustfulness; rash confidence
    \item[creō (1)] \marginnote{*}procreate; bring into being, create, produce; institute, establish (an idea or custom)
    \item[crēscō, crēscere, crēvī, crētum] \marginnote{*}come into being, spring up; develop; lengthen, increase, swell, expand; progress, advance; increase in numbers, amount, or the like; multiply
    \item[cucurbita, cucurbitae, f.] a gourd, a pumpkin; a cupping glass (from the shape); a dolt, `pumpkin-head'
    \item[culpa, culpae, f.] \marginnote{*}responsibility, blame, fault; wrongdoing, offence, midsconduct; failure, neglect, error, mistake
    \item[cum] \marginnote{*}(preposition with ablative) with, together with, along with
    \item[cum] \marginnote{*}(conjunction with indicative) at that time, when; from that time, since; (conjunction with subjunctive) under the circumstances when, when; because, since; although
    \item[cunctor (1)] delay, hang back; hesitate; tarry, linger
    \item[cunctus, cuncta, cunctum] \marginnote{*}the whole of, all; total, complete
    \item[cupiō, cupere, cupīvī/cupiī, cupītum] \marginnote{*}wish for, desire, want; wish well to, favor, be well disposed to (plus dative)
    \item[cūra, cūrae, f.] \marginnote{*}anxiety, worry, care, distress; carefulness, pains, (serious) attention; solicitude, concern
    \item[cūrō (1)] \marginnote{*}watch over, look after, care for; tend, rear; treat (a sick person, an illness, a wound), tend to, cure; undertake, have done, see to it (that)
    \item[dē] \marginnote{*}(preposition with ablative) down from; away from, off; from; concerning, about; of, out of
    \item[dēbeō (2)] \marginnote{*}be under an obligation, be indebted, owe; ought, should; must
    \item[dēceptor, dēceptōris, m.] a betrayer; a deceiver
    \item[dēcipiō, dēcipere, dēcēpī, dēceptum] deceive, mislead, dupe; frustrate, foil, cheat; escape the notice of, elude
    \item[dēfīgō, dēfīgere, dēfīxī, dēfīxum] plant, embed, sink, bury; affix, attach; keep (one's eyes or thoughts) directed (on), fix, focus; fix (with a glance); petrify, dumbfound, render incapable of thought or movement
    \item[dēgō, dēgere, dēgī, ———] spend, pass (time, one's life); remain alive, live on, continue
    \item[deinceps] in succession, in turn; after that, after this, next, then
    \item[deinde] \marginnote{*}afterwards, then, next; from there; in the next (second, third, etc.) place
    \item[dēlābor, dēlābī, dēlāpsus] fall, drop; sink, slip down; descend, fly or glide down; flow down; fail, lose strength
    \item[dēlīberō (1)] engage in careful thought, weigh the pros and cons of; deliberate; take counsel (with someone), consult; consider carefully, think over, ponder
    \item[dēlūdō, dēlūdere, dēlūsī, dēlūsum] deceive, dupe, delude
    \item[dēmō, dēmere, dēmpsī, dēmptum] remove, take away, take off, subtract
    \item[dēmōnstrātiō, dēmōnstrātiōnis, f.] (the action of) showing, bringing to notice, displaying; a display, show, representation; demonstration, proof; exposition, declaration
    \item[dēmōnstrō (1)] point out, indicate, designate; show, prove, demonstrate; represent, describe, mention
    \item[dēnegō (1)] reject, refuse, deny
    \item[dēnique] \marginnote{*}finally, at last, at length, in the end; lastly; in sum, in short, to sum up; in point of fact, indeed
    \item[dēnuō] anew, over again, from a fresh start; for a second time, once more, again; in turn, then again
    \item[dēpendeō, dēpendēre, dēpendī, ———] hang down (from); proceed or derive from; depend (on)
    \item[dēpōnō, dēpōnere, dēposuī, dēpositum] put down, lay down; give up, surrender, shed; let rest; deposit, lodge, give for safekeeping; abandon, resign, drop
    \item[dēprehendō, dēprehendere, dēprehendī, dēprehēnsum] intercept, seize, catch; comprehend, perceive, understand; find, discover, detect, recognize; take by surprise, overtake
    \item[dēscrībō, dēscrībere, dēscrīpsī, dēscrīptum] draw, mark out, describe, trace out; write down, transcribe, copy down; prescribe, establish
    \item[dēsidia, dēsidiae, f.] idleness, slackness, inactivity; leisure, freedom (from work or responsibility)
    \item[dēsignō (1)] mark out, trace out, draw in outline; mark; indicate, denote, designate; earmark, assign; appoint, elect; order, plan
    \item[dēsinō, dēsinere, dēs(i)ī/dēsīvī, dēsitum] \marginnote{*}leave off, desist, finish, stop, cease; come to an end, end (with or in), end
    \item[dēsuēscō] unlearn, forget about, become unaccustomed to, become unused to
    \item[dētorqueō, dētorquēre, dētorsī, dētortum] turn away, deflect; turn aside, divert, sway, change (often for the worse), pervert; bend out of shape, twist, distort
    \item[dētrahō, dētrahere, dētrāxī, dētrāctum] detach (by pulling), pull, draw, remove; drag away, draw off; force down, induce to come down; pull down, demolish; detach, dislodge; derive; detract (in speech), say in disparagement (of); exclude, omit, subtract, cut out
    \item[deus, deī, m.] \marginnote{*}a god, a divinity
    \item[dēveniō, dēvenīre, dēvēnī, dēventum] come (to a destination), arrive; go; ( of places) extend (to); fall to the lot of; have recourse to, turn (to)
    \item[dēvinciō, dēvincīre, dēvinxī, dēvīnctum] bind fast, tie up (tightly); fix in position, hold fast; oblige, constrain, bind (morally, emotionally, etc.); subjugate, bring into subjection; unite, bind together
    \item[dīcō, dīcere, dīxī, dictum] \marginnote{*}say, speak, utter; tell, mention, relate; affirm, declare, state, assert
    \item[differō, differre, distulī, dīlātum] \marginnote{*}carry away in different directions, scatter, disperse; confound, bewilder, distract; spread around, publish, make known; postpone, defer, put off, keep waiting; differ, be different
    \item[difficilis, difficile] \marginnote{*}hard, difficult, troublesome; intractable, obdurate, inflexible, hard to manage
    \item[difficultās, difficultātis, f.] difficulty, situation involving trouble, trouble
    \item[diffīdentia, diffīdentiae, f.] mistrust, distrust; diffidence, lack of confidence
    \item[dīiūdicātiō, dīiūdicātiōnis, f.] (the action of) judging, deciding; a judgment, a decision
    \item[dīligēns, dīligentis] fond (of), devoted (to); careful, attentive, diligent, scrupulous; thrifty, economical
    \item[dīmoveō, dīmovēre, dīmōvī, dīmōtum] cleave, part; cause to part, disperse; displace, remove; set aside, dismiss, divert
    \item[disciplīna, disciplīnae, f.] instruction, teaching, training, education; a branch of study, discipline; a philosophical school or sect; discipline, orderly conduct based on moral training, order (maintained in a group of people)
    \item[dispār, disparis] unequal; different (in character or kind), dissimilar
    \item[disputō (1)] argue one's case or point of view; reason out, debate, argue
    \item[distinctiō, distinctiōnis, f.] (the action of) distinguishing, discriminating; a difference, distinction; separation, division (in a piece of writing or speech)
    \item[distinctus, distincta, distinctum] \marginnote{*}different, distinct; definite, precise, not vague; clear, lucid
    \item[distinguō, distinguere, distīnxī, disctīnctum] separate, divide, part; mark off, distinguish, separate (from); punctuate (a process), relieve, interrupt; distinguish (from another), make or perceive a difference between; define, specify; resolve, settle
    \item[diūturnitās, diūturnitātis, f.] passage of a long period of time, lapse of time; long duration, permanence, durability, longevity
    \item[dīvellō, dīvellere, dīvellī/dīvulsī, divulsum] tear open, tear apart, tear to pieces; pull away (from), tear away (from); break up, sunder, disrupt
    \item[dīversus, dīversa, dīversum] turned, pointed, or facing in different directions; set apart, separate; distant, remote; different, differing, distinct, divergent, inconsistent
    \item[dīvīsibilis, dīvīsibile] divisible, separable
    \item[dō, dare, dedī, datum] \marginnote{*}give, offer, grant; impose, assign, appoint; concede, allow
    \item[dōnec] \marginnote{*}until, up to the time at which; as long as, while
    \item[dormiō (4)] \marginnote{*}sleep, be asleep, fall asleep
    \item[dubitātiō, dubitātiōnis, f.] uncertainty, doubt, perplexity; wavering, hesitation
    \item[dubitō (1)] \marginnote{*}be in doubt, be uncertain; doubt, waver, hesitate over
    \item[dubius, dubia, dubium] \marginnote{*}hesitatnt, undecided, wavering, faltering; uncertain, doubtful, dubious
    \item[dūcō, dūcere, dūxī, ductum] \marginnote{*}lead, conduct, draw, bring forward, guide; consider, reckon, hold, account, esteem, believe
    \item[dulcēdō, dulcēdinis, f.] sweetness; pleasantness, charm; a pleasure, something sweet or pleasant
    \item[dum] \marginnote{*}while, as long as; until
    \item[duo, duae, duo] \marginnote{*}two; a couple, a pair
    \item[dūrō (1)] make hard, harden, solidify; become hard, become solid; harden onself, steel oneself; hold out, endure, last; endure, bear with
    \item[ē, ex] \marginnote{*}out of, from within, from
    \item[ecce] \marginnote{*}behold! look! see!
    \item[ēdūcō, ēdūcere, ēdūxī, ēductum] \marginnote{*}lead or bring out or away; lead forth; draw out, extract; elicit, evoke; drain away, draw away; bring forth, conceive
    \item[efficiō, efficere, effēcī, effectum] \marginnote{*}manufacture, make, construct; compose; cause to occur, bring about; bring it about that, be the cause that; yield, produce, bear; make up, constitute; make or perform completely, finish; deduce from premises, prove, (passive) follow
    \item[effingō, effingere, effīnxī, effictum] shape, mould, fashion, form; portray, depict, represent; reproduce, copy, imitate
    \item[ēmanō (1)] pour fourth, flow (out, forth); arise, emanate; become diffused, spread (out, about, around)
    \item[ēmittō, ēmittere, ēmīsī, ēmissum] \marginnote{*}send out, send forth, dispatch; make known, publish; release, free, let go, discharge, let loose; let fly, launch, shoot; give off, emit; utter (words, sounds)
    \item[ēnatō] swim away, escape by swimming; float up or forth
    \item[enim] \marginnote{*}(usually postpositive) for; for instance, namely, that is to say; I mean, in fact
    \item[ēnītor, ēnītī, ēnīsus/ēnīxus sum] struggle up or out; take pains, strive, exert onself; give birth to, produce
    \item[ēns, entis, n.] entity, thing, being; essence; existence
    \item[eō] \marginnote{*}(adverb) for that reason, consequently, therefore; (with comparatives, often correlative to \textit{quo} or another relative) by that degree, so much, to such a degree; there, to that place
    \item[eō, īre, iī/īvī, itum] \marginnote{*}go, proceed, make one's way, move, pass
    \item[equidem] (with the first person singular or in replies about the speaker or writer) I for my part, I personally, I and no one else (often better left untranslated); (as an emphatic particle not with the first person singular) indeed, in truth
    \item[ergō] \marginnote{*}for that reason, therefore, then, so, accordingly; (in questions) in that case, then; well then, all right then; (marking a concession and objection) yes, but; ah, but
    \item[errō (1)] \marginnote{*}wander about, roam; float, flit, drift; be in doubt, be uncertain, waver; go astray, wander from the course, go wrong; think or act in error, be mistaken, be incorrect; go wrong morally
    \item[error, errōris, m.] \marginnote{*}a wandering, straying, strolling; uncertainty, doubt, perplexity; mistake, error; moral lapse
    \item[et] \marginnote{*}and; even, also; \textbf{et\dots et} both\dots and
    \item[etenim] and indeed, the fact is, for
    \item[etiam] \marginnote{*}still, yet, even now; yet again; also, in addition, as well, too; (indicating an affirmative answer to a question) indeed, yes
    \item[etsī] even if, although, though
    \item[ēvadō, ēvadere, ēvāsī, ēvāsum] come or go out, go forth; go up, mount up, ascend; get away, escape; turn out, fall out, end, become, result; pass over, go past, pass beyond
    \item[ēversiō, ēversiōnis, f.] the action of overturning or upsetting; expulsion, turning out; destruction
    \item[ēvertō, ēvertere, ēvertī, ēversum] overturn, turn upside down, reverse, throw back; throw down, cause to fall violently, bring down; ruin, overthrow, destroy, upset
    \item[ēvidēns, ēvidentis] perceptible; clear, obvious, plain, evident, apparent; open, unconcealed
    \item[ex] see \textit{ē}
    \item[excitō (1)] call out, summon forth, bring out; wake, rouse; excite, stir, agitate, rouse to vigorous physical or mental activity
    \item[exclūdō, exclūdere, exclūsī, exclūsum] shut out, keep out, exclude; cut off, remove, separate; debar, hinder, prevent; leave out, omit
    \item[excōgitō (1)] think out, contrive, devise, invent
    \item[exemplum, exemplī, n.] \marginnote{*}a sample, specimen; example, instance; precedent, pattern, model, parallel; a copy, reproduction, transcript
    \item[exeō, exīre, exiī/exīvī, exitum] \marginnote{*}go out, go forth; go away, depart, move away; withdraw, retire
    \item[existentia, existentiae, f.] see \textit{ex(s)istentia}
    \item[existō, existere, extitī, extitum] see \textit{ex(s)istō}
    \item[exiguus, exigua, exiguum] small (in size, amount, or quantity), scanty, little; short, brief; trivial, slight, petty
    \item[expectō (1)] see \textit{ex(s)pectō}
    \item[expergiscor, expergiscī, experrectus sum] wake up, become awake; rouse oneself
    \item[expīrō (1)] see \textit{ex(s)pīrō}
    \item[explicātiō, explicātiōnis, f.] (the act of) unwinding, straightening; explication, explanation; deployment, exercise
    \item[explicō, explicāre, explicāvī/explicuī, explicātum/explicitum] unfold, untwine, straighten; level out, smooth; disentangle, settle, solve; make clear, reveal to view, make known, give an account of (the first conjugation forms of the perfect are common in early Latin; the alternative forms replace them after Cicero)
    \item[explōrō (1)] reconnoitre, inspect; inquire into, investigate; ascertain (that); ensure (that); \textit{explōrātum habēre} know for certain, be sure
    \item[expōnō, expōnere, exposuī, expositum] put out, set out; expose (i.e., leave a child out to die); set on shore, disembark; leave exposed or unprotected, expose; set forth, relate, explain, exhibit
    \item[ex(s)istentia, ex(s)istentiae, f.] a being, living thing; substance, material; existence; reality (opposed to appearance)
    \item[ex(s)istō, ex(s)istere, ex(s)titī, ex(s)titum] appear, arise; come forward, present oneself; come into being, emerge, arise; prove to be, show oneself
    \item[ex(s)olvō, ex(s)olvere, ex(s)oluī, ex(s)olūtum] unfasten, undo, loose; open, solve; set free, release; put an end to, do away with; perform, discharge; pay; award
    \item[ex(s)pectō (1)] wait for, await; look forward to, hope for; expect; wait to see or know
    \item[ex(s)pīrō (1)] breathe out, exhale, emit; breathe one's last, die, expire, come to an end, perish;
    \item[extendō, extendere, extendī, extentum/extēnsum] make taut, stretch; stretch out, thrust out; lie full length; prolong, extend the duration of, continue; strain, exert oneself
    \item[extēnsiō, extēnsiōnis, f.] distance covered by stretching out, span, extent
    \item[externus, externa, externum] outward, external, situated on the outside; extraneous, not intrinsic; foreign, coming from abroad, alien
    \item[extrā] \marginnote{*}(adverb or preposition with the accusative) on the outside, externally; out, outward; beyond the boundaries of, outside, out of, beyond the scope of
    \item[facilis, facile] \marginnote{*}easy, not requiring great effort; straightforward, simple; involving no difficulties, tractable, light, tolerable
    \item[faciō, facere, fēcī, factum] \marginnote{*}make, build, construct; produce, bring forth, cause to grow; write, compose, form; appoint, institute, create; do, cause, bring about
    \item[facultās, facultātis, f.] ability, power, capacity, skill; strength, support; a particular skill, faculty, talent, endowment; power, potency, property (as an ability, not a piece of land); possibility, opportunity, chance
    \item[fallō, fallere, fefellī, falsum] \marginnote{*}deceive, trick, mislead; fail to support, prove treacherous; belie, dissapoint, fail to come up to, fail to execute; escape the notice of, be unperceived by; go unnoticed; avoid, elude, escape
    \item[falsitās, falsitātis, f.] falsity, the quality of being false; a falsehood, a false statement, belief, or similar (non-classical)
    \item[familiāritās, familiāritātis, f.] close friendship, intimacy, familiarity; close relationship, kinship (of family or friends); the state of being well-known, familiarity
    \item[fateor, fatērī, fassus sum] \marginnote{*}concede, acknowledge, admit; admit guilt, confess; profess, declare, avow; assent, say yes, agree to
    \item[fatīgō (1)] tire out, weary, exhaust; work at persistently, keep at (something or someone); press hard, harass; assail, worry, plague; expend, use up, exhaust
    \item[favus, favī, m.] a honeycomb; hexagonal paving stone
    \item[fenestra, fenestrae, f.] window; any hole, opening, aperture
    \item[ferē] (adverb) approximately, roughly; almost, nearly, about; 
    \item[feriō, ferīre, ———, ———] strike, hit; flot, beat; knock
    \item[ferō, ferre, tulī, lātum] \marginnote{*}bring, carry, take, convey, lead; lift up, raise; bear, endure, sustain; allege, claim; relate, tell, report
    \item[ferveō/fervō, fervēre/fervere, ferbuī/fervī, fervitum] be intensely hot, be boiling hot; boil; be warm, be hot; be inflamed, be feverish; be active or busy, move swiftly or in an agitated manner; be eager or enthusiastic
    \item[fictilis, fictile] made of clay, earthenware, terracotta
    \item[fictītius, fictītia, fictītium]  (in classical Latin, the stem is \textit{fictīc}-) artificial, not natural; unreal, made up
    \item[figmentum, figmentī, n.] an image, something made up; fiction, invention, unreality
    \item[fīgō, fīgere, fīxī, fīctum] fix, fasten, drive, thrust in; attach, affix; post, erect, set up
    \item[figūra, figūrae, f.] a form, shape, figure, outline; posture, pose, attitude
    \item[figūrō (1)] form, fashion, shape, mould; transform; arrange; depict, represent, make a likeness of; (sometimes with \textit{sibi}) form a mental image of, imagine
    \item[fingō, fingere, fīnxī, fictum] \marginnote{*}imagine, conceive, suppose, assume; make up, invent, fabricate (mentally); feign, simulate; form, fashion, create (physically)
    \item[fīō, fīerī, factus sum] \marginnote{*}(infinitive also \textit{fierī}) take place, occur, arise, happen, come about; be done, be made, be created, be produced, be prepared
    \item[firmus, firma, firmum] strong, durable; robust, sturdy; sound, strong, in good health; firm, steady, secure
    \item[flexibilis, flexibile] pliant, flexible, yielding; adaptable; tractable, pliable, open to influence or change
    \item[flōs, flōris, m.] \marginnote{*}flower, blossom, bloom
    \item[focus, focī, m.] a fireplace, hearth; (as the symbol of life in a home) one's fireside, one's home; sacrifical hearth or altar
    \item[fōns, fontis, m.] \marginnote{*}a spring, fountain; well, source
    \item[for, fārī, fātus sum] \marginnote{*}speak, talk; say, tell; tell of, reveal (something)
    \item[forma, formae, f.] \marginnote{*}a form, contour, figure, shape; appearance, looks; beauty, good looks; a geometrical figure
    \item[formō (1)] shape, fashion; form, build; change the appearance of, transform; adapt, modify
    \item[forsan] \marginnote{*}perhaps, possibly, maybe, it may be
    \item[fortasse] \marginnote{*}perhaps, possibly, maybe, it may be
    \item[fortassis] perhaps, possibly, maybe, it may be
    \item[forte] by chance, by accident, accidentally, fortuitously, at random; maybe, perhaps, conceivably; (introducing an example) say
    \item[fortis, forte] \marginnote{*}strong, hardy, vigorous, tough; healthy, in good physical health, robust; brave, bold, resolute; (of arguments or ideas) convincing, strong, forceful; (colloquial of people and conduct) honorable, decent, worthy
    \item[frāgrantia, frāgrantiae, f.] odor, scent, a smell; fragrance, sweet smell
    \item[frequenter] in large numbers, in crowds; densely, thickly; on many occasions or at frequent intervals, repeatedly, often, frequently; commonly, generally, widely
    \item[frigidus, frigida, frigidum] cold, chilly, cool; sluggish, torpid
    \item[frōns, frontis, f.] forehead, brow, front; front, façade, forepart (of anything); outside, exterior; external quality, appearance; \textit{prīmā fronte} at first appearance, at first glance
    \item[fruor, fruī, frūctus sum] \marginnote{*}enjoy the produce of or proceeds from, derive advantage from; avail oneself of, enjoy; be blessed with, enjoy; delight in, find enjoyable, enjoy
    \item[frustrā] \marginnote{*}to no purpose, in vain, to no avail; without reason or purpose, mistakenly, needlessly
    \item[fundāmentum, fundāmentī, n.] a foundation, substructure for a building; a basis, foundation; a fundamental necessity, something indispensible, a requirement
    \item[funditus] (\textit{adverb}) from the very bottom, from the foundations, by the roots; to the ground (in context of destruction); utterly, completely, without exception
    \item[fūsē] (in space) widely, extensively; widely in scope, at length, fully; without rigid distinction, loosely, roughly
    \item[gaudeō, gaudēre, gāvīsus sum] \marginnote{*}rejoice, be glad, be joyful
    \item[generālis, generāle] of or belonging to a kind, shared by a class, common to a group or type; of universal application, general
    \item[genius, geniī, m.] guardian spirit of a person or family; spirit, inclination, genius, inner nature
    \item[genus, generis, n.] \marginnote{*}stock, descent, origin; family line; nationality, race, nation; a generation, age; a class, type, kind, variety, group; gender
    \item[geōmetra/geōmetrēs, geōmetrae, m.] a geometer, geometrician
    \item[geōmetria, geōmetriae, f.] geometry
    \item[gurges, gurgitis, m.] a swirling mass of water, whirlpool, eddy; waters of a river, sea, etc., a flood
    \item[gustus, gustūs, m.] the action of tasting; the sense of taste; a flavor, taste; a small portion (of food or drink), a taste; a portion, a specimen, sample
    \item[habēna, habēnae, f.] a halter, bridle, rein
    \item[habeō, habēre, habuī, habitum] \marginnote{*}have, own, possess; hold, have in hand, have under one's control; have on, wear, carry; conduct, hold; regard, consider, look on, treat as, hold (mentally)
    \item[hāctenus] to this point in space, so far; to this point in time, up until now; to this extent or degree, so far
    \item[haereō, haerēre, haesī, haesum] cling, adhere, stick, stick fast; be fixed, hold on tightly; be contiguous in space, join on (to); be inherent (in) or connected (with); remain in place, stay put, linger; persist, continue; be at a loss, be in difficulties, be stuck
    \item[hālitus, hālitūs, m.] exhalation, vapor; breath, the air that one breathes in
    \item[haud] \marginnote{*}not, not at all, by no means
    \item[hauriō, haurīre, hausī/hauriī, haustum/haurītum] draw up, draw out, draw, scoop up; derive, draw from; wound (in such a way as to draw blood); drink, imbibe, drain; swallow; take in, absorb; consume, devour; swallow up, engulf; use up, consume
    \item[hesternus, hesterna, hesternum] of yesterday, yesterday's
    \item[hīc] \marginnote{*}in this place, here; in the present case or circumstances, in this case, in this situation
    \item[hic, haec, hoc] \marginnote{*}this, these (of something near in place, time, or uppermost in thought); the last, the recent; the latter (as opposed to \textit{ille} meaning `the former'); the following (referring to something about to be said)
    \item[hodiē] \marginnote{*}today
    \item[homō, hominis, m.] \marginnote{*}a human being, person
    \item[hūmānus, hūmāna, hūmanum] \marginnote{*}of or belonging to a person or people in general, human; civilized, humane (as opposed to nature or wild animals); cultured, cultivated; kindly, considerate, morally worthy of humanity; merciful, indulgent
    \item[hyemālis, hyemāle] (classical \textit{hiemāl}-) of or belonging to winter, winter-; wintry, stormy
    \item[iaceō, iacēre, iacuī, iacitum] \marginnote{*}lie, lie down; rest, recline; lie dead, be killed, die
    \item[iam] \marginnote{*}already now, just now; soon now
    \item[idcircō] on that account, for that reason, therefore
    \item[idea, ideae, f.] form, idea, archetype (as a translation of Plato's technical term \textgreek{ἰδέα}); an idea (generally); form, image, likeness
    \item[īdem, eadem, idem] \marginnote{*}the same, identical (as something previously mentioned or under discussion)
    \item[ideō] \marginnote{*}for that reason, therefore
    \item[igitur] \marginnote{*}(almost always postpositive) in that case, then; consequently, therefore, then, so; accordingly, so then
    \item[ignis, ignis, m.] \marginnote{*}(ablative singular is normally \textit{igni} but sometimes (especially later) \textit{igne}) fire; (metaphorically) fire or glow of passion; lightning, a lightning flash; fever, high temperature
    \item[ignoscō, ignoscere, ignōvī, ignōtum] (with accusative of offence and dative of offender) to pardon, forgive, excuse; overlook, allow, indulge, make allowance
    \item[īlicō] (also \textit{illicō}) on the spot, just here (there); at that moment, at once, there and then
    \item[ille, illa, illud] \marginnote{*}that, those; the famous, the well-known (or infamous); the former (as opposed to \textit{hic} as the latter)
    \item[illūsiō, illūsiōnis, f.] a mocking, the action of ridicule; saying the opposite of what is meant, irony; (post-classical) a deception of any kind, an illusion
    \item[illūstrō (1)] light up, make light, illuminate; make clear to the mind, clear up, elucidate, illustrate, explain; make famous, renowned, illustrious
    \item[imāginārius, imāgināria, imāginārium] unreal, imaginary; fictitious, pretend
    \item[imāginātiō, imāginātiōnis, f.] (the faculty of) imagination; (an act of) imagination; a mental image, something imagined
    \item[imāginātrix, imāginātricis] of or related to imagination, productive of imagination, causing imagination
    \item[imāginor (1)] form a mental picture of, imagine; give or produce an image
    \item[imāgō, imāginis, f.] (mental) image, idea, conception, thought, representation in one's mind; a representation, picture, likeness, image; an ancestral death mask used in Roman funerals; a reflection; an echo; illusory apparition, ghost, phantom, hallucination; description, sketch; semblance, imitation; duplicate, copy, replica; model, example
    \item[immittō, immittere, immīsī, immissum] cause to go, send (to a place); send (against or into); set (against); cast, direct; put in, insert, introduce; grant entry, let in, admit
    \item[immō] \marginnote{*}(used to introduce a correction of something said or implied previously) rather, more correctly, more precisely, on the contrary, no indeed
    \item[immōbilis, immōbile] immovable, fixed; unmoving, motionless; unalterable, fixed, unchanging; slow to move; imperturbable, steadfast, emotionally steady
    \item[immortālitās, immortālitātis, f.] immortality
    \item[imō] see \textit{immō}
    \item[imperfectiō, imperfectiōnis, f.] imperfection, flaw
    \item[imperfectus, imperfecta, imperfectum] unfinished, not completed; imperfect
    \item[imprōvidē] without forethought, unwarily; thoughtlessly, without thought for the future
    \item[imprōvīsus, imprōvīsa, imprōvīsum] unforeseen, unexpected; \textit{ex imprōvīsō} (adverbially) unexpectedly; suddenly
    \item[imprūdēns, imprūdentis] having no knowledge of something, ignorant; unaware of what one is doing, unwitting; unintentional; unaware of what will happen, not foreseeing; foolish, incautious, lacking in judgment or discretion
    \item[impōnō, impōnere, imposuī, impositum] \marginnote{*}place, put, or set on or; impose, force, inflict; assign, confer; post, station
    \item[in] \marginnote{*}(with accusative) into, onto; against; (with ablative) in, on
    \item[incertus, incerta, incertum] not clearly ascertained, doubtful; that cannot be known beforehand, subject to chance, unpredictable, uncertain; not specified or defined; not dependable, unsure, unsafe; (in active sense) having no certain knowledge, uninformed; not certain what to do, hesitant, irresolute
    \item[inchohō (1)] see \textit{incohō}
    \item[incidō, incidere, incidī, incāsum] \marginnote{*}fall on, onto, or into; fall, settle, impinge (on); fall over, stumble against; throw onself, rush (upon); fall into the possession or power of; happen on, chance to meet; fall into, happen into (a state or situation), enter inadvertently into, slip into; arise, occur, happen
    \item[incipiō, incipere, incēpī, inceptum] \marginnote{*}take in hand, begin, start, embark on; originate, take rise
    \item[incohō (1)] begin, commence, start, initiate; establish, found
    \item[inconcussus, inconcussa, inconcussum] unshaken, firm, steady; unbroken, secure, untroubled; steadfast, unwavering
    \item[incorruptibilis, incorruptibile] incorruptible; imperishable, indestructible
    \item[incumbō, incumbere, incubuī, incubitum] be intent on, apply oneself (vigorously) to, engage (eagerly) in \textit{plus dative}; settle on, sit on, lie on; press on, lean on; (of troubles, evils, etc.) befall, affect, afflict; (of emotion) come over; (of a task, duty, responsibility, etc.) to fall on, be incumbent on
    \item[incurrō, incurrere, incucurrī, incursum] rush or charge (at), make an attack (on); run or rush in; encounter, meet with, meet, run into
    \item[incēdō, incēdere, incessi, incessum] advance, march, proceed; stride, strut; advance, extend; go into or onto, enter; arise, come on, befall
    \item[incōnsīderantia, incōnsīderantiae, f.] lack of reflection or forethought, recklessness; inattentiveness, absent-mindedness
    \item[indicium, indiciī, n.] information, disclosure of information;evidence, an indication, token, symbol; omen, portent, warning
    \item[indīvīsibilis, indīvisibile] indivisible, inseparable
    \item[indubitātus, indubitāta, indubitātum] that cannot be doubted, certain, unquestionable; not hesitated over, confident
    \item[indulgeō, indulgēre, indulsī, indultum] be indulgent or lenient (to); allow (someone) to have their way; make allowance for; look favorably on, show kindness to; indulge, give free rein to; take pleasure in, indulge in, devote oneself to (plus dative)
    \item[industria, industriae, f.] diligence, purposeful activity, industry, zeal; an example of diligence, a purposeful activity
    \item[industrius, industria, industrium] active, dilgent, zealous, assiduous
    \item[induō, induere, induī, indūtum] put on; assume, adopt; clothe, dress in (often with ablative)
    \item[ineptiō, ineptīre, ineptīvī, ———] be foolish, be silly
    \item[ineptus, inepta, ineptum] without sense of what is fitting, lacking in judgment; foolish, silly
    \item[inextrīcābilis, inextrīcābile] impossible to disentangle or sort out; pathless, inescapable; insoluble
    \item[ineō, inīre, iniī/inīvī, initum] go into, enter; enter upon, commence, begin
    \item[īnfirmō (1)] weaken (physically or mentally); lessen destroy; refute, deny; annul, invalidate
    \item[īnfundō, infundere, infūdī, infūsum] pour in or into, pour on or over; (with ablative) fill, moisten, wet; pour down, shower; cause to extend, impart; stretch out or relax
    \item[īnfīgō, infīgere, infīxī, infīxum] implant, drive in; fasten, affix, attach; set firmly in place, plant, implant, impress; fix in the mind or memory, impress, implant
    \item[īnfīnītus, infīnīta, infīnītum] not limited, infinite, endless, boundless, unlimited; not specified, indefinite; unrestricted, absolute
    \item[ingenium, ingeniī, n.] natural disposition, character; mental ability, intelligence, intellect; (by metonymy) an intelligent person or person of great ability; a clever scheme, trick, ruse, snare, trap; contrivance, device, machine
    \item[ingēns, ingentis] huge, vast, very large (in size, number, or extent); very great (in degree or intensity); notable, momentous, of great importance
    \item[ingredior, ingredī, ingressus sum] \marginnote{*}go into or onto; enter upon, commence, begin; walk, proceed (on foot); advance, assail, attack
    \item[initium, initiī, n.] \marginnote{*}beginning, start, commencement
    \item[innumerābilis, innumerābile] countless, that cannot be counted, limitless
    \item[innōtēscō, innōtēscere, innōtuī, ———] become known or familiar; become famous or celebrated
    \item[inquam] \marginnote{*}say (usually introducing a direct quotation)
    \item[īnsaniō (4)] be out of one's mind, be mad, be insane; behave like someone insane, rave, act crazily
    \item[īnsidiae, īnsidiārum, f.] (plural in Latin with a singular meaning in English) an ambush, a surprise attack; treachery, a plot; a snare, a trap
    \item[īnsomnium, īnsomniī, n.] wakefulness, sleeplessness (usually plural); a vision, a dream, something seen in a dream or a trance
    \item[īnspectiō, īnspectiōnis, f.] the action of looking (at or into); a visual examination, inspection; theoretical examination, inquiry, investigation
    \item[īnstar, ———, n.] (found only in nominative and accusative singular) an equivalent in measure, appearance, effect, condition, etc., an equal, a likeness;
    \item[īnstituō, īnstituere, īnstituī, īnstitūtum] set in being, organize, put into operation; put up, erect; establish, set up; appoint; institute, originate, establish; train, instruct
    \item[īnsuper] \textit{adverb and preposition with accusative and ablative} on the top of, upon, over; (with accusative) on to the top of, upon, over (implying the breaking of a boundary); (with ablative) above, over (without the breaking of a boundary); in addition to
    \item[integer, integra, integrum] \marginnote{*}untouched, untried, fresh; undecided; open-minded, unprejudiced; whole, complete, (in an) undiminished (state); undamaged, whole, healthy, sound, unbroken, unscathed, unimpaired
    \item[intellectiō, intellectiōnis, f.] perception or recognition (by the mind), the act of understanding; faculty of understanding, the intellect; a concept, notion, idea
    \item[intellectuālis, intellectuāle] of or related to the understanding, intellectual; apprehended by the intellect; possessing intellect, intelligent
    \item[intellego] to come to know, see into, perceive, understand, discern, comprehend, gather
    \item[intelligō, intelligere, intellēxī, intellēctum] (present stem also \textit{intelleg}-) grasp mentally, understand, realize; understand by inference, deduce; understand the value of, appreciate; (intransitive) have or exercise powers of understand, possess intelligence
    \item[inter] \marginnote{*}(preposition with accusative) among, in the presence of, amid; among, along with; beetween; during, in the middle of, while busy with, amid
    \item[interdum] at times, from time to time, now and then; occasionally; in the meantime, for the time being
    \item[intereō, interīre, interīvī/interiī, interitum] go among, mingle with; perish, go to ruin, decay, die
    \item[interim] \marginnote{*}meanwhile, in the meantime; for the time being, for the present; all the while, at the same time; from time to time, occasionally, sometimes
    \item[interitus, interitūs, m.] death, destruction, perishing
    \item[intrā] \marginnote{*}(preposition with accusative) within, inside, privately, at home; in one's own country; \textit{intrā sē} to oneself, privately; (of time) within the space of, inside, before; on this side of, without passing beyond, short of; (adverb) inside, within; to the inside, inwards
    \item[intueor, intuērī, intuitus sum] look at, watch, gaze at; be a witness of, observe; (of things) be turned towards, face, have a view of; reflect upon, consider, contemplate
    \item[inveniō, invenīre, invēnī, inventum] \marginnote{*}encounter, meet, come across; find, find out, discover; devise, contrive, plan
    \item[investīgō (1)] track down, find by following a trail; search out; search out, track down, search after
    \item[invītus, invīta, invītum] not wishing, unwilling, reluctant
    \item[inūsitātus, inūsitāta, inūsitātum] unusual, uncommon, extraordinary, very rare; unfamiliar, strange
    \item[ipse, ipsa, ipsum] \marginnote{*}self, actual, very, exact (intensive not reflexive)
    \item[is, ea, id] \marginnote{*}(pronoun) he, she, it, they; (adjective) this or that (a weaker demonstrative pronoun than \textit{hic} or \textit{ille})
    \item[iste, ista, istud] \marginnote{*}that (of yours), that (often but not always with negative connotation)
    \item[ita] \marginnote{*}thus, so; in this way, in this manner, in such a manner
    \item[itaque] \marginnote{*}and so, and thus, and accordingly; therefore, for that reason, consequently
    \item[iūdicium, iūdiciī, n.] \marginnote{*}judicial investigation, trial, legal process; sentence; exercise of judgment, judging or deciding; a judgment, a decision; assessment, appraisal
    \item[iūdicō (1)] \marginnote{*}examine judicially, judge, be a judge, pass judgment; sentence, condemn; decide, evaluate, appraise, determine
    \item[iungō, iungere, iunxī, iunctus] \marginnote{*}join together, connect, attach, fasten, yoke, harness; (metaphorically) bring together, unite, join
    \item[iūs, iūris, n.] \marginnote{*}law, authority, right; a law, a rule; a legal procedure
    \item[iuvō, iuvāre, iūvī, iūtum] \marginnote{*}help, aid, assist; further, serve, support, benefit; strengthen, improve; (impersonal with an infinitive) profit, benefit
    \item[labefaciō, labefacere, labefēcī, labefactum] weaken, shake, loosen, undermine; overthrow, ruin, destroy
    \item[labōriōsus, labōriōsa, labōriōsum] involving much effort or work, hard, difficult, laborious; hard-working, industrious, diligent; involving hardship or suffering, painful, distressing
    \item[lateō, latēre, latuī] \marginnote{*}hide, be in hiding, go into hiding; take refuge, shelter; be out of sight, be invisible; be latent, lie below the surface, be concealed, lie hidden; escape notice, be overlooked, go unobserved
    \item[latus, lateris, n.] \marginnote{*}the side (of the upper body or chest), flank; a side; an extremity or edge
    \item[laxus, laxa, laxum] wide; loose, open; spacious, roomy
    \item[lector, lectōris, m.] a reader
    \item[lentus, lenta, lentum] pliant, flexible; tough, tenacious; sticky, viscous
    \item[levitās, levitātis, f.] lightness; lack of intensity, mildness; triviality, unreliability, fickleness, shallowness
    \item[līberē] without restriction, at will, freely; open, frankly, boldly; wantonly, shamelessly
    \item[līberō (1)] free, release, deliver, extricate; absolve or acquit (in a court); clear, pass, traverse, cross over
    \item[lībertās, lībertātis, f.] \marginnote{*}freedom, independence; opportunity; frankness of speech, outspokenness; lack of restraint, impertinence, licence
    \item[licet, licēre, licuit/licitum est] \marginnote{*}it is permitted, one may
    \item[līmēs, līmitis, m.] a boundary, limit; a piece of land within a boundary; lane, path, track, road; course, route
    \item[liquēscō, liquēscere, licuī, ———] become liquid, melt, liquefy; flow or melt away, dissipate
    \item[liquidus, liquida, liquidum] flowing, fluid, liquid; limpid, clear, unclouded
    \item[locus, locī, m. (pl. both locī, m. and loca, n.] \marginnote{*}a place, spot, neighborhood, locale, region
    \item[longus, longa, longum] \marginnote{*}long (in length or time); far
    \item[loquor, loquī, locūtus sum] \marginnote{*}speak, talk, say; tell, mention
    \item[lūceō, lūcēre, lūxī, ———] be light or clear; to shine, beam, emit light; be bright, glitter; be evident, be clearly known or felt; dawn, become light
    \item[lūdificātiō, lūdificātiōnis, f.] jeering, derision, mocking, mockery, mocking game
    \item[lūx, lūcis, f.] \marginnote{*}light; daylight, light of day; brightness
    \item[māchina, māchinae, f.] machine; device, plan, contrivance
    \item[magis] \marginnote{*}more, to a greater extent
    \item[magnitūdō, magnitūdinis, f.] \marginnote{*}greatness, size, bulk, magnitude
    \item[magnus, magna, magnum] \marginnote{*}great in size or extent, big, vast; great in number or amount, much, large; (of age) old, aged; notable, famous, of great consequence or importance
    \item[malignus, maligna, malignum] ungenerous, mean, grudging; scanty, pour; ill-disposed, spiteful, unkind; harmful
    \item[mālō, mālle, māluī, ———] \marginnote{*}prefer, want more
    \item[malus, mala, malum] \marginnote{*}bad; unpleasant, nasty; wicked, evil; harmful, painful, distressing
    \item[maneō, manēre, mānsī, mānsum] \marginnote{*}stay, remain (in one place); have patience, wait; wait for; be in store for, await
    \item[manifestus, manifesta, manifestum] clear, plain, apparent; evident, manifest; exposed, caught in the act, flagrant
    \item[manus, manūs, f.] \marginnote{*}hand (of people), paw (of animals); a band, a troop, company, faction; (by metonymy) handwork, workmanship
    \item[māteria, māteriae, f.]  stuff, matter, materials (from which anything is composed); the wood of a tree, timber; matter, subject, topic, theme (of a science, book, speech, etc.)
    \item[māteriālis, māteriāle]  material, of or pertaining to matter
    \item[mātūrus, mātūra, mātūrum] ripe, full-grown, adult (of plants, fruits, or animals); experienced, mature; advanced in age, aged, old; occurring at the proper time, timely
    \item[medicīna, medicīnae, f.] the art or practice of healing, medicine, the knowledge or science of medicine; a treatment, a cure, a remedy
    \item[meditātiō, meditātiōnis, f.] \marginnote{*}reflection, contemplation, pondering; the subject of reflection or thought, an idea, a thought; planning, devising; practicing, rehearsal; exercise, practice
    \item[meditor (1)] contemplate, ponder, reflect; have in mind, intend; devise, plan, think out; go over, say to oneself, work over, practice
    \item[mel, mellis, n.] honey; something sweet, pleasant, or agreeable
    \item[membrum, membrī, n.] \marginnote{*}a limb, part of the body, member; (pl.) limbs, the body (as a whole); one of the main divisions or component parts of something, a part; a clause, part of a larger unit of speech or writing
    \item[memoria, memoriae, f.] \marginnote{*}the power or faculty of remembering, memory; a memory, a recollection; repute, what is remembered about someone or something
    \item[mendax, mendācis] prone to lie, untruthful; lying, false, deceiving
    \item[mēns, mentis, f.] \marginnote{*}mind; disposition, character; purpose, design, intention; significance, meaning; frame of mind, attitude
    \item[mentior (4)] lie, tell a falsehood; make a false promise, promise falsely; give a false account, misrepresent; invent, fabricate; feign, simulate
    \item[meus, mea, meum] \marginnote{*}of me, my, mine, belonging to me, my own
    \item[minimus, minima, minimum] \marginnote{*}smallest (in size, extent, duration, etc.); least; very small, very little
    \item[minus] \marginnote{*}less, to a smaller extent or degree
    \item[minūtus, minūta, minūtum] small (in length, size, duration, etc.), tiny, meager, brief; fine, consisting of fine particles
    \item[mīror (1)] \marginnote{*}be surprised, amazed, bewildered; marvel at; hold in awe, admire, revere
    \item[mīrus, mīra, mīrum] extraordinary, remarkable, astonishing, amazing
    \item[modus, modī, m.] \marginnote{*}a measure, extent, quantity; way, manner
    \item[moneō, monēre, monuī, monitum] remind, bring to mind, bring to one's recollection; admonish, advise, warn; instruct, teach; tell, inform, point out
    \item[mortālis, mortāle] mortal, subject to death or destruction; (substantive) human 
    \item[moveō, movēre, mōvī, mōtum] \marginnote{*}(make) move, impel, set in motion; shake, agitate; wield, ply, exercise; shift, move; oust, expel, dislodge; stir, provoke, rouse
    \item[multus, multa, multum] \marginnote{*}numerous, many; (with partitive genitive) many (of); (with a singular noun) an abundance of, much
    \item[mundus, mundī, m.] \marginnote{*}the heavens, the sky; the universe; the world, the earth
    \item[mūtābilis, mūtābile] liable to change, changeable, fluctuating, uncertain; that can be changed, alterable, mutable; (of people) fickle, unsteadfast
    \item[mūtātiō, mūtātiōnis, f.] a changing, altering, a change, alteration, mutation; give and take, exchange; substitution, replacement; alternation interchange; conversion, translation
    \item[mūtō (1)] \marginnote{*}give and receive, exchange; substitute (for), take or put (one thing) in place (of another); replace, change (one thing by another); change (in some way), make different, modify, alter
    \item[mūtuō] mutually, reciprocally; in return; jointly, alike;
    \item[nam] \marginnote{*}(affirmative or in assent) certainly, to be sure; (explanatory) for, because (generally explaining something that was just said or implied); (introducing an example or illustration) for instance
    \item[nātiō, nātiōnis, f.] birth of a child; a people, race, nation; nationality
    \item[nātūra, nātūrae, f.] \marginnote{*}nature (very broadly understood: of a person, of an object, of the universe as a whole); character, temperment, innate ability
    \item[-ne] \marginnote{*}(introduces a direct `yes' or `no' question)
    \item[necdum] (also \textit{nequedum}; sometimes written as two words) and not yet; but not yet; \textit{necdum\dots necdum} not yet either\dots or; (adverb) not yet
    \item[necessāriō] necessarily; unavoidably, inevitably
    \item[necessārius, necessāria, necessārium] essential, necessary requisite, needed; compelling, exercising compulsion, unavoidable
    \item[necesse] \marginnote{*}indispensable, essential, necessary; inevitable, determined by natural law; necessarily true; forced, compulsory
    \item[necne] (in direct questions) Or not?; (in indirect questions) or not
    \item[negō (1)] \marginnote{*}say (that\dots not); deny (that), deny (something), deny (the existence of); say no, refuse, decline, withold (something); forbid
    \item[nēmō, nēminis m./f.] \marginnote{*}nobody,  no one; (following a negative, usually an emphatic positive assertion) everyone, everybody; a person of no importance, a nobody
    \item[nempe] certainly, without doubt, assuredly, of course, as everybody knows; (in answers to questions or picking up another speaker's words) Why, clearly; (conceding something) admittedly, it is true
    \item[nec/neque] \marginnote{*}and not, nor, and\dots not (for purposes of translation, you will often want to put the \textit{not} much later in the clause or sentence than the \textit{and}; \textbf{nec\dots nec} neither\dots nor
    \item[nequeō, nequīre, nequiī/nequīvī, nequitum] be unable
    \item[nequidem] not even (in classical Latin, this would be \textit{ne X quidem}, but Descartes somestimes writes \textit{nequidem X})
    \item[nesciō (4)] \marginnote{*}not to know, not know, be ignorant, be unaware of; \textit{nescio an\dots} I am inclined to think that, perhaps, probably
    \item[nihil/nīl] \marginnote{*}nothing, not anything (indeclinable)
    \item[nihildum] (often written as two words) nothing so far
    \item[nihilum/nīlum, nihilī/nīlī, n.] nothing
    \item[nihilōminus/nīlōminus] none the less, notwithstanding just the same; likewise, as well
    \item[nihilum/nīlum, nihilī/nīlī, n.] nothing, not anything; \textit{dē nihilō} without reason, for nothing; \textit{ad nihilum venīre} come to nothing, have no effect; (in expressions as genitive of value) \textit{nihilī pendere}, \textit{nihilī facere}, \textit{nihilī putare} (or similar) set no value on, care nothing about
    \item[nisi] \marginnote{*}if\dots not, unless, except if
    \item[niteō, nitēre, nituī, ———] be radiant, shine; sparkle, glitter, be bright with reflected light; have a healthy look (of animals, land, people); be resplendent, be well decked out (of a home or building)
    \item[nītor, nītī, nīxus sum] \marginnote{*}rest one's weight, lean (on), support oneself, lean, incline; put one's faith (in), rely (on); be based (on), rest or depend (on); strive, exert oneself, direct efforts towards a goal or purpose
    \item[nocturnus, nocturna, nocturnum] of or belonging to the night; operating by night, appropriate to night, nocturnal
    \item[nōlō, nōlle, nōluī] \marginnote{*}wish\dots not, want\dots not, not want, be unwilling; refuse, decline
    \item[nōmen, nōminis, n.] \marginnote{*}name; designation, title; term, expression; a noun; fame, reputation
    \item[nōn] \marginnote{*}not
    \item[nōndum] \marginnote{*}not yet
    \item[nōnne] \marginnote{*}(introduces a yes or no question expecting the answer yes); Surely\dots ?; Isn't it the case that\dots ?
    \item[nōnnihil/nōnnīl/nōnnihilum] (with a partitive genitive) a certain amount, some, a number of; (alone) something; not a little, not a few; (adverb) to a certain extend, in some measure
    \item[nōnnisi] (often written as two words) not unless, not except, only
    \item[nōnnūllus, nōnnūlla, nōnnūllum] (often written as two words) a certain amount, not a little; some, several, a number of
    \item[noscō, noscere, nōvī, nōtum] \marginnote{*}get knowledge of, become acquainted with, come to know, learn, discern; (in perfect tenses) know (i.e. `I have learned' is equivalent to `I know' and `I had learned' is equivalent to `I knew'); examine, study, inspect; familiarize oneself with, make the acquaintance of
    \item[noster, nostra, nostrum] \marginnote{*}our, our own, ours, of us
    \item[nōtitia, nōtitiae, f.] acquaintance, practical knowledge, familiarity; knowledge, understanding, awareness, cognizance; fame, celebrity; (also in a bad sense) notoriety, infamy
    \item[notō (1)] mark, to designate with a mark, brand, stamp; censure, stigmatize (since Roman censors would put a mark beside a dishonored person's name); pick out (among a group), distinguish; single out, designate; indicate (by a sign); point out, represent, denote, indicate
    \item[nōtus, nōta, nōtum] \marginnote{*}known, familiar; widely or generally known, noted
    \item[novus, nova, nobum] \marginnote{*}new; strange, unfamiliar; surprising, unforeseen; strange, alien, subversive, seditious; \textit{rēs novae}, \textit{motus novī} revolution
    \item[nox, noctis, f.] \marginnote{*}night, evening
    \item[nūdus, nūda, nūdum] \marginnote{*}naked, bare, unclothed; uncovered, exposed
    \item[nūllus, nūlla, nūllum] \marginnote{*}not any, none, no; (with a following negative, the meaning is a strong positive) \textit{nūllus nōn} every, any, all
    \item[numerō (1)] add up, count; reckon, compute; enumerate, catalogue, specify in a list
    \item[numerus, numerī, m.] \marginnote{*}sum, total; a number; a (large or small) quantity or amount
    \item[numquam] \marginnote{*}at no time, never
    \item[numquid]  (introducing a question where agreement is expected, often a rhetorical question) Surely\dots ? Isn't it the case that\dots ? (This meaning does not appear in classical Latin, but it is common in Descartes.); (introducing questions where a negative answer is expected) Is it really possible that\dots ? Surely\dots not?; (introducing rhetorical questions where the implied answer is no) Can it be said that\dots ?; (introducing an indirect question) whether, if;
    \item[nunc] \marginnote{*}now, at this time, at present; now, under these circumstances, in view of this; \textit{nunc\dots nunc} at one moment\dots at another moment, in some cases\dots in other cases
    \item[nunquam] see \textit{numquam}
    \item[nunquid] see \textit{numquid}
    \item[nūper, nūpera, nūperum] not long ago, recently, just now; (in reference to past time) not long before
    \item[nūtriō (4)] suckle, feed at the breast; support with food, nourish, feed; foster, bring up, rear; support, build up, increase, encourage, look after, care for
    \item[obdormiō (4)] fall asleep
    \item[obfirmō (1)] (often \textit{off}-) make firm against attack, secure; make firm, make inflexible; make up one's mind about or persist in
    \item[obiectiō, obiectiōnis, f.] reproach, criticism; objection
    \item[obiectīvus, obiectīva, obiectīvum] objective
    \item[oblīviscor, oblīviscī, oblītus sum] forget, forget about, forget how
    \item[obscūritās, obscūritātis, f.] darkness, obscurity; lack of clarity, unintelligibility; the condition of being unknown, obscurity
    \item[obstinātus, obstināta, obstinātum] stubborn, obstinate, resolute; hardened (in vice or crime)
    \item[obstupēscō, obstupēscere, obstupuī, ———] be struck dumb, be stunned, dazed, or astounded (with any powerful emotion)
    \item[occāsiō, occāsiōnis. f] an opportunity, chance; contingency, accident
    \item[occupō (1)] \marginnote{*}take possession of, occupy; seize control of; invade, seize, cover; seize hold of, take hold of; forestall, catch first, take by surprise; be the first to act, anticipate others, get in first
    \item[occurrō, occurrere, occurrī, occursum] \marginnote{*}run to meet, hurry to meet; arrive, turn up; meet or confront (in a hostile manner), go to oppose; counteract, check; meet accidentally, run into , come upon, happen upon, become involved; (of an idea or situation) present itself, occur (to a person); occur, crop up, confront one
    \item[oculus, oculī, m.] \marginnote{*}the eye, an eye; (usually plural) operation of the eyes, sight; (usually plural) gaze, look, regard
    \item[odor, odōris, m.] smell, scent, odor; pleasant smell, fragrance; unpleasant smell, stink, stench; a whiff, hint, suggestion
    \item[odorātus, odorātūs, m.] action or sense of smelling, smell
    \item[ōlim] \marginnote{*}at that time, some time ago; once upon a time, once; formerly, of old
    \item[omnīnō] in every respect, entirely, absolutely, altogether; in all, all told, altogether; (with negatives) at all, in any way or circumstance; in general, as a whole
    \item[omnis, omne] \marginnote{*}all, the entire amount, the whole of; every one, each, every
    \item[opīniō, opīniōnis, f.] (an) opinion, belief, view; expectation; imagination (as a faculty or a mental picture); reputation
    \item[opīnor (1)] think, believe, suppose; have in mind, imagine, conceive; look for, expect; express an opinion, opine
    \item[opportūnus, opportūna, opportūnum] advantageous, convenient, favorable; timely, opportune; appropriate (to an occasion) seasonable; available when wanted, ready to hand; liable, exposed, susceptible (to something, generally something bad)
    \item[ops, opis, f.] \marginnote{*}power, ability, might; forces, troops; power, dominion, influence; aid, assistance, help; (usually plural) financial resources, wealth, property
    \item[opus, operis, n.] \marginnote{*}work, a task, an undertaking; occupation, employment; activity, working, effort, strenuous activity; an achievement, accomplishment, product of work or labor (e.g., a work of art, a finished building, a literary work, etc.)
    \item[ōrdō, ōrdinis, m.] a row, line, series, arrangement, order, sequence; a line or rank (of soldiers); a band, troop, company (of soldiers); an order (i.e., a rank or class of citizens); class, rank, station, condition
    \item[orīgō, orīginis, f.] a beginning, commencement; source, start; descent, lineage, birth, origin
    \item[orior, orīrī, ortus sum] arise, rise, get up; be born, originate; come forth, spring, descend
    \item[ostendō, ostendere, ostendī, ostentum/ostensum] \marginnote{*}hold out, present; show, display, reveal; exhibit (for sale, use, inspection, etc.); point out, mention; make clear, show, evince, demonstrate, disclose, make known
    \item[ōtium, ōtiī, n.] \marginnote{*}unoccupied or spare time, leisure time; leisure, freedom from work, rest, relaxation, ease; peace, tranquility, calm; inactivity, idleness; temporary cessation, respite, lull
    \item[pactum, pactī, n.] an agreement, compact; means, manner, method, grounds, consideration
    \item[pactus, pacta, pactum] agreed, settled, determined; contracted, stipulated
    \item[pars, partis, f.] \marginnote{*}a portion, part, piece, share
    \item[particulāris, particulāre] particular; partial, of or concerning a part
    \item[partim] in part, partly, to a certain degree or extent
    \item[parum] (as a noun) an insufficient amount, too little, not enough; as an adverb) insufficiently, too little, not enough
    \item[parvus, parva, parvum] \marginnote{*}small (in size, extent, duration, amount, quantity, etc.); insignificant, unimportant
    \item[patior, patī, passus sum] \marginnote{*}to be subjected to, undergo, experience; put up with, tolerate; endure, make do; allow, permit; admit of, allow, grant, accept
    \item[paucus, pauca, paucum] \marginnote{*}few, little
    \item[paulus, paula, paulum] \marginnote{*}little, small (in size or quantity)
    \item[pauper, pauperis] \marginnote{*}poor, not wealthy; of little worth, cheap; meager, poor, unproductive
    \item[peccatum, peccatī, n.] sin, transgression; wrong, injustice; (in a medical context) illness
    \item[pendeō, pendēre, pependī] hang, be suspended; hang down, hang onto; overhang; be suspended, be left incomplete or hanging; depend, hinge, be based (on), result (from)
    \item[pēnsitō (1)] weigh (in a set of scales); pay, weight out (in payment); weigh in one's mind, ponder, consider; (with \textit{cum}) weigh against, compare (with)
    \item[per] \marginnote{*}(preposition with accusative) through, across; along, over, along with; through the length of, along (some part of); all over, throughout; in the course of, during (a time); through in succession; as far as (some person or authority) is concerned; as a result of, by reason of, through
    \item[perceptiō, perceptiōnis, f.] the action of taking or taking possession; gathering; the right to gather; mental grasp, perception
    \item[percipiō, percipere, percēpī, perceptum] take, take possession of; harvest, earn, reap, acquire; catch hold of; perceive, apprhend, notice, take in or grasp with the mind
    \item[percurrō, percurrere, per(cu)currī, percursum] run, move quickly over or through; travel quickly from end to end, pass through, visit in quick succession; run over, skim over; run over (in words or thought), survey, review; run through in sequence
    \item[pereō, perīre, perīvī/periī, peritum] pass away, come to nothing; vanish, disappear; perish, die, be destroyed; be lost, be wasted; be ruined, be undone
    \item[perfacile] very easily; very willingly
    \item[perfacilis, perfacile] very easy; very peaceful (i.e., very at ease)
    \item[perfectus, perfecta, perfectum] finished, complete; perfect, excellent; accomplished, exquisite
    \item[perficiō, perficere, perfēcī, perfectum] bring (an action or process) to its end, complete, finish; make perfect, bring to perfection; carry out, execute; bring about, achieve, effect; destroy, kill; use up, exhaust
    \item[pergō, pergere, perrexī, perrectum] \marginnote{*}proceed, move onward; go on, lead
    \item[perīculum, perīculī, n.] \marginnote{*}test, trial, proof; danger, risk; liability, responsibility for damage or loss
    \item[permisceō, permiscēre, permiscuī, permixtum] mix or blend thoroughly, mix together; bring into association, combine in a group; mix up, convound; involve, embroil; disturb, throw into confusion, confuse
    \item[permittō, permittere, permīsī, permissum] \marginnote{*}(literal meaning very rare) send through; let run free, relax, allow full scope to, give rein to; cede, relinquish, surrender; commit, entrust; leave (to another) to decide or do, refer; permit, allow, sanction
    \item[permultus, permulta, permultum] very much, very many, a great many
    \item[persecūtiō, persecūtiōnis, f.] (the action of) chasing, pursuing; a chase, pursuit; prosecution (of a legal case); persecution, attack, harassment
    \item[perspicuus, perspicua, perspicuum] transparent, pellucid; clearly visible, conspicuous; lucid, clear
    \item[persuādeō, persuādēre, persuāsī, persuāsum] \marginnote{*}persuade, prevail upon, convince
    \item[pertineō, pertinēre, pertinuī, ———] \marginnote{*}extend, reach, stretch; be aimed (at), be directed (towards); tend, be conducive (to); relate or pertain (to), have to do (with); be a concern (to), concern, be the business (of)
    \item[perveniō, pervenīre, pervēnī, perventum] \marginnote{*}come (to), get (to), arrive (at); get through (to the end or conclusion); extend or reach (to); pass into the hands of, become the property of
    \item[pēs, pedis, m.] \marginnote{*}foot (of a person); foot (as a measure of distance); foot (in poetry)
    \item[petō, petere, petīvī/petiī/petī, petitum] \marginnote{*}make for, go towards, direct one's course to; go for, go after, attack; seek and bring, fetch, procure; seek, aim at, strive for, strive after, pursue
    \item[physica, physicae, f.] natural science, the study of physical nature, physics (though in a much broader sense than ours)
    \item[physica, physicōrum, n. pl.] natural science, the study of physical nature, physics (though in a much broader sense than ours)
    \item[pictor, pictōris, m.] a painter
    \item[pīleus, pīleī, m.] (also with the stem \textit{pill}-) hat
    \item[pingō, pingere, pīnxī, pīctum] paint, tint, adorn with colors or colored designs; paint (a picture); paint or draw a picture of, depict; decorate, embellish
    \item[placidus, placida, placidum] kindly, indulgent; tame, quiet, friendly; peaceful, calm, free from stress
    \item[plācō (1)] conciliate, placate; reconcile (someone with someone else or something); calm, soothe, make calm
    \item[plānē] evenly; simply, plainly, clearly, distinctly; wholly, entirely, utterly, completely, thoroughly; by all means, assuredly
    \item[plānus, plāna, plānum] even, level, flat, plane; simple, plain; obvious, clear, manifest
    \item[platēa, platēae, f.] a street
    \item[plērusque, plēraque, plērumque] \marginnote{*}greater part of, greater number of, most of; (plural) a great number, very many; (plural substantive) most people, the majority
    \item[pondus, ponderis, n.] \marginnote{*}weight, heaviness; a heavy object, a mass; a burden or load (literal or figurative); importance, value, weight, influence
    \item[pōnō, pōnere, posuī, positum] \marginnote{*}place, set, put; station, post; put in position, set up, pitch (of a military camp); build, construct, found; put down, lay down
    \item[porrō] straight on, forward, onward, ahead; further off, beyond; hereafter, later; further, more (referring to continuing action); in turn; on top of that, next, besides; furthermore, again, moreover
    \item[positiō, positiōnis, f.] the action of placing; planting (of crops); layout; disposition, lie (of land); position, situation, mental position, attitude; condition, state
    \item[possum, posse, potuī, ———] \marginnote{*}be able (to), be capable (of), can; be possible
    \item[post] \marginnote{*}\textit{adverb and preposition with accusative} (adverb) behind, back; at a later time, afterwards; (with accusative) behind; beyond, over; after; second to, of less value or importance than, after 
    \item[posteā] \marginnote{*}subsequently, afterwards, thereafter; hereafter, in future, later; next (in time); next (in value, importance, rank, etc.)
    \item[posterus, postera, posterum] \marginnote{*}future, later; next, following; (plural substantive) descendants, posterity, future generations (one's own or in general)
    \item[postquam] \marginnote{*}after, when; ever since, from the time that, since
    \item[potēns, potentis] \marginnote{*}endowered, provided (with); having got possession (of); having or exercising power (over); capable, powerful, influential
    \item[potestās, potestātis, f.] \marginnote{*}power, (possession of) control or command; position of power, office, magistracy; jurisdiction, authority; opportunity, chance, right
    \item[potis, pote] \marginnote{*}able, capable; liable; possible
    \item[praecīdō, praecīdere, praecīdī, praecīsum] shorten, cut back, cut or break the front off; cut down or sever (a part from the main body of something or someone); cut short, bring to a sudden end, break off; deprive of
    \item[praecipuus, praecipua, praecipuum] taken before other things; particular, special; peculiar; principal, excellent, distinguished, foremost, chief
    \item[praeclārus, praeclāra, praeclārum] very clear, brilliant, bright, radiant; splendid, magnificent, grand; outstanding (in achievement or reputation), brilliant, glorious
    \item[praeiūdicium, praeiūdiciī, n.] previous legal judgment or ruling, prejudgment; a precedent (in legal or other contexts); a preconceived opinion, a previous judgment; a preconception, prejudice, presumption
    \item[praemissa, praemissae, f.] premise
    \item[praemittō, praemittere, praemīsī, praemissum] send forward, send in advance; set in front, set before; say or write by way of preface; place in front, prefix; give as a premise
    \item[praerequīrō, praerequīrere, praerequīsīvī/praerequīsiī, praerequīsītum] require in advance, have as a prerequisite
    \item[praesertim] especially, chiefly, principally, particularly
    \item[praeter] \marginnote{*}\textit{preposition with accusative} passing, past, across; beyond, surpassing, exceeding; to a greater degree than; out of line with, contrary to, at variance with; in addition to, as well as, besides; other than, with the exception of, except, but, save
    \item[praeterquam] beyond, besides, except, save, apart from
    \item[praetereā] \marginnote{*}in addition (to that), besides, as well; moreover, furthermore
    \item[prāvus, prāva, prāvum] crooked, not straight, awry; twisted, distorted, misshapen, deformed; corrupt, debased, evil; (in a weaker sense) wrong-headed, misguided; defective, faulty, wrong; incompetent, bad
    \item[prīmus, prīma, prīmum] \marginnote{*}the first, first, earliest; furthest in front, leading; furthest out, uttermost, extreme; primary, fundamental
    \item[prīncipium, prīncipiī, n.] \marginnote{*}action or fact of beginning, starting, founding; origin; first part (of anything), head, source; original position, starting point; beginning, opening; (guiding) principle, basis, premiss, starting point
    \item[prior, prius] \marginnote{*}former, previous, earlier, prior, preceding; in front, leading; immediately preceding, the last; in anticipation or advance of someone else, earlier, first
    \item[prō] \marginnote{*}\textit{preposition with ablative} before, in front of; on behalf of, in the interests of, as representitive of; in favor of, on the side of; in place of, instead of; in proportion to, according to; in relation to, considering, with regard to; in view of, having regard to, to judge from
    \item[probābilis, probābile] commendable, acceptable; plausible, credible; probable, likely
    \item[probātiō, probātiōnis, f.] (the act of) testing or proving; a trial, inspection, examination; approval, assent; a proof, evidence
    \item[probō (1)] \marginnote{*}approve, commend; think well of, esteem; give assent to, authorize, santion; examine, test, put to the test (in order to approve or not); win approval for; demonstrate, prove, show to be real or true
    \item[prōcūrō (1)] look after, attend to; administer, have charge of; expiate, avert by sacrifices; (intransitive) perform sacrifices (in order to avert something)
    \item[profectō] actually, indeed, really, truly, surely, assuredly, certainly, by all means, undoubtedly
    \item[prōferō, prōferre, prōtulī, prōlātum] bring forth, bring out; (reflexive or passive as intransitive) come forth, emerge; show, display; bring into existence, put forth; utter, pronounce; make known, publish, disclose; carry or move forwards (reflexive or passive as intransitive) advance, come forward; extend, prolong; postpone, defer
    \item[prōficiō, prōficere, prōfēcī, prōfectum] make headway, gain results, be successful; do good, help; advance, gain ground; increase (in size or extent), rise (of prices); progress, develop, improve
    \item[profundus, profunda, profundum] extnding a long way down, (very) deep, bottomless; deep, thick, profound; insatiable, having an immense capacity; secret, abstruse, mysterious, remote from general knowledge; absorbing, intense, profound
    \item[proinde] according (as), in proportion (as); in a corresponding manner or degree, accordingly; in the same way or degree (as); equally, similarly, likewise; so then, accordingly
    \item[prōnūntiō (1)] proclaim, announce, state publicly; pronounce (a decision, a verdict, etc.); affirm, declare; state as a fact, tell, relate; assert; utter, speak, express; recite, declaim
    \item[prōnus, prōna, prōnum] leaning or bending forward, tilted forward, bending down; lying prone, prone; sloping, having a downward incline; inclined (to, towards), disposed (to), liable (to)
    \item[prōpōnō, prōpōnere, prōposuī, prōpositum] set forth, set out, lay out; place before, display, expose to view; propose, plan, intend, design; offer, propose (as a reward)
    \item[prōpositiō, prōpositiōnis, f.] (the action of) setting forth, proposing, representing; a proposition, representation; a design, purpose, resolution, determination; principal subject, theme
    \item[proprius, propria, proprium] \marginnote{*}one's own, personal, private; peculiar (to someone, something), particular (to one person or thing), special, specific; characteristic, personal; lasting, permanent, continuous
    \item[propter] \marginnote{*}\textit{adverb and preposition with accusative} (adverb) near, close by, close at hand; (with accusative) near, close to; in view of, because of, as a result of, on account of, thanks to; for the sake of, out of consideration for
    \item[prōrsus] forward, straight ahead; without interruption, straight; right through to the end; thoroughly, in every respect, altogether, quite, absolutely; (usually after a connective) more than that, even, indeed; (as a sentence connective) indeed, in fact; all in all
    \item[prout] according as, in proportion as; in so far as, inasmuch as, to the extent that
    \item[prūdēns, prūdentis] well aware of what one does or of the consequences of one's action, acting deliberately, open-eyed; aware (of), having foreknowledge (of); knowing (that), well aware (that); exercising foresight, prudent, discreet; characterized by prudence or good sense; having good understanding, clever, having a good practical understanding or skill (in)
    \item[prūdentia, prūdentiae, f.] practical understanding or wisdom, shrewdness, good sense; proficiency (in some field), practical grasp; foreknowledge
    \item[pudeō (2)] (often impersonal in the \nth{3} person singular) fill with shame, make ashamed; (with a personal subject) feel shame, be ashamed
    \item[pulsō (1)] strike (with repeated blows), beat, assault, hit; knock; push, impel, drive; send on one's way, send away, dispel
    \item[pūnctum, pūnctī, n.]  "small hole
    \item[pungō, pungere, pupugī/pepugī, pūnctum] prick, puncture, sting, jab, poke; trouble, vex, disturb
    \item[purgō (1)] free from impurities or dirt, clean, purify; clear a space (of any obstructions, pests, or unwanted people); remove the outer covering, husk, or shell (of something); free from troubles, anxieties, etc.; absolve, exonerate, free (someone from charges); apologize (i.e., purge onesel of an offense)
    \item[purpura, purpurae, f.] shellfish yielding a purple dye; purple dye; purple-dyed cloth (associated with senators, knights, and royalty); purple color
    \item[pūrus, pūra, pūrum] clean, free from dirt, pure; unadulterated, unmixed, absolute, natural; unconditional, without exception, absolute; faultless
    \item[putō (1)] \marginnote{*}clean, make clean; go over in the mind, ponder; estimate, assess; consider (to be), regard (as), deem; think, suppose, believe (that)
    \item[quadrātum, quadrātī, n.] a (geometric) square; a square object, something square
    \item[quaerō, quaerere, quaes(i)ī/quaesīvī, quaesītum] \marginnote{*}try to find, search for, hunt for, seek, look for; ask to see, ask for; try to obtain, strive for, seek; require, demand, need; try to bring about, aim at, intend, try (to); seek to know about, inquire about, inquire into, examine, consider; ask a question; hold a judicial inquiry into, investigate, try a case
    \item[quaestiō, quaestiōnis, f.] the act of searching; examination, interrogation; judicial investigation, inquiry; investigation, research, inquiry; subject of discussion or dispute, problem, question, issue
    \item[quālis, quāle] \marginnote{*}of what sort, kind, or nature; what kind of a, what sort of a
    \item[quamdiū] (interrogative) For how long?; (exclamatory) How long! What a long time!; (relative) for what length of time, (as) long as, until, during
    \item[quamquam] \marginnote{*}(introducing a subordinate concessive clause) however much, although; (introducing a main clause) admittedly, to be sure
    \item[quamvis] (adverb) as much as you will, as much as you like, however much; very much, exceedingly, as much as possible; (conjunction) although, even though, however much, despite the fact that
    \item[quandōquidem] inasmuch as, seeing that, since
    \item[quantitās, quantitātis, f.] magnitude, multitude, quantity, degree, size, amount
    \item[quantus, quanta, quantum] \marginnote{*}(interrogative) of what size, amount, quantity, degree, importance? how much? how many? how big?; (relative) of what size, degree, amount, number, importance
    \item[quantusvīs, quantavīs, quantumvīs] of whatever size, amount, or degree (you wish); however great, however much
    \item[quapropter] Interrog., for what, wherefore, why
    \item[quārē] \marginnote{*}(interrogative) in what way? how?, why? for what reason?; (relative) by which means, whereby, because of which, for which reason, why; for the reason that, because; (introducing a new sentence) therefore, hence, for this reason
    \item[quārtus, quārta, quārtum] fourth
    \item[quasi] \marginnote{*}as if, just as, as though; as for example, say; (qualifying expressions of number, measurement, or the like) as good as, practically, more or less
    \item[quātenus] to what point, how far; as far as, to the distance that; until when, how long; how far, to what extent; in so far as, inasmuch as
    \item[quattuor] \marginnote{*}four
    \item[-que] \marginnote{*}and
    \item[quemadmodum] \marginnote{*}(interrogative) in what way? how?; (relative) in the manner in which, in which manner, just as, as
    \item[queō, quīre, quīvī/quiī, quitum] be able, can
    \item[quī, quae, quod] \marginnote{*}who, which, that
    \item[quia] \marginnote{*}because
    \item[quīcumque, quaecumque, quodcumque] \marginnote{*}\textit{also -cunque} whoever, whatever; anyone whatever, anything whatever
    \item[quīdam, quaedam, quoddam] a certain, some; a certain number, some number; a certain amount, some amount; (substantive neuter) \textit{quiddam}
    \item[quidem] \marginnote{*}certainly, indeed; if nothing else, at any rate (particularizing and emphasizing a preceding word or phrase; often found with personal pronouns, especially \textit{ego}
    \item[quiēs, quiētis, f.] (the rest of) sleep; rest, repose, relaxation; idleness, inactivity, inaction; absence of noise, stillness; a calm, peaceful state (of mind or of the world)
    \item[quiēscō, quiēscere, quiēvī, quiētum] \marginnote{*}(rest in) sleep, fall asleep; rest, relax (from work, pain, etc.), take a rest; take no action, do nothing, be still, stand by; say nothing, be quiet; be peaceful, make no disturbance, be calm
    \item[quīlibet, quaelibet, quodlibet/quidlibet] whatever, whichever, any (without distinction of any kind); a certain, some
    \item[quīn] \marginnote{*}(interrogative) why not?; (as an adverbial particle) indeed, in fact; (with \textit{etiam} or \textit{et}) yes, and\dots, and furthermore; (conjunction) so as to prevent, so that\dots not
    \item[quīnque] \marginnote{*}five
    \item[quīntus, quīnta, quīntum] fifth
    \item[quis, quid] \marginnote{*}who? which (one)? what?
    \item[quisnam, quaenam, quidnam] who (in fact)?, what (in fact)?, who (ever)?, what (ever)? (strengthened form of \textit{quis, quid})
    \item[quispiam, quaepiam, quodpiam] one or other, an unspecified, some; a particular (but unspecified), a certain, some
    \item[quisquam, quicquam/quidquam] \marginnote{*}any, anyone, anything
    \item[quisquis, quidquid/quicquid] \marginnote{*}anyone who, anything that; everyone who, everything that; all who, all that; whoever, whatever
    \item[quīvīs, quaevīs, quodvīs] whatever person you please, whatever thing you please; whoever, whatever; anyone, anything
    \item[quōdammodo/quōdam modo] in a certain manner, in a certain measure, to some degree, to a certain degree; so to speak
    \item[quōmodō/quōmodo] (interrogative) in what way? how?; in the manner in which, as; to the extent to which, as far as
    \item[quoniam] \marginnote{*}as soon as, after; seeing that, now that; since, inasmuch as, because
    \item[quoque] \marginnote{*}in the same way, too, likewise, no less; besides, as well, also, too
    \item[quotiēns] \marginnote{*}(interrogative) how often? how many times? (relative) as often as, whenever; the number of times that, as many as the times that
    \item[ratiō, ratiōnis, f.] \marginnote{*}reckoning, calculation (the action or the result of the action); proportion, relation; reasoning, reckoning; theory; an explanation, reason (for), ground; an account; (exercise or faculty of) reason; an affair, concern, business; plan of action, policy, scheme; condition, nature, kind, fashion, way
    \item[ratiōnālis, ratiōnāle] derived from or concerned with reason, theoretical, dialectical; possessing reason, rational
    \item[reālitās, reālitātis, f.] real nature, reality, the quality of being real or having actual existence
    \item[reāliter] really, in fact, in reality
    \item[recēnseō, recēnsēre, recēnsuī, recēnsum] count, enumerate, make a review or census of
    \item[recordor (1)] call to mind, recollect, remember
    \item[recurrō, recurrere, recurrī, recursum] run back, hurry back, return; run in reverse; return, come back, recur; revert, go back (in condition); have recourse to, fall back on
    \item[reddō, reddere, reddidī, redditum] \marginnote{*}give back, restore, return (something), repay, put back; restore; throw back, reflect, echo; say in reply, answer; reproduce, repeat; pay, discharge (a debt)
    \item[redeō, redīre, rediī, reditum] \marginnote{*}come or go back, return, go back; revert, return, be restored (to a state or condition); recur, come back, return
    \item[redūcō, redūcere, redūxī, reductum] lead back, bring back,  conduct back, escort back; pull back, draw back; recall, restore, bring back (to a condition, state, situation, etc.); bring, reduce (to a state); bring down, reduce (in degree or quality)
    \item[referō, referre, retulī, relātum] \marginnote{*}bring back, carry home; bring again; move or force back, withdraw, bring or draw back; (reflexive and passive as intransitive) go back, return; revert, return (to something, to a situation, etc.); report, bring back (news, a message, or reply); record, enter, write down; ascribe, refer, put down; give back, give in return, restore, pay back, render; recall, mention, relate
    \item[regō, regere, rēxī, rēctum] \marginnote{*}keep straight, direct, guide; manage, steer, guide (the course of); direct the activities of, control, direct, govern, rule, command
    \item[rēiciō, rēicere, rēiēcī, rēiectum] throw, drive, thrust, or turn back; drive away, repulse, beat off; repel, deter, prevent; remove, send out of the way; throw away, discard, abandon; dismiss, cast aside; refuse (to accept, admit, or adopt), spurn, rebuff, reject; refer, hand over, transfer; put off, postpone
    \item[relābor, relābī, relāpsus sum] move or fall gradually back, slip or slide back; recede, ebb; relapse, revert, slip back, fall back
    \item[reliquiae, reliquiārum, f. pl.] what remains, the remnants, the remains, the rest; remaining members, survivors; vestiges, traces (what remains after an activity, action, condition, etc.)
    \item[reliquus, reliqua, reliquum] \marginnote{*}the rest of, the remaining, the other; left, remaining (in some place, condition, or state), surviving; future, further
    \item[remaneō, remanēre, remānsī, remānsum] stay or remain behind; remain in position, stay (where one is); be left, remain, continue to be, presist, endure
    \item[removeō, removēre, remōvī, remōtum] \marginnote{*}move back or away, remove; banish, do away with, remove; debar, disqualify; set aside, leave out of account
    \item[reor, rērī, ratus sum] \marginnote{*}think, imagine, suppose, deem, hold a belief or opinion
    \item[reperiō, reperīre, rep(p)erī, repertum] \marginnote{*}find by looking, discover, find by inquiry or consideration; light upon, acquire, get; discover, get to know, find; find to be, find (in a condition or situation); make up, devise, invent
    \item[repetō, repetere, repetīvī, repetiī, repetītum] \marginnote{*}return to, make for again, make one's way back (to); attack again, go for again; attack or go after in retaliation; resort again to; repeat; take steps to recover, get back, seek in return, seek to restore; demand, claim back, call back, recall
    \item[repleō, replēre, replēvī, replētum] replenish, refill, fill again, reoccupy; restore (to full number, strength, etc.); make up, supply (a deficiency); fill up, occupy the whole of; sate, stuff, satiate
    \item[repraesentō (1)] resent to view, exhibit, show, present; revive, bring back into the present; serve as teh equivalent of, represent; represent (in art), portray; represent (in thought or words), portray; make immediately available, bring on at once; pay (a sum), pay at once
    \item[repugnō (1)] offer resistance (to) fight back, defend oneself; fight or rebel (against), struggle (against doing something), strive (to prevent); object (to) quarrel (with), protest (against); be contrary to, be inconsistent (with), be hostile (to), clash (with)
    \item[requīrō, requīrere, requīsīvī/requīsiī, requīsītum] try to find, seek, look for; ask or inquire about; ask, demand; try to obtain or bring about, seek; look for, expect to find; need, stand in need of; feel the loss of, miss
    \item[rēs, reī, f.] \marginnote{*}property, wealth; goods, possession; a thing; matter, situation, affair, business, (often in this sense) a court case; fact, deed; activity, practice; purpose, object
    \item[resolvō, resolvere, resolvī, resolūtum] untie, unfasten, unbind, loose, loosen, release; disclose, show, reveal; annul, cancel, abolish, destroy
    \item[respiciō, respicere, respexī, respectum] \marginnote{*}look back, look away (from something), look round; look back and see, notice behind one; look around for (something or someone needed), look to (for help or protection); review, look back on; turn one's thoughts or attention to; take notice of, take account of, heed; have regard for, show concern for; have reference (to), relate (to) be the concern (of)
    \item[respondeō, respondēre, respondī, respōnsum] \marginnote{*}reply, respond (by voice or in writing), answer; answer a charge, speak in defense, say in refutation, reply to an argument, offer an opposite point of view; answer a summons, present oneself; be consistent with, agree or accord (with), conform (to)
    \item[respōnsiō, respōnsiōnis, f.] reply, answer, response; refutation
    \item[retineō, retinēre, retinuī, retentum] \marginnote{*}hold back, hold fast, detain, confine, keep from escaping, prevent from being taken away; hold back, stop, check, delay, restrain; keep hold of, grasp, cling to; continue to use, continue to have, retain, continue to observe, keep up, maintain
    \item[rēvērā] \marginnote{*}in fact, in reality, truly
    \item[revolvō, revolvere, revoluī, revolūtum] roll back, roll aside; unroll, roll back (to the start of); go back over (something in thought or speech), think or speak over; cause to move in a circular course, make revolve, turn (something) around; cause to return, bring around again; fall back again, relapse; fall back on; revert to, come back (to a topic or argument)
    \item[revertor, revertī, reversus sum] turn back, move back, return, come back; have recourse (to), fall back (on); go back, revert, return, change back; come around again, recur, be renewed; relate back
    \item[rotundus, rotunda, rotundum] round, circuluar, spherical, rounded; smooth, well-rounded (positive term for rhetorical style)
    \item[rūrsus] \marginnote{*}back, backwards; in reverse; once again, a second time; in one's turn; on top of that, in addition, besides; on the other hand, on the contrary; now again, at another moment
    \item[saltem] at least, at all events, anyhow; (in negative sentences) even, so much as
    \item[sanguis, sanguinis, m.] \marginnote{*}blood; bloodshed
    \item[sānus, sāna, sānum] \marginnote{*}physically sound, healthy; wholesome, causing health, healthy; undamaged, unimpaired; mentally sound, sane, sensible, reasonable, sober
    \item[sapiō, sapere, sapīvī/sapiī, ———] taste (like something); have a good taste; smell of; (figurative) be a sign of, give indication of; have taste or discernment; be intelligent, show good sense; know, understand; be in one's right mind, be sane
    \item[sapor, sapōris, m.] taste, flavor; distinctive quality, distinctive character; sense of taste; smell, odor
    \item[satis/sat] (adjectival) enough, sufficient, satisfactory; (adverbial) enough, sufficiently
    \item[satyriscus, satyriscī] little satyr, satyr
    \item[scientia, scientiae, f.] \marginnote{*}knowledge, awareness, mental grasp; knowledge (as opposed to mere belief); understanding, expert knowledge; a particular branch of knowledge, an art, skill, or technical expertise; learning, erudition, wide knowledge
    \item[scīlicet] it is evident or clear (that), one may be sure (that); it is obvious; naturally, you may depend on it; as is apparent, evidently, certainly, no doubt; (ironic) to be sure, doubtless; I mean, of course; that is to say, namely
    \item[sciō, scīre, scīvī/sciī, scītum] \marginnote{*}know, understand, perceive, be aware; have (certain) knowledge of (as opposed to mere belief);  have knowledge of, be skilled in
    \item[scopus, scopī, m.] goal, target, aim
    \item[scrībō, scrībere, scrīpsī, scrīptum] write; scratch, engrave, draw
    \item[scrīptum, scrīptī, n.] a work of writing, treatise, book, work, text; literal meaning
    \item[sēcēdō, sēcēdere, sēcessī, sēcessum] detach oneself, withdraw, move away (often to a private place); secede (i.e., withdraw in protest or revolt), dissociate oneself; withdraw from Rome to the country (for vacation or in retirement from public life)
    \item[sēcernō, sēcernere, sēcrēvī, sēcrētum] sunder, sever, separate, divide; distinguish, discern; set aside, reject, separate
    \item[sēcius/sētius] to a lesser degree, less readily; \textit{nihilō sēcius/sētius} none the less, for all that, just the same
    \item[secundus, secunda, secundum] \marginnote{*}following; favorable, supporting, helpful, propitious; second
    \item[sēcūrus, sēcūra, sēcūrum] \marginnote{*}free from care, fear, anxiety, untroubled, undisturbed, tranquil; negligent, indifferent, nonchalant, careless, perfunctory (i.e., lacking proper concern or care); free from danger, safe, immune (from something negative)
    \item[sed] \marginnote{*}but; however
    \item[semel] \marginnote{*}once, a single time, the first time; only once, just once, once and for all; at any one time, once, ever; at one and the same time, simultaneously
    \item[semper] \marginnote{*}ever, always, at all times, continually, perpetually, forever
    \item[sēnsus, sēnsūs, m.] sensation, capacity to perceive by the senses; any one of the five senses; a sensation, any impression by a sense; the faculties of perception; self-awareness, consciousness, awareness (in general or of something other than oneself); judgment, the faculty of making distinctions, sensibility, perception of what is appropriate or right; a mental feeling, emotion, thought, idea; character, disposition
    \item[sentiō, sentīre, sēnsī, sēnsum] \marginnote{*}perceive by one of the senses; (less literally) perceive, feel, discern, recognize, become aware of; hold or express a given belief or opinion, think, feel, opine; mean, intend, have in mind
    \item[sēparō (1)] divide (off), separate, split up, divide up; keep separate, cut off, isolate, part; debar, exclude; separate (in thought or writing) treat as distinct, exclude from consideration
    \item[sequor, sequī, secūtus (sequutus) sum] \marginnote{*}follow, go after, go behind; pursue, chase, follow; follow in time or sequence, succeed, come after; follow from, follow logically; escort, accompany, attend
    \item[seriēs, seriēī, f] a series, a row, a line, procession, connected system; a continuous series, succession, sequence; a line of ancestors or descendants
    \item[sēriō] seriously, in earnest, soberly, not lightly or playfully
    \item[sērius, sēria, sērium] weighty, important, serious; grave or serious in appearance or character
    \item[sētius/sēcius] to a lesser degree, less readily; \textit{nihilō sētius/sēcius} none the less, for all that, just the same
    \item[seu/sīve] \marginnote{*}or if; (repeated) whether\dots or; if\dots or
    \item[sextus, sexta, sextum] sixth
    \item[sī] \marginnote{*}if, supposing that
    \item[sīc] \marginnote{*}so, thus, in this manner, just so; as follows; in this way, so\dots (that), in such a way (that)
    \item[sīcut/sīcutī] just as, as, so as, even as
    \item[sīcutī] see \textit{sīcut}
    \item[significātiō, significātiōnis, f.] the action of giving signs or signals; action or fact of conveying information; outward sign, expression, intimation, indication; suggestion, hint; the meaning, sense (of a word, expression, work, etc.)
    \item[signum, signī, n.] sign, mark, token, indication
    \item[similis, simile] \marginnote{*}like, similar; \textit{vērī/vērō similis} resembling the truth, likely, reasonable, plausible, probable
    \item[similitūdō, similitūdinis, f.] similarity, likeness, resemblance; a similar thing, that which resembles something, a likeness; a simile, analogy, comparison
    \item[simplex, simplicis] having a single layer, fold, etc., single, onefold; simple, uncompounded, undecorated, plain; alone, standing alone, separate; free from complications, straightforward, simple; artless, ingenuous, innocent, direct, candid
    \item[simul] \marginnote{*}at the same time, together; at once, as soon as
    \item[sine] \marginnote{*}\textit{preposition with ablative} without
    \item[singulī, singulae, singula] \marginnote{*}one apiece, one to each; each one, every single; taken separately, one by one, individual
    \item[sīquidem] (as a strong conditional) if it is really possible that, if indeed; (concessive) even supposing; (adding a caveat or rider) at any rate if, always assuming that; (qualifying an assertion) if it is really the case that
    \item[sīrēna] a siren
    \item[sīve/seu] \marginnote{*}or if; (repeated) whether\dots or; if\dots or
    \item[soleō, solēre, solitus sum] \marginnote{*}be accustomed (to), make it a practice (to); be liable (to), be likely (to), be apt (to); (with neuter pronoun or after \textit{ut}) be the common practice, be the norm, be the usual case
    \item[solus, sola, solum] \marginnote{*}alone, only; lonely; only one, single, sole
    \item[solvō, solvere, solvī, solūtus] \marginnote{*}release, set free; dissolve, take apart; resolve, settle, solve
    \item[somniō (1)] dream; daydream; have idle thoughts about, have delusions about
    \item[somnium, somniī, n.] a dream, vision; idle hope, fantasy, delusion, daydream
    \item[somnus, somnī, m.] \marginnote{*}sleep; sleepiness, drowsiness; (euphemistically) (eternal) rest, death
    \item[sonus, sonī, m.] sound, noise; articulate sound, speech
    \item[sōpiō (4)] cause to sleep, put to sleep, overcome with sleep; render unconscious, knock out
    \item[spatium, spatiī, n.] \marginnote{*}ground used for athletics or horse racing, a course; a circuit or lap of a racecourse; area, space, room; surface area, extent, size; a stretch of time, period, term, duration
    \item[spectō (1)] look at, gaze at, watch, observe; (of places) to look, face, lie; examine, test, try, consider
    \item[speculātīvus, speculātīva, speculātīvum] watchful, attentive, vigilant; contemplative, characterized by or engaged in contemplation; speculative, theoretical, characterized by, engaged in, or obtained by (purely) intellectual inquiry
    \item[spērō (1)] \marginnote{*}hope (for), look forward (to); hope (that); anticipate, apprehend
    \item[sponte] voluntarily, of one's own will, without prompting, of one's own accord; deliberately, purposely; unaided, without help; spontaneously, of itself
    \item[stabiliō (4)] make firm, make steady; fix or establish firmly
    \item[statim] \marginnote{*}immediately, at once, without delay; \textit{with ē/ex} immediately after, straight from
    \item[statuō, statuere, statuī, statūtum] \marginnote{*}cause to stand, set up, set, establish; build, put up (a structure); establish, found (a city); decide, determine, judge, deem
    \item[sternō, sternere, strāvī, strātum] smooth (something), make (something) level or smooth; strike down, stretch lifeless, lay low; overthrow, defeat utterly
    \item[strātum, strātī, n.] bed-covering, quilt; bed-sheet
    \item[strepitus, strepitūs, m.] a sound, a noise (inexpressive or meaningless sound or noise, not speech); clamor, uproar; din, turmoil
    \item[studeō, studēre, studuī, ———] \marginnote{*}give attention, be eager, be zealous; take pains, be  diligent, be busy with, be devoted, apply oneself; strive after, pursue, desire, wish
    \item[stupor, stupōris, m.] numbness, insensibility; bewilderment, stupefaction; dullness of apprehension, stupidity; (metonymically) a stupid person, a clod
    \item[sub] \marginnote{*}\textit{preposition with accusative and ablative} (with accusative) to a position below or underneath, under; down to, down into, down under; up to, to the edge of; just before (a time or event); until, up to; directly after, in response to, as a consequence of; (with ablative) under, below, beneath, underneath; at the foot of, below; immediately behind, next to
    \item[subdūcō, subdūcere, subdūxī, subductum] draw up, raise, lift; withdraw, draw off, lead away, take away, subtract, remove; steal
    \item[substantia, substantiae, f.] essence, contents, material, substance
    \item[subtīlitās, subtīlitātis, f.] fineness, extreme slenderness; fineness of detail, delicate work; fineness of perception, acuteness, refinement; attentiveness to finer points, subtlety, minute thoroughness
    \item[succēdō, succēdere, successī, successum] move below, come under; come to the foot (of), come up (to), come as far (as), approach close (to); advance to a higher level, move upwards, advance up; move up into the position (of), move up (somewhere) as a replacement or relief; become successor (to someone), take over (from), succeed (to an office, etc.), take the place (of); come after; succeed, turn out well, prosper
    \item[sufficiō, sufficere, suffēcī, suffectum] supply, provide (especially as a replacement); appoint to a magistracy (in place of another or in case of vacancy); substitute (one thing for another); have sufficent strength, be equal (to), stand up (to); be sufficient (in quantity, extent, degree, etc.), suffice
    \item[suffodiō, suffodere, suffōdī, suffossum] dig or tunnel under; undermine (i.e., dig under a wall so as to weaken its foundations); pierce below
    \item[suī] \marginnote{*}(\nth{3} person reflexive pronoun) himself, herself, itself, themselves
    \item[sum, esse, fuī, futūrum] \marginnote{*}be, exist; be real, be true
    \item[summus, summa, summum] highest, topmost, uppermost; the top or summit (of); latest (in time or sequence), final; developed to the height of excellence, perfect; highest in rank, supreme, most exalted
    \item[sūmō, sūmere, sūmpsī, sūmptum] \marginnote{*}\marginnote{*}take, take up, take in hand; choose, select, adopt; lay hold of; assume, assume, maintain, suppose, affirm, take for certain or for granted; undertake, begin
    \item[super] \marginnote{*}\textit{adverb and preposition with accusative and ablative} (adverb) over, above, in a higher position; on the surface or upper part, on top (of); in addition, besides (= \textit{insuper}); after what has been taken, left over, remaining (often elliptical for \textit{superest}); to an excess degree, more than sufficiently, too much, in excess; (with accusative) over, above, beyond; in close succession, (soon) after; in addition to, over and above, besides; beyond, more than, to a greater degree or extent than, above; (with ablative) higher than, over; on the top of, on the uppermost part of; about, concerning, on; in close succession to, on top of; in addition to, over and above
    \item[superaedificātus, superaedificāta, superaedificātum] built on top of (not in \textbf{OLD})
    \item[superextruō, superextruere, superextrūxī, superextūctum] build or pile on top of or onto (not in \textbf{OLD})
    \item[supersum, superesse, superfuī, superfutūrum] \marginnote{*}be higher (than), be on top (of); be set (over); be superior (to); be additional to the requirements or needs (of), be superfluous, be in excess (of), be beyond the capacity (of); remain, be left over, survive, remain in existence; remain (to be done, performed, handled, etc.)
    \item[suppōnō, suppōnere, supposuī, suppositum] suppose (for the sake of argument or discussion), hypothesize; believe, suppose, presume (These are not classical meanings of the verb, but Descartes uses the verb this way.); put below, place under; substitute false, counterfeit, falsify
    \item[suprā] \marginnote{*}\textit{preposition with ablative} on the upper side (of), on the top (of), above; earlier than; more than, exceeding, beyond; in charge of, over, in command of
    \item[suspiciō, suspiciōnis, f.] a suspicion, mistrustful feeling; a slight idea, inkling; a faint indication, suggestion, trace
    \item[suspicor (1)] form an idea of, guess, imagine, infer; suspect, have an inkling of (something wrong); suspect, believe (a person) guilty (of something), be suspicious of, mistrust
    \item[suus, sua, suum] \marginnote{*}of oneself, belonging to oneself; (emphatic) one's very own, belonging to someone (and no other), particularly associated or characteristic of one; his own, her own, their own; his, her, its, their
    \item[taceō (2)] \marginnote{*}be silent, not speak, say nothing; say nothing about, omit mention of, pass over in silence (with \textit{dē} or accusative direct object)
    \item[tactiō, tactiōnis, f.] a touching, the act of touch
    \item[tactus, tactūs, m.] action or fact of physical contact, touch; sense of touch; tactile qualities (i.e. the touch or feel of something); contact, influence
    \item[tālis, tāle] \marginnote{*}of such a character, kind, or type; of such an exceptional (for good or bad) sort, such (a)
    \item[tam] \marginnote{*}to such a degree, to such an extent, to that extent, so, so much
    \item[tamdiū] (for) so long, all this time
    \item[tamen] \marginnote{*}all the same, nevertheless, yet, just the same, in spite of what has been said
    \item[tamquam] \marginnote{*}in the same way, to the same degree, just as; a kind of, quasi- (when applying a term to something improperly); (with conditional clause) just as (if); as for example; (with subjunctive) in the same way as if, as though; (with subjunctive) as though (introducing a hypothesis or something contrary-to-fact); (indicating a circumstance as the basis for some action) on the ground that
    \item[tandem] \marginnote{*}(for emphasis, expressing a strong sense of protest or impatience) really, I ask you, after all; after some time, at last, at length, finally
    \item[tangō, tangere, tetigī, tāctum] \marginnote{*}touch; be immediately next to, border on; arrive at, reach; (of feelings, etc.) touch, affect, affect with emotion; make slight mention of, touch on; (colloquial) deprive fraudulently, steal by cheating
    \item[tantum] \marginnote{*}to such an extent or degree; for such a time, for such a distance; only, just, merely
    \item[tantummodo] \marginnote{*}only, merely, just
    \item[tantus, tanta, tantum] \marginnote{*}so big, so great (in size, importance, degree, etc.), so much; (plural) so many, so vast a number of
    \item[temptō/tentō (1)] see \textit{tentō}
    \item[tempus, temporis, n.] \marginnote{*}(moment or period of) time; the time, the date (for something), (appointed) time; season, any recurrent period or phase; (usually plural) a period in history, times; proper or due time; favorable time, opportunity; circumstances (existing at a particular time), moment, occasion
    \item[tendō, tendere, tetendī, tēnsum/tentum] \marginnote{*}extend outwards or upwars, stretch or hold out, offer; direct, aim; stretch out, extend (in time or space), spread out; pitch camp; exert strain on, pull tight; direct (one's steps, course, etc.), proceed; reach (to or as far as); progress, move on (to another stage, condition, etc.); (intransitive) press on, insist; (intransitive) make an effort, exert oneself, strain; (with infinitive) strive, aim (to do)
    \item[tenebrae, tenebrārum, f.] \marginnote{*}darkness; (figurative) mental darkness, ignorance, lack of knowledge or understanding; obscurity, concealment, a condition where something is unknown or unobserved
    \item[tentō/temptō (1)] \marginnote{*}handle, touch, feel; test, seek to discover the state of; test, try out, attempt, try (to do); examine, try to find out about; make an attempt on, try to get possession of
    \item[tenuis, tenue] thin, slender, narrow, fine, fine-meshed; (of various substances) watery, rarefied, insubstantial; pure, clear, fine
    \item[terminō (1)] mark boundaries of; form a boundary of, border; define, delimit, determine the limits of; limit, restrict; fix or lay down a limit; bring to a close or end, conclude; settle, decide
    \item[terra, terrae, f.] \marginnote{*}earth; land, ground, soil
    \item[tertius, tertia, tertium] third
    \item[timeō (2)] \marginnote{*}fear, be afraid, be fearful, be apprehensive, be afraid of, dread, apprehend
    \item[toga, togae, f.] a covering, clothing; Roman outer garment, toga
    \item[tollō, tollere, sustulī, sublātum] \marginnote{*}pick up, raise, lift, hoist; climb up, ascend; take (on to a ship or vehicle), pick up, take on board; raise the spirits or morale of, hearten, rouse; pick up and remove, take away, carry off; carry away, reap; take, steal; take out, remove, exclude; get rid of, remove, eliminate, destroy, kill, do away with, eliminate
    \item[tōtus, tōta, tōtum] \marginnote{*}the whole of, all, complete, every part of, throughout the whole, all over the; free from defect or damage, unimpaired, entire
    \item[tractō (1)] keep on pulling or dragging, drag about; handle, work (with the hands), manipulate, treat manually; have dealings with, have to do with, deal with, treat (in some manner); (reflexive) conduct oneself; manage, handle, employ (affairs, means, resources); carry out, practice, perform; examine, consider; (intransitive) deliberate, carry on a discussion; deal with, discuss, treat (a theme, subject, idea)
    \item[trānseō, trānsīre, trānsīvī/trānsiī, trānsitum] \marginnote{*}come or go across, cross over; move on; transfer allegiance, go over; change one's nature, appearance, etc., be transformed; proceed, be in transit, pass through; go through, run through; go past, pass by; overtake, pass, pass beyond, go farther than
    \item[trānsferō, trānsferre, trānstulī, trānslātum] bear across, carry or bring over; change the location of, transfer, transpose, shift, transplant; transfer (something) from one person (place, etc.) to another, transfer control or possession of; translate; bring (someone, something) over to (something new); change, transform
    \item[trēs, tria] \marginnote{*}three
    \item[triangulāris, triangulāre] triangular
    \item[tribuō, tribuere, tribūtum] share out, divide, apportion; grant, bestow, award; allocate, devote, apply; attribute or ascribe (something to someone), impute, attribute (to something), ascribe (to a cause); place value (on), pay regard (to); give credit, pay respect (to)
    \item[tum/tunc] \marginnote{*}then, at that time, moment, date; next, after that; in addition, moreover, besides
    \item[turbō (1)] \marginnote{*}(intransitive) act turbulently, riot, revolt; (transitive) agitate, stir up, disturb
    \item[ubī/ubi] \marginnote{*}in what place? where?; where; when
    \item[ūllus, ūlla, ūllum] \marginnote{*}any, any at all; anyone, anything
    \item[ulterius] in or to a more distant place, farther away, furter; at or to a further point in time, stage, degree, or extent, further, more
    \item[ultimus, ultima, ultimum] \marginnote{*}most distant, farthest away, endmost; remotest in time, earliest; latest in time, final, ultimate, last (in sequence); final, critical, decisive
    \item[umquam] \marginnote{*}at any time, ever; at all
    \item[unde] \marginnote{*}from what place? where\dots from? whence? from which place, whence, from which (point, situation, source, etc.), from whom
    \item[ūniversālis, ūniversāle] having general application, universal, applying to all
    \item[ūnus, ūna, ūnum] \marginnote{*}one, a single
    \item[ūnusquisque, ūnaquaeque, ūnumquidque/ūnumquicque/ūnumquodque] each one, every single one
    \item[ūsitātus, ūsitāta, ūsitātum] familiar, everyday, commonly used or practiced
    \item[ut] \marginnote{*}as; when; how; (introducing various subordinate clauses)
    \item[ūter, ūtra, ūtrum] which (of two)?
    \item[ūtilitās, ūtilitātis, f.] use, usefulness, utility; benefit, profit, advantage
    \item[ūtor, ūtī, ūsus sum] \marginnote{*}use, make use of, put to use, employ; manage, handle, control; exercise, engage in, practice; (with adverb or predicate adjective) put to (such-and-such a) use; experience, undergo, enjoy
    \item[ūtrimque] from or on both sides or ends
    \item[vacō (1)] be vacant, empty, or unfilled; be without occupants or an owner; be destitute or devoid (of), be free (from); be left free for, be available to; be unengaged, have leisure, be free, have time to spare
    \item[validus, valida, validum] \marginnote{*}strong, powerful, robust, sturdy; thriving, flourishing; fit, in sound health; solid, substantial; powerful, vehement, intense
    \item[vapor, vapōris, m.] steam, exhalation, vapor; heat, warmth
    \item[varietās, varietātis, f.] difference, diversity, variety
    \item[vel] \marginnote{*}(non-exclusive) or; \textit{vel\dots vel} either\dots or
    \item[velut] \marginnote{*}as for example, for instance; in the same way that, just as, just like; as it were, so to speak; as if, as though; (giving a justification) as being
    \item[veniō, venīre, vēnī, ventum] \marginnote{*}come, approach, arrive
    \item[verbum, verbī, n.] \marginnote{*}a word; (sometimes specifically) a verb
    \item[vereor, verērī, veritus sum] \marginnote{*}fear, be afraid of; show reverence or respect for, be in awe of
    \item[vērisimilis, vērisimile] seeming true, appearing true, consistent with the truth, like the truth
    \item[veritās, veritātis, f.] truth; reality; the state of being real or actual
    \item[vertō, vertēre, vertī, versum] \marginnote{*}(cause to) turn, spin; depend on, turn on, hinge on; overturn, knock down, ruin; turn around, invert, reverse, transpose; (cause to) turn the other way; (cause to) turn tail, make flee, put to flight; turn or change the position or direction of (something)
    \item[vērumtamen] but even so, still, nevertheless
    \item[vērus, vēra, vērum] \marginnote{*}real, genuine, actual; true; proper
    \item[vestiō (4)] cover with a garment, provide with clothing, dress, clothe, cover
    \item[vestis, vestis, f.] an item of clothing, garment, piece of clothing; clothes (collectively); garments, clothing
    \item[vetō, vetāre, vetuī, vetitum] \marginnote{*}forbid, prohibit, hinder, prevent
    \item[vetus, veteris] \marginnote{*}old, having lived a long time; belonging to the past, old-time
    \item[via, viae, f.] \marginnote{*}road, track, path (made for travel); passage, channel, duct, course; a journey, march, fact or instance of traveling
    \item[vidēlicet] it is plain to see, it is clear (that); evidently, plainly; (ironic) of course, no doubt, obviously; that is to say, that is
    \item[videō, vidēre, vīdī, vīsum] \marginnote{*}see; notice, observe; be a witness of; meet, see (people, events, etc.); appreciate, perceive, note with understanding
    \item[vigil, vigilis, m.] a sentry, guard, person who keeps watch
    \item[vigil, vigilis] awake, watchful, wakeful; alert, vigilant, paying watchful attention
    \item[vigilia, vigiliae, f.] wakefulness, sleeplessness, lying awake; action or fact of keeping watch, a patrol, a guard; watchful attention, vigilance
    \item[vigilō] stay awake; be watchful, be alert
    \item[vīs, vis, f.] \marginnote{*}force, violence; compulsion, constraint; power, influence, strength
    \item[vīsiō, vīsiōnis, f.] the act or sense of seeing, vision; an appearance, sight; mental image
    \item[vīsus, vīsūs, m.] faculty or power of seeing, sight, vision; action of seeing, glance, gaze, sight; that which his seen, a sight
    \item[vīta, vītae, f.] \marginnote{*}life; a way of life, a mode of life
    \item[vītō (1)] \marginnote{*}move out of the way, avoid, dodge, keep out of the way of, keep clear of; steer clear of, shun, avoid
    \item[vitrum, vitrī, n.] glass; something made of glass
    \item[vix] \marginnote{*}with difficulty; hardly, scarcely, barely
    \item[vocō (1)] \marginnote{*}call, summon, invite; invoke, call upon; call together, convoke; designate, call by name, call (something)
    \item[volō, velle, voluī, ———] \marginnote{*}wish, want, desire; be willing, be prepared (to); be about (to), be on the point (of)
    \item[voluntās, voluntātis, f.] \marginnote{*}will, volition; one's will or wish, what one wants to happen; readiness (to do or agree to someting), willingness, approval; choice, option (as opposed to compulsion); deliberate choice; intention, inclination; disposition; favorable disposition, goodwill, sympathy
    \item[vox, vōcis, f.] \marginnote{*}(the human) voice; a sound (produced by voice), utterance; sound (in general); (singular or plural) spoken utterance, words
    \item[vulgō] \marginnote{*}publicly; in the usual way, according to the general rule or practice, generally, commonly; commonly, habitually, regularly; all together, en masse; far and wide, all over the place, at wide
    \item[vulgus, vulgī, n. (m.)] \marginnote{*}the common people, general public; a multitude, crowd (often derogatory); a flock or group of animals
    \item[vultus, vultūs, m.] \marginnote{*}facial expression, look, countenance; face, front of the head; a surface, a face (of an object); one's gaze, one's view; appearance of a face, looks, features; (of a physical object) outward appearance, face; (of an abstract thing) aspect, appearance
\end{description}
% -]] Vocabulary


\backmatter
    \markboth{Bibliography}{Bibliography}
    \printbibliography

\end{document}
% -]] Document
